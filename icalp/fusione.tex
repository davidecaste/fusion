\documentclass[a4paper,UKenglish,cleveref,pdftex, thm-restate,numberwithinsect,anonymous]{lipics}
\newif\ifreport
\reporttrue
%\reportfalse
 
% full includes some additional material
\newif\iffull
\fulltrue
%\fullfalse

% additional text
\iffull
\newcommand{\full}[1]{#1}
%\newcommand{\short}[1]{}
\else 
\newcommand{\full}[1]{}
%\newcommand{\short}[1]{#1}
\fi


\ifreport
\nolinenumbers %uncomment to disable line numbering
\hideLIPIcs  %uncomment to remove references to LIPIcs series (logo, DOI, ...), e.g. when preparing a pre-final version to be uploaded to arXiv or another public repository
\else
%\relatedversion{An extended version is available at \url{https://arxiv.org/abs/2007.04213}.} %optional, e.g. full version hosted on arXiv, HAL, or other respository/website
%\relatedversion{A full version of the paper is available at \url{...}.}
\fi

\bibliographystyle{abbrv}% the mandatory bibstyle



\usepackage{hyperref}
\usepackage[textsize=tiny]{todonotes}
%\usepackage{etex}
\usepackage[all]{xy}
\SelectTips{cm}{}
% derivations and labels
\usepackage{proof}
\newcommand{\lab}[1]{\ensuremath{\mathsf{{#1}}}}
\newcommand{\slab}[1]{\ensuremath{\scriptstyle{\mathsf{{#1}}}}}
\usepackage{wrapfig}


%frecce
\newcommand{\mor}{\mathsf{Mor}}
\newcommand{\mon}{\mathsf{Mono}}
\newcommand{\reg}{\mathsf{Reg}}

% ``boxed'' \infer command
\newcommand{\binfer}[3][]{
  \mbox{\infer[#1]{#2}{#3}}}

% Command for labels on the left side of the rule
%       \inferL{<name>}{<post>}{<pre>}
% generates:
%             <pre>
%      <name> ------
%             <post>
%
\newlength{\myheight}
\newcommand{\inferL}[3]
  {\settoheight{\myheight}{\mbox{${#2}$}}
   \raisebox{\myheight}{{#1}}
   \makebox[1mm]{}
   \mbox{\infer{#2}{#3}}
}

\usepackage{amssymb,graphicx,epsfig,color}
%\usepackage[scriptsize]{subfigure}
\usepackage{subcaption}
\usepackage{wrapfig}

% re-stating 
%\usepackage{thm-restate}

% showing labels
%\usepackage[inline]{showlabels}

\usepackage{pgf}
\usepackage{tikz}
\usepackage{tikz-cd}
\usetikzlibrary{arrows,shapes,snakes,automata,backgrounds,petri,fit,positioning}
\tikzstyle{node}=[circle, draw=black, minimum size=1mm]
\tikzstyle{trans}=[font=\scriptsize]
\tikzstyle{lab}=[font=\small]


\newcommand{\pgfBox}{
  \begin{pgfonlayer}{background} 
    \fill[blue!2,thick,draw=black!50,rounded corners,inner sep=3mm] ([xshift=-1.5pt,yshift=-1.5pt]current bounding box.south west) rectangle ([xshift=1.5pt,yshift=1.5pt]current bounding box.north east);
  \end{pgfonlayer}
}
\usepackage{scalerel}
\newcommand{\smallmin}{\scaleobj{0.6}{-}}
\newcommand{\Deltamin}{\Delta^{\hspace{-1pt}\downarrow\hspace{-1pt}}}
\newcommand{\Rrel}[1]   {\stackrel{{#1}}{\Longrightarrow}}
\newcommand{\oa}{\overline a}
\newcommand{\ob}{\overline b}
\newcommand{\oc}{\overline c}
\newcommand{\od}{\overline d}
\newcommand{\rec}{\emph{rec}}
\newcommand{\fn}[1]{{\mathtt{fn}}(#1)}
%\usepackage{latexsym}
\usepackage{stmaryrd}
\def\encodep#1{\llfloor#1\rrfloor}

\newcommand{\cat}[1]{\ensuremath{\mathbf{#1}}}

\newcommand{\dpo}{\textsc{dpo}}

% base classes of categories for adhesive and quasi adhesive case
\newcommand{\bAdh}{\ensuremath{\mathbb{B}}}
\newcommand{\bQAdh}{\ensuremath{\mathbb{QB}}}

%from pawel
\usepackage{amsmath}
\usepackage{amssymb}
\usepackage{amsthm}
\usepackage{enumerate}
\usepackage{xspace}
\usepackage{amsfonts}
\usepackage{mathrsfs}
\usepackage{cite}
\usepackage{float}
\usepackage{fancybox}
\usepackage{proof-at-the-end}
\usepackage{cleveref}


%\spnewtheorem*{notation}{Notation}{\bfseries}{\rmfamily}

%%%%%%%% MATHEMATICAL NOTATION %%%%%%%%%%%%%%%%%%%%%%%%%%%%%%%%%%%%%%%%%

%symbol for natural numbers
\newcommand{\nat}{\ensuremath{\mathbb{N}}}

% finite subset
\newcommand{\sfin}{\ensuremath{\subseteq_{\mathit{fin}}}}

% flattening of a multiset
\newcommand{\flt}[1]{\ensuremath{[\![{#1}]\!]}}

% compact elements
\newcommand{\compact}[1]{\ensuremath{\mathop{\mathsf{K}({#1})}}}

% principal ideal
\newcommand{\principal}[1]{\ensuremath{\mathop{\downarrow\!{#1}}}}

% ideal completion
\newcommand{\ideal}[1]{\ensuremath{\mathsf{Idl}({#1})}}

% complete prime elements
\newcommand{\pr}[1]{\ensuremath{\mathop{\mathit{pr}({#1})}}}
\newcommand{\wpr}[1]{\ensuremath{\mathop{\mathit{wpr}({#1})}}}

% irreducible elements
\newcommand{\ir}[1]{\ensuremath{\mathop{\mathit{ir}({#1})}}}

% difference of irreducible elements
\newcommand{\diff}[2]{\ensuremath{\delta({#1},{#2})}}

% immediate precedence

% abbreviation for event structure
% \newcommand{\esabbr}{event structure}
\newcommand{\esabbr}{\textsc{es}}
\newcommand{\esnabbr}{\textsc{esnb}}
\newcommand{\esnmabbr}{\textsc{esn}}
\newcommand{\eseqabbr}{\textsc{epes}}

% predecessor of an irreducible
\newcommand{\pred}[1]{\ensuremath{\mathit{p}({#1})}}

% irreducible elements in es domains
\newcommand{\esir}[2]{\ensuremath{\langle{#1}, {#2}\rangle}}

% equivalence classes [of irreducibles]
\newcommand{\eqclass}[2][]{\ensuremath{[{#2}]_{\scriptscriptstyle {#1}}}}
% union of the equivalence classes of the elements in a set
\newcommand{\eqclasscup}[2]{\ensuremath{{#2}_{\scriptscriptstyle {#1}}}}

\newcommand{\eqclassir}[1]{\ensuremath{\eqclass[\leftrightarrow^*]{#1}}}

% quotient of set wrt a relation
\newcommand{\quotient}[2]{\ensuremath{{#1}_{\scriptscriptstyle {#2}}}}

% category of event structures 
\newcommand{\es}{\ensuremath{\mathsf{ES}}}
% category of stable event structures 
\newcommand{\ses}{\ensuremath{\mathsf{sES}}}
% category of prime event structures 
\newcommand{\pes}{\ensuremath{\mathsf{pES}}}
% category of prime event structures with equivalence
\newcommand{\epes}{\ensuremath{\mathsf{epES}}}

% category of connected event structures 
\newcommand{\ces}{\ensuremath{\mathsf{cES}}}

% category of weak prime algebraic domains domains 
\newcommand{\WDom}{\ensuremath{\mathsf{wDom}}}
% category of domains
\newcommand{\Dom}{\ensuremath{\mathsf{Dom}}}
% category of prime algebraic domains
\newcommand{\PDom}{\ensuremath{\mathsf{pDom}}}


%%%%% NON BINARY CONFLICT

% category of event structures 
\newcommand{\esn}{\ensuremath{\mathsf{ES_{nb}}}}
% category of stable event structures 
\newcommand{\sesn}{\ensuremath{\mathsf{sES_n}}}

% category of connected event structures 
\newcommand{\cesn}{\ensuremath{\mathsf{cES_{nb}}}}

% category of prime event structures 
\newcommand{\pesn}{\ensuremath{\mathsf{pES_n}}}

% category of fusion domains 
\newcommand{\WDomb}{\ensuremath{\mathsf{wDom_b}}}
% category of domains
\newcommand{\Domb}{\ensuremath{\mathsf{Dom_b}}}
% category of prime algebraic domains
\newcommand{\PDomb}{\ensuremath{\mathsf{pDom_b}}}

%%%%% END NON BINARY CONFLICT


% slice category
\newcommand{\slice}[2]{\ensuremath{({#1} \downarrow {#2})}}


% event structure for a domain
\newcommand{\zev}[0]{\ensuremath{\mathcal{E}}}
\newcommand{\ev}[1]{\ensuremath{\zev({#1})}}

% from general to connected event structures
\newcommand{\zconnes}[0]{\ensuremath{\mathcal{C}}}
\newcommand{\connes}[1]{\ensuremath{\zconnes({#1})}}
% and inclusion
\newcommand{\zinces}[0]{\ensuremath{\mathcal{I}}}
\newcommand{\inces}[1]{\ensuremath{\zinces({#1})}}


% stable version
\newcommand{\zsev}[0]{\ensuremath{\mathcal{E}_S}}
\newcommand{\sev}[1]{\ensuremath{\zsev({#1})}}

% with equivalence
\newcommand{\zeveq}[0]{\ensuremath{\mathcal{E}_{eq}}}
\newcommand{\eveq}[1]{\ensuremath{\zeveq({#1})}}

% es with equivalence to es and vice
\newcommand{\zfuse}[0]{\ensuremath{\mathcal{M}}}
\newcommand{\fuse}[1]{\ensuremath{\zfuse({#1})}}
\newcommand{\zunf}[0]{\ensuremath{\zunf}}
\newcommand{\unf}[1]{\ensuremath{\mathcal{U}({#1})}}



% Winskel/Droste version
\newcommand{\zevwd}[0]{\ensuremath{\mathcal{E}_{wd}}}
\newcommand{\evwd}[1]{\ensuremath{\zevwd({#1})}}




% configurations of an event structure
\newcommand{\conf}[1]{\ensuremath{\mathit{Conf}({#1})}}
% finite configurations
\newcommand{\conff}[1]{\ensuremath{\mathit{Conf_F}({#1})}}

% product of the sets of minimal enablinsg
\newcommand{\pmin}[1]{\ensuremath{U_{#1}}}

% connectectedness of minimal enablinsg
\newcommand{\conn}[1]{\ensuremath{\stackrel{#1}{\frown}}}


% domain for an event structure or graph grammar
\newcommand{\zdom}[0]{\ensuremath{\mathcal{D}}}
\newcommand{\dom}[1]{\ensuremath{\zdom({#1})}}


\newcommand{\zdomeq}[0]{\ensuremath{\mathcal{D}_{eq}}}
\newcommand{\domeq}[1]{\ensuremath{\zdomeq({#1})}}


% partial order for a graph grammar
\newcommand{\poset}[1]{\ensuremath{\mathcal{P}({#1})}}

% stable version
\newcommand{\pdom}[1]{\ensuremath{\mathcal{D}_S({#1})}}
\newcommand{\ppdom}[0]{\ensuremath{\mathcal{D}_S}}


% powerset
\newcommand{\Pow}[1]{\ensuremath{\mathbf{2}^{#1}}}

% powerset of finite subsets
\newcommand{\Powfin}[1]{\ensuremath{\mathbf{2}_\mathit{fin}^{#1}}}

% powerset of subsets of cardinality <= 1
\newcommand{\Powone}[1]{\ensuremath{\mathbf{2}_1^{#1}}}

% integer interval
\newcommand{\interval}[2][1]{\ensuremath{[{#1},{#2}]}}

% domain interval
\newcommand{\dint}[2]{\ensuremath{[{#1},{#2}]}}

% set of intervals
\newcommand{\IntSet}[1]{\ensuremath{\mathop{\mathit{Int}({#1})}}}

% intervals to irreducibles and vice
\newcommand{\inir}{\ensuremath{\mathop{\mathit{\zeta}}}}
\newcommand{\irin}{\ensuremath{\mathop{\mathit{\iota}}}}

% permutations
\newcommand{\perm}{\sigma}

% causes
\newcommand{\causes}[1]{\ensuremath{\lfloor {#1})}}

%%% GRAPH GRAMMARS


\newcommand{\Abs}[1]{\ensuremath{\mathsf{Abs}({#1})}}
\newcommand{\tr}[1]{\ensuremath{\mathsf{Tr}({#1})}}
% fusion safe traces
\newcommand{\trs}[1]{\ensuremath{\mathsf{Tr}_s({#1})}}
%\newcommand{\graph}{\ensuremath{\mathsf{Graph}}}
\newcommand{\tgraph}[1]{\ensuremath{\mathsf{Graph}_{#1}}}
\newcommand{\can}[1]{\ensuremath{\mathsf{C}({#1})}}
% source and target of a derivation
\newcommand{\source}[1]{\ensuremath{\mathsf{s}({#1})}}
\newcommand{\target}[1]{\ensuremath{\mathsf{t}({#1})}}
\newcommand{\col}[1]{\ensuremath{\mathsf{col}({#1})}}

% left decorated trace
\newcommand{\ltrace}[1]{\ensuremath{\langle {#1}\rangle_c}}

\newcommand{\bx}[1]{\phantom{\big(}#1{\phantom{\big)}}}
\newcommand{\bxx}[1]{\,#1\,}
\newcommand{\cycl}[1]{\ensuremath{\mbox{\textcircled{\scriptsize{$#1$}}}}}
\renewcommand{\iff}{\ensuremath{\Leftrightarrow}}

%%%%GENERAL CATEGORICAL NOTATION

\newcommand{\gph}[1]{\mathcal{G}\textbf{\textup{-Graph}}}

\newcommand{\dph}{\mathsf{dph}}
%identità
\newcommand{\id}[1]{\mathsf{id}_{#1}}
%codominio
\newcommand{\cod}[1]{\mathsf{cod}({#1})}
	%Variabili categorie
\def\A{\textbf {\textup{A}}}
%mono
\newcommand{\mto}[0]{\scalebox{1}{$\rightarrowtail$}}


\newcommand{\sk}{\mathsf{sk}_{\X}}


\def\R{\mathsf{R}}
\def\B{\textbf {\textup{B}}}
\def\C{\textbf {\textup{C}}}
\def\D{\textbf {\textup{D}}}
\def\X{\textbf {\textup{X}}}
\def\Y{\textbf {\textup{Y}}}
\def\G{\textbf {\textup{G}}}

\newcommand{\ske}{\mathsf{sk}(\X)}
\renewcommand{\P}{\textbf {\textup{P}}}

%\derivazioni

\newcommand{\dder}[1]{\mathscr{#1}}
\newcommand{\sder}[2]{S_{i_1,i_2}(\mathscr{#1}, \mathscr{#2})}
\newcommand{\der}[1]{\underline{\dder{#1}}}
\def\dpo{\mathsf{C}^{\X}_R}
\def\gpo{\mathsf{G}^{\X}_R}
\def\dpi{[\mathsf{C}]^{\X}_R}
\def\gpi{[\mathsf{G}]^{\X}_R}

\newcommand{\ider}[1]{\mathscr{I}_{#1}}

%categorie
\def\Set{\textbf {\textup{Set}}}

%comma
\newcommand{\comma}[2]{#1\hspace{1pt} {\downarrow}\hspace{1pt} #2}
\newcommand{\cma}[2]{\mathcal{#1}\hspace{1pt} {\downarrow}\hspace{1pt} \mathcal{#2}}

%derivazioni
\newcommand{\lpro}{\langle \hspace{-1.85pt}[}
\newcommand{\rpro}{]\hspace{-1.85pt}\rangle}
\newcommand{\tpro}[1]{\lpro \der{#1}\rpro}
\newcommand{\tproi}[2]{\lpro \der{#1}_{#2}\rpro}
\newcommand{\lgh}[0]{\mathsf{lg}}

%%% NEW

\usepackage{xparse}

% inversions
\newcommand{\inv}[1]{\ensuremath{inv}({#1})}

% direct shift
\newcommand{\shiftdir}[1][]{\ensuremath{\mathrel{{\rightsquigarrow}^{\mathit{sh}}_{#1}}}}

% shift preorder
\newcommand{\shiftpre}[1][]{\ensuremath{\mathrel{{\sqsubseteq}^{\mathit{sh}}_{#1}}}}

% shift equivalence
\newcommand{\shifteq}[1][]{\ensuremath{\mathrel{{\equiv}^\mathit{sh}_{#1}}}}

% transp{source}[target]: if target not specified source+1
\NewDocumentCommand{\transp}{m o}{%
  \ensuremath{({#1},%
  \IfNoValueTF{#2}%
    {{#1}+1}%
    {#2}%
    )}
}

\NewDocumentCommand{\mycommand}{o}{%
  % <code>
  \IfNoValueTF{#1}
    {code when no optional argument is passed}
    {code when the optional argument #1 is present}%
  % <code>
}
% interchange
\newcommand{\IC}[1]{\ensuremath{\mathit{IC}({#1})}}

%%%%% Ambienti matematici  %%%%%%
%\newtheorem{theorem}{Theorem}[section]
%\newtheorem{proposition}[theorem]{Proposition}
%\newtheorem{lemma}[theorem]{Lemma}
%\newtheorem{corollary}[theorem]{Corollary}

%\theoremstyle{definition}
%\newtheorem{definition}[theorem]{Definition}
\newtheorem*{notation}{Notation}
%\newtheorem{remark}[theorem]{Remark}
%\newtheorem{example}[theorem]{Example}


% commands for restructuring
\newcommand{\rem}[2]{{\color{blue}#1}{\color{red}#2}}
\renewcommand{\rem}[2]{}


\title{Left-linear rules in adhesive categories}

\author{Paolo Baldan} 
{Department of Mathematics, University of Padova, Italy}
{baldan@math.unipd.it}{}{}
\author{Davide Castelnovo}
{Department of Mathematics, University of Padova, Italy}
{davide.castelnovo@math.unipd.it}{}{}
\author{Andrea Corradini}
{Department of Computer Science, University of Pisa, Italy}
{andrea.corradini@unipi.it}{}{}

\author{Fabio Gadducci}
{Department of Computer Science, University of Pisa, Italy}
{fabio.gadducci@unipi.it}{}{}


\authorrunning{P.~Baldan, D.~Castelnovo, A.~Corradini, F.~Gadducci}
%
\Copyright{Paolo Baldan, Davide Castelnovo, Andrea Corradini and Fabio Gadducci}
%
\ccsdesc[500]{Theory of Computation~Models of computation}
%
\keywords{
Adhesive categories, double-pushout rewriting, left-linear rules, switch equivalence, local Church-Rosser property.
}

\begin{document}
\maketitle



\begin{abstract}
When two sequences of steps executed by a computational device can be considered equivalent?
This is a relevant question if one considers concurrent and distributed systems, since the independence
of two steps means that they could be executed on any order, and potentially in parallel.
%
We investigate the issue in the context of adhesive categories, a very
general instance of the DPO approach to rewriting. More precisely, we
focus on left-linear rules, which not only consume, preserve and generate entities,
as linear rules do, but may ``merge'' parts of the state.
%
An apparently minimal, yet technically hard enhancement, which allows to properly extend the 
formalisms that can be modelled in the adhesive framework, and for which we prove a 
Church-Rosser property.
%
% We refine the established notion of sequential independence between steps, and ...
 \end{abstract}

\tableofcontents

\iffalse
\section*{Summary (with dependencies)}

\subsection*{REWRITING}
\begin{itemize}
\item DPO Rewriting in $\mathcal{M}$-adhesive categories, with left-$\mathcal{M}$-linear rules (from now on shortened in ``linear'')\\
  Observe that rewriting is deterministic (by uniqueness of
  PO-complement Corollary~\ref{lem:radj})
  
\item Category of abstract derivation for DPO rewriting system $(\X, \R)$
  \begin{itemize}
  \item \emph{DPO-derivation category} $\dpo$: objects are graphs,
    arrows between possibly empty, derivations.
  \item \emph{DPO-abstract decorated derivation category} $\dpi$
  \end{itemize}
  
  
  
\item \emph{Consistent permutations} (Definition~\ref{def:permcon}):
  permutation on the steps of two abstract decorated derivations and
  isomorphism between the colimits such that arrows into the colimits
  ``commutes''.
  \begin{enumerate}
  \item It is unique for consuming
    rewriting systems (Corollary~\ref{cor:unique})
  \item They can be composed
    (Lemma~\ref{lem:sum})
  \end{enumerate}
\item
  
  
\item Notions of independence\\ We have several
  possibile notions of independence for two direct derivations $\dder{D}
  ; \dder{D'}$
  \begin{enumerate}
    
  \item \emph{Existence of a filler} There
is a filler for $\dder{D} ; \dder{D'}$ (Definition~\ref{def:filler})

\item \emph{Switchable} There is
  $\dder{D}_1' ; \dder{D}_1$ which is a \emph{switch} of $\dder{D} ;
  \dder{D'}$ (Definition~\ref{de:switch}, being a switch is defined
  axiomatically) [actually, Definition~\ref{de:good-switchable} provides
  a definition of switchable which looks different to me, I would avoid
  this, to be checked]
  
\item \emph{(Weakly) Sequential
    independent} There is an \emph{independence pair}
  (Definition~\ref{de:sequential-independence}) They are called
  \emph{strongly sequential independent} if the independence pair is
  unique.
\end{enumerate}

We have that $(1) \Rightarrow (2)$ (Theorem~\ref{prop:fil}) and
$(2) \Rightarrow (3)$ while we expect that in general
$(3) \not\Rightarrow (1)$ (but no counterexample). This is instead
known to hold for linear rules.


Some terminology (Definition~\ref{de:good-switchable})
\begin{itemize}
  
  
\item \emph{Tame rewriting systems}\\ An
  independence pairs induced by a filler is called \emph{good} A
  rewriting system where all indepedence pairs are good is called
  \emph{tame}

  NOTE: the categories in which
  condition $B$ from~\cite{baldan2011adhesivity} identifies a large
  class of (quasi)adhesive categories where left-linear systems are all
  tame. Adapted here to M-adhesivity.
  
\item \emph{Very Tame rewriting
    systems}\\ Tame and existing indepence pairs are unique (hence
  switching is unique) (see later for examples)
\end{itemize}


\item \emph{Switch equivalent derivations}
  (concrete)\\
  Definition~\ref{de:switchable}
  
  THEOREM: Two switch equivalent derivations admit a consistent
  permutation (Lemma~\ref{lem:consperm} ??).
  
  A consequence of this used for proving that the slice of traces is a
  pre-order.
  
  The converse holds only when rules are linear [proof clear, but not
  written]  
\end{itemize}

\subsection*{CONCURRENCY}

This part is dealt with for very tame rewriting systems and depends on this assumption. Note that
\begin{itemize}
\item Linear rewriting systems on adhesive categories are very tame
\item Left-linear rewritint systems on graph-like categories with generalisation of (i) no isolated nodes in $L$ and (ii) right leg injective on nodes is very tame
\item Graphs with equivalences (Egg) with left-leg regular mono, right leg mono is very tame
\end{itemize}

We show
\begin{itemize}
  

\item Three-steps lemma (Lemma~\ref{lem:indep-global-left}), which is a weakened version of ``indepedence is global''.
  \begin{itemize}
    
  \item If in $abc$ I have that $ab$ are independent and $c$ is anticipated, they remain independent.
  \item  If in $abc$ I have that $bc$ are independent and $a$ is moved to the end, they remain independent only if there is an additional condition (essentially requiring that $a$ was not a cause of $c$)
  \end{itemize}

  
  
\item No need of useless switchs (if two derivations are switch equivalent I can obtain one from another applying only inversions)
  
\item weak prime domain e connected es semantics
\end{itemize}
\fi

\section{Introduction}

%\todo{A VERY NICE INTRODUCTION}
%
%SOME IDEAS (BADLY WRITTEN, JUST TO START TO PUT IDEAS IN ORDER)

One of the key pay-off of concurrency theory is the idea that the behaviour of 
a computational device can be modelled by abstracting its sequences of steps
%performed by the device 
via a suitable equivalence. Intuitively, this equivalence 
captures when steps are causally unrelated and thus could be executed in any 
order, and possibly concurrently. 
%
Two seminal contributions to this line of research have been
Mazurkiewicz's traces~\cite{Mazurkiewicz86} and Winskel's event structures~\cite{NPW:PNES}
%
Concerning formalisms based on the rewriting paradigm, i.e. where the
behaviour is dictated by a set of rules which have to be instantiated with
the system at hand, concurrency often boils down to having a notion of 
independence between two consecutive steps.
%thus viewing a sequence of steps as a concurrent computation. 
As noteworthy examples, we have 
interchange law~\cite{Mes92} in term rewriting and permutation 
equivalence~\cite{JJL80} in $\lambda$-calculi. 

Among rewriting-based formalisms, those manipulating graph-like structures
proved useful in many settings for representing the dynamics of systems: the 
graph serves to represent the entities and their relations within states, while 
rewriting steps, modifying the graph, model computation steps that results 
in changes to the system state. The DPO (double-pushout) approach~\cite{CMREHL:AAGT} 
is nowadays considered the standard one for these structures, thanks to its locality 
(rule application has a local effect on a state, differently e.g. from the SqPO 
approach~\cite{CorradiniHHK06}) and most importantly to its flexibility: the rules allow to 
specify that,  in a rewriting step, some entities in the current state are removed, 
some others are needed for the rewriting step to occur but remain unaltered and 
some new entities can be generated. Concerning concurrency, the chosen notion 
is shift equivalence, and independence is formally captured by the so-called 
Church-Rosser theorem of DPO rewriting~\cite[Section~xxx]{CMREHL:AAGT}.

A noteworthy advantage is that DPO rewriting can be formulated over
adhesive categories and their variants~\cite{lack2005adhesive,ehrig2006weak}. 
%
Operating within the framework of adhesive categories allows one to devise 
the foundations of a rewriting theory which can be subsequently instantiated 
to the context of interest, thus avoiding the need of repeatedly proving
similar ad-hoc results in for each specific setting. In fact, the framework is
so general that it allowed to recast in the DPO approach many additional formalisms,
ranging from xxx to string diagrams~\cite{bonchi2022string}.

So far, most of the theoretical results focussed on what are called \emph{linear rules}: 
intuitively, a step is required only to consume, preserve and generate entities. 
However, in some situations it is also necessary to ``merge'' parts of the state. 
This happens already in the context of string diagram mentioned above, but also 
in graphical implementations of nominal calculi where, as a result of name passing, 
the received name is identified with a local one at the
receiver~\cite{CVY:ESSPE,Gad07} or in the visual modelling of bonding in
biological/chemical processes~\cite{PUY:MBPE}. A similar mechanisms is
at the core of e-graphs (equality graphs), recently used for
rewrite-driven compiler optimisations~\cite{WNW:egg}: rather than
explicitly merging nodes corresponding to equal expression, the idea
here - conceptually and, as we will see, also mathematically similar -
is to maintain an equivalence\todo{P: more precisely a congruence}
over the nodes.  \todo{P: la sequenza di lavori su questo \`e ampia,
  vedere se c'\`e qualcosa d'altro}

% =========DA QUI SUPERSEDED\\
% Rewriting systems over graphs are used in uncountably many setting for
% representing the dynamics of software systems: the graph serves to represent
% the entities and their relations within states , while rewriting steps,
% modifying the graph, model computational steps that results in
% changes to the system state.

% Rewriting over many kind of graphs is elegantly captured by the
% so-called \emph{double pushout} (DPO) approach to rewriting. A
% noteworthy advantage is that DPO rewriting can be formulated over
% adhesive categories and their
% variants~\cite{lack2005adhesive,ehrig2006weak}. Operating within the
% framework of adhesive categories allows one to devise the foundations
% of a rewriting theory which can be subsequently instantiated to the
% context of interest, thus avoiding the need of repeatedly proving
% similar ad-hoc results in for each specific setting.

% Rewriting over adhesive categories has been widely studied in the last
% two decades and most of the research focused on the so-called linear
% rewriting rules. Roughly speaking, linear rewriting rules can specify
% that, in a rewriting step, some entities in the current state are
% removed, some other are needed for the rewriting step to occur but
% remain unaltered and some new entities can be generated.

% The computation described by a rewriting system can be naturally
% interpreted as a distributed computation and a notion of independence
% between rewriting steps naturally emerges. When different rules are
% applied at ``distinct'' parts of the state, possibly overlapping on
% preserved items, the rewriting steps can be naturally deemed as
% independent and their application can occur in any order and possibly
% concurrently. This phenomenon is formally captured by the so-called
% Church-Rosser theorem of DPO rewriting~\cite{CMREHL:AAGT}.
% %
% Having a notion of independence between computation steps sits at the
% core of the possibility of viewing an execution trace as a concurrent
% computation. This is obtained by considering traces up to an
% equivalence that identifies traces differing only for the order of
% independent steps, referred to as shift equivalence~\cite{CMREHL:AAGT}
% (similar to interchange law~\cite{Mes92} in term rewriting or
% permutation equivalence~\cite{JJL80} in the $\lambda$-calculus).

% In some situations rewriting with linear rules are not sufficiently
% expressive for capturing the dynamic of a system. This is the case
% when a computational step is required not only to consume, preserve
% and generated, but also ``merge'' parts of the state. This happens,
% e.g., in nominal calculi where, as a result of name passing, the
% received name is identified with a local one at the
% receiver~\cite{CVY:ESSPE,Gad07} or in the modelling of bonding in
% biological/chemical processes~\cite{PUY:MBPE}. A similar mechanisms is
% at the core of e-graphs (equality graphs), recently used for
% rewrite-driven compiler optimisations~\cite{WNW:egg}. Rather than
% explicitly merging nodes corresponding to equal expression, the idea
% here - conceptually and, as we will see, also mathematically similar -
% is to maintain an equivalence\todo{P: more precisely a congruence}
% over the nodes.  \todo{P: la sequenza di lavori su questo \`e ampia,
%   vedere se c'\`e qualcosa d'altro}\\
% =========

Technically, to acquire the expressiveness needed for expressing
fusions, i.e., the merge of some elements in the state, requires to
move from linear to left-linear rewriting rules. While left-linear
rules have been considered in various instances (mainly in categories
of graphs) and some results concerning the corresponding rewriting
theory have been put forward, to the best of our knowledge the
phenomena arising when rewriting with left-linear rules in the general
setting of adhesive and quasi-adhesive categories (which capture most
``applications'') have not been systematically explored. ADD REFS

Here we perform this exploration by observing that for left-linear
rules a number of fundamental properties which play a role in the
corresponding theory of rewriting need a deep rethinking.

Firstly, Church-Rosser theorem does not work out of the box. In
particular, sequential independence between rewriting steps~\cite{}
expressed in terms of the existence of a so-called independence pair
is a canonical and well-understood notion of independence in the
linear case. \todo{P: Too technical? Forse qui e' piu' esplicativo un
  esempio} Whenever two steps are sequential independent
\begin{itemize}
\item they can be switched and 
\item the result of the switch is uniquely determined.
\end{itemize}

Here we observe that, in general, both property can fail: sequential
independent steps could be not switchable and different independent
pairs could exist leading to different switches of the same rewriting
steps.  A refined notion of independence can be considered to ensure
that independent rewriting steps can be switched, based in the concept
of a so-called filler. The problem was identified already
in~\cite{baldan2011adhesivity} where a suitable subclass of adhesive
categories is identified where independence as induced by a filler
coincide with sequential independence (as induced by an independence
pair). Here this is extended to $\mathcal{M}$-adhesive categories.


We face the problem in the setting of $\mathcal{M}$-adhesive
categories (which capture adhesive and quasi-adhesive categories).
The natural idea, put forward in~\cite{baldan2017domains}, of limiting
sequential independence to those cases in which the switch is possible
and uniquely determined falls short.

Rather we focus on a class of rewriting systems, the so-called
\emph{very tame rewriting systems} in $\mathcal{M}$-adhesive
categories where sequential independence ensures existence and
uniqueness of the switch and argue that this is the right setting for
dealing with left-linear rewriting systems.

On the one hand we argue that very tame rewriting systems capture most
categories of interest for rewriting. In particular, we single out a
class of left-linear rewriting systems on graph-like structures which
are very tame and we show that rewriting systems on (term)graphs with
equivalences resempling e-graphs are also very tame. As a sanity
check, also linear rewriting systems on $\mathcal{M}$-adhesive
categories are very tame.

On the other hand we show that an appropriate theory of rewriting can
be developed for very tame rewriting systems, recovering, sometimes in
weakened form, results which hold in the linear case.  Local
Curch-Rosser is fully recovered for very tame rewriting systems.

For linear rewriting systems sequential independence of derivations
can be also characterised in terms of the existence of a suitable
isomorphism between the colimit of the derivations, the so-called
consistent permutations (which are also strictly related to the notion
of process). This result, which is folklore for categories of graphs,
is explicitly proved here for $\mathcal{M}$-adhesive categories. Based on such result one can deduce two fundamental consequences:
\begin{enumerate}
\item when swicthing a derivation consisting of several steps, the result only depends on the overall permutation, not on the order of the swicthes
\item independence of rewriting steps is a ``global property'', independent of the position in which these steps are in the derivation.
\end{enumerate}

We show that for left-linear rewriting systems the characterisation of
shift equivalence based on consistent permutations fails: the
existence of a consistent permutation is only necessary but not
sufficient, hence it cannot be used for proving (1) and (2). However property (1) can be still proved to holds. Instead only a weak form of (2) can be proved. This is intimately related to the possibility of expressing disnjuctive forms of causality.

\todo[inline]{
This fails for left-linear rules where, as observed
in~\cite{baldan2017domains}, for graph rewriting, the possibility of
merging parts of the state leads to a sort of disjunctive causality,
i.e., a rewriting step, say $r_c$ can be enabled by two different
rewriting steps $r_a$ and $r_b$, in a way that in a derivation
$r_a r_b r_c$ steps $r_b$ and $r_c$ are independent since the presence
of $r_a$ in the past is sufficient for enabling $r_a$, while in
$r_b r_c r_a$ , the steps $r_b$ and $r_c$ are not independent. }


\section{$\mathcal{M}$-adhesive categories and DPO rewriting systems}\label{sec:ade}

This first section is devoted to recall the definition and the basic theory of \emph{$\mathcal{M}$-adhesive categories} \cite{azzi2019essence,ehrig2012,ehrig2014adhesive,lack2005adhesive,heindel2009category}.

\begin{notation}
  We stipulate some notational conventions that will be used
  throughout the paper.
  \begin{itemize}
  \item
    Given a category $\X$ we will not distinguish notationally
    between $\X$ and its class of objects: so that ``$X\in \X$'' means
    that $X$ belongs to the class of objects of $\X$.

  \item
    If $1$ is a terminal object in $\X$, the unique
    arrow $X\to 1$ from an object $X$ will be denoted by
    $!_X$. Similarly, if $0$ is initial in $\X$ then $?_X$ will denote
    the unique arrow $0\to
    X$. %When $\X$ is $\Set$ and $1$ is a singleton, $\delta_x$ will denote the arrow $1\to X$ with value $x\in X$.

  \item $\mor(\X)$, $\mon(\X)$ and $\reg(\X)$ will denote the class of
    all arrows, monos and regular monos of $\X$, respectively.

  \item Given an integer $n\in \mathbb{Z}$, $[0,n]$ will denote the
    set of natural numbers less or equal than $n$, assuming that
    $[0,n]=\emptyset$ whenever $n<0$.
  \end{itemize}
\end{notation}


\subsection{$\mathcal{M}$-adhesivity}\label{subsec:ade}
The key property that $\mathcal{M}$-adhesive categories enjoy is given
by the so-called \emph{Van Kampen condition}
\cite{brown1997van,johnstone2007quasitoposes,lack2005adhesive}. We
will recall it and examine some of its consequences. First of all we
need to recall some terminology and facts regarding subclasses of
$\mor(\X)$.

\begin{definition}
  Let $\X$ be a category and $\mathcal{A}$ a subclass of
  $\mor(\X)$. We say that $\mathcal{A}$ is
  \begin{itemize}
    \parbox{11cm}{\item
      \emph{stable under pushouts (pullbacks)} if for every pushout (pullbacks) square as the one on the right, 	if $m \in \mathcal{A}$ ($n\in \mathcal{A}$) then $n \in \mathcal{A}$ ($m \in \mathcal{A}$);
    \item \emph{closed under composition} if $g, f\in \mathcal{A}$ implies $g\circ f\in \mathcal{A}$ whenever $g$ and $f$ are composable.{\tiny }}\hfill
    \parbox{2cm}{$\xymatrix{A\ar[r]^f  \ar[d]_{m}& B \ar[d]^n \\ C \ar[r]_g & D}$}
    \parbox{11cm}{}\hfill
  \end{itemize}
\end{definition}

So equipped, we can introduce the notion of \emph{$\mathcal{A}$-Van Kampen square}.

\noindent
\parbox{10cm}{
  \begin{definition}[Van Kampen property]
    Let $\X$ be a category and consider the two diagrams aside.
 %   
%    \hspace{15pt}
    Given  a class of arrows $\mathcal{A}\subseteq \mor(\X)$, we say that the bottom square
    % \emph{has the Van Kampen property relatively to $\mathcal{A}$}, or that it
    is a \emph{$\mathcal{A}$-Van Kampen square} if
    \begin{enumerate}
    \item is a pushout square;
    \item 	whenever the cube above has pullbacks as back and left faces and the vertical arrows belong to $\mathcal{A}$, then its top face is a pushout 
    if and only if the front and right faces are pullbacks.
    \end{enumerate}
    Pushout squares which enjoy the ``if'' half of item (2) above are called \emph{$\mathcal{A}$-stable}.
    
    We will call a $\mor(\X)$-Van Kampen ($\mor(\X)$-stable) square simply a \emph{Van Kampen} (\emph{stable}) square.
  \end{definition}}
\parbox{2cm}{$\xymatrix@C=10pt@R=10pt{&A'\ar[dd]|\hole_(.65){a}\ar[rr]^{f'} \ar[dl]_{m'} && B' \ar[dd]^{b} \ar[dl]_{n'} \\ C'  \ar[dd]_{c}\ar[rr]^(.7){g'} & & D' \ar[dd]_(.3){d}\\&A\ar[rr]|\hole^(.65){f} \ar[dl]^{m} && B \ar[dl]^{n} \\C \ar[rr]_{g} & & D \\A \ar[dd]_{m}\ar[rr]^{f}&&B\ar[dd]^{n}\\ \\C\ar[rr]_{g} &&D}$ }


We can now define $\mathcal{M}$-adhesive categories.

\begin{definition}[$\mathcal{M}$-adhesive category]
  Let $\X$ be a category and $\mathcal{M}$ a subclass of
  $\mon(\X)$
  % \begin{enumerate}
  % \item $\mathcal{M}$
  including  all isomorphisms, closed under composition,
  % \item $\mathcal{M}$
  and stable under pullbacks and pushouts.
  % \end{enumerate} 
  The category
  % 
  $\X$ is said to be \emph{$\mathcal{M}$-adhesive} if
  \begin{enumerate}
  \item it has \emph{$\mathcal{M}$-pullbacks}, i.e. pullbacks along arrows of $\mathcal{M}$;
  \item it has \emph{$\mathcal{M}$-pushouts}, i.e. pushouts along arrows of $\mathcal{M}$;
  \item  $\mathcal{M}$-pushouts are $\mathcal{M}$-Van Kampen squares.
  \end{enumerate}
  
  A category $\X$ is said to be \emph{strictly $\mathcal{M}$-adhesive}
  if $\mathcal{M}$-pushouts are Van Kampen squares.
\end{definition}
We will use $m\colon X\mto Y$ to denote that an arrow $m\colon X\to Y$ belongs to $\mathcal{M}$.


\begin{remark}
  \label{rem:salva}
  \emph{Adhesivity} and \emph{quasiadhesivity} 
  \cite{lack2005adhesive,garner2012axioms} coincide with strict
  $\mon(\X) $-adhesivity and strict $\reg(\X)$-adhesivity,
  respectively, for $\reg(\X)$ the class of regular monos.
\end{remark}

\begin{example}
  \label{rem:iso}
  Let $\mathsf{Iso}$ be the class of all
  isomorphisms, then any category $\X$ is $\mathsf{Iso}$-adhesive.
\end{example}

%\todo{P: qui probabilmente si puo' compattare tenendo solo cio' che si usa}
%\todo{Si usa tutto per la classe B+ [DC]}
%$\mathcal{M}$-adhesivity is well-behaved with respect to  the comma construction \cite{mac2013categories}, as shown by the following theorem.
%\begin{theorem}[\cite{ehrig2006fundamentals,lack2005adhesive}]\label{lem:comma}
  %Let $\A$ and $\B$ be respectively an $\mathcal{M}$-adhesive and an
  %$\mathcal{M}'$-adhesive category. Let also $L:\A\rightarrow \C$ be a
  %functor that preserves $\mathcal{M}$-pushouts, and
  %$R:\B\rightarrow \C$ be a functor which preserves pullbacks.Then
 % $\comma{L}{R}$ is $\cma{M}{M'}$-adhesive, where
  %\[
    %\cma{M}{M}':=\{(h,k)\in \mathcal{A}(\comma{L}{R}) \mid h\in
    %\mathcal{M}, k\in \mathcal{M}'\}\]
%\end{theorem}
$\mathcal{M}$-adhesivity is well-behaved with respect to  the construction of slice and functor caregories, \cite{mac2013categories}, as shown by the following theorems \cite{ehrig2006fundamentals,lack2005adhesive}.
   --ò-à

%\begin{theorem}
%[\cite{ehrig2006fundamentals,lack2005adhesive}]
%\label{lem:comma}
  %Let $\A$ and $\B$ be respectively an $\mathcal{M}$-adhesive and an
 % $\mathcal{M}'$-adhesive category. Let also $L:\A\rightarrow \C$ be a
  %functor that preserves $\mathcal{M}$-pushouts, and
  %$R:\B\rightarrow \C$ a functor that preserves pullbacks.Then
  %$\comma{L}{R}$ is $\cma{M}{M'}$-adhesive, where
  %\[
    %\cma{M}{M}':=\{(h,k)\in \mathcal{A}(\comma{L}{R}) \mid h\in
   % \mathcal{M}, k\in \mathcal{M}'\}\]
%\end{theorem}

\begin{theorem}\label{cor:slice}
  Let $X$ be an object of an $\mathcal{M}$-adhesive category
  $\X$. Then $\X/X$ and $X/\X$ are, respectively, $\mathcal{M}/X$-
  and $X/\mathcal{M}$-adhesive, where
  \[\mathcal{M}/X:=\{m\in \mathcal{A}(\X/X) \mid m\in \mathcal{M} \}
    \qquad X/\mathcal{M}:=\{m\in \mathcal{A}(X/\X) \mid m\in
    \mathcal{M} \}
  \]
\end{theorem}


\begin{theorem}
  \label{thm:functors}
  Let $\X$ be an $\mathcal{M}$-adhesive category. Then for every small
  category $\Y$, the category $\X^\Y$ of functors $\Y\to \X$ is
  $\mathcal{M}^{\Y}$-adhesive, where
  \[\mathcal{M}^{\Y}:=\{\eta \in \mathcal{A}(\X^\Y) \mid \eta_Y \in
    \mathcal{M} \text{ for every } Y\in \Y\}\]
\end{theorem}

We can list various examples of $\mathcal{M}$-adhesive categories (see
\cite{castelnovo2023thesis,CastelnovoGM22,lack2006toposes}).


\begin{example}
  \label{ex:adhesive}
  Every topos is adhesive~\cite{lack2006toposes}, thus in particular
  $\cat{Set}$ is adhesive. By the closure properties recalled before
  every presheaf $[\cat{X},\cat{Set}]$ is adhesive and in particular
  the category $\cat{Graph} = [ E \rightrightarrows V, \cat{Set}]$ is
  adhesive where $E \rightrightarrows {V}$ is the two object category
  with two morphisms $s,t \colon{E} \to {V}$. In a similar way,
  various categories of hypergraphs can be shown to be adhesive, while
  the category $\cat{sGraphs}$ of simple graphs, i.e., graphs without
  parallel edges is quasi-adhesive, i.e.,
  $\reg{(\cat{sGraphs})}$-adhesive.\todo{Add hierarchical graphs and term graphs\cite{castelnovo2023thesis,CastelnovoGM22}).}
\end{example}


A number of properties of adhesive categories which play a role in the
theory of rewriting generalise to $\mathcal{M}$-adhesive
categories. These include $\mathcal{M}$-pushout-pullback decomposition
and uniqueness of pushouts complements (details and proofs are in
\Cref{app:ade}).

\subsection{DPO rewriting systems derivations}\label{subsec:DPO}

$\mathcal{M}$-adhesive categories are the right context in which to
perform abstract rewriting using the so-called ``double pushout
approach'' (DPO). We will recall the basic definitions and properties
of this approach to abstract rewriting.


\begin{definition}[\cite{habel2012mathcal,heindel2009category}]
  Let $\X$ be an $\mathcal{M}$-adhesive category. A \emph{left
    $\mathcal{M}$-linear} rule $\rho$ is a pair $(l,r)$ of arrows with
  $l: K \to L$ and $r: K \to R$, such that $l$ belongs to
  $\mathcal{M}$.  The rule $\rho$ is said
  \emph{$\mathcal{M}$-linear} if $r\in \mathcal{M}$ too. A rule $\rho$
  is said to be \emph{consuming} if $l$ is not an isomorphism.  The
  object $L$ is the \emph{left-hand side}, $R$ is the \emph{right-hand
    side}, and $K$ the \emph{glueing object}.

  A \emph{left-linear DPO rewriting system} is a pair $(\X, \R)$ where
  $\X$ is an $\mathcal{M}$-adhesive category and $R$ is a set of left
  $\mathcal{M}$-linear rules. The rewiriting system $(\X, \R)$ is
  called \emph{linear} (\emph{consuming}) if every rule in $R$ is
  $\mathcal{M}$-linear (consuming).

  \noindent
  \parbox{10cm}{\hspace{15pt}Given two objects $G$ and $H$ and a rule
    $\rho=(l,r)$ in $\R$, a \emph{direct derivation $\mathscr{D}$ from
      $G$ to $H$ applying the rule $\rho$}, is a diagram as the one
    aside, in which both squares are pushouts. The arrow $m$ is called
    the \emph{match} of the derivation, while $h$ is its
    \emph{back-match}.  We will denote a direct derivation $\dder{D}$
    between $G$ and $H$ as $\dder{D}\colon G\Mapsto H$. }
  \parbox{3cm}{$\xymatrix{L \ar[d]_{m}& K \ar[d]^{k}\ar@{>->}[l]_{l}
      \ar[r]^{r} & R \ar[d]^{h}\\G & \ar@{>->}[l]^{f} D \ar[r]_{g}&
      H}$}
\end{definition}


\full{
  %
  If we look to direct derivations as
  transitions, it is natural to consider them as edges in a direct
  graph. Taking objects as vertices leds us to the following
  definition \cite{heindel2009category}.

  \begin{definition}
    Let  $\X$ be an $\mathcal{M}$-adhesive category and 
    $(\X, \R)$ a left-linear DPO rewriting system. 
    The \emph{DPO-derivation graph} of
    $(\X, \R)$ is the (large) directed graph $\gpo$ having as vertices
    the objects of $\X$ and in which an edge between $G$ and $H$ is a
    direct derivation $\dder{D}\colon G\Mapsto H$. A \emph{derivation}
    $\der{D}$ between two objects $G$ and $H$ is a path between them
    in $\gpo$. The \emph{source} and \emph{target} of $\der{D}$ are,
    respectively, $G$ and $H$.
  \end{definition}
}

A derivation can now defined simply as a sequence of direct derivations.

\begin{definition}[Derivation]
  Let  $\X$ be an $\mathcal{M}$-adhesive category and 
  $(\X, \R)$ a left-linear DPO rewriting system. 
  Given $G, H \in \X$, a \emph{derivation}
  $\der{D}$ with source $G$ and target $H$, written
  $\der{D}: G \Mapsto H$ is defined as a sequence
  $\{\dder{D}_i\}_{i=0}^n$ of direct derivations such that
  \begin{enumerate}
  \item for every index $i$, $\dder{D}_i$ is a direct derivation $G_i \Mapsto G_{i+1}$;
  \item $G_0=G$ and $G_{n+1}=H$.
  \end{enumerate}
  Moreover, for each $G \in \X$ we have $G : G \Mapsto G$ is a
  derivation, called an \emph{empty derivation}.

 
  We will call the number $n+1$ the \emph{length} of the derivation,
  denoted by $\lgh(\der{D})$. We will also say that an empty
  derivation has length $0$.

  Moreover, if every $\dder{D}_i$ applies the rule $\rho_i\in R$, then
  we can define an associated sequence of rules as $r(\der{D})$ as
  $\{\rho_i\}_{i=0}^n$.
\end{definition}

Derivations naturally compose: given $\der{D}: G \Mapsto H$ and $\der{E}: H \Mapsto F$ we can consider their composition $\der{D} \cdot \der{E}: G \to F$, defined in the obvious way.

% \todo[inline]{P: se non impatta sul resto forse sarebbe forse piu' carino definire:\\
%   - for each $G \in \X$, $G : G \Mapsto G$ is a derivation\\  
%   - for each direct derivation $\dder{D}: G \Mapsto H$ and derivation $\der{E} : H \Mapsto F$ we have that $\dder{D}; \der{E} : G \Mapsto F$ is a derivation
% }

\rem{Suppongo questo serva quando poi si considerano le permutazioni consistenti, spostare? }{
\begin{remark}
  \label{rem:func}
  Consider a derivation $\der{D}$ in a left-linear DPO rewriting
  system $(\X, \R)$. We can take the subcategory $\Delta(\der{D})$ of
  $\X$ given by the arrows appearing in $\der{D}$. This subcategory
  comes equipped with an inclusion functor
  $I(\der{D})\colon \Delta(\der{D})\to \X$. Moreover, we can further
  define $\Deltamin(\der{D})$ as the subcategory of $\Delta(\der{D})$
  containing only the bottom row of the derivation.\todo{Used?}
\end{remark} 

\noindent
\parbox{10cm}{%
  \begin{notation}
    Let $\der{D}=\{\dder{D}_i\}_{i=0}^n$ be a derivation. We will
    depict the $i^\text{th}$ element $\dder{D}_i$ of $\der{D}$ as in
    the diagram on the right. Notice that, in particular, if $\der{D}$
    is a derivation between $G$ and $H$, then $G_0=G$ and
    $G_{n+1}=H$. When $\der{D}$ has length $1$ we will suppress the
    indexes. In such case, we will also identify $\der{D}$ with its
    only element.
  \end{notation} }
\parbox{3cm}{
  $\xymatrix{L_i \ar[d]_{m_i}& K_i
    \ar[d]^{k_i}\ar@{>->}[l]_{l_i} \ar[r]^{r_i} & R_i \ar[d]^{h_i}\\
    G_{i} & \ar@{>->}[l]^{f_{i}} D_{i} \ar[r]_{g_{i}}& G_{i+1} }$
}
}

Often, we are interested in derivations only up to some notion of coherent isomorphism between them. This leads us to the following definition. 

\noindent
\parbox{8cm}{
  \begin{definition}
    Let  $\X$ be an $\mathcal{M}$-adhesive category and 
    $(\X, \R)$ a left-linear DPO rewriting system. 
    An \emph{abstraction equivalence} between two derivations $\der{D}$
    and $\der{D'}$ with the same length and
    $r(\der{D})=r(\der{D}')$, is a family of isomorphisms
    $\{\phi_X\}_{X\in \Deltamin(\der{D})}$ such that the diagram on the right commutes
    for every $i\in [0, \lgh(\der{D})]$. We
    will say that $\der{D}$are $\der{D}'$ are \emph{abstraction
      equivalent} if there exists an abstraction equivalence between
    them.
  \end{definition}
}
\parbox{4cm}{\vspace{-1.35em}$\xymatrix@C=40pt{G'_i&D'_i \ar[r]^{g'_i}
    \ar@{>->}[l]_{f'_i}&G'_{i+1}\\ L_i \ar[u]^{m'_i} \ar[d]_{m_i}& K_i
    \ar[u]^{k'_i} \ar[d]_{k_i} \ar[r]^{r_i} \ar@{>->}[l]_{l_i}
    &R\ar[u]^{h'_i} \ar[d]_{h_i}\\G_i
    \ar@/_.45cm/[uu]_(.35){\phi_{G_i}}|\hole&D_i\ar@{>->}[l]^{f_i}\ar@/_.45cm/[uu]_(.35){\phi_{D_i}}|\hole
    \ar[r]_{g_i}&G_{i+1}\ar@/_.45cm/[uu]_{\phi_{H_i}}}$}



$\mathcal{M}$-adhesivity of $\X$ guarantes the essential uniqueness of
the result obtained rewriting an object, as shown by the next
proposition.


\begin{restatable}{proposition}{propUnique}
  \label{prop:unique}
  Let $\X$ be a $\mathcal{M}$-adhesive category. Suppose that the two
  direct derivations $\dder{D}$ and $\dder{D}'$ below, with the same
  match and applying the same left-linear rule $\rho$ are given
  \[\xymatrix{L \ar[d]_{m}& K \ar[d]^{k}\ar@{>->}[l]_{l} \ar[r]^{r} &
      R \ar[d]^{h} & L \ar[d]_{m}& K \ar[d]^{k'}\ar@{>->}[l]_{l}
      \ar[r]^{r} & R \ar[d]^{h'}\\G & \ar@{>->}[l]^{f} D \ar[r]_{g}& H
      & G & \ar@{>->}[l]^{f'} D' \ar[r]_{g'}& H'}\]
  Then there an abstraction equivalence between $\dder{D}$ and
  $\dder{D}'$, whose first component is $\id{G}$.
\end{restatable}

[Proof in \Cref{propUnique-proof}]


\iffalse
\subsection{Consistent permutations}

Given DPO rewriting system $(\X, \R)$,  a derivation $\der{D}$ determines a diagram
$\Delta(\der{D})$ in $\X$. Using $\mathcal{M}$-adhesivity, it can be shown  that such diagram has a colimit $(\tpro{D}, \{\iota_X\}_{X\in \Delta(\der{D})})$ and that, if $(\X, \R)$ is linear, then every coprojection is in $\mathcal{M}$ ( see \Cref{lem:colim} for details and a proof of these statements).


So equipped, we can introduce the notion of \emph{consistent permutation}.

\begin{definition}[Consistent permutation]
  \label{def:permcon}
  Let $\X$ be an $\mathcal{M}$-adhesive category and consider a
  left-linear DPO rewriting system $(\X, \R)$ on it.  Take two
  non-empty decorated derivations $(\der{D}, \alpha, \omega)$ and
  $(\der{D}', \alpha', \omega')$ with the same length and with
  isomorphic sources and targets.

  If $\der{D}=\{\dder{D}_i\}_{i=0}^n$ and
  $\der{D}'=\{\dder{D}'_i\}_{i=0}^n$ with associated sequence of rules
  $r(\der{D})=\{\rho_i\}_{i=0}^n$ and
  $r(\der{D}')=\{\rho'_i\}_{i=0}^n$, a \emph{consistent permutation}
  between $(\der{D}, \alpha, \omega)$ and
  $(\der{D}', \alpha', \omega')$ is a permutation
  $\sigma\colon [0,n]\to [0,n]$ such that, for every $i\in [0,n]$,
  $\rho_i=\rho'_{\sigma(i)}$ and, moreover, there exists a
  \emph{mediating isomorphism}
  $\xi_\sigma\colon \tpro{D} \to \lpro \der{D}' \rpro$ fitting in the
  following diagrams, where $m_i, m'_i, h_i$ and $h'_i$ are,
  respectively, the matches and back-matches of $\dder{D}_i$ and
  $\dder{D}'_i$.

  \[
    \xymatrix@C=30pt{\pi(G_0)\ar[r]^{\alpha} \ar[d]_{\alpha'} & G_0
      \ar[r]^{\iota_{G_0}} &\tpro{D} \ar[d]^{\xi_\sigma}& \pi(G_{n+1})
      \ar[r]^{\omega} \ar[d]_{\omega'} & G_{n+1}
      \ar@{>->}[r]^{\iota_{G_{n+1}}} &\tpro{D} \ar[d]^{\xi_\sigma}\\
      G'_0 \ar[rr]_{\iota'_{G'_0}} & &\lpro \der{D}' \rpro& G'_{n+1}
      \ar@{>->}[rr]_{\iota'_{G'_{n+1}}} & &\lpro \der{D}' \rpro\\L_i
      \ar[r]^{m_i} \ar[d]_{m'_{\sigma(i)}}& G_i \ar[r]^{\iota_{G_i}}
      &\tpro{D} \ar[d]^{\xi_\sigma} & R_i \ar[r]^{h_i}
      \ar[d]_{h'_{\sigma(i)}}& G_{i+1} \ar[r]^{\iota_{G_{i+1}}}
      &\tpro{D} \ar[d]^{\xi_\sigma} \\G'_{\sigma(i)}
      \ar[rr]_{\iota'_{G'_{\sigma(i)}}}&& \lpro \der{D}' \rpro&
      G'_{\sigma(i)+1} \ar[rr]_{\iota'_{G'_{\sigma(i)+1}}}&& \lpro
      \der{D}' \rpro}
  \]
\end{definition}

\begin{remark}
  The commutativity of the last two rectangles in \Cref{def:permcon},
  is equivalent to the commutativity of the following bigger diagram.
  \[
    \xymatrix@C=40pt{
      G_i\ar@/^1cm/[rrr]^{\iota_{G_{i}}}&D_i\ar@/^.4cm/[rr]^(.35){\iota_{D_i}}
      \ar[r]_{g_i} \ar@{>->}[l]^{f_i}&G_{i+1}
      \ar[r]_{\iota_{G_{i+1}}}&\tpro{D} \ar[dd]^{\xi_\sigma}\\ L_i
      \ar[u]^{n_i} \ar[d]_{n'_{\sigma(i)}}& K_i
      \ar[d]^{k'_{\sigma(i)}} \ar[u]_{k_i} \ar[r]^{r_i}
      \ar@{>->}[l]_{l_i} &R\ar[u]^{h_i}
      \ar[d]_{h'_{\sigma(i)}}\\G'_{\sigma(i)}\ar@/_1cm/[rrr]_{\iota'_{G'_{\sigma(i)}}}
      &D'_{\sigma(i)}\ar@{>->}[l]_{f'_{\sigma(i)}} \ar[r]^{g'_i}
      \ar@/_.4cm/[rr]_(.35){\iota'_{D'_{\sigma(i)}}}&G'_{\sigma(i)+1}
      \ar[r]^{\iota'_{G'_{\sigma{i}+1}}}& \lpro\der{D}' \rpro }
  \]
  This follows at once from the following chain of identities:
  \begin{align*}
    \xi_\sigma \circ \iota_{D_i}\circ k_i & =\xi_\sigma \circ \iota_{G_{i}} \circ f_i \\&= \xi_\sigma \circ \iota_{G_{i}}\circ m_i \circ l_i\\&=\iota_{G'_{\sigma(i)}}\circ m'_i\circ l_i\\&=\iota_{G'_{\sigma(i)}}\circ f'_{\sigma(i)}\circ k'_i\\&=\iota_{D'_\sigma(i)}\circ k_i'
  \end{align*}
\end{remark}

\begin{remark}\label{rem:coproj}
  Notice that, in particular, the previous diagram entails
  \[\xi_\sigma \circ \iota_{L_i}=\iota'_{L_{\sigma(i)}} \quad \xi_\sigma \circ \iota_{K_i}=\iota'_{K_{\sigma(i)}} \quad \xi_\sigma \circ \iota_{R_i}=\iota'_{R_{\sigma(i)}} \]
\end{remark}

The previous remark allows us to prove the following.

\begin{proposition}
  \label{prop:isouno}
  For every consistent permutation $\sigma$ between
  $(\der{D}, \alpha, \omega)$ and $(\der{D}', \alpha', \omega')$, the
  mediating isomorphism
  $\xi_\sigma\colon \tpro{D}\to \lpro \der{D}'\rpro$ is unique.
\end{proposition}
\begin{proof} Let $\xi'_\sigma$ be another mediating isomorphism, then
  by \Cref{rem:coproj} we have
  \[\begin{split}
      \xi_\sigma \circ \iota_{L_i}&=\iota'_{L_{\sigma(i)}}\\&=\xi'_\sigma\circ \iota_{L_i}
    \end{split} \qquad \begin{split}
      \xi_\sigma \circ \iota_{K_i}&=\iota'_{K_{\sigma(i)}}\\&=\xi'_\sigma\circ \iota_{K_i}
    \end{split} \qquad \begin{split}
      \xi_\sigma \circ \iota_{R_i}&=\iota'_{R_{\sigma(i)}}\\&=\xi'_\sigma\circ \iota_{R_i}
    \end{split}\]
  
  Now, notice that
  \begin{align*}
    \xi_\sigma \circ \iota_{G_0} & =\iota'_{G'_0}\circ \alpha'\circ \alpha^{-1} \\&=\xi'_\sigma \circ \iota_{G_0}
  \end{align*}
  
  If $\lgh(\der{D})=0$ this is enough to conclude, otherwise we are going to prove by induction that, for every $i\in [0, \lgh(\der{D})-1]$
  \[
    \xi_\sigma \circ \iota_{G_i}=\xi'_\sigma\circ \iota_{G_i}
  \]
  
  \smallskip \noindent $i=0$. This is simply the result obtained before.
  
  \smallskip \noindent $i >0$. If $i>0$, we know that there is a pushout square
  \[\xymatrix{K_{i-1}\ar[r]^{r_{i-1}} \ar[d]_{k_{i-1}}& R_{i-1} \ar[d]^{h_{i-1}}\\ D_{i-1} \ar[r]_{g_{i-1}}& G_i}\]
  By \Cref{rem:coproj} and the induction hypothesis we know that
  \[\begin{split}
      \xi_\sigma \circ \iota_{G_i}\circ h_{i-1}&=  \xi_\sigma\circ \iota_{R_{i-1}}\\&=\iota'_{R_{\sigma(i-1)}}\\&=\xi'_{\sigma}\circ \iota_{R_{i-1}}\\&=\xi'_{\sigma} \circ \iota_{G_i}\circ h_{i-1}\\&
    \end{split} \qquad
    \begin{split}
      \xi_\sigma \circ \iota_{G_i}\circ g_{i-1}&=\xi_{\sigma}\circ \iota_{D_{i-1}}\\&=\xi_{\sigma}\circ \iota_{G_{i-1}} \circ f_{i-1}\\&=\xi'_{\sigma}\circ \iota_{G_{i-1}} \circ f_{i-1}\\&=\xi'_{\sigma}\circ \iota_{D_{i-1}} \\&=\xi'_{\sigma}\circ \iota_{G_{i}} \circ g_{i-1}
    \end{split}
  \]
  
  Since $\xi_\sigma \circ \iota_{D_i}$ must be $\xi_\sigma \circ \iota_{G_i}\circ f_i$, we also have
  \[\xi_\sigma \circ \iota_{D_i}=\xi'_\sigma \circ \iota_{D_i}\]
  and the thesis follows.
\end{proof}


\begin{remark}
  \label{rem:inversa}
  Let $\sigma\colon [0,n]\to [0,n]$ be a consistent permutation
  between $(\der{D},\alpha, \omega)$ and
  $(\der{D}', \alpha', \omega')$, then its inverse $\sigma^{-1}$ is a
  consistent permutation between $(\der{D}', \alpha', \omega')$ and
  $(\der{D},\alpha, \omega)$. Indeed, it is enough to consider, as
  mediating isomorphism, the inverse $\xi^{-1}_\sigma$ of
  $\xi_\sigma$.
\end{remark}

\begin{remark}
  \label{rem:comp}
  Consistent permutations can be composed. Indeed, given decorated
  derivations $(\der{D}, \alpha, \omega)$,
  $(\der{D}', \alpha', \omega')$ and
  $(\der{\hat{D}}, \hat{\alpha}, \hat{\omega})$ all of length $n$, if
  $\sigma$ is a consistent permutation between the first two and
  $\tau$ one between the second and the third, then we have diagrams
  \[
    \xymatrix@C=18pt{ & G_0 \ar[rr]^{\iota_{G_0}} &&\tpro{D}
      \ar[d]^{\xi_\sigma}& & G_{n} \ar@{>->}[rr]^{\iota_{ G_{n}}}
      &&\tpro{D} \ar[d]^{\xi_\sigma} \\ \pi(G_0)
      \ar@/_.2cm/[dr]_{\hat{\alpha}}\ar@/^.2cm/[ur]^{\alpha}
      \ar[r]^{\alpha'} &G'_0 \ar[rr]^{\iota'_{G'_0}} &&\lpro \der{D}'
      \rpro \ar[d]^{\xi_{\tau}}&\pi({
        G_{n}})\ar@/_.2cm/[dr]_{\hat{\omega}}\ar@/^.2cm/[ur]^{\omega}
      \ar[r]^{\omega'} & { G'_{n}} \ar@{>->}[rr]^{\iota'_{{ G'_{n}}}}
      && \lpro \der{D}' \rpro \ar[d]^{\xi_{\tau}}\\ &\hat{G}_0
      \ar[rr]_{\hat{\iota}_{\hat{G}_0}}&& \lpro \hat{\der{D}}\rpro &&
      \hat{G}_n \ar@{>->}[rr]_{\hat{\iota}_{\hat{G}_n}}&& \lpro
      \hat{\der{D}} \rpro \\& G_i \ar[rr]^{\iota_{G_i}} &&\tpro{D}
      \ar[d]^{\xi_\sigma}& & G_{i+1} \ar[rr]^{\iota_{G_{i+1}}}
      &&\tpro{D} \ar[d]^{\xi_\sigma} \\ L_i
      \ar@/_.2cm/[dr]_{\hat{m}_{\tau(\sigma(i))}}\ar@/^.2cm/[ur]^{m_i}
      \ar[r]^{m'_{\sigma(i)}} &G'_{\sigma(i)}
      \ar[rr]^{\iota'_{G'_{\sigma(i)}}} &&\lpro \der{D}' \rpro
      \ar[d]^{\xi_{\tau}}&R_i
      \ar@/_.2cm/[dr]_{\hat{h}_{\tau(\sigma(i))}}\ar@/^.2cm/[ur]^{h_i}
      \ar[r]^{h'_{\sigma(i)}} & G'_{\sigma(i)+1}
      \ar[rr]^{\iota'_{G'_{\sigma(i)+1}}} && \lpro \der{D}' \rpro
      \ar[d]^{\xi_{\tau}}\\ &\hat{G}_{\tau(\sigma(i))}
      \ar[rr]_-{\hat{\iota}_{\hat{G}_{\tau(\sigma(i))}}}&& \lpro
      \hat{\der{D}}\rpro && \hat{G}_{\tau(\sigma(i))+1}
      \ar[rr]_-{\hat{\iota}_{\hat{G}_{\tau(\sigma(i))+1}}}&& \lpro
      \hat{\der{D}} \rpro }\] We deduce at once that
  $\tau\circ \sigma$ is a consistent permutation with mediating
  isomorphism given by $\xi_{\tau} \circ \xi_\sigma$.
\end{remark}

\begin{remark} \label{rem:abscons}Let $(\der{D}, \alpha, \omega)$ and $(\der{D}', \alpha', \omega')$ be two abstraction equivalent decorated derivations of length $n$. Then the identity permutation $\id{[0,n-1]}\colon [0,n-1]\to [0,n-1]$ is a consistent permutation between them. In such a case we can take as $\xi_{\id{[0,n-1]}}\colon \tpro{D}\to \lpro \der{D}'\rpro$ simply the isomorphism induced by any abstraction equivalence $\{\phi_X\}_{X\in \Delta(X)}$.

  In particular, given a decorated derivation $(\der{D}, \alpha, \omega)$ with source $G$ and target $H$ and two isomorphisms, $\phi\colon G'\to G$ $\psi\colon H\to H'$ be two isomorphisms, we know, by \Cref{rem:absequi}, that \[(\der{D}, \alpha, \omega)\equiv_a (\phi*\der{D}, \phi^{-1}\circ \alpha, \omega ) \qquad (\der{D}, \alpha, \omega)\equiv_a (\der{D}*\psi, \alpha, \psi \circ \omega )\]
  In these cases, if the cocones $(\tpro{D}, \{\iota_X\}_{X\in \Delta(\der{D})})$, $(\lpro \phi *\der{D} \rpro, \{\iota'_X\}_{X\in \Delta(\phi *\der{D})})$ and $(\lpro \der{D}*\psi \rpro, \{\iota'_X\}_{X\in \Delta(\der{D}*\psi)})$ are colimiting, we can take as mediating isomorphisms, respectively, the unique arrows $\Gamma^{\phi} \colon \tpro{D}\to \lpro \phi*\der{D} \rpro$ and $\Gamma_\psi\colon \tpro{D}\to \lpro \der{D}*\psi \rpro$  such that, for every $X\in \Deltamin(\der{D})$:
  \[\Gamma^\phi \circ \iota_X:=\begin{cases}
      \iota'_{G'}\circ \phi^{-1} & X=G              \\
      \iota'_X                   & \text{otherwise}
    \end{cases} \qquad \Gamma_\psi \circ \iota_X:=\begin{cases}
      \hat{\iota}_{H'}\circ \psi & X=H              \\
      \hat{\iota}_X              & \text{otherwise}
    \end{cases}\]
\end{remark}


\begin{remark}\label{ex:empty}
  If $\der{D}$ and $\der{D}'$ are empty, then the converse also holds. If $\id{\emptyset}$ is a consistent permutation, then a mediating isomorphism provides an abstraction equivalence.
\end{remark}

\begin{example}\todo{identità consistente non implica astrazione}Notice that the \Cref{ex:empty} cannot, in general, be generalized to non-empty derivations. A counterexample is the following one.
\end{example}


\begin{remark}\label{rem:dett}
  Let $(\der{D}, \alpha, \omega)$ be the composite $(\der{D}_1, \alpha_1, \omega_1)\cdot (\der{D}_2, \alpha_2, \omega_2)$ of two derivation of length $l_1$ and $l_2$. Suppose that:
  \[r(\der{D}_1)=\{\rho_{1,i}\}_{i=0}^{l_1-1} \quad r(\der{D}_2)=\{\rho_{2,i}\}_{i=0}^{l_2-1} \quad  r(\der{D})=\{\rho_{i}\}_{i=0}^{l_1+l_2-1}\]
  and let  $G_{1,0}, G_{2,0}$ and $G_{0}$ be the sources of $\der{D}_1$, $\der{D}_2$,  and $\der{D}$.  Denote, also by $G_{1,i+1}, G_{2,i+1}$ and $G_{i+1}$ are the results of the application of, respectively, $\rho_{1,i}, \rho_{2,i}, \rho_i$ to $G_{1,i}, G_{2,i}$ and $G_{i}$  in the corresponding derivation.
  
  Then we always have arrows $q_1\colon \lpro \der{D}_1\rpro \to \tpro{D}$, $q_2\colon \lpro \der{D}_2\rpro \to \tpro{D}$ such that:
  \begin{itemize}
  \item
    $q_1\circ \iota_{G_{1,0}}\circ \alpha_1 = \iota_{G_{0}}\circ
    \alpha$ and, for every $i\in [0, l_1-1]$
    \[q_1\circ \iota_{1, G_{1,i}}=\iota_{G_{i}}\]
  \item
    $q_2\circ \iota_{G_{2,l_2}}\circ \omega_2 =
    \iota_{G_{l_1+l_2}}\circ \omega$ and, for every
    $i\in [l_1+1, l_1+l_2]$
    \[q_2\circ \iota_{2, G_{2,i}}=\iota_{G_{i+l_1}}\]
  \end{itemize}
  
  Indeed, we have three cases:
  \begin{itemize}
  \item $\lgh(\der{D}_1)=0$. Then $(\der{D}, \alpha, \omega)$ is
    $(\der{D}_2, \alpha_2\circ \omega_1^{-1}\circ \alpha_1, \omega_2)$
    thus we can take as $q_2$ the identity on $\lpro
    \der{D}_2\rpro$. In this case $\iota_{1,G_{1,0}}$ is an
    isomorphism, so that the only possible choice for $q_1$ is
    $\iota_{G_0}\circ \alpha_2\circ \omega^{-1}_1 \circ
    \iota^{-1}_{1,G_{1,0}}$
  \item $\lgh(\der{D}_1)\neq 0$ and $\lgh(\der{D}_2)=0$ . Hence
    $(\der{D}, \alpha, \omega)$ is
    $(\der{D}_1, \alpha_1, \omega_1 \circ \alpha^{-1}_2\circ
    \omega_2)$. We can define $q_1$ as $\id{\lpro \der{D}_1\rpro}$,
    while, since $\iota_{2,G_{2,0}}$ is an isomorphism, $q_2$ must be
    $\iota_{G_{l_1+l_2}}\circ \omega_1 \circ \alpha^{-1}_2 \circ
    \iota^{-1}_{2,G_{2,0}}$.
  \item $\lgh(\der{D}_1)\neq 0$ and $\lgh(\der{D}_2)\neq 0$. In this
    case we have that $(\der{D}, \alpha, \omega)$ is
    $(\der{D}_1*\omega_1^{-1}\cdot \alpha_2*\der{D}_2, \alpha_1,
    \omega_2)$. By the second point of \Cref{lem:colim} and by
    \Cref{rem:abscons} we get the following diagram, in which the
    central square is a pushout:
    \[\xymatrix@C=40pt{&&G_{2,0} \ar@/_.3cm/[dl]_{\alpha^{-1}_2}
        \ar@/^.3cm/[dr]^{\iota_{2, G_{2,0}}}\\&\pi(G_{2,0})
        \ar[dr]^{\iota_{G_{l_1}}}\ar[r]^-{\iota'_{2, \pi(G_{2,0})}}
        \ar@{>->}[d]_-{\iota'_{1, \pi(G_{1,l_1})}} & \lpro
        \alpha_2*\der{D}_2 \rpro \ar@{>->}[d]^{p_2} &\tproi{D}{2}
        \ar[l]_-{\Gamma^{\alpha_2}} \ar@{>.>}@/^.2cm/[dl]^{q_2}\\
        G_{1,l_1} \ar@/^.3cm/[ur]^{\omega^{-1}_1}
        \ar@{>->}@/_.3cm/[dr]_{\iota_{1, G_{1, l_1}}}& \lpro
        \der{D}_1*\omega_1^{-1} \rpro \ar[r]_{p_1}& \tpro{D} \\
        &\tproi{D}{1} \ar[u]^{\Gamma_{\omega^{-1}_1}}
        \ar@{.>}@/_.2cm/[ur]_{q_1}}\] Let us define $q_1$ as
    $p_1\circ \Gamma_{\omega^{-1}_1}$ and $q_2$ as
    $p_2\circ \Gamma^{\alpha_2}$. Then, for every $i\in [0,l_1-1]$ and
    $j\in [l_1+1, l_1+l_2]$
    \[\begin{split}
        q_1\circ \iota_{1, G_{1,i}}&=p_1\circ
        \Gamma_{\omega^{-1}_1}\circ \iota_{1, G_{1,i}}\\&=p_1\circ
        \iota'_{1, G_{1,i}} \\&=\iota_{G_{i}}
      \end{split} \qquad \begin{split} q_2\circ \iota_{2,
          G_{2,j}}&=p_2\circ \Gamma^{\alpha_2}\circ \iota_{2,
          G_{2,j}}\\&=p_2\circ \iota'_{2, G_{2,j}}
        \\&=\iota_{G_{j+l_1}}
      \end{split}\] Moreover, we also have:
    \begin{align*}
      q_1\circ \iota_{G_{1,0}}\circ \alpha_1 & = p_1\circ \Gamma_{\omega^{-1}_1}\circ \iota_{G_{1,0}}\circ \alpha_1 \\&=p_1\circ \iota'_{1, G_{1,0}} \circ \alpha_1 \\&= \iota_{G_{0}}\circ \alpha
    \end{align*}
    and
    \begin{align*}
      q_2\circ \iota_{G_{2,l_2}}\circ \omega_2 & = p_2\circ \Gamma^{\alpha_2}\circ \iota_{G_{2,l_2}}\circ \omega_2 \\&=p_2\circ \iota'_{1, G_{2,l_2}} \circ \omega_2 \\&= \iota_{G_{l_1+l_2}}\circ \omega
    \end{align*}
  \end{itemize}

  It is worth noticing that, for every $i\in [0,l_1-1]$ and
  $j\in [l_1+1, l_1+l_2]$ the previous equalities also entail that,
  \[\begin{split}q_1\circ \iota_{1, D_{1,i}}&=\iota_{D_{i}}\\q_2\circ \iota_{2, D_{2,j}}&=\iota_{D_{j+l_1}} 	\\q_1\circ \iota_{1, R_{1,i-1}}&=\iota_{R_{i-1}} \\
      q_2\circ \iota_{2, R_{2,j-1}}&=\iota_{R_{j+l_1-1}}
    \end{split} \qquad
    \begin{split} q_1\circ \iota_{1, L_{1,i}}&=\iota_{L_{i}}
      \\q_2\circ \iota_{2, L_{2,j}}&=\iota_{L_{j+l_1}}\\ q_1\circ
      \iota_{1, K_{1,i}}&=\iota_{K_{i}}\\q_2\circ \iota_{2,
        K_{2,j}}&=\iota_{K_{j+l_1}}\end{split}\]
\end{remark}

We can now turn our attention to the case in which a consistent
permutation between two composite derivations is given.

\begin{lemma}[Restriction Lemma]\label{prop:uniqu}
  Let $\sigma\colon [0,n]\to [0,n]$ be a consistent permutations
  between $(\der{D}, \alpha, \omega)$ and
  $(\der{D}', \alpha', \omega')$ and suppose that
  \[(\der{D}, \alpha, \omega)=(\der{D}_1, \alpha_1, \omega_1) \cdot (\der{D}_2, \alpha_2, \omega_2) \qquad (\der{D}', \alpha', \omega')=(\der{D}'_1, \alpha'_1, \omega'_1) \cdot (\der{D}'_2, \alpha'_2, \omega'_2) \]
  
  Let $l_1$ be the length of $\der{D}_1$, and suppose that there exists a consistent permutation $\tau:[0,l_1-1]\to [0, l_1-1]$.  If $q_1\colon \lpro \der{D}_1\rpro \to \lpro \der{D}\rpro$  and $q'_1\colon \lpro \der{D}'_1\rpro \to \lpro \der{D}'\rpro$ are the canonical arrows defined above, then the following hold true:
  \begin{enumerate}
  \item  if $l_1=0$, then the following diagram commutes:
    \[\xymatrix{\lpro \der{D}_1\rpro \ar[r]^{\xi_\tau} \ar[d]_{q_1}& \lpro \der{D}'_1 \ar[d]^{q'_1}\rpro\\ \lpro \der{D}\rpro \ar[r]_{\xi_\sigma} & \lpro \der{D}'\rpro } \]
    % \[\xi_\sigma \circ q_1 \circ \iota_{1, G_{1,0}} = q'_1\circ \xi_{\tau} \circ  \iota_{1, G_{1,0}}\]
  \item if $l_1\neq 0$, so that $G_{1,0}=G_0$, then \[\xi_\sigma \circ \iota_{G_0} = q'_1\circ \xi_{\tau} \circ  \iota_{1, G_{1,0}}\]
  \end{enumerate}

  Moreover, let $E_{\sigma, \tau}$ be the set
  \[E_{\sigma, \tau}:=\{i\in  [0, l_1-1] \mid \sigma(j)=\tau(j) \text{ for every } j \leq i \}\]
  then also the following hold true:
  \begin{enumerate}
    \setcounter{enumi}{2}
  \item  if $E_{\sigma, \tau}\neq \emptyset$,  for every index $i\in [0, \max(E_{\sigma, \tau})]$ we have
    \[\xi_\sigma \circ \iota_{G_i}=q'_1 \circ \xi_{\tau} \circ \iota_{1, G_i}\]
  \item if $l_1-1\in E_{\sigma, \tau}$ and $l_2=0$ so that  $G_{l_1}=G_{1, l_1}$, then
    \[\xi_\sigma \circ \iota_{G_{l_1}}=q'_1 \circ \xi_{\tau} \circ \iota_{1, G_{l_1}}\]
  \item if $l_1-1\in E_{\sigma, \tau}$ and $l_2\neq 0$, so that $G_{l_1}=\pi(G_{2,0})$, then
    \[\xi_\sigma \circ \iota_{G_{l_1}}=q'_1 \circ \xi_{\tau} \circ \iota_{1, G_{l_1}}\circ \omega_1 \]
  \item if $l_1-1\in E_{\sigma, \tau}$, then the following diagram commutes:
    \[\xymatrix{\lpro \der{D}_1\rpro \ar[r]^{\xi_\tau} \ar[d]_{q_1}& \lpro \der{D}'_1 \ar[d]^{q'_1}\rpro\\ \lpro \der{D}\rpro \ar[r]_{\xi_\sigma} & \lpro \der{D}'\rpro } \]
  \end{enumerate}
\end{lemma}
\begin{proof}\begin{enumerate}
  \item Recall that, in this case \[q_1=\iota_{G_0}\circ \alpha_2\circ \omega^{-1}_1 \circ \iota^{-1}_{1,G_{1,0}} \qquad q'_1=\iota'_{G'_0}\circ \alpha'_2\circ (\omega'_1)^{-1} \circ (\iota'_{1,G'_{1,0}})^{-1}\]
    
    If we compute we have
    \begin{align*}
      \xi_\sigma \circ q_1 & =\xi_\sigma \circ \iota_{G_0}\circ \alpha_2\circ \omega^{-1}_1 \circ \iota^{-1}_{1,G_{1,0}} \\&=\iota'_{G'_0}\circ \alpha'\circ \alpha^{-1}\circ \alpha_2\circ \omega^{-1}_1\circ \iota^{-1}_{1,G_{1,0}}
    \end{align*}
    Now, since $l_1=0$ we also have that
    \[\alpha=\alpha_2\circ \omega_1^{-1}\circ \alpha_1 \qquad \alpha'=\alpha'_2\circ (\omega'_1)^{-1}\circ \alpha'_1\]
    therefore, using \Cref{rem:empty,ex:empty}
    \begin{align*}
      \alpha'\circ \alpha^{-1} & =\alpha'_2\circ (\omega'_1)^{-1}\circ \alpha'_1\circ \alpha^{-1}_1\circ \omega_1\circ \alpha^{-1}_2 \\&=\alpha'_2\circ (\omega'_1)^{-1}\circ \omega'_1\circ \omega^{-1}_1\circ \omega_1\circ \alpha^{-1}_2\\&=\alpha'_2\circ \alpha^{-1}_2
    \end{align*}
    Hence, again by \Cref{rem:empty,ex:empty}:
    \begin{align*}
      \xi_\sigma \circ q_1 & =\iota'_{G'_0}\circ \alpha'\circ \alpha^{-1}\circ \alpha_2\circ \omega^{-1}_1\circ \iota^{-1}_{1,G_{1,0}} \\&=\iota'_{G'_0}\circ \alpha'_2\circ \alpha^{-1}_2\circ \alpha_2\circ \omega^{-1}_1\circ \iota^{-1}_{1,G_{1,0}}\\&=\iota'_{G'_0}\circ \alpha'_2\circ \omega^{-1}_1\circ \iota^{-1}_{1,G_{1,0}}\\&=\iota'_{G'_0}\circ \alpha'_2\circ (\omega'_1)^{-1} \circ  \omega'_1\circ \omega^{-1}_1\circ \iota^{-1}_{1,G_{1,0}}\\&=\iota'_{G_0}\circ \alpha'_2\circ (\omega'_1)^{-1} \circ \alpha'_1\circ \alpha^{-1}_1\circ \iota^{-1}_{1,G_{1,0}}\\&=\iota'_{G'_0}\circ \alpha'_2\circ (\omega'_1)^{-1} \circ (\iota'_{1,G_{1,0}})^{-1}\circ \iota'_{1,G'_{1,0}}\circ \alpha'_1\circ \alpha^{-1}_1\circ \iota^{-1}_{1,G_{1,0}}\\&=q'_1\circ \iota'_{1,G'_{1,0}}\circ \alpha'_1\circ \alpha^{-1}_1\circ \iota^{-1}_{1,G_{1,0}}\\&=q'_1\circ \xi_{\tau}
    \end{align*}
                              
    
    
  \item Start noticing that
    \[\xi_\sigma \circ \iota_{G_0}=\iota'_{G'_0}\circ \alpha'\circ \alpha^{-1} \qquad \xi_{\tau} \circ \iota_{1,G_{1,0}}= \iota'_{1, G'_{1,0}} \circ \alpha'_1\circ \alpha^{-1}_1\]
    Therefore:
    \begin{align*}
      q'_1\circ \xi_{\tau} \circ \iota_{1, G_{1,0}} & =q'_1 \circ \iota'_{1, G'_{1,0}} \circ \alpha'_1\circ \alpha^{-1}_1 \\ &= \iota'_{G'_{0}}\circ \alpha' \circ \alpha^{-1}_1
    \end{align*}
    By hypothesis $\lgh(\der{D}_1) \neq 0$, thus $\alpha_1=\alpha$ and we can conclude.
    
  \item  From $E_{\sigma, \tau}\neq \emptyset$ we deduce that $l_1\neq 0$ and that $0\in E_{\sigma, \tau}$.   We proceed by induction on $i\in [0,\max(E_{\sigma, \tau})]$.
    
    \smallskip \noindent  If $i=0$ the thesis follows from point $1$.
    
    \smallskip \noindent If $i>0$, we proceed as in the proof of \Cref{prop:isouno} considering the pushout square
    \[\xymatrix{K_{i-1}\ar[r]^{r_{i-1}} \ar[d]_{k_{i-1}}& R_{i-1} \ar[d]^{h_{i-1}}\\ D_{i-1} \ar[r]_{g_{i-1}}& G_i}\]
    Since $i$ belongs to $E_{\sigma, \tau}$, then $i-1\in E_{\sigma, \tau}$ too. By definition, $i\leq l_1-1$, so that, using \Cref{rem:coproj} we have
    \[\begin{split}
        \xi_\sigma \circ \iota_{G_i}\circ h_{i-1}&=  \xi_\sigma\circ \iota_{R_{i-1}}\\&=\iota'_{R'_{\sigma(i-1)}}\\&=\iota'_{R'_{\tau(i-1)}}\\&=q'_1\circ \iota'_{1, R'_{1,\tau(i-1)}}\\&=q'_1\circ \xi_{\tau}\circ \iota_{1, R_{i-1}}\\&=q'_1\circ \xi_{\tau} \circ \iota_{1, G_{1,i}}\circ h_{1, i-1}\\&=q'_1\circ \xi_\tau \circ \iota_{1, G_i} \circ h_{i-1}
      \end{split} \]
    
    \Cref{rem:dett} and the induction hypothesis give us
    \begin{align*}
      \xi_\sigma \circ \iota_{G_i}\circ g_{i-1} & =\xi_{\sigma}\circ \iota_{D_{i-1}} \\&=\xi_{\sigma}\circ \iota_{G_{i-1}} \circ f_{i-1}\\&=q'_1 \circ \xi_{\tau} \circ \iota_{1, G_{i-1}} \circ f_{i-1}\\&=q'_1 \circ \xi_{\tau} \circ \iota_{1, G_{i-1}} \circ f_{1, i-1}\\&=q'_1 \circ \xi_{\tau} \circ \iota_{1, D_{i-1}}\\&=q'_1 \circ \xi_{\tau} \circ \iota_{1, G_i}\circ g_{1, i-1}\\&=q'_1 \circ \xi_{\tau} \circ \iota_{1, G_i}\circ g_{i-1}
    \end{align*}
    and the thesis now follows.
  \item Since $l_1-1\in E_{\sigma, \tau}$ we have $l_1\neq 0$ and $E_{\sigma, \tau}=[0,l_1-1]$.  We can start with the pushout
    \[\xymatrix{K_{l_1-1}\ar[r]^{r_{l_1-1}} \ar[d]_{k_{l_1-1}}& R_{i-1} \ar[d]^{h_{l_1-1}}\\ D_{l_1-1} \ar[r]_{g_{l_1-1}}& G_{l_1}}\]
    By \Cref{def:conc}, we have
    \[h_{l_1-1}=h_{1, l_1} \qquad g_{l_1-1}=g_{1,l_1}\]
    which, since $q'_1=\id{\lpro \der{D}'_1\rpro}$, in turn, implies that
    \[\iota'_{R_{\tau(l_1-1)}}=q'_1\circ \iota'_{1, R'_{1,\tau(l_1-1)}}\]
    We can thus repeat the same argument of the previous point to deduce
    \begin{align*}
      \xi_\sigma \circ \iota_{G_{l_1}}\circ h_{l_1-1}&=  \xi_\sigma\circ \iota_{R_{l_1-1}}\\&=\iota'_{R'_{\sigma(l_1-1)}}\\&=\iota'_{R'_{\tau(l_1-1)}}\\&=q'_1\circ \iota'_{1, R'_{1,\tau(l_1-1)}}\\&=q'_1\circ \xi_{\tau}\circ \iota_{1, R_{l_1-1}}\\&=q'_1\circ \xi_{\tau} \circ \iota_{1, G_{1,l_1}}\circ h_{1, l_1-1}\\&=q'_1\circ \xi_\tau \circ \iota_{1, G_{l_1}} \circ h_{l_1-1}
    \end{align*}
    
    By hypothesis $l_1-1\in E_{\sigma, \tau}$, so that pint $3$ of this lemma to get
    \[\xi_\sigma \circ \iota_{G_{l_1-1}}=q'_1 \circ \xi_{\tau} \circ \iota_{1, G_{l_1-1}}\]
    
    From this the thesis follows after a brief computation:
    \begin{align*}
      \xi_\sigma \circ \iota_{G_{l_1}}\circ g_{l_1-1} & =\xi_{\sigma}\circ \iota_{D_{l_1-1}} \\&=\xi_{\sigma}\circ \iota_{G_{l_1-1}} \circ f_{l_1-1}\\&=q'_1 \circ \xi_{\tau} \circ \iota_{1, G_{l_1-1}} \circ f_{l_1-1}\\&=q'_1 \circ \xi_{\tau} \circ \iota_{1, G_{l_1}-1} \circ f_{1, l_1-1}\\&=q'_1 \circ \xi_{\tau} \circ \iota_{1, D_{l_1-1}}\\&=q'_1 \circ \xi_{\tau} \circ \iota_{1, G_{l_1}}\circ g_{1, l_1-1}\\&=q'_1 \circ \xi_{\tau} \circ \iota_{1, G_{l_1}}\circ g_{l_1-1}
    \end{align*}
    
  \item In this case, we can consider the following diagram, in which the inner rectangle and the outer border are pushouts
    \[\xymatrix@C=35pt{K_{1,l_1-1}\ar[r]^{r_{1,l_1-1}} \ar[d]_{k_{1,l_1-1}}& R_{i-1} \ar[d]_{h_{1,l_1-1}} \ar@/^.3cm/[ddr]^{h_{l_1-1}}\\ D_{1,l_1-1} \ar[r]_{g_{1,l_1-1}} \ar@/_.3cm/[drr]_{g_{l_1-1}}& G_{1,l_1} \ar[dr]^{\omega_1^{-1}}\\ &&\pi(G_{2,0})}\]
    
    
    \[	\xi_\sigma \circ \iota_{G_{l_1}}=q'_1 \circ \xi_{\tau} \circ \iota_{1, G_{l_1}}\circ \omega_1\]
    We can start noticing that
    \[q'_1\circ \iota'_{1, G'_{1,l_1}}=\iota'_{G'_{l_1}} \circ \omega_1^{-1} \]
    Therefore
    \begin{align*}q'_1\circ \iota'_{1, R'_{1,l_1}} & =q'_1\circ \iota'_{1, G'_{1,l_1}}\circ  h'_{1, l_1} \\&=q'_1\circ \iota'_{1, G'_{1,l_1}}\circ \omega_1 \circ  h'_{ l_1}\\&= \iota'_{G'_{l_1}}\circ h'_{l_1}\\&=\iota'_{R'_{l_1-1}}
    \end{align*}
                              
    The argument now proceeds almost verbatim like the one of the previous two point.
    On the one hand we have
    \begin{align*}
      \xi_\sigma \circ \iota_{G_{l_1}}\circ h_{l_1-1}&=  \xi_\sigma\circ \iota_{R_{l_1-1}}\\&=\iota'_{R'_{\sigma(l_1-1)}}\\&=\iota'_{R'_{\tau(l_1-1)}}\\&=q'_1\circ \iota'_{1, R'_{1,\tau(l_1-1)}}\\&=q'_1\circ \xi_{\tau}\circ \iota_{1, R_{l_1-1}}\\&=q'_1\circ \xi_{\tau} \circ \iota_{1, G_{1,l_1}}\circ h_{1, l_1-1}\\&=q'_1\circ \xi_\tau \circ \iota_{1, G_{1, l_1}} \circ \omega_1 \circ  h_{l_1-1}			\end{align*}
    On the other hand, by point $3$ of this lemma we have
    \begin{align*}
      \xi_\sigma \circ \iota_{G_{l_1}}\circ g_{l_1-1} & =\xi_{\sigma}\circ \iota_{D_{l_1-1}} \\&=\xi_{\sigma}\circ \iota_{G_{l_1-1}} \circ f_{l_1-1}\\&=q'_1 \circ \xi_{\tau} \circ \iota_{1, G_{l_1}} \circ f_{l_1-1}\\&=q'_1 \circ \xi_{\tau} \circ \iota_{1, G_{l_1}} \circ f_{1, {l_1}-1}\\&=q'_1 \circ \xi_{\tau} \circ \iota_{1, D_{{l_1}-1}}\\&=q'_1 \circ \xi_{\tau} \circ \iota_{1, G_{l_1}}\circ g_{1, l_1-1}\\&=q'_1 \circ \xi_{\tau} \circ \iota_{1, G_{l_1}}\circ \omega_1\circ g_{l_1-1}
    \end{align*}

  \item  The thesis follows from points $3$ and $4$ if $l_2=0$, from $3$ and $5$ if $l_2\neq 0$.	\qedhere
  \end{enumerate}
\end{proof}

We are now ready to prove the central result of this section.
\begin{lemma}\label{lem:impo}
  Let $(\X,R)$ be a left-linear DPO rewriting system and let also
  $(\der{D}_1, \alpha_1, \omega_1)$, $(\der{D}_2, \alpha_2, \omega_2)$
  be two composable decorated derivations of length, respectively,
  $l_1$ and $l_2$. Suppose that
  $\sigma:[0, l_1+l_2-1]\to [0, l_1+l_2-1]$ is a consistent
  permutation between
  $(\der{D}_1, \alpha_1, \omega_1)\cdot (\der{D}_2, \alpha_2,
  \omega_2)$ and
  $(\der{D}'_1, \alpha'_1, \omega'_1)\cdot (\der{D}'_2, \alpha'_2,
  \omega'_2)$ for some other two composable decorated derivations
  $(\der{D}_1, \alpha_1, \omega_1)$ and
  $(\der{D}_2, \alpha_2, \omega_2)$.  If
  $\tau:[0,l_1-1]\to [0, l_1-1]$ is a consistent permutation between
  $(\der{D}_1, \alpha_1, \omega_1)$ and
  $(\der{D}'_1, \alpha'_1, \omega'_1)$. Suppose that the set
  \[D_{\sigma, \tau}:=\{i\in [0, l_1-1]\mid \sigma(i)\neq \tau(i)\}\]
  is non-empty and let $j$ be its minimum. Let also $r(\der{D}_1)$ be $\{\rho_i\}_{i=0}^{l_1-1}$. Then the following hold true:
  \begin{enumerate}
  \item if $j=0$, then the rule $\rho_0$ is not consuming;
  \item if $j\neq 0$ then the rule $\rho_{j-1}$ is not consuming.
  \end{enumerate}
\end{lemma}
\begin{remark}\label{rem:minmax}It is worth to notice that the hypothesis $D_{\sigma, \tau} \neq \emptyset$ entails $l_1\neq 0$. Moreover, if $j\neq 0$, then $j-1$ is the maximum of $E_{\sigma, \tau}$.
\end{remark}

\begin{proof}
  \begin{enumerate}
  \item Let $k$ be $\sigma^{-1}(\tau(0))$ and notice that, since
    $\sigma(0)\neq \tau(0)$, then $0< k$.  By the second point of
    \Cref{prop:uniqu} we can consider the diagram
    \[\xymatrix@C=40pt{&& L_0 \ar[d]_{m'_{\tau(0)}}
        \ar@/^.2cm/[dr]^{m_{k}} \ar@/_.4cm/[dll]_{m_{0}} \\G_{0}
        \ar@/^.2cm/[drrr]^(.4){\iota_{G_0}}|(.67)\hole
        \ar[d]_{\iota_{1,G_{1,0}}}& &G'_{\tau(0)}
        \ar[d]^(.45){\iota'_{G'_{\tau(0)}}} & G_{k}
        \ar[d]^{\iota_{G_k}}\\ \lpro\der{D}_1 \rpro
        \ar[r]_{\xi_{\tau}} &\lpro \der{D}'_1\rpro \ar[r]_{q'_1} &
        \lpro \der{D'}\rpro & \tpro{D} \ar[l]^{\xi_{\sigma}}}\] From
    \Cref{cor:ele}, we can conclude that there exists
    $c\colon L_0\to D_0$ such that $f_0\circ c=m_0$. We thus have the
    solid part of the commutative diagram below.
    \[\xymatrix{L_0 \ar@/^.3cm/[drr]^{\id{L_0}} \ar@{.>}[dr]_{t}
        \ar@/_.3cm/[ddr]_{c}\\ & K_0 \ar[d]_{k_0}\ar@{>->}[r]^{l_0}&
        L_0 \ar[d]^{m_0} \\& D_0 \ar@{>->}[r]_{f_0} & G_0} \]
    The internal square is an $\mathcal{M}$-pushout and thus a
    pullback, by \Cref{prop:pbpoad}, so that we have the existence of
    the dotted $t\colon L_0\to K_0$. Therefore $\id{L_0}=l_0\circ t$,
    proving that $l_0$ is an epimorphism. The thesis now follows from
    \Cref{cor:rego}.
  \item Let $k$ be $\sigma^{-1}(\tau(j-1))$ and notice that
    $\rho_{j-1}=\rho_k$. We have already noticed in \Cref{rem:minmax}
    that $j-1$ is the maximum of $E_{\sigma, \tau}$.  Hence, by the
    third point of \Cref{prop:uniqu} we have
    \[\xymatrix@C=40pt{&& L_{j-1} \ar[d]_{m'_{\tau({j-1})}}
        \ar@/^.2cm/[dr]^{m_{k}} \ar@/_.4cm/[dll]_{m_{{j-1}}}
        \\G_{1,{j-1}}
        \ar@/^.2cm/[drrr]^(.4){\iota_{G_{j-1}}}|(.67)\hole
        \ar[d]_{\iota_{1,G_{1,j-1}}}& &G'_{\tau({j-1})}
        \ar[d]^{\iota'_{G'_{\tau({j-1})}}} & G_{k}
        \ar[d]^{\iota_{G_k}}\\ \lpro\der{D}_1 \rpro
        \ar[r]_{\xi_{\tau}} &\lpro \der{D}'_1\rpro \ar[r]_{q'_1} &
        \lpro \der{D'}\rpro & \tpro{D} \ar[l]^{\xi_{\sigma}}}\]
    Let
    $a$ be $\min(j-1, k)$, by \Cref{cor:ele}, there exists
    $c\colon L_{j-1}\to D_a$ such that $f_a\circ c=m_a$. As before
    this yields the solid part of the following diagram, in which the
    square is an $\mathcal{M}$-pushout.
    \[\xymatrix{L_{j-1} \ar@/^.3cm/[drr]^{\id{L_{j-1}}}
        \ar@{.>}[dr]_{t} \ar@/_.3cm/[ddr]_{c}\\ & K_{j-1}
        \ar[d]_{k_a}\ar@{>->}[r]^{l_{j-1}}& L_{j-1} \ar[d]^{m_a} \\&
        D_a \ar@{>->}[r]_{f_a} & G_a} \]
    The existence of the dotted $t\colon L_{j-1}\to K_{j-1}$ follows
    from \Cref{prop:pbpoad} and we can conclude. \qedhere
  \end{enumerate}
\end{proof}


\begin{corollary}[Uniqueness of consistent
  permutation]\label{cor:unique}
  Let $(\X,\R)$ be a consuming left-linear DPO rewriting system. Then,
  for every two decorated derivations $(\der{D}, \alpha, \omega)$ and
  $(\der{D}', \alpha', \omega')$, there exists at most one consistent
  permutation between them.
\end{corollary}
\begin{proof}
  Let
  $\sigma, \tau: [0, \lgh(\der{D})-1]\rightrightarrows [0,
  \lgh(\der{D})-1]$ be two consistent permutations. Let $H$ and $H'$
  be, respectively, the target of $\der{D}$ and $\der{D}'$. Take two
  isomorphisms $\gamma:\pi(H)\to \gamma$, $\gamma':\pi(H')\to H'$,
  then, according to \Cref{def:conc}:
  \begin{align*}(\der{D}, \alpha, \omega) & =(\der{D}, \alpha,
    \omega)\cdot (\der{D}_2, \gamma, \gamma) \\(\der{D}', \alpha',
    \omega')&=(\der{D}', \alpha', \omega')\cdot (\der{D}'_2, \gamma',
    \gamma')
  \end{align*}
  where $\der{D}_2$ and $\der{D}'_2$ are the empty derivation on $H$
  and $H'$. Since $(\X, \R)$ is consuming, then \Cref{lem:impo}
  entails $D_{\sigma, \tau}=\emptyset$, from which the thesis follows.
\end{proof}



\begin{lemma}\label{lem:presuffix} Let $(\X, \R)$ be a consuming
  left-linear DPO rewriting system.  Let also
  $(\der{D}_1, \alpha_1, \omega_1)$, $(\der{D}_2, \alpha_2, \omega_2)$
  be two composable decorated derivations of length, respectively,
  $l_1$ and $l_2$. Suppose that
  $\sigma\colon [0, l_1+l_2-1]\to [0, l_1+l_2-1]$ is a consistent
  permutation between
  $(\der{D}_1, \alpha_1, \omega_1)\cdot (\der{D}_2, \alpha_2,
  \omega_2)$ and
  $(\der{D}'_1, \alpha'_1, \omega'_1)\cdot (\der{D}'_2, \alpha'_2,
  \omega'_2)$ for some other two composable decorated derivations
  $(\der{D}_1, \alpha_1, \omega_1)$ and
  $(\der{D}_2, \alpha_2, \omega_2)$.  If
  $\tau\colon [0,l_1-1]\to [0, l_1-1]$ is a consistent permutation
  between $(\der{D_1}, \alpha_1, \omega_1)$ and
  $(\der{D}'_1, \alpha'_1, \omega'_1)$. Then the following hold true:
  \begin{enumerate}
  \item suppose that $l_1$ and $l_2$ are different from $0$ and
    consider the following two pushout squares given by
    \Cref{rem:dett}:
    \[\xymatrix@C=45pt@R=30pt{\pi(G_{2,0}) \ar@{>->}[d]_{\iota_{1,
            G_{1,n+1}} \circ \omega_1}\ar[r]^{\iota_{2, G_{2,0}} \circ
          \alpha_2}& \lpro \der{D}_2 \rpro \ar@{>->}[d]^{q_2} &
        \pi(G'_{2,0}) \ar@{>->}[d]_{\iota'_{1, G'_{1,n+1}} \circ
          \omega'_1} \ar[r]^{\iota'_{2, G'_{2,0}} \circ \alpha'_2}&
        \lpro \der{D}'_2 \rpro \ar@{>->}[d]^{q'_2} \\ \lpro \der{D}_1
        \rpro \ar[r]_{q_1} & \tpro{D} & \lpro \der{D}'_1\rpro
        \ar[r]_{q'_1} & \lpro \der{D}'\rpro }\] then, for every
    $i\in [0, l_2]$ there exists a unique
    $\zeta_i\colon G_{2,i}\to \lpro \der{D}_2\rpro $ fitting in the
    diagram below
    \[\xymatrix@R=35pt{G_{2,i} \ar[r]^{\iota_{2,G_{2,i}}} \ar@{.>}[d]_{\zeta_i}&\lpro \der{D}_2 \rpro \ar@{>->}[r]^{q_2}& \tpro{D} \ar[d]^{\xi_{\sigma}} \\
        \lpro \der{D}'_2 \rpro \ar@{>->}[rr]_{q'_2}&& \lpro
        \der{D}'\rpro }\]
  \item the permutation
    \[\varrho\colon [0,l_2-1]\to [0, l_2-1] \qquad i \mapsto
      \sigma(i+l_1)-l_1\] is a consistent one.
  \end{enumerate}
\end{lemma}

\begin{proof}\begin{enumerate}
  \item We can start noticing that by \Cref{lem:impo},
    $\sigma(i)=\tau(i)$ for every $i\in [0, l_1-1]$. By the second
    point of \Cref{prop:uniqu} we can deduce that
    \[\xi_\sigma \circ \iota_{G_{i}}=\]Let us prove the existence of
    $\zeta_i$ by induction on $i\in [0, l_2]$.

    \smallskip \noindent $i=0$. We already know that $G_{l_1}$ and
    $G'_{l_1}$ are both equal to $\pi(G_{2,0})$.  Moreover, we also
    know that
    \begin{align*}
      q_2\circ \iota_{2, G_{2,0}}=\iota_{G_{l_1}} \circ \alpha^{-1}_2 \qquad
    \end{align*}

    Thus we have a diagram
    \[\xymatrix{&G_{2,0} \\
        G_{l_1} \\
        G'_{2,0}&}\]

    Moreover, $\pi(mj$
    \begin{align*}
      \xi_\sigma \circ p_2\circ \iota_{2, G_{2,0}} & =\xi_\sigma \circ \iota_{G_{n+1}}\circ \alpha^{-1}_2 \\&g
    \end{align*}

    \smallskip \noindent $i>0$.


    The uniqueness half of the thesis follows at once since $p'_2$ is
    in $\mathcal{M}$.

  \item Let us split the cases.

    \smallskip \noindent $l_1=0$. then $\varrho=\sigma$ and there is
    nothing to proof.

    \smallskip \noindent $l_2=0$ and $l_1\neq0$. Then
    $\varrho=\id{\emptyset}$ and

    \smallskip \noindent $l_1\neq0$ and $l_2\neq 0$.
    \[\xymatrix{xy code}\]

    Consistency now follows at once. \qedhere
  \end{enumerate}
\end{proof}


Finally, we can show that consistent permutations between composable abstract derivations yield a consistent permutation between the composites.

\begin{lemma}\label{lem:sum} Let $(\der{D}, \alpha, \omega)$
  and $(\der{D}', \alpha', \omega')$ be two abstract decorated
  derivations such that
  \[(\der{D}, \alpha, \omega)=(\der{D}_1, \alpha_1,
    \omega_1)\cdot (\der{D}_2, \alpha_2, \omega_2) \qquad
    (\der{D}', \alpha', \omega')=(\der{D}'_1, \alpha'_1,
    \omega'_1)\cdot (\der{D}'_2, \alpha'_2, \omega'_2)\] Given
  a consistent permutation
  $\sigma\colon [0, \lgh(\der{D}_1)-1]\to [0,
  \lgh(\der{D}_1)-1]$ between
  $(\der{D}_1, \alpha_1, \omega_1)$ and
  $(\der{D}'_1, \alpha'_1, \omega'_1)$ and another
  $\tau\colon [0, \lgh(\der{D}_2)-1]\to [0,
  \lgh(\der{D}_2)-1]$ between
  $(\der{D}_2, \alpha_2, \omega_2)$ and
  $(\der{D}'_2, \alpha'_2, \omega'_2)$, then
  \[\sigma+\tau\colon[0, \lgh(\der{D})-1]\to[0,
    \lgh(\der{D})-1] \quad i \mapsto \begin{cases}
      \sigma(i)                               & i < \lgh(\der{D}_1)   \\
      \lgh(\der{D}_1)+\tau(i-\lgh(\der{D}_1)) & i\geq
      \lgh(\der{D}_1)
    \end{cases}\] is a consistent permutation between
  $(\der{D}, \alpha, \omega)$ and
  $(\der{D}', \alpha', \omega')$.
\end{lemma}

\begin{proof}
  Let us start fixing some notation. Given $X_1\in \Delta(\der{D}_1)$,
  $X_2\in \Delta(\der{D}_2)$, $X'_1\in \Delta(\der{D}'_1)$,
  $X'_2\in \Delta(\der{D}'_2)$, we will denote their coprojections
  into, respectively, $\lpro \der{D}_1\rpro$, $\lpro \der{D}'_2\rpro$,
  $\lpro \der{D}'_1\rpro$, $\lpro \der{D}'_2\rpro$ by
  $\iota_{1, X_1}\colon X_1\to \lpro\der{D}_1\rpro$,
  $\iota_{1, X_2}\colon X_2\to \lpro\der{D}_1\rpro$,
  $\iota'_{1, X'_1}\colon X'_1\to \lpro\der{D}'_1\rpro$ and
  $\iota'_{2, X'_2}\colon X'_2\to \lpro\der{D}_1\rpro$. Similarly,
  given $X\in \Delta(\der{D})$ and $\X'\in \Delta(\der{D}')$ we will
  use $\iota_X\colon X\to \tpro{D}$ and
  $\iota'_{X'}\colon X'\to \lpro \der{D}' \rpro$ for the
  coprojections.

  Now that these preliminaries matters have been addressed, we will proceed with the rest of the proof.. We split the proof in three cases.
  \begin{itemize}
  \item $\lgh(\der{D}_1)=0$. Thus $\lgh(\der{D}'_1)=0$ too  and we have
    \begin{align*}
      (\der{D}, \alpha, \omega) & =(\der{D}_2, \alpha_2\circ \omega_1^{-1}\circ \alpha_1, \omega_2) \\ (\der{D}', \alpha', \omega')&=(\der{D}'_2, \alpha'_2\circ (\omega'_1)^{-1}\circ \alpha'_1, \omega'_2)
    \end{align*}
    Moreover, in this case $\sigma$ must be $\id{\emptyset}$ and $\sigma+\tau$ must be equal to $\tau$. The thesis now follows from the commutativity of the following diagram.
    \[\xymatrix@C=35pt{& G_{1,0}\ar[dd]^{\xi_{\id{\emptyset}}} \ar[dr]^{\omega^{-1}_1}&&G_{2,0}\ar[r]^-{\iota_{2, G_{2,0}}} & \lpro \der{D}_2\rpro \ar[dd]^{\xi_{\tau}}\\\pi(G_{1,0})  \ar[ur]^{\alpha_1} \ar[dr]_{\alpha'_1}&& \pi(G_{2,0}) \ar[ur]^{\alpha_2} \ar[dr]_{\alpha'_2}\\& G'_{1,0} \ar[ur]_{(\omega'_1)^{-1}}&&G'_{2,0} \ar[r]_-{\iota'_{2, G_{2,0}}}& \lpro \der{D}'_2\rpro}\]

  \item $\lgh(\der{D}_2)=0$.  As before this implies that also $\lgh(\der{D}'_2)$ is $0$.  Applying \Cref{def:conc} we get
    \begin{align*}
      (\der{D}, \alpha, \omega) & =(\der{D}_1, \alpha_1, \omega_1 \circ (\alpha_2)^{-1}\circ \omega_2) \\ (\der{D}', \alpha', \omega')&=(\der{D}'_1, \alpha'_1, \omega'_1 \circ (\alpha'_2)^{-1}\circ \omega'_2)
    \end{align*}
    Since $\der{D}_2$ is empty, then $\tau=\id{\emptyset}$ and
    $\sigma+\tau=\sigma$. Let $n$ be $\lg(\der{D}_1)$, as before the
    thesis follows from the diagram below.
    \[\xymatrix@C=35pt{& G_{2,0}\ar[dd]^{\xi_{\id{\emptyset}}}
        \ar[dr]^{\alpha^{-1}_2}&&G_{1,n}\ar@{>->}[r]^-{\iota_{1,
            G_{1,n}}} & \lpro \der{D}_2\rpro
        \ar[dd]^{\xi_{\sigma}}\\\pi(G_{2,0}) \ar[ur]^{\omega_2}
        \ar[dr]_{\omega'_2}&& \pi(G_{1,n}) \ar[ur]^{\omega_1}
        \ar[dr]_{\omega'_1}\\& G'_{2,0}
        \ar[ur]_{(\alpha'_1)^{-1}}&&G'_{1,n} \ar@{>->}[r]_-{\iota'_{1,
            G_{1,n}}}& \lpro \der{D}'_2\rpro}\]

  \item $\lgh(\der{D}_1)\neq0$ and $\lgh(\der{D}_2)\neq 0$. Thus
    $\der{D}'_1$ and $\der{D}'_2$ are non-empty too. In this case we
    have:
    \begin{align*}
      (\der{D}, \alpha, \omega)    & =(\der{D}_1*\omega_1^{-1}\cdot \alpha_2*\der{D}_2, \alpha_1, \omega_2)         \\
      (\der{D}', \alpha', \omega') & =(\der{D}'_1*(\omega'_1)^{-1}\cdot \alpha'_2*\der{D}'_2, \alpha'_1, \omega'_2)
    \end{align*}

    Let us assume that $\der{D}_1$, $\der{D}'_1$, $\der{D}_2$ and
    $\der{D}'_2$ are given by
    \[\der{D}_1=\{\dder{D}_{1,i}\}_{i=0}^n \quad
      \der{D}'_1=\{\dder{D}'_{1,i}\}_{i=0}^n \quad
      \der{D}_2=\{\dder{D}_{2,i}\}_{i=0}^t \quad
      \der{D}_2'=\{\dder{D}'_{2,i}\}_{i=0}^t\] By definition of
    consistent permutation, the rule applied by $\dder{D}_{1,i}$ and
    the one applied in $\dder{D}_{2,i}$ must coincide with,
    respectively, the one applied in $\dder{D}'_{1,i}$ and the one
    applied $\dder{D}'_{2,1}$. Let $\dder{D}_{1,i}$,
    $\dder{D}'_{1,i}$, $\dder{D}_{2,i}$ and $\dder{D}'_{2,i}$ be
    given, by the following four diagrams.
    \[\xymatrix{L_{1,i} \ar[d]_{m_{1, i}}& K_{1,i} \ar[d]_{k_{1, i}}
        \ar[r]^{r_{1,i}} \ar@{>->}[l]_{l_{1,i}} & R_{1,i}\ar[d]^{h_{1,
            i}} &L_{1,i} \ar[d]_{m'_{1, i}}& K_{1,i} \ar[d]_{k'_{1,
            i}} \ar[r]^{r_{1,i}} \ar@{>->}[l]_{l_{1,i}} &
        R_{1,i}\ar[d]^{h'_{1, i}} \\G_{1,i} & D_{1,i} \ar[r]_{g_{1,i}}
        \ar@{>->}[l]^{f_{1,i}} & G_{1,i+1} & G'_{1,i} &
        D'_{1,i}\ar[r]_{g'_{1,i}} \ar@{>->}[l]^{f'_{1,i}} &
        G'_{1,i+1}}\]
    \[\xymatrix{L_{2,i} \ar[d]_{m_{2, i}}& K_{2,i} \ar[d]_{k_{2, i}}
        \ar[r]^{r_{2,i}} \ar@{>->}[l]_{l_{2,i}} & R_{2,i}\ar[d]^{h_{2,
            i}} &L_{2,i} \ar[d]_{m'_{2, i}}& K_{2,i} \ar[d]_{k'_{2,
            i}} \ar[r]^{r_{2,i}} \ar@{>->}[l]_{l_{2,i}} &
        R_{2,i}\ar[d]^{h'_{2, i}} \\G_{2,i} & D_{2,i} \ar[r]_{g_{2,i}}
        \ar@{>->}[l]^{f_{2,i}} & G_{2,i+1} & G'_{2,i} &
        D'_{2,i}\ar[r]_{g'_{2,i}} \ar@{>->}[l]^{f'_{2,i}} &
        G'_{2,i+1}}\]

    % Let $(\lpro \der{D}_1*(\omega_1)^{-1}\rpro, \{j_X\}_{X\in \Delta( \der{D}_1*(\omega_1)^{-1})})$, $(\lpro \der{D}'_1*(\omega'_1)^{-1}\rpro, \{j'_X\}_{X\in \Delta( \der{D}'_1*(\omega'_1)^{-1})})$, $(\lpro \der{D}_1*(\omega_1)^{-1}\rpro, \{j_X\}_{X\in \Delta( \der{D}_1*(\omega_1)^{-1})})$ and $(\lpro \der{D}'_1*(\omega'_1)^{-1}\rpro, \{j'_X\}_{X\in \Delta( \der{D}'_1*(\omega'_1)^{-1})})$ be colimiting cocones. 
    By \Cref{rem:dett} we get the following diagram, in which the two solid squares are pushouts:
    \[\xymatrix@C=45pt@R=30pt{\pi(G_{2,0}) \ar@/^.4cm/[rrr]^{\id{\pi(G_{2,0})}}\ar@{>->}[d]_{\iota_{1, G_{1,n+1}} \circ \omega_1}\ar[r]_{\iota_{2, G_{2,0}} \circ \alpha_2}& \lpro \der{D}_2 \rpro \ar@{>->}[d]^{q_2}\ar[r]_{\xi_\tau} & \lpro \der{D}'_2 \rpro \ar@{>->}[d]_{q'_2} &\pi(G'_{2,0}) \ar@{>->}[d]^{\iota'_{1, G'_{1,n+1}} \circ \omega'_1} \ar[l]^{\iota'_{2, G'_{2,0}} \circ \alpha'_2}\\ \lpro \der{D}_1 \rpro \ar@/_.4cm/[rrr]_{\xi_\sigma} \ar[r]^{q_1} & \tpro{D} \ar@{.>}[r]^{\xi_{\sigma+\tau}} & \lpro \der{D}'\rpro & \lpro \der{D}'_1\rpro  \ar[l]_{q'_1}}\]

    Moreover, we can point out that, for every index $i\in [0,n+t+2]$ we have
    
    \[\iota_{G_i}=\begin{cases}q_1\circ \iota_{1, G_{1,i}}                & i \leq n \\
        q_1\circ \iota_{1, G_{1, i}}\circ \omega_1 & i=n+1    \\
        q_2\circ \iota_{2, G_{2,i-(n+1)}}          & i > n+1
      \end{cases} \qquad \iota'_{G'_i}=\begin{cases}q'_1\circ \iota'_{1, G'_{1,i}}                 & i \leq n \\
        q'_1\circ \iota'_{1, G'_{1, i}}\circ \omega'_1 & i=n+1    \\
        q'_2\circ \iota'_{2, G'_{2,i-(n+1)}}           & i > n+1
      \end{cases}\]
    
    Now, on the one hand, the existence of the dotted arrow $\xi_{\sigma+\tau}\colon \tpro{D}\to \lpro \der{D}'\rpro$ in the diagram above follows from the following computation:
    \begin{align*}
      q'_1\circ \xi_\sigma \circ \iota_{1, G_{1,n+1}} \circ \omega_1 & =q'_1\circ \iota'_{1,G'_{1, n+1}} \circ \omega'_1 \\&= q'_2 \circ \iota'_{2, G'_{2,0}} \circ \alpha'_2\\&=q'_2\circ \xi_\tau \circ \iota_{2, G_{2,0}} \circ \alpha_2
    \end{align*}
    On the other hand, we can repeat the same computation for $\xi^{-1}_\sigma$ and $\xi^{-1}_\tau$ to get
    \begin{align*}
      q_1\circ \xi^{-1}_\sigma \circ \iota'_{1, G'_{1,n+1}} \circ \omega'_1 & =q_1\circ \iota_{1,G_{1, n+1}} \circ \omega_1 \\&= q_2 \circ \iota_{2, G_{2,0}} \circ \alpha_2\\&=q_2\circ \xi^{-1}_\tau \circ \iota'_{2, G'_{2,0}} \circ \alpha'_2
    \end{align*}
    Hence there exists a $\psi\colon \lpro \der{D}'\rpro \to \tpro{D}$ such that:
    \[\psi \circ q'_1=q_1\circ \xi^{-1}_\sigma \qquad \psi \circ q'_2=q_2\circ \xi^{-1}_\tau\]
    Furthermore, the computations below show that $\psi$ is the inverse of $\xi_\sigma$.
    \[\begin{split}
        \psi \circ \xi_{\sigma+\tau}\circ  q_1&=\psi \circ q'_1\circ \xi_\sigma \\&=q_1\circ \xi^{-1}_{\sigma}\circ \xi_\sigma\\&=q_1\\\\
        \xi_{\sigma+\tau}\circ \psi \circ q'_1&=\xi_{\sigma+\tau}\circ q_1\circ \xi^{-1}_\sigma \\&=q'_1\circ \xi_{\sigma}\circ \xi^{-1}_\sigma\\&=q'_1
      \end{split}\qquad \begin{split}
        \psi \circ \xi_{\sigma+\tau}\circ  q_2&=\psi \circ q'_2\circ \xi_\tau \\&=q_2\circ \xi^{-1}_{\tau}\circ \xi_\tau\\&=q_2\\ \\ 	\xi_{\sigma+\tau}\circ \psi \circ q'_2&=\xi_{\sigma+\tau}\circ q_2\circ \xi^{-1}_\tau \\&=q'_2\circ \xi_{\tau}\circ \xi^{-1}_\tau\\&=q'_2
      \end{split}
    \]
    To conclude we have to show that $\xi_{\sigma+\tau}$ fits in the
    diagrams of \Cref{def:permcon}. For the first two diagrams we
    have:
    \begin{align*}
      \xi_{\sigma+\tau} \circ \iota_{G_{0}}\circ \alpha & =\xi_{\sigma+\tau} \circ q_1 \circ \iota_{1,G_{1,0}}\circ \alpha_1 \\&=q_1'\circ \xi_\sigma \circ  \iota_{1,G_{1,0}} \alpha_1\\&=q'_1\circ \iota'_{1, G'_{1,0}}\circ \alpha'_1\\&=\iota'_{G'_0}\circ \alpha'
    \end{align*}
    and
    \begin{align*}
      \xi_{\sigma+\tau} \circ \iota_{G_{n+t+2}}\circ \omega & =\xi_{\sigma+\tau} \circ q_2 \circ \iota_{2,G_{2,t+1}}\circ \omega_2 \\&=q'_2\circ \xi_\tau \circ  \iota_{2,G_{2,t+1}} \omega_2\\&=q'_2\circ \iota'_{2, G'_{2,t+1}}\circ \omega'_2\\&=\iota'_{G'_{n+t+2}}\circ \omega'
    \end{align*}

    Let $i$ be an index in $[0, n+t+1]$, we proceed splitting the cases.
    \begin{itemize}
    \item $i$ is in $[0,n]$. Then we have
      \begin{align*}
        \xi_{\sigma+\tau}\circ \iota_{G_i} \circ m_i & = \xi_{\sigma+\tau} \circ q_1\circ \iota_{1, G_{1,i}}\circ m_{1,i} \\&=q'_1 \circ \xi_\sigma \circ \iota_{1, G_{1,i}}\circ m_{1,i}\\&= q'_1\circ \iota'_{1, G'_{1,\sigma(i)}}\circ m'_{1,\sigma(i)}\\&=\iota'_{G'_{\sigma(i)}}\circ m'_{\sigma(i)}
      \end{align*}
      Now, suppose that $i\neq n$, then
      \begin{align*}
        \xi_{\sigma+\tau}\circ \iota_{G_{i+1}} \circ h_i & = \xi_{\sigma+\tau} \circ q_1\circ \iota_{1, G_{1,{i+1}}}\circ h_{1,i} \\&=q'_1\circ \xi_\sigma \circ \iota_{1, G_{1,{i+1}}}\circ h_{1,i}\\&= q'_1\circ \iota'_{1, G'_{1,\sigma(i)+1}}\circ h'_{1,\sigma(i)}
      \end{align*}
      and we have two subcases. If $\sigma(i)\neq n$, then \[h'_{1, \sigma(i)}=h'_{\sigma(i)} \qquad q'_1\circ \iota'_{1, G'_{1,\sigma(i)+1}}=
        \iota'_{G'_{\sigma(i)+1}}\]
      and we can conclude. Otherwise, we have
      \begin{align*}q'_1\circ \iota'_{1, G'_{1,n+1}}\circ h'_{1,n} & =q'_1 \circ \iota'_{1,G'_{1,n+1}} \circ \omega'_1\circ (\omega')^{-1}_1 \circ h'_{1,n} \\&=\iota'_{G'_{n+1}}\circ h'_n
      \end{align*}
      yielding again the wanted equality
      \[\xi_{\sigma+\tau}\circ \iota_{G_{i+1}} \circ h_i=\iota'_{G'_{\sigma(i)+1}}\circ h'_{\sigma(i)}\]
      Finally, if $i=n$ we get
      \begin{align*}
        \xi_{\sigma+\tau}\circ \iota_{G_{n+1}} \circ h_n & = \xi_{\sigma+\tau} \circ \iota_{\pi(G_{1,n+1})} \circ \omega^{-1}_1\circ h_{1,n} \\&= \xi_{\sigma+\tau} \circ q_1\circ \iota_{1, G_{1,n+1}} \circ \omega_1 \circ \omega^{-1}_1 \circ h_{1,n}\\&=q'_1 \circ \xi_\sigma \circ \iota_{1, G_{1,n+1}} \circ h_{1,n}\\&=q'_1\circ \iota'_{1, G_{1,\sigma(n)+1}} \circ h'_{1,\sigma(n)}
      \end{align*}
      We have again two cases. If $\sigma(n)\neq n$ then we can conclude at once since
      \[h'_{1, \sigma(n)}=h'_{\sigma(n)} \qquad q'_1\circ \iota'_{1, G'_{1,\sigma(n)+1}}=
        \iota'_{G'_{\sigma(n)+1}}\]
      While, if $\sigma(n)=n$, then we can use the identities
      \[(\omega'_1)^{-1}\circ h'_{1, n}=h'_{n} \qquad q'_1\circ \iota'_{1, G'_{1, n+1}}=
        \iota'_{G'_{n+1}}\circ (\omega'_1)^{-1}\]
      to get
      \begin{align*}
        \xi_{\sigma+\tau}\circ \iota_{G_{n+1}} \circ h_n & = q'_1\circ \iota'_{1, G_{1,n+1}} \circ h'_{1,n} \\&=\iota'_{G'_{n+1}}\circ (\omega'_1)^{-1}\circ h'_{1,n}\\&=\iota'_{G'_{n+1}} h'_{n}
      \end{align*}
    \item $i$ is in $[n+1, n+t+1]$- We proceed  similarly to the point above. First of all we have
      \begin{align*}
        \xi_{\sigma+\tau}\circ \iota_{G_{i+1}} \circ h_i & = \xi_{\sigma+\tau} \circ q_2\circ \iota_{1, G_{2,i-n}}\circ h_{2,i-(n+1)} \\&=q'_2 \circ \xi_\tau \circ \iota_{2, G_{2,i-n}}\circ h_{2,i-(n+1)}\\&= q'_2\circ \iota'_{2, G'_{2,\tau(i-n)}}\circ h'_{2,\tau(i-(n+1))}\\&=\iota'_{G'_{\tau(i-n)}}\circ h'_{n+1+\tau(i-(n+1))}
      \end{align*}
      Next, supposing that $i\neq n+1$:
      \begin{align*}
        \xi_{\sigma+\tau}\circ \iota_{G_{i}} \circ m_i & = \xi_{\sigma+\tau} \circ q_2\circ \iota_{2, G_{2,i-(n+1)}}\circ m_{2,i-(n+1)} \\&=q'_2\circ \xi_\tau \circ \iota_{2, G_{2,{i-(n+1)}}}\circ m_{2,i-(n+1)}\\&= q'_2\circ \iota'_{2, G'_{2,\tau(i-(n+1))}}\circ m'_{2,\tau(i-(n+1))}
      \end{align*}
      and we have again two possibilities. If $\tau(i-(n+1))\neq 0$, then the thesis follows from the equalities \[m'_{2, \tau(i-(n+1))}=m'_{(\sigma+\tau)(i)} \qquad q'_2\circ \iota'_{2, G'_{2,\tau(i-(n+1))}}=
        \iota'_{G'_{(\sigma+\tau)(i)}}\]
      Suppose, instead, that $\tau(i-(n+1))= 0$, then we have
      \begin{align*} q'_2\circ \iota'_{2, G'_{2,0}}\circ m'_{2,0} & =q'_2 \circ \iota'_{2,G'_{2,0}} \circ \alpha'_2 \circ \alpha'^{-1}_2 \circ m'_{2,0} \\&=q'_1\circ \iota'_{1, G'_{1, n+1}}\circ \omega'_1\circ \alpha'^{-1}_2 \circ m'_{2,0}\\&=\iota'_{G'_{n+1}}\circ m'_{n+1}
      \end{align*}
      allowing us to conclude again.
      
      We are left with the case $i=n+1$. Let us compute
      \begin{align*}
        \xi_{\sigma+\tau}\circ \iota_{G_{n+1}} \circ m_{n+1} & = \xi_{\sigma+\tau} \circ \iota_{\pi(G_{2,0})} \circ (\alpha'_2)^{-1}\circ m_{2,0} \\&=\xi_{\sigma+\tau} \circ q_1\circ \iota_{1, G_{1,n+1}} \circ \omega_1 \circ (\alpha'_2)^{-1}\circ m_{2,0}\\&=\xi_{\sigma+\tau} \circ q'_2 \circ \iota'_{2,G'_{2,0}} \circ \alpha'_2 \circ (\alpha'_2)^{-1}\circ m_{2,0} \\&=q'_2 \circ \xi_\tau \circ \iota_{2, G_{2,0}} \circ m_{2,0}\\&=q'_2\circ \iota'_{2, G_{2,\tau(0)}} \circ m'_{2,\tau(0)}
      \end{align*}
      We have again two cases. If $\tau(0)\neq 0$ then $(\sigma+\tau)(i)\neq n+1$ and we can conclude at once since
      \[m'_{2, \tau(0)}=m'_{(\sigma+\tau)(n+1)} \qquad q'_2\circ \iota'_{2, G'_{2,\tau(0)}}=
        \iota'_{G'_{(\sigma+\tau)(0)}}\]
      If $\tau(0)=0$, so that  $(\sigma+\tau)(0)= n+1$ we appeal to the equalities
      \[(\alpha')^{-1}_2\circ m'_{2, 0}=m'_{n+1} \qquad q'_2\circ \iota'_{2, G'_{2, 0}}=
        \iota'_{G'_{n+1}}\circ (\alpha'_2)^{-1}\]
      to get
      \begin{align*}
        \xi_{\sigma+\tau}\circ \iota_{G_{n+1}} \circ m_{n+1} & = q'_2\circ \iota'_{2, G_{2,\tau(0)}} \circ m'_{2,\tau(0)} \\&=	\iota'_{G'_{n+1}}\circ (\alpha'_2)^{-1}\circ m'_{2,0}\\&=\iota'_{G'_{n+1}}\circ  m'_{n+1}
      \end{align*}
    \end{itemize}
    The thesis now follows at once.	 \qedhere
  \end{itemize} \end{proof}
\fi




\section{Independence in rewriting}
\label{sec:equi}

In this section we discuss the notion of independence between
consecutive rewriting steps, which is a fundamental ingredient of the
rewriting theory, which comes into play when viewing a derivation
sequence as a concurrent computation, the idea being that of
abstracting out the order of execution of independent rewritings. We
observe that the shift to left-linear rules leads to the failure of
various basic properties of the classical notion of independence and
we single out a framework in which these can be re-established.


\subsection{Sequentially independent and switchable derivations }\label{subsec:switch}



In the DPO approach the canonical notion of independence is that of sequential independence.
%recalled below.

\begin{definition}[Sequential independence]
  \label{de:sequential-independence}
  Let with $\X$ an $\mathcal{M}$-adhesive category 
  and $(\X, \R)$ a left-linear DPO rewriting system. 
  Let also
  $\der{D}=\{\dder{D}_i\}_{i=0}^1$ be a derivation of length $2$, with
  $\dder{D}_i$ using rule $\rho_i = (l_i,r_i)$.  An \emph{independence
    pair} between $\dder{D}_0$ and $\dder{D}_1$, is a pair of arrows
  $i_0\colon R_0\to D_1$ and $i_1\colon L_1\to D_0$ such that the
  following diagram commutes
  \[\xymatrix@C=15pt{L_0 \ar[d]_{m_0}&& K_0
      \ar[d]_{k_0}\ar@{>->}[ll]_{l_0} \ar[r]^{r_0} & R_0
      \ar@{.>}@/^.35cm/[drrr]|(.3)\hole_(.4){i_0}
      \ar[dr]|(.3)\hole_{h_0} && L_1 \ar@{.>}@/_.35cm/[dlll]^(.4){i_1}
      \ar[dl]|(.3)\hole^{m_1}& K_1 \ar[d]^{k_1}\ar@{>->}[l]_{l_1}
      \ar[rr]^{r_1} && R_1 \ar[d]^{h_1}\\G_0 && \ar@{>->}[ll]^{f_0}
      D_0 \ar[rr]_{g_0}&& G_1 && \ar@{>->}[ll]^{f_1} D_1
      \ar[rr]_{g_1}&& G_2}\] We will say that $\dder{D}_0$ and
  $\dder{D}_1$ are
  % \emph{weakly sequentially independent} if an independence pair
  % exists. If such independence pair is unique we will say that
  % $\dder{D}$ and $\dder{D'}$ are
  \emph{sequentially independent}.\todo{Eliminata la nozione di weakly
    indepedendent, giusto?}
\end{definition}

Intutitively, the existence of the independence pair is intended to
capture the possibility of switching the application of the two rules:
$f_0 \circ i_1$ is a potential match for $\rho_1$ in $G_0$ and the
existence of $i_0 : R_0 \to D_1$ indicates that $\rho_1$ do not delete
items in the image of $K_0$ and thus $\rho_1$ can be applied after
$\rho_0$.

More precisely, let us formalise what it means for a derivation being the switch of another.

% We want to be able to to switch the application of two rules used by two consecutive weakly independent direct derivations.
% This leads us to the following definition.

\begin{definition}[Switch]
  \label{def:switch}
  Let with $\X$ an $\mathcal{M}$-adhesive category 
  and $(\X, \R)$ a left-linear DPO rewriting system. 
  Let also
  $\der{D}=\{\dder{D}_i\}_{i=0}^1$ be a derivation of length $2$ made
  by two sequentially independent derivations $\dder{D}_0$,
  $\dder{D}_1$, with independence pair $(i_0, i_1)$, as in the diagram
  on the left below. A \emph{switch} of $\der{D}$ along $(i_0,i_1)$ is
  a derivation $\der{E}=\{\dder{E}_i\}_{i=0}^1$, between the same objects and using the same rules
  in reverse order, as in the diagram on the right below, 
  \[
    \def\objectstyle{\scriptstyle}\def\labelstyle{\scriptstyle}
    \xymatrix@C=1.1mm@R=8mm{
      {L_0} \ar[d]_{m_0}
      & & {K_0} \ar@{>->}[ll]_{l_0} \ar[rr]^{r_0} \ar[d]_{k_0}
      & & {R_0} \ar[drr]|(.35)\hole_(.6){h_0}  \ar@/^.22cm/[drrrrrr]|(.3)\hole^(.75){i_0}
      & & & & 
      % 
      {L_1}\ar[dll]|(.35)\hole^(.6){m_1} \ar@/_.22cm/_(.75){i_1}[dllllll]
      & & {K_1} \ar@{>->}[ll]_{l_1} \ar[rr]^{r_1} \ar[d]^{k_1}
      & & {R_1} \ar[d]^{h_1} \\
      % 
      % 
      {G_0}
      & & {D_0} \ar@{>->}[ll]^{f_0} \ar[rrrr]_{g_0}
      & & & & {G_1} & &
      & &  {D_1} \ar@{>->}[llll]^{f_1} \ar[rr]_{g_1}
      & & {G_0}
    }
    % 
    \xymatrix@C=1.1mm@R=8mm{
      {L_1} \ar[d]_{m_0'}
      & & {K_1} \ar@{>->}[ll]_{l_1} \ar[rr]^{r_1} \ar[d]_{k_0'}
      & & {R_1} \ar[drr]|(.35)\hole_(.6){h_0'}  \ar@/^.22cm/@{.>}^(.75){i_0'}|(.3)\hole[drrrrrr]
      & & & & 
      % 
      {L_0}\ar[dll]|(.35)\hole^(.6){m_1'} \ar@/_.22cm/@{.>}_(.75){i_1'}[dllllll] 
      & & {K_0} \ar@{>->}[ll]_{l_0} \ar[rr]^{r_0} \ar[d]^{k_1'}
      & & {R_0} \ar[d]^{h_1'} \\
      % 
      % 
      {G_0}
      & & {D_1'} \ar@{>->}[ll]^{f_0'} \ar[rrrr]_{g_0'}
      & & & & {G'_1} & &
      & &  {D_0'} \ar@{>->}[llll]^{f_1'} \ar[rr]_{g_1'}
      & & {G_2}  }
    %
  \]
  such that there is an independence pair $(i_0', i_1')$ between
  $\dder{E}_0$ and $\dder{E}_1$ and 
  \begin{center}   
    $m_0=f_0' \circ i_1'$
    \qquad $h_1=g_1' \circ i_0'$
    \qquad $m_0'= f_0 \circ i_1$
    \qquad $h_1'= g_{1}\circ i_0$.
  \end{center}


  If a switch of $\dder{D}_0$ and $\dder{D}_1$ exists we will say that
  they are \emph{switchable}.

  % \[
  %   \xymatrix@C=7pt{L_0 \ar[d]_{m_0}&& K_0
  %     \ar[d]_{k_0}\ar@{>->}[ll]_{l_0} \ar[r]^{r_0} & R_0
  %     \ar@/^.35cm/[drrr]|(.3)\hole_(.4){i_0} \ar[dr]|(.3)\hole_{h_0}
  %     && L_1 \ar@/_.35cm/[dlll]^(.4){i_1} \ar[dl]|(.3)\hole^{m_1}& K_1
  %     \ar[d]^{k_1}\ar@{>->}[l]_{l_1} \ar[rr]^{r_1} && R_1
  %     \ar[d]^{h_1}\\G_0 && \ar@{>->}[ll]^{f_0} D_0 \ar[rr]_{g_0}&& G_1
  %     && \ar@{>->}[ll]^{f_1} D_1 \ar[rr]_{g_1}&& G_2}\]

  % Take another derivation $\der{E}=\{\dder{E}_i\}_{i=0}^1$ as below
  % \[\xymatrix@C=18pt{L_{\der{E},0} \ar[d]_{m_{\der{E},0}}&&
  %     K_{\der{E},0}
  %     \ar[d]_{k_{\der{E},0}}\ar@{>->}[ll]_{l_{\der{E},0}}
  %     \ar[r]^{r_{\der{E},0}} & R_{\der{E},0} \ar[dr]_{h_{\der{E},0}}
  %     && L_{\der{E},1} \ar[dl]^{m_{\der{E},1}}& K_{\der{E},1}
  %     \ar[d]^{k_{\der{E},1}}\ar@{>->}[l]_{l_{\der{E},1}}
  %     \ar[rr]^{r_{\der{E},1}} && R_{\der{E},1}
  %     \ar[d]^{h_{\der{E},1}}\\G_{0} && \ar@{>->}[ll]^{f_{\der{E},0}}
  %     D_{\der{E},0} \ar[rr]_{g_{\der{E},0}}&& G_{\der{E},1} &&
  %     \ar@{>->}[ll]^{f_{\der{E},1}} D_{\der{E},1}
  %     \ar[rr]_{g_{\der{E},1}}&& G_{2}}\] We say that $\der{E}$ is a
  % \emph{switch} of $\dder{D}_0$ and $\dder{D}_1$ along $(i_1, i_2)$ if
  % \begin{enumerate}
  % \item $\dder{E}_0$ applies the rule used by $\dder{D}_1$ and
  %   $\dder{E}_1$ the one used by $\dder{D}_0$, so that
  %   \[(l_{\der{E},0}, r_{\der{E},0})=(l_1,r_1) \qquad (l_{\der{E},1},
  %     r_{\der{E},1})=(l_0,r_0)\]
  % \item there exists an independence pair $(j_0, j_1)$ between
  %   $\dder{E}_0$ and $\dder{E}_1$ such that
  %   \[ m_0=f_{\der{E},0} \circ j_1 \qquad h_1=g_{\der{E},1} \circ
  %     j_0\]
  % \item $m_{\der{E},0}= f_0\circ i_1$ and
  %   $h_{\der{E},1}= g_{1}\circ i_0$.
  % \end{enumerate}

  % If a switch of $\dder{D}_0$ and $\dder{D}_1$ exists we will say that
  % they are \emph{switchable}.
\end{definition}

It can be shown that switches along the same independence pair are unique up to abstraction
equivalence (see Proposition~\ref{thm:switch_uni}).

\begin{example}
  \label{ex:seq-ind}
  Consider a rewriting system in $\cat{Graph}$, the category of
  directed graphs and graph morphisms, which is adhesive. The set of
  rules $\R = \{ \rho_0, \rho_1, \rho_2\}$ consists of three rules as
  depicted in Fig.~\ref{fi:rules}, where numbers are used to represent
  the mappings from the interface to the left- and right-hand
  sides. Rules $\rho_0$ and $\rho_1$ are linear: $\rho_0$ generates a
  new node, while $\rho_1$ delete an edge. Instead, $\rho_2$ is non
  linear as it ``merge'' two nodes.

  \begin{figure}
      \begin{center}
    %%% TYPE GRAPH 
    % 
    % \subcaptionbox*{$T$}{
    %   \begin{tikzpicture}[node distance=2mm, baseline=(current bounding box.center)]
    %     \node () {
    %       \begin{tikzpicture}
    %         \node at (0,0) [node] (1) {}
    %         edge [in=95, out=65, loop] ();        
    %         % 
    %         \pgfBox
    %       \end{tikzpicture}
    %     };
    %   \end{tikzpicture}
    % }
    % % 
    % \hfill
    %
    %%% RULES 
    % 
%    \subcaptionbox*{}{
    % RULE P0
    %
  \begin{tikzpicture}[node distance=2mm, font=\small]
    %, baseline=(current bounding box.center), font=\small]
      \node (l) {
      \begin{tikzpicture}
      %
        \node at (0,0.53) {};
        \node at (0,0) [node, label=below:$1$] (1) {} ;        
      %
      \pgfBox
      \end{tikzpicture} 
    };
    \node[font=\scriptsize, above] at (l.north) {$L_0$};
    \node [right=of l] (k) {
      \begin{tikzpicture}
        %
        \node at (0,0.53) {}; 
        \node at (0,0) [node, label=below:$1$] (1) {};
      %
      \pgfBox
      \end{tikzpicture} 
    };
    \node[font=\scriptsize, above] at (k.north) {$K_0$};
    \node[below] at (k.south) {$\rho_0$};
    \node  [right=of k] (r) {
      \begin{tikzpicture}
        \node at (0,0.53) {}; 
        \node at (0,0) [node, label=below:$1$] (1) {};
        \node at (.5,0) [node, label=below:$2$] (2) {};
        \draw[->, color=blue] (1) to[out=20, in=160] (2);
        %
        \pgfBox
      \end{tikzpicture}
    };
    \node[font=\scriptsize, above] at (r.north) {$R_0$};
    \path (k) edge[->] node[trans, above] {} (l);
    \path (k) edge[->] node[trans, above] {} (r);    
    \end{tikzpicture}
%  }
    %
    \hspace{1cm}
%  \hfill
  %
%  \subcaptionbox*{}{
    % RULE P1
    %
    \begin{tikzpicture}[node distance=2mm, font=\small]
      \node (l) {
      \begin{tikzpicture}
        %
        \node at (0,0.53) {}; 
        \node at (0,0) [node, label=below:$1$] (1) {} ;
        % 
      \pgfBox
      \end{tikzpicture} 
    };
    \node[font=\scriptsize, above] at (l.north) {$L_1$};
    \node [right=of l] (k) {
      \begin{tikzpicture}
        %
        \node at (0,0.53) {}; 
        \node at (0,0) [node, label=below:$1$] (1) {};
        % 
        \pgfBox
      \end{tikzpicture} 
    };
    \node[font=\scriptsize, above] at (k.north) {$K_1$};
    \node[below] at (k.south) {$\rho_1$};
    \node  [right=of k] (r) {
      \begin{tikzpicture}
        \node at (0,0) [node, label=below:$1$] (1) {}
         edge [in=55, out=85, loop] ();
        %
        \pgfBox
      \end{tikzpicture}
    };
    \node[font=\scriptsize, above] at (r.north) {$R_1$};
    \path (k) edge[->] node[trans, above] {} (l);
    \path (k) edge[->] node[trans, above] {} (r);
    \end{tikzpicture}
%  }
  % 
    \hspace{1cm}
    %\hfill
  %
%    \subcaptionbox*{}{
    % RULE P2
    %
    \begin{tikzpicture}[node distance=2mm, font=\small]
      \node (l) {
      \begin{tikzpicture}
        %
        \node at (0,0.53) {}; 
        \node at (0,0) [node, label=below:$1$] (1) {} ;
        \node at (0.5,0) [node, label=below:$2$] (2) {} ;
      %
      \pgfBox
      \end{tikzpicture} 
    };
    \node[font=\scriptsize, above] at (l.north) {$L_2$};
    \node [right=of l] (k) {
      \begin{tikzpicture}
        %
        \node at (0,0.53) {}; 
        \node at (0,0) [node, label=below:$1$] (1) {} ;
        \node at (0.5,0) [node, label=below:$2$] (2) {} ;
      %
      \pgfBox
      \end{tikzpicture} 
    };
    \node[font=\scriptsize, above] at (k.north) {$K_2$};
    \node[below] at (k.south) {$\rho_2$};
    \node  [right=of k] (r) {
      \begin{tikzpicture}
        \node at (0,0.53) {}; 
        \node at (0,0) [node, label=below:$12$] (12) {};
        %
        \pgfBox
      \end{tikzpicture}
    };
    \node[font=\scriptsize, above] at (r.north) {$R_2$};
    \path (k) edge[->] node[trans, above] {} (l);
    \path (k) edge[->] node[trans, above] {} (r);
    \end{tikzpicture}
 % }
  %
\end{center}

%%% Local Variables:
%%% mode: latex
%%% TeX-master: t
%%% End:

    
    \caption{Rules of the rewriting system grammar of the example.}
    \label{fi:rules}
  \end{figure}
  
  A derivation $\der{D}$ consisting of three steps $\dder{D}_0$,
  $\dder{D}_1$, and $\dder{D}_2$, each applying the corresponding rule
  $\rho_i$ ,is depicted in in Fig.~\ref{fi:derD}. The second and third
  steps $\dder{D}_1$ and $\dder{D}_2$ are clearly sequential
  independent, via an independence pairs that map nodes consistently
  with their numbering.\todo{P: having names for the graphs would
    probably help}
  
  \begin{figure}
      \begin{center}
    \begin{tikzpicture}[node distance=2mm, font=\small, baseline=(current bounding box.center)]      
      \node (L1) at (0,2) {
        \begin{tikzpicture}
          % 
          \node at (0,0.53) {};
          \node at (0,0) [node, label=below:$1$] (1) {} ;
          % 
          \pgfBox
        \end{tikzpicture} 
      };
      \node [right=of L1] (K1) {
        \begin{tikzpicture}
          % 
          \node at (0,0.53) {}; 
          \node at (0,0) [node, label=below:$1$] (1) {};
          % 
          \pgfBox
        \end{tikzpicture} 
      };
      \node [above] at (K1.north) {$\rho_0$};
 %     \node [above=of K1] {$\rho_0$};
      \node [right=of K1](R1) {
        \begin{tikzpicture}
          \node at (0,0.53) {}; 
          \node at (0,0) [node, label=below:$1$] (1) {};
          \node at (.5,0) [node, label=below:$2$] (2) {}; 
          % 
          \pgfBox
        \end{tikzpicture}
      };
      \path (K1) edge[->] node[trans, above] {} (L1);
      \path (K1) edge[->] node[trans, above] {} (R1);

      \node at (4,2) (L2) {
        \begin{tikzpicture}
          % 
          \node at (0,0.53) {}; 
          \node at (0,0) [node, label=below:$2$] (2) {} ;
          % 
          \pgfBox
        \end{tikzpicture} 
      };
      \node [right=of L2] (K2) {
        \begin{tikzpicture}
          % 
          \node at (0,0.53) {}; 
          \node at (0,0) [node, label=below:$2$] (2) {};
          % 
          \pgfBox
        \end{tikzpicture} 
      };
 %     \node [above=of K2] {$\rho_1$};
      \node [above] at (K2.north) {$\rho_1$};
      \node [right=of K2] (R2) {
        \begin{tikzpicture}
          \node at (0,0) [node, label=below:$2$] (2) {}
          edge [in=95, out=65, loop] ();        
          % 
          \pgfBox
        \end{tikzpicture}
      };
      \path (K2) edge[->] node[trans, above] {} (L2);
      \path (K2) edge[->] node[trans, above] {} (R2);

      \node at (8,2) (L3) {
        \begin{tikzpicture}
          % 
          \node at (0,0.53) {}; 
          \node at (0,0) [node, label=below:$1$] (1) {} ;
          \node at (0.5,0) [node, label=below:$2$] (2) {} ;
          % 
          \pgfBox
        \end{tikzpicture} 
      };
      \node [right=of L3] (K3) {
        \begin{tikzpicture}
          % 
          \node at (0,0.53) {}; 
          \node at (0,0) [node, label=below:$1$] (1) {} ;
          \node at (0.5,0) [node, label=below:$2$] (2) {} ;
          % 
          \pgfBox
        \end{tikzpicture} 
      };
%      \node [above=of K3] {$\rho_2$};
      \node [above] at (K3.north) {$\rho_2$};
      \node [right=of K3] (R3) {
        \begin{tikzpicture}
          \node at (0,0.53) {}; 
          \node at (0,0) [node, label=below:$12$] (12) {};
          % 
          \pgfBox
        \end{tikzpicture}
      };
      \path (K3) edge[->] node[trans, above] {} (L3);
      \path (K3) edge[->] node[trans, above] {} (R3);

      %%%%%% second row
      \node at (0,0) (G1) {
        \begin{tikzpicture}
          % 
          \node at (0,0.53) {};
          \node at (0,0) [node, label=below:$1$] (1) {} ;
          % 
          \pgfBox
        \end{tikzpicture} 
      };
      \node [right=of G1] (D1) {
        \begin{tikzpicture}
          % 
          \node at (0,0.53) {}; 
          \node at (0,0) [node, label=below:$1$] (1) {};
          % 
          \pgfBox
        \end{tikzpicture} 
      };
      \node at (3,0) (G2) {
        \begin{tikzpicture}
          % 
          \node at (0,0.53) {}; 
          \node at (0,0) [node, label=below:$1$] (1) {} ;
          \node at (0.5,0) [node, label=below:$2$] (2) {} ;
          % 
          \pgfBox
        \end{tikzpicture} 
      };
      \path (D1) edge[->] node[trans, above] {} (G1);
      \path (D1) edge[->] node[trans, above] {} (G2);
      \path (L1) edge[->] node[trans, above] {} (G1);
      \path (K1) edge[->] node[trans, above] {} (D1);
      \path (R1) edge[->] node[trans, above] {} (G2);

      \node at (5,0) (D2) {
        \begin{tikzpicture}
          % 
          \node at (0,0.53) {};
          \node at (0,0) [node, label=below:$1$] (1) {} ;
          \node at (0.5,0) [node, label=below:$2$] (2) {} ;                    
          % 
          \pgfBox
        \end{tikzpicture} 
      };
      \node at (7,0) (G3) {
        \begin{tikzpicture}
          % 
          \node at (0,0.53) {}; 
          \node at (0,0) [node, label=below:$1$] (1) {};
          \node at (0.5,0) [node, label=below:$2$] (2) {}
                edge [in=95, out=65, loop] ();                    
          % 
          \pgfBox
        \end{tikzpicture} 
      };

      \path (D2) edge[->] node[trans, above] {} (G2);
      \path (D2) edge[->] node[trans, above] {} (G3);
      \path (L2) edge[->] node[trans, above] {} (G2);
      \path (K2) edge[->] node[trans, above] {} (D2);
      \path (R2) edge[->] node[trans, above] {} (G3);
      
      \node at (9.46,0) (D3) {
        \begin{tikzpicture}
          % 
          \node at (0,0) [node, label=below:$1$] (1) {};
          \node at (0.5,0) [node, label=below:$2$] (2) {}
                edge [in=95, out=65, loop] () ;
          % 
          \pgfBox
        \end{tikzpicture} 
      };
      \node [right=of D3] (G4) {
        \begin{tikzpicture}
          % 
          \node at (0,0) [node, label=below:$12$] (12) {}
                edge [in=95, out=65, loop] ();
          % 
          \pgfBox
        \end{tikzpicture} 
      };
      \node[font=\scriptsize, below] at (G1.south) {$G_0$};
      \node[font=\scriptsize, below] at (D1.south) {$D_0$};      
      \node[font=\scriptsize, below] at (G2.south) {$G_1$};
      \node[font=\scriptsize, below] at (D2.south) {$D_1$};      
      \node[font=\scriptsize, below] at (G3.south) {$G_2$};
      \node[font=\scriptsize, below] at (D3.south) {$D_2$};      
      \node[font=\scriptsize, below] at (G4.south) {$G_3$};      
      \path (D3) edge[->] node[trans, above] {} (G3);
      \path (D3) edge[->] node[trans, above] {} (G4);
      \path (L3) edge[->] node[trans, above] {} (G3);
      \path (K3) edge[->] node[trans, above] {} (D3);
      \path (R3) edge[->] node[trans, above] {} (G4);
    \end{tikzpicture}
  %
\end{center}
%%% Local Variables:
%%% mode: latex
%%% TeX-master: t
%%% End:

    \caption{The derivation $\der{D}$.}
    \label{fi:derD}
  \end{figure}

  The independence pair can be used to construct a switch $\der{E}$, as
  depicted in Fig.~\ref{fi:derE}.
  \begin{figure}
      \begin{center}
    \begin{tikzpicture}[node distance=2mm, font=\small,  baseline=(current bounding box.center)]      
      \node (L1) at (0,2) {
        \begin{tikzpicture}
          % 
          \node at (0,0.53) {};
          \node at (0,0) [node, label=below:$1$] (1) {} ;
          % 
          \pgfBox
        \end{tikzpicture} 
      };
      \node [right=of L1] (K1) {
        \begin{tikzpicture}
          % 
          \node at (0,0.53) {}; 
          \node at (0,0) [node, label=below:$1$] (1) {};
          % 
          \pgfBox
        \end{tikzpicture} 
      };
 %     \node [above=of K1] {$\rho_0$};
      \node [above] at (K1.north) {$\rho_0$};
      \node [right=of K1](R1) {
        \begin{tikzpicture}
          \node at (0,0.53) {}; 
          \node at (0,0) [node, label=below:$1$] (1) {};
          \node at (.5,0) [node, label=below:$2$] (2) {}; 
          \draw[->, color=blue] (1) to[out=20, in=160] (2);
          % 
          \pgfBox
        \end{tikzpicture}
      };
      \path (K1) edge[->] node[trans, above] {} (L1);
      \path (K1) edge[->] node[trans, above] {} (R1);


      \node at (4,2) (L3) {
        \begin{tikzpicture}
          % 
          \node at (0,0.53) {}; 
          \node at (0,0) [node, label=below:$1$] (1) {} ;
          \node at (0.5,0) [node, label=below:$2$] (2) {} ;
          % 
          \pgfBox
        \end{tikzpicture} 
      };
      \node [right=of L3] (K3) {
        \begin{tikzpicture}
          % 
          \node at (0,0.53) {}; 
          \node at (0,0) [node, label=below:$1$] (1) {} ;
          \node at (0.5,0) [node, label=below:$2$] (2) {} ;
          % 
          \pgfBox
        \end{tikzpicture} 
      };
 %     \node [above=of K3] {$\rho_2$};
      \node [above] at (K3.north) {$\rho_2$};
      \node [right=of K3] (R3) {
        \begin{tikzpicture}
          \node at (0,0.53) {}; 
          \node at (0,0) [node, label=below:$12$] (12) {};
          % 
          \pgfBox
        \end{tikzpicture}
      };
      \path (K3) edge[->] node[trans, above] {} (L3);
      \path (K3) edge[->] node[trans, above] {} (R3);

      \node at (8,2) (L2) {
        \begin{tikzpicture}
          % 
          \node at (0,0.53) {}; 
          \node at (0,0) [node, label=below:$12$] (1) {} ;
          % 
          \pgfBox
        \end{tikzpicture} 
      };
      \node [right=of L2] (K2) {
        \begin{tikzpicture}
          % 
          \node at (0,0.53) {}; 
          \node at (0,0) [node, label=below:$12$] (1) {};
          % 
          \pgfBox
        \end{tikzpicture} 
      };
 %     \node [above=of K2] {$\rho_1$};
      \node [above] at (K2.north) {$\rho_1$};

      \node [right=of K2] (R2) {
        \begin{tikzpicture}
          \node at (0,0) [node, label=below:$12$] (1) {}
          edge [in=55, out=85, loop] ();
          % 
          \pgfBox
        \end{tikzpicture}
      };
      \path (K2) edge[->] node[trans, above] {} (L2);
      \path (K2) edge[->] node[trans, above] {} (R2);

      
      %%%%%% second row
      \node at (0,0) (G1) {
        \begin{tikzpicture}
          % 
          \node at (0,0.53) {};
          \node at (0,0) [node, label=below:$1$] (1) {} ;
          % 
          \pgfBox
        \end{tikzpicture} 
      };
      \node [right=of G1] (D1) {
        \begin{tikzpicture}
          % 
          \node at (0,0.53) {}; 
          \node at (0,0) [node, label=below:$1$] (1) {};
          % 
          \pgfBox
        \end{tikzpicture} 
      };
      \node at (3,0) (G2) {
        \begin{tikzpicture}
          % 
          \node at (0,0.53) {}; 
          \node at (0,0) [node, label=below:$1$] (1) {} ;
          \node at (0.5,0) [node, label=below:$2$] (2) {} ;
          \draw[->, color=blue] (1) to[out=20, in=160] (2);
          % 
          \pgfBox
        \end{tikzpicture} 
      };
      \path (D1) edge[->] node[trans, above] {} (G1);
      \path (D1) edge[->] node[trans, above] {} (G2);
      \path (L1) edge[->] node[trans, above] {} (G1);
      \path (K1) edge[->] node[trans, above] {} (D1);
      \path (R1) edge[->] node[trans, above] {} (G2);

      \node at (5.5,0) (D2) {
        \begin{tikzpicture}
          % 
          \node at (0,0.53) {};
          \node at (0,0) [node, label=below:$1$] (1) {} ;
          \node at (0.5,0) [node, label=below:$2$] (2) {} ;
          \draw[->, color=blue] (1) to[out=20, in=160] (2);
          % 
          \pgfBox
        \end{tikzpicture} 
      };
      \node at (7.4,0) (G3) {
        \begin{tikzpicture}
          % 
          \node at (0,0.53) {}; 
          \node at (0,0) [node, label=below:$12$] (12) {}
          edge [in=125, out=155, color=blue, loop] ();
          % 
          \pgfBox
        \end{tikzpicture} 
      };

      \path (D2) edge[->] node[trans, above] {} (G2);
      \path (D2) edge[->] node[trans, above] {} (G3);
      \path (L3) edge[->] node[trans, above] {} (G2);
      \path (K3) edge[->] node[trans, above] {} (D2);
      \path (R3) edge[->] node[trans, above] {} (G3);
      
      \node at (9.1,0) (D3) {
        \begin{tikzpicture}
          % 
          \node at (0,0.53) {}; 
          \node at (0,0) [node, label=below:$12$] (12) {}
          edge [in=125, out=155, color=blue, loop] ();
          % 
          \pgfBox
        \end{tikzpicture} 
      };
      \node [right=of D3] (G4) {
        \begin{tikzpicture}
          % 
          \node at (0,0) [node, label=below:$12$] (12) {}
          edge [in=55, out=85, loop] ()
          edge [in=125, out=155, color=blue, loop] ();
          % 
          \pgfBox
        \end{tikzpicture} 
      };

      \node[font=\scriptsize, below] at (G1.south) {$G_0$};
      \node[font=\scriptsize, below] at (D1.south) {$D_0$};      
      \node[font=\scriptsize, below] at (G2.south) {$G_1$};
      \node[font=\scriptsize, below] at (D2.south) {$D_1'$};      
      \node[font=\scriptsize, below] at (G3.south) {$G_2'$};
      \node[font=\scriptsize, below] at (D3.south) {$D_2'$};      
      \node[font=\scriptsize, below] at (G4.south) {$G_3$};      

      \path (D3) edge[->] node[trans, above] {} (G3);
      \path (D3) edge[->] node[trans, above] {} (G4);
      \path (L2) edge[->] node[trans, above] {} (G3);
      \path (K2) edge[->] node[trans, above] {} (D3);
      \path (R2) edge[->] node[trans, above] {} (G4);
    \end{tikzpicture}
  %
\end{center}
%%% Local Variables:
%%% mode: latex
%%% TeX-master: t
%%% End:

    \caption{The derivation $\der{E}$.}
    \label{fi:derE}
  \end{figure}
\end{example}


As already mentioned, for linear rewriting systems, when viewing
derivations as concurrent computations, it is canonical to identify
derivations which are equal ``up to switching''. The same notion can
be given for left-linear rewriting systems relying on notion of
switch.

\begin{definition}[Switch equivalence]
  \label{de:switch-equivalence}
  Let $(\X, \R)$ be a left-linear DPO rewriting system.  Let
  $\der{D}, \der{E} : G \Mapsto H$ be derivations with the same
  length, $\der{D}=\{\dder{D}_{i}\}_{i=0}^n$ and
  $\der{E}=\{\dder{E}_{i}\}_{i=0}^n$. If
  $\dder{D}_i \cdot \dder{D}_{i+1}$ is a switch of
  $\dder{E}_i \cdot \dder{E}_{i+1}$ for some $i \in [0,n-1]$ we write
  $\der{D} \leftrightsquigarrow_{(i,i+1)} \der{E}$. Then \emph{switch
    equivalence} is the least equivalence on derivation such that if
  $\der{D} \leftrightsquigarrow_{(i,i+1)} \der{E}$ then
  $\der{D} \equiv^s \der{E}$.
\end{definition}



\subsection{Church-Rosser: Relating sequential indepedence and switchability}\label{subsec:CR}


The fact that sequential indepedence implies switchability always
holds for linear rules (see \Cref{prop:equi}). The result is so
fundamental that it is not even stated, in the sense that
switchability is not introduced axiomatically as in
Definition~\ref{def:switch}, but it is based on the explicit
construction of a switch.

Unfortunately, when working with left-linear rewriting systems,
sequential independence does not imply switchability (while the
converse implication clearly holds), as shown by the example below.

\full{
  \begin{remark}
    \label{rem:fact}
    Notice that, since $f_0'$ is in $\mathcal{M}$, there is
    at most one $i_0' \colon L_0\to D_0'$ such that
    $m_0=f_0' \circ i_0'$. Moreover, we already know that
    there is at most one $i_1' \colon R_1\to D_1'$ such that
    $m_1'=f_1' \circ i_1'$, therefore, the
    independence pair $(i_1',i_0')$ between the components of a switch is
    uniquely determined.
  \end{remark}
}

\begin{example}
  \label{ex:diff1}
  Given two posets $(P, \leq)$ and $(Q, \leq)$, define
  $(P, \leq)\oplus(Q,\leq)$ as the poset which has as underlying set
  the disjoint union $P+Q$ and in which every element of $Q$ is
  greater than every element of $P$, while elements of $P$ and $Q$ are
  compared using the original orders.

  Take now the set $S:=\{a,b,c\}$ and endow it with the smallest order
  such that $a\leq b$ and $a\leq c$.  Take also the set $\mathbb{N}$
  ordered by $\geq$, so that $0$ is its maximum.  Let $\X$ be the
  category associated to the order
  $(S, \leq)\oplus (\mathbb{N}, \geq)$, which by \Cref{rem:iso} is
  $\mathsf{Iso}$-adhesive. Notice that in this category the arrows
  $(a,b)\colon a\to b$ $(a,c)\colon a\to c$ do not have a pushout
  because $b$ and $c$ do not have a supremum.  Consider a rewriting
  system with set of rules $\R$ containing the following two rules
  \[\xymatrix{a & a \ar[r]^{(a, 0)} \ar[l]_{\id{a}} & 0 & a & a
      \ar[r]^{(a, b)} \ar[l]_{\id{a}} & b }\]
  We can then consider the  derivation
  $\der{D}=\{\dder{D}_i\}_{i=0}^1$ of length $2$. 
  \[
    \xymatrix@C=30pt{a \ar[d]_{m_0}&& a \ar[d]_{k_0}\ar[ll]_{\id{a}}
      \ar[r]^{(a,0)} & 0 \ar@/^.35cm/[drrr]|(.315)\hole_(.45){\id{0}}
      \ar[dr]|(.3)\hole_{\id{0}} && a \ar@/_.35cm/[dlll]^(.45){(a,c)}
      \ar[dl]|(.3)\hole^{(a,0)}& a \ar[d]^{(a,0)}\ar@{>->}[l]_{\id{a}}
      \ar[rr]^{(a,b)} && b \ar[d]^{(b,0)}\\c &&
      \ar@{>->}[ll]^{(\id{c})} c \ar[rr]_{(c,0)}&& 0 &&
      \ar@{>->}[ll]^{\id{0}} 0 \ar[rr]_{\id{0}}&& 0}\]

  \noindent
  \parbox{10cm}{ \hspace{15pt} Notice that the two
    direct derivations are sequential independent.
    %
    On the other hand, the rule
    applied by $\dder{D}_1$ cannot be applied to $c$. In fact, the
    unique morphism $a\to c$ is $(a,c)$, thus we get a diagram
    aside. But $b$ and $c$ do not have a supremum in the poset
    underlying $\X$, thus we do not get a direct derivation from
    $c$.}  \parbox{3cm}{
    \vspace{-.3cm}$\xymatrix{a \ar[d]_{(a,c)}& a
      \ar[d]_{(a,c)}\ar@{>->}[l]_{\id{a}} \ar[r]^{(a,b)} & b \\c
      & c \ar[l]^{\id{c}} }$}  
\end{example}

\iffalse 

Switches along the same independence pair are unique up to abstraction
equivalence (see Proposition~\ref{thm:switch_uni}).

\begin{example}
  \label{ex:seq-ind}
  Consider a rewriting system in $\cat{Graph}$, the category of
  directed graphs and graph morphisms, which is adhesive. The set of
  rules $\R = \{ \rho_0, \rho_1, \rho_2\}$ consists of three rules as
  depicted in Fig.~\ref{fi:rules}, where numbers are used to represent
  the mappings from the interface to the left- and right-hand
  sides. Rules $\rho_0$ and $\rho_1$ are linear: $\rho_0$ generates a
  new node, while $\rho_1$ delete an edge. Instead, $\rho_2$ is non
  linear as it ``merge'' two nodes.

  \begin{figure}
      \begin{center}
    %%% TYPE GRAPH 
    % 
    % \subcaptionbox*{$T$}{
    %   \begin{tikzpicture}[node distance=2mm, baseline=(current bounding box.center)]
    %     \node () {
    %       \begin{tikzpicture}
    %         \node at (0,0) [node] (1) {}
    %         edge [in=95, out=65, loop] ();        
    %         % 
    %         \pgfBox
    %       \end{tikzpicture}
    %     };
    %   \end{tikzpicture}
    % }
    % % 
    % \hfill
    %
    %%% RULES 
    % 
%    \subcaptionbox*{}{
    % RULE P0
    %
  \begin{tikzpicture}[node distance=2mm, font=\small]
    %, baseline=(current bounding box.center), font=\small]
      \node (l) {
      \begin{tikzpicture}
      %
        \node at (0,0.53) {};
        \node at (0,0) [node, label=below:$1$] (1) {} ;        
      %
      \pgfBox
      \end{tikzpicture} 
    };
    \node[font=\scriptsize, above] at (l.north) {$L_0$};
    \node [right=of l] (k) {
      \begin{tikzpicture}
        %
        \node at (0,0.53) {}; 
        \node at (0,0) [node, label=below:$1$] (1) {};
      %
      \pgfBox
      \end{tikzpicture} 
    };
    \node[font=\scriptsize, above] at (k.north) {$K_0$};
    \node[below] at (k.south) {$\rho_0$};
    \node  [right=of k] (r) {
      \begin{tikzpicture}
        \node at (0,0.53) {}; 
        \node at (0,0) [node, label=below:$1$] (1) {};
        \node at (.5,0) [node, label=below:$2$] (2) {};
        \draw[->, color=blue] (1) to[out=20, in=160] (2);
        %
        \pgfBox
      \end{tikzpicture}
    };
    \node[font=\scriptsize, above] at (r.north) {$R_0$};
    \path (k) edge[->] node[trans, above] {} (l);
    \path (k) edge[->] node[trans, above] {} (r);    
    \end{tikzpicture}
%  }
    %
    \hspace{1cm}
%  \hfill
  %
%  \subcaptionbox*{}{
    % RULE P1
    %
    \begin{tikzpicture}[node distance=2mm, font=\small]
      \node (l) {
      \begin{tikzpicture}
        %
        \node at (0,0.53) {}; 
        \node at (0,0) [node, label=below:$1$] (1) {} ;
        % 
      \pgfBox
      \end{tikzpicture} 
    };
    \node[font=\scriptsize, above] at (l.north) {$L_1$};
    \node [right=of l] (k) {
      \begin{tikzpicture}
        %
        \node at (0,0.53) {}; 
        \node at (0,0) [node, label=below:$1$] (1) {};
        % 
        \pgfBox
      \end{tikzpicture} 
    };
    \node[font=\scriptsize, above] at (k.north) {$K_1$};
    \node[below] at (k.south) {$\rho_1$};
    \node  [right=of k] (r) {
      \begin{tikzpicture}
        \node at (0,0) [node, label=below:$1$] (1) {}
         edge [in=55, out=85, loop] ();
        %
        \pgfBox
      \end{tikzpicture}
    };
    \node[font=\scriptsize, above] at (r.north) {$R_1$};
    \path (k) edge[->] node[trans, above] {} (l);
    \path (k) edge[->] node[trans, above] {} (r);
    \end{tikzpicture}
%  }
  % 
    \hspace{1cm}
    %\hfill
  %
%    \subcaptionbox*{}{
    % RULE P2
    %
    \begin{tikzpicture}[node distance=2mm, font=\small]
      \node (l) {
      \begin{tikzpicture}
        %
        \node at (0,0.53) {}; 
        \node at (0,0) [node, label=below:$1$] (1) {} ;
        \node at (0.5,0) [node, label=below:$2$] (2) {} ;
      %
      \pgfBox
      \end{tikzpicture} 
    };
    \node[font=\scriptsize, above] at (l.north) {$L_2$};
    \node [right=of l] (k) {
      \begin{tikzpicture}
        %
        \node at (0,0.53) {}; 
        \node at (0,0) [node, label=below:$1$] (1) {} ;
        \node at (0.5,0) [node, label=below:$2$] (2) {} ;
      %
      \pgfBox
      \end{tikzpicture} 
    };
    \node[font=\scriptsize, above] at (k.north) {$K_2$};
    \node[below] at (k.south) {$\rho_2$};
    \node  [right=of k] (r) {
      \begin{tikzpicture}
        \node at (0,0.53) {}; 
        \node at (0,0) [node, label=below:$12$] (12) {};
        %
        \pgfBox
      \end{tikzpicture}
    };
    \node[font=\scriptsize, above] at (r.north) {$R_2$};
    \path (k) edge[->] node[trans, above] {} (l);
    \path (k) edge[->] node[trans, above] {} (r);
    \end{tikzpicture}
 % }
  %
\end{center}

%%% Local Variables:
%%% mode: latex
%%% TeX-master: t
%%% End:

    
    \caption{Rules of the rewriting system grammar of the example.}
    \label{fi:rules}
  \end{figure}
  
  A derivation $\der{D}$ consisting of three steps $\der{D}_0$,
  $\der{D}_1$, and $\der{D}_2$, each applying the corresponding rule
  $\rho_i$ ,is depicted in in Fig.~\ref{fi:derD}. The second and third
  steps $\der{D}_1$ and $\der{D}_2$ are clearly sequential
  independent, via an independence pairs that map nodes consistently
  with their numbering.\todo{P: having names for the graphs would
    probably help}
  
  \begin{figure}
      \begin{center}
    \begin{tikzpicture}[node distance=2mm, font=\small, baseline=(current bounding box.center)]      
      \node (L1) at (0,2) {
        \begin{tikzpicture}
          % 
          \node at (0,0.53) {};
          \node at (0,0) [node, label=below:$1$] (1) {} ;
          % 
          \pgfBox
        \end{tikzpicture} 
      };
      \node [right=of L1] (K1) {
        \begin{tikzpicture}
          % 
          \node at (0,0.53) {}; 
          \node at (0,0) [node, label=below:$1$] (1) {};
          % 
          \pgfBox
        \end{tikzpicture} 
      };
      \node [above] at (K1.north) {$\rho_0$};
 %     \node [above=of K1] {$\rho_0$};
      \node [right=of K1](R1) {
        \begin{tikzpicture}
          \node at (0,0.53) {}; 
          \node at (0,0) [node, label=below:$1$] (1) {};
          \node at (.5,0) [node, label=below:$2$] (2) {}; 
          % 
          \pgfBox
        \end{tikzpicture}
      };
      \path (K1) edge[->] node[trans, above] {} (L1);
      \path (K1) edge[->] node[trans, above] {} (R1);

      \node at (4,2) (L2) {
        \begin{tikzpicture}
          % 
          \node at (0,0.53) {}; 
          \node at (0,0) [node, label=below:$2$] (2) {} ;
          % 
          \pgfBox
        \end{tikzpicture} 
      };
      \node [right=of L2] (K2) {
        \begin{tikzpicture}
          % 
          \node at (0,0.53) {}; 
          \node at (0,0) [node, label=below:$2$] (2) {};
          % 
          \pgfBox
        \end{tikzpicture} 
      };
 %     \node [above=of K2] {$\rho_1$};
      \node [above] at (K2.north) {$\rho_1$};
      \node [right=of K2] (R2) {
        \begin{tikzpicture}
          \node at (0,0) [node, label=below:$2$] (2) {}
          edge [in=95, out=65, loop] ();        
          % 
          \pgfBox
        \end{tikzpicture}
      };
      \path (K2) edge[->] node[trans, above] {} (L2);
      \path (K2) edge[->] node[trans, above] {} (R2);

      \node at (8,2) (L3) {
        \begin{tikzpicture}
          % 
          \node at (0,0.53) {}; 
          \node at (0,0) [node, label=below:$1$] (1) {} ;
          \node at (0.5,0) [node, label=below:$2$] (2) {} ;
          % 
          \pgfBox
        \end{tikzpicture} 
      };
      \node [right=of L3] (K3) {
        \begin{tikzpicture}
          % 
          \node at (0,0.53) {}; 
          \node at (0,0) [node, label=below:$1$] (1) {} ;
          \node at (0.5,0) [node, label=below:$2$] (2) {} ;
          % 
          \pgfBox
        \end{tikzpicture} 
      };
%      \node [above=of K3] {$\rho_2$};
      \node [above] at (K3.north) {$\rho_2$};
      \node [right=of K3] (R3) {
        \begin{tikzpicture}
          \node at (0,0.53) {}; 
          \node at (0,0) [node, label=below:$12$] (12) {};
          % 
          \pgfBox
        \end{tikzpicture}
      };
      \path (K3) edge[->] node[trans, above] {} (L3);
      \path (K3) edge[->] node[trans, above] {} (R3);

      %%%%%% second row
      \node at (0,0) (G1) {
        \begin{tikzpicture}
          % 
          \node at (0,0.53) {};
          \node at (0,0) [node, label=below:$1$] (1) {} ;
          % 
          \pgfBox
        \end{tikzpicture} 
      };
      \node [right=of G1] (D1) {
        \begin{tikzpicture}
          % 
          \node at (0,0.53) {}; 
          \node at (0,0) [node, label=below:$1$] (1) {};
          % 
          \pgfBox
        \end{tikzpicture} 
      };
      \node at (3,0) (G2) {
        \begin{tikzpicture}
          % 
          \node at (0,0.53) {}; 
          \node at (0,0) [node, label=below:$1$] (1) {} ;
          \node at (0.5,0) [node, label=below:$2$] (2) {} ;
          % 
          \pgfBox
        \end{tikzpicture} 
      };
      \path (D1) edge[->] node[trans, above] {} (G1);
      \path (D1) edge[->] node[trans, above] {} (G2);
      \path (L1) edge[->] node[trans, above] {} (G1);
      \path (K1) edge[->] node[trans, above] {} (D1);
      \path (R1) edge[->] node[trans, above] {} (G2);

      \node at (5,0) (D2) {
        \begin{tikzpicture}
          % 
          \node at (0,0.53) {};
          \node at (0,0) [node, label=below:$1$] (1) {} ;
          \node at (0.5,0) [node, label=below:$2$] (2) {} ;                    
          % 
          \pgfBox
        \end{tikzpicture} 
      };
      \node at (7,0) (G3) {
        \begin{tikzpicture}
          % 
          \node at (0,0.53) {}; 
          \node at (0,0) [node, label=below:$1$] (1) {};
          \node at (0.5,0) [node, label=below:$2$] (2) {}
                edge [in=95, out=65, loop] ();                    
          % 
          \pgfBox
        \end{tikzpicture} 
      };

      \path (D2) edge[->] node[trans, above] {} (G2);
      \path (D2) edge[->] node[trans, above] {} (G3);
      \path (L2) edge[->] node[trans, above] {} (G2);
      \path (K2) edge[->] node[trans, above] {} (D2);
      \path (R2) edge[->] node[trans, above] {} (G3);
      
      \node at (9.46,0) (D3) {
        \begin{tikzpicture}
          % 
          \node at (0,0) [node, label=below:$1$] (1) {};
          \node at (0.5,0) [node, label=below:$2$] (2) {}
                edge [in=95, out=65, loop] () ;
          % 
          \pgfBox
        \end{tikzpicture} 
      };
      \node [right=of D3] (G4) {
        \begin{tikzpicture}
          % 
          \node at (0,0) [node, label=below:$12$] (12) {}
                edge [in=95, out=65, loop] ();
          % 
          \pgfBox
        \end{tikzpicture} 
      };
      \node[font=\scriptsize, below] at (G1.south) {$G_0$};
      \node[font=\scriptsize, below] at (D1.south) {$D_0$};      
      \node[font=\scriptsize, below] at (G2.south) {$G_1$};
      \node[font=\scriptsize, below] at (D2.south) {$D_1$};      
      \node[font=\scriptsize, below] at (G3.south) {$G_2$};
      \node[font=\scriptsize, below] at (D3.south) {$D_2$};      
      \node[font=\scriptsize, below] at (G4.south) {$G_3$};      
      \path (D3) edge[->] node[trans, above] {} (G3);
      \path (D3) edge[->] node[trans, above] {} (G4);
      \path (L3) edge[->] node[trans, above] {} (G3);
      \path (K3) edge[->] node[trans, above] {} (D3);
      \path (R3) edge[->] node[trans, above] {} (G4);
    \end{tikzpicture}
  %
\end{center}
%%% Local Variables:
%%% mode: latex
%%% TeX-master: t
%%% End:

    \caption{The derivation $\der{D}$.}
    \label{fi:derD}
  \end{figure}

  The category $\cat{Graph}$ is tame by the result in ...  Thus,
  sequential independence ensure the existence of a switch $\der{E}$,
  depicted in Fig.~\ref{fi:derE}.
  \begin{figure}
      \begin{center}
    \begin{tikzpicture}[node distance=2mm, font=\small,  baseline=(current bounding box.center)]      
      \node (L1) at (0,2) {
        \begin{tikzpicture}
          % 
          \node at (0,0.53) {};
          \node at (0,0) [node, label=below:$1$] (1) {} ;
          % 
          \pgfBox
        \end{tikzpicture} 
      };
      \node [right=of L1] (K1) {
        \begin{tikzpicture}
          % 
          \node at (0,0.53) {}; 
          \node at (0,0) [node, label=below:$1$] (1) {};
          % 
          \pgfBox
        \end{tikzpicture} 
      };
 %     \node [above=of K1] {$\rho_0$};
      \node [above] at (K1.north) {$\rho_0$};
      \node [right=of K1](R1) {
        \begin{tikzpicture}
          \node at (0,0.53) {}; 
          \node at (0,0) [node, label=below:$1$] (1) {};
          \node at (.5,0) [node, label=below:$2$] (2) {}; 
          \draw[->, color=blue] (1) to[out=20, in=160] (2);
          % 
          \pgfBox
        \end{tikzpicture}
      };
      \path (K1) edge[->] node[trans, above] {} (L1);
      \path (K1) edge[->] node[trans, above] {} (R1);


      \node at (4,2) (L3) {
        \begin{tikzpicture}
          % 
          \node at (0,0.53) {}; 
          \node at (0,0) [node, label=below:$1$] (1) {} ;
          \node at (0.5,0) [node, label=below:$2$] (2) {} ;
          % 
          \pgfBox
        \end{tikzpicture} 
      };
      \node [right=of L3] (K3) {
        \begin{tikzpicture}
          % 
          \node at (0,0.53) {}; 
          \node at (0,0) [node, label=below:$1$] (1) {} ;
          \node at (0.5,0) [node, label=below:$2$] (2) {} ;
          % 
          \pgfBox
        \end{tikzpicture} 
      };
 %     \node [above=of K3] {$\rho_2$};
      \node [above] at (K3.north) {$\rho_2$};
      \node [right=of K3] (R3) {
        \begin{tikzpicture}
          \node at (0,0.53) {}; 
          \node at (0,0) [node, label=below:$12$] (12) {};
          % 
          \pgfBox
        \end{tikzpicture}
      };
      \path (K3) edge[->] node[trans, above] {} (L3);
      \path (K3) edge[->] node[trans, above] {} (R3);

      \node at (8,2) (L2) {
        \begin{tikzpicture}
          % 
          \node at (0,0.53) {}; 
          \node at (0,0) [node, label=below:$12$] (1) {} ;
          % 
          \pgfBox
        \end{tikzpicture} 
      };
      \node [right=of L2] (K2) {
        \begin{tikzpicture}
          % 
          \node at (0,0.53) {}; 
          \node at (0,0) [node, label=below:$12$] (1) {};
          % 
          \pgfBox
        \end{tikzpicture} 
      };
 %     \node [above=of K2] {$\rho_1$};
      \node [above] at (K2.north) {$\rho_1$};

      \node [right=of K2] (R2) {
        \begin{tikzpicture}
          \node at (0,0) [node, label=below:$12$] (1) {}
          edge [in=55, out=85, loop] ();
          % 
          \pgfBox
        \end{tikzpicture}
      };
      \path (K2) edge[->] node[trans, above] {} (L2);
      \path (K2) edge[->] node[trans, above] {} (R2);

      
      %%%%%% second row
      \node at (0,0) (G1) {
        \begin{tikzpicture}
          % 
          \node at (0,0.53) {};
          \node at (0,0) [node, label=below:$1$] (1) {} ;
          % 
          \pgfBox
        \end{tikzpicture} 
      };
      \node [right=of G1] (D1) {
        \begin{tikzpicture}
          % 
          \node at (0,0.53) {}; 
          \node at (0,0) [node, label=below:$1$] (1) {};
          % 
          \pgfBox
        \end{tikzpicture} 
      };
      \node at (3,0) (G2) {
        \begin{tikzpicture}
          % 
          \node at (0,0.53) {}; 
          \node at (0,0) [node, label=below:$1$] (1) {} ;
          \node at (0.5,0) [node, label=below:$2$] (2) {} ;
          \draw[->, color=blue] (1) to[out=20, in=160] (2);
          % 
          \pgfBox
        \end{tikzpicture} 
      };
      \path (D1) edge[->] node[trans, above] {} (G1);
      \path (D1) edge[->] node[trans, above] {} (G2);
      \path (L1) edge[->] node[trans, above] {} (G1);
      \path (K1) edge[->] node[trans, above] {} (D1);
      \path (R1) edge[->] node[trans, above] {} (G2);

      \node at (5.5,0) (D2) {
        \begin{tikzpicture}
          % 
          \node at (0,0.53) {};
          \node at (0,0) [node, label=below:$1$] (1) {} ;
          \node at (0.5,0) [node, label=below:$2$] (2) {} ;
          \draw[->, color=blue] (1) to[out=20, in=160] (2);
          % 
          \pgfBox
        \end{tikzpicture} 
      };
      \node at (7.4,0) (G3) {
        \begin{tikzpicture}
          % 
          \node at (0,0.53) {}; 
          \node at (0,0) [node, label=below:$12$] (12) {}
          edge [in=125, out=155, color=blue, loop] ();
          % 
          \pgfBox
        \end{tikzpicture} 
      };

      \path (D2) edge[->] node[trans, above] {} (G2);
      \path (D2) edge[->] node[trans, above] {} (G3);
      \path (L3) edge[->] node[trans, above] {} (G2);
      \path (K3) edge[->] node[trans, above] {} (D2);
      \path (R3) edge[->] node[trans, above] {} (G3);
      
      \node at (9.1,0) (D3) {
        \begin{tikzpicture}
          % 
          \node at (0,0.53) {}; 
          \node at (0,0) [node, label=below:$12$] (12) {}
          edge [in=125, out=155, color=blue, loop] ();
          % 
          \pgfBox
        \end{tikzpicture} 
      };
      \node [right=of D3] (G4) {
        \begin{tikzpicture}
          % 
          \node at (0,0) [node, label=below:$12$] (12) {}
          edge [in=55, out=85, loop] ()
          edge [in=125, out=155, color=blue, loop] ();
          % 
          \pgfBox
        \end{tikzpicture} 
      };

      \node[font=\scriptsize, below] at (G1.south) {$G_0$};
      \node[font=\scriptsize, below] at (D1.south) {$D_0$};      
      \node[font=\scriptsize, below] at (G2.south) {$G_1$};
      \node[font=\scriptsize, below] at (D2.south) {$D_1'$};      
      \node[font=\scriptsize, below] at (G3.south) {$G_2'$};
      \node[font=\scriptsize, below] at (D3.south) {$D_2'$};      
      \node[font=\scriptsize, below] at (G4.south) {$G_3$};      

      \path (D3) edge[->] node[trans, above] {} (G3);
      \path (D3) edge[->] node[trans, above] {} (G4);
      \path (L2) edge[->] node[trans, above] {} (G3);
      \path (K2) edge[->] node[trans, above] {} (D3);
      \path (R2) edge[->] node[trans, above] {} (G4);
    \end{tikzpicture}
  %
\end{center}
%%% Local Variables:
%%% mode: latex
%%% TeX-master: t
%%% End:

    \caption{The derivation $\der{E}$.}
    \label{fi:derE}
  \end{figure}
\end{example}
\fi 

The conditions guaranteeing switchability are inspired
by the notion of \emph{canonical filler}
\cite{heindel2009category}.

\todo{P: la chiamiamo ``strong indepedence pair''? o ``filler pair''}

\begin{definition}[Good independence pair]
  \label{def:filler}
  Let with $\X$ an $\mathcal{M}$-adhesive category 
  and $(\X, \R)$ a left-linear DPO rewriting system. 
  Let also $(i_0, i_1)$ be an independence pair between two direct
  derivations $\dder{D}_0$, $\dder{D}_1$
  as in the solid part of the diagram below
  \[
    \xymatrix@C=18pt{L_0 \ar[d]_{m_0}&& K_0
      \ar[d]_{k_0}\ar@{>->}[ll]_{l_0} \ar[r]^{r_0} \ar@{->}@/^.20cm/[ddrr]|(.32)\hole|(.57)\hole_(.7){u_0} & R_0
      \ar@/^.35cm/[drrr]|(.3)\hole^(.2){i_0} \ar[dr]|(.3)\hole_{h_0}
      && L_1 \ar@/_.35cm/[dlll]_(.2){i_1} \ar[dl]|(.3)\hole^{m_1}& K_1
      \ar[d]^{k_1}\ar@{>->}[l]_{l_1} \ar[rr]^{r_1} \ar@{->}@/_.20cm/[ddll]|(.32)\hole|(.57)\hole^(.7){u_1} && R_1 \ar[d]^{h_1} \\
      %
      G_0 &&
      \ar@{>->}[ll]^{f_0} D_0 \ar[rr]_(.3){g_0}&& G_1 &&
      \ar@{>->}[ll]^(.3){f_1} D_1 \ar[rr]_{g_1}&& G_2\\
      %
      & & & & P \ar@/^.3cm/@{>.>}[ull]^{p_0} \ar@/_.3cm/@{.>}[urr]_{p_1}
    }
  \]

  Consider the pullback of $g_0$ and $f_1$, getting $p_i : P \to D_i$
  ($i \in \{0,1\}$) and the mediating arrows $u_0\colon K_0\to P$ and
  $u_1\colon K_1\to P$ into the pullback given by \Cref{prop:tec} .
  %
  We will say that $(i_0, i_1)$ is a \emph{good independence pair} if
  the first two squares depicted  
  below are pushouts and if the pushout of $r_1 : K_1 \to R_1$ and $u_1 : K_1 \to P$ exists.
  \[
    \xymatrix{
      K_0 \ar[r]^{r_0} \ar[d]_{u_0}& R_0 \ar[d]^{i_0} & K_1
      \ar@{>->}[r]^{l_1} \ar[d]_{u_1}& L_1 \ar[d]^{i_1}
       &K_1 \ar[r]^{r_1} \ar[d]_{u_1}& R_1\ar@{.>}[d]^{j_0}
      \\
      %
      P \ar[r]_{p_1} & D_1 &P \ar[r]_{p_0} & D_0
       & P \ar@{.>}[r]_{q_1} & Q_1
    }
  \]
\end{definition}


% \begin{definition}[Good independence pair]\label{def:filler}
%   Let $(i_0, i_1)$ be an independence pair between two direct
%   derivations $\dder{D}_0$, $\dder{D}_1$ in $(\X, \R)$ be a
%   left-linear DPO rewriting system with $\X$ $\mathcal{M}$-adhesive.
%   \[
%     \xymatrix@C=15pt{L_0 \ar[d]_{m_0}&& K_0
%       \ar[d]_{k_0}\ar@{>->}[ll]_{l_0} \ar[r]^{r_0} & R_0
%       \ar@/^.35cm/[drrr]|(.3)\hole_(.4){i_0} \ar[dr]|(.3)\hole_{h_0}
%       && L_1 \ar@/_.35cm/[dlll]^(.4){i_1} \ar[dl]|(.3)\hole^{m_1}& K_1
%       \ar[d]^{k_1}\ar@{>->}[l]_{l_1} \ar[rr]^{r_1} && R_1 \ar[d]^{h_1}
%       && P \ar@{>->}[rr]^{p_0} \ar[d]_{p_1}&& D_0\ar[d]^{g_0}\\G_0 &&
%       \ar@{>->}[ll]^{f_0} D_0 \ar[rr]_{g_0}&& G_1 &&
%       \ar@{>->}[ll]^{f_1} D_1 \ar[rr]_{g_1}&& G_2 && D_1
%       \ar@{>->}[rr]_{f_1} && G_1}
%   \]

%   Let also the right square above be a pullback. Let
%   $u_0\colon K_0\to P$ and $u_1\colon K_1\to P$ be the arrows of
%   \Cref{prop:tec}, we will say that $(i_0, i_1)$ is a \emph{good
%     independence pair} if the first and second square below are
%   pushouts and if the third square exists and it is a pushout too.
%   \[\xymatrix{K_0 \ar[r]^{r_0} \ar[d]_{u_0}& R_0 \ar[d]^{i_0} & K_1
%       \ar@{>->}[r]^{l_1} \ar[d]_{u_1}& L_1 \ar[d]^{i_1}&K_1
%       \ar[r]^{r_1} \ar[d]_{u_1}& R_1\ar@{.>}[d]^{j_0} \\ P
%       \ar[r]_{p_1} & D_1 &P \ar[r]_{p_0} & D_0&P \ar@{.>}[r]_{q_1} &
%       Q_1}\]
% \end{definition}


We can now prove a Local Church-Rosser Theorem for good independence pairs.


\begin{restatable}[Local Church-Rosser Theorem]{theorem}{thmChurch}
  \label{thm:church}
  Let $(i_0, i_1)$ be a good independence pair
  between $\dder{D}_0$ and $\dder{D}_1$. Then $\dder{D}_0$ and
  $\dder{D}_1$ are switchable.
\end{restatable}
[Proof in \Cref{thmChurch-proof}]



\todo[inline]{Forse non introdurrei neppure la classe dei tame, tanto usiamo solo i very tame, o no?}

\todo[inline]{Io invece la introdurrei proprio come punto teorico che mi sembra parecchio importante: qua li condizioni mi garantiscono che gli independence pair implicano la scambiabilità? [DC]}

The correspondence between sequential independence and switchability
is fundamental.  A very large class of DPO rewriting systems - in the
line of \cite{baldan2011adhesivity} - can be identified where all
independence pairs are good for left-linear rewriting systems. The details are in 
\Cref{app:fill}. Here it is important only to observe that such class includes $\cat{Set}$ and it is closed by comma and functor category constructions. As such it includes essentially all categories (e.g., presheaves over set) which are typically considered for modelling purposes.



%\rem{Non introdurrei una classe, come detto sopra}{
\begin{definition}[Tame rewriting systems]A left-linear DPO rewriting system is \emph{tame} if every independence pair between two direct derivations is good.
\end{definition}

\begin{example}
  \label{ex:diff2}
  In light of \Cref{thm:church}, \Cref{ex:diff1} provides an example
  of an independence pair which is not good, and thus of a non tame
  adhesive rewriting system. Instead the category $\cat{Graph}$ is
  tame by the result in ....
\end{example}

\begin{remark}
  By \Cref{thm:church} the existence of a good independence pair
  entails switchability which, in turn, entails weak sequential
  independence by construction. Tame rewriting systems are exactly
  those rewriting systems in which these three notions coincide.
\end{remark}
%}



%\rem{Questo e' quanto gia' asserito sopra, mettere in appendice e riferire}

A source of tame left-linear DPO rewriting systems is given by the linear ones, as shown by the following proposition.

\begin{restatable}{proposition}{propEqui}
  \label{prop:equi}
  Every linear DPO rewriting system $(\X, \R)$ is tame.
\end{restatable}
[Proof in \Cref{propEqui-proof}]

\subsection{Very tame rewriting systems}\label{subsec:verytame}

Even if we work in a setting where sequential independence ensures switchability, dealing with left-linear rules there is a further - possibly more serious - issue, i.e., the fact that there can be more than one independence pairs between the same derivations. This can hinder the core idea of using sequential independence as a basis of a theory of concurrency for rewriting systems, since exchanges performed using different independence pairs may lead to derivations that are not abstraction equivalent, and thus one would equate computations that 
should definitively be taken apart, as shown in the example below.

\begin{example}
  Consider the derivation $\der{E}$ from Example~\ref{ex:seq-ind}.
  %
  The last two steps are sequential independent but one easily sees
  that there are two distinct indepedence pair, as the left-hand side
  of $\rho_1$ can be mapped either to node $1$ or to node
  $2$.\todo{Sarebbe bello dire di quale grafo} Correspondingly, there
  are two switches of $\der{E}$: one is derivation $\der{D}$ in
  FIgure~\ref{fi:derD} we started from, the other is derivation
  $\der{D}'$ Figure~\ref{fi:derD1}.
  
  \begin{figure}
      \begin{center}
    \begin{tikzpicture}[node distance=2mm, font=\small, baseline=(current bounding box.center)]      
      \node (L1) at (0,2) {
        \begin{tikzpicture}
          % 
          \node at (0,0.53) {};
          \node at (0,0) [node, label=below:$1$] (1) {} ;
          % 
          \pgfBox
        \end{tikzpicture} 
      };
      \node [right=of L1] (K1) {
        \begin{tikzpicture}
          % 
          \node at (0,0.53) {}; 
          \node at (0,0) [node, label=below:$1$] (1) {};
          % 
          \pgfBox
        \end{tikzpicture} 
      };
%      \node [above=of K1] {$\rho_0$};
      \node [above] at (K1.north) {$\rho_0$};
      \node [right=of K1](R1) {
        \begin{tikzpicture}
          \node at (0,0.53) {}; 
          \node at (0,0) [node, label=below:$1$] (1) {};
          \node at (.5,0) [node, label=below:$2$] (2) {};
          \draw[coloredge] (1) to[out=20, in=160] (2);
          % 
          \pgfBox
        \end{tikzpicture}
      };
      \path (K1) edge[->] node[trans, above] {} (L1);
      \path (K1) edge[->] node[trans, above] {} (R1);

      \node at (4,2) (L2) {
        \begin{tikzpicture}
          % 
          \node at (0,0.53) {}; 
          \node at (0,0) [node, label=below:$1$] (1) {} ;
          % 
          \pgfBox
        \end{tikzpicture} 
      };
      \node [right=of L2] (K2) {
        \begin{tikzpicture}
          % 
          \node at (0,0.53) {}; 
          \node at (0,0) [node, label=below:$1$] (1) {};
          % 
          \pgfBox
        \end{tikzpicture} 
      };
%      \node [above=of K2] {$\rho_1$};
            \node [above] at (K2.north) {$\rho_1$};
      \node [right=of K2] (R2) {
        \begin{tikzpicture}
          \node at (0,0) [node, label=below:$1$] (1) {}
          edge [in=55, out=85, loop] ();        
          % 
          \pgfBox
        \end{tikzpicture}
      };
      \path (K2) edge[->] node[trans, above] {} (L2);
      \path (K2) edge[->] node[trans, above] {} (R2);

      \node at (8,2) (L3) {
        \begin{tikzpicture}
          % 
          \node at (0,0.53) {}; 
          \node at (0,0) [node, label=below:$1$] (1) {} ;
          \node at (0.5,0) [node, label=below:$2$] (2) {} ;
          % 
          \pgfBox
        \end{tikzpicture} 
      };
      \node [right=of L3] (K3) {
        \begin{tikzpicture}
          % 
          \node at (0,0.53) {}; 
          \node at (0,0) [node, label=below:$1$] (1) {} ;
          \node at (0.5,0) [node, label=below:$2$] (2) {} ;
          % 
          \pgfBox
        \end{tikzpicture} 
      };
 %     \node [above=of K3] {$\rho_2$};
      \node [above] at (K3.north) {$\rho_2$};
      \node [right=of K3] (R3) {
        \begin{tikzpicture}
          \node at (0,0.53) {}; 
          \node at (0,0) [node, label=below:$12$] (12) {};
          % 
          \pgfBox
        \end{tikzpicture}
      };
      \path (K3) edge[->] node[trans, above] {} (L3);
      \path (K3) edge[->] node[trans, above] {} (R3);

      %%%%%% second row
      \node at (0,0) (G1) {
        \begin{tikzpicture}
          % 
          \node at (0,0.53) {};
          \node at (0,0) [node, label=below:$1$] (1) {} ;
          % 
          \pgfBox
        \end{tikzpicture} 
      };
      \node [right=of G1] (D1) {
        \begin{tikzpicture}
          % 
          \node at (0,0.53) {}; 
          \node at (0,0) [node, label=below:$1$] (1) {};
          % 
          \pgfBox
        \end{tikzpicture} 
      };
      \node at (3,0) (G2) {
        \begin{tikzpicture}
          % 
          \node at (0,0.53) {}; 
          \node at (0,0) [node, label=below:$1$] (1) {} ;
          \node at (0.5,0) [node, label=below:$2$] (2) {} ;
          \draw[coloredge] (1) to[out=20, in=160] (2);
          % 
          \pgfBox
        \end{tikzpicture} 
      };
      \path (D1) edge[->] node[trans, above] {} (G1);
      \path (D1) edge[->] node[trans, above] {} (G2);
      \path (L1) edge[->] node[trans, above] {} (G1);
      \path (K1) edge[->] node[trans, above] {} (D1);
      \path (R1) edge[->] node[trans, above] {} (G2);

      \node at (5,0) (D2) {
        \begin{tikzpicture}
          % 
          \node at (0,0.53) {};
          \node at (0,0) [node, label=below:$1$] (1) {} ;
          \node at (0.5,0) [node, label=below:$2$] (2) {} ;
          \draw[coloredge] (1) to[out=20, in=160] (2);
          % 
          \pgfBox
        \end{tikzpicture} 
      };
      \node at (7,0) (G3) {
        \begin{tikzpicture}
          % 
          \node at (0,0.53) {}; 
          \node at (0,0) [node, label=below:$1$] (1) {}
          edge [in=55, out=85, loop] ();
          \node at (0.5,0) [node, label=below:$2$] (2) {} ;
          \draw[coloredge] (1) to[out=20, in=160] (2);
          % 
          \pgfBox
        \end{tikzpicture} 
      };

      \path (D2) edge[->] node[trans, above] {} (G2);
      \path (D2) edge[->] node[trans, above] {} (G3);
      \path (L2) edge[->] node[trans, above] {} (G2);
      \path (K2) edge[->] node[trans, above] {} (D2);
      \path (R2) edge[->] node[trans, above] {} (G3);
      
      \node at (9.46,0) (D3) {
        \begin{tikzpicture}
          % 
          \node at (0,0) [node, label=below:$1$] (1) {}
          edge [in=55, out=85, loop] ();
          \node at (0.5,0) [node, label=below:$2$] (2) {} ;
          \draw[coloredge] (1) to[out=20, in=160] (2);
          % 
          \pgfBox
        \end{tikzpicture} 
      };
      \node [right=of D3] (G4) {
        \begin{tikzpicture}
          % 
          \node at (0,0) [node, label=below:$12$] (12) {}
          edge [in=55, out=85, loop] ()
          edge [in=125, out=155, colorloop] ();
          % 
          \pgfBox
        \end{tikzpicture} 
      };
      \node[font=\scriptsize, below] at (G1.south) {$G_0$};
      \node[font=\scriptsize, below] at (D1.south) {$D_0$};      
      \node[font=\scriptsize, below] at (G2.south) {$G_1$};
      \node[font=\scriptsize, below] at (D2.south) {$D_1''$};      
      \node[font=\scriptsize, below] at (G3.south) {$G_2''$};
      \node[font=\scriptsize, below] at (D3.south) {$D_2''$};      
      \node[font=\scriptsize, below] at (G4.south) {$G_3$};      

      \path (D3) edge[->] node[trans, above] {} (G3);
      \path (D3) edge[->] node[trans, above] {} (G4);
      \path (L3) edge[->] node[trans, above] {} (G3);
      \path (K3) edge[->] node[trans, above] {} (D3);
      \path (R3) edge[->] node[trans, above] {} (G4);
    \end{tikzpicture}
  %
\end{center}
%%% Local Variables:
%%% mode: latex
%%% TeX-master: t
%%% End:

    \caption{The derivation $\der{D}'$.}
    \label{fi:derD1}
  \end{figure}
  
  As a consequence $\der{D}$ and $\der{D}'$ are switch equivalent, but
  this is not acceptable when viewing equivalence classes of
  derivations as concurrent computations: in $\der{D}$ the first two
  steps are not sequential independent, while in $\der{D}'$ they are
  (intuitively because in $\der{D}$ rule $\rho_1$ uses the node
  generated by $\rho_0$, while in $\der{D}'$ rule $\rho_1$ uses the
  node which was in the initial graph. From the technical point of
  view, the property of being switch equivalent is not closed by
  prefix, something which prevents to derive a sensible concurrent
  sematics: if we limit to the first two steps, derivations
  $\der{D}_0; \der{D}_1$ and $\der{D}_0'; \der{D}_1'$ are not switch
  equivalent while $\der{D}$ and $\der{D}'$ would be
\end{example} 

Limiting sequential independence to the case in which the independence
pair is unique (as suggested in~ ...) brings to the same problems.

For these reasons we support the idea that a theory of rewriting for
left-linear rules in adhesive categories should be developed for
systems where uniqueness of the independence pair is always ensured.


%\subsubsection{Graphical rewriting systems}\label{subsubsec:graphical}


\begin{definition}[Very tame rewriting systems]
A left-linear DPO rewriting system $(\X, \R)$ is \emph{very tame} if it is very tame and, for every derivation $\der{D}:=\{\dder{D}_{i}\}_{i=0}^1$, there is at most one independence paire between $\dder{D}_0$ and $\dder{D}_1$.
\end{definition}

The previous definition might sound restrictive. First, it is immediate to see that linear rewriting systems are all very tame (see Proposition~\ref{pr:weak}).
%
Moreover we next introduce a class of rewriting systems, comprising
all the ones used in concrete modelling applications of DPO rewriting
(e.g., in the encoding of process calculi ..., ...) are \todo{INSERIRE ESEMPI QUI}
actually very tame.

\begin{definition}
Let $\mathcal{G}=(V, E, s, t)$ be a finite directed acyclic graph, we define the \emph{depth} of a vertex $v\in V$ in the following way:
\[\dph(v)=\begin{cases}
0 & \text{there is no $e\in E$ such that $t(e)=v$}\\
m+1 &\text{where } m=\min\{\dph(s(e)) \mid e\in E \text{ such that } t(e)=v \} 
\end{cases}\]
\end{definition}


\begin{remark}
Since $\mathcal{G}$ is acyclic and finite, then $\dph(v)$ is a natural number.
\end{remark}

\begin{definition}[Node-merging graphical rewriting systems]\todo{inserire referenze qui}
Let $\mathcal{G}=(V, E, s, t)$ be a directed acyclic graph, the category  $\gph{G}$ of \emph{$\mathcal{G}$-graphs} is the category $\Set^{\G}$, where $\G$ is the free category over $\mathcal{G}$. A \emph{norde-merging graphical rewriting systems} is a left-linear DPO-rewriting system $(\gph{G}, \R)$ such that, for every rule $(l,r)$, with $l\colon K\to L$ and $r\colon K\to R$  in $\R$, the following hold true:
\begin{enumerate}
	\item for every $v\in V$ and $x\in L(v)$, there exists a vertex $w$ with depth $0$ and an arrow $p\colon w\to v$ (i.e.~a path in $\mathcal{G}$) such that  $x$ belongs to the image of $L(p)\colon L(w)\to L(v)$.
	\item $r\colon K\to R$ is mono on vertices of depth $0$: for every $v\in V$ such that $\dph(v)=0$, the component $r_v:K(v)\to R(v)$ is mono.
\end{enumerate}
\end{definition}

\begin{remark}\label{rem:po}
	Notice that $\gph{G}$ is adhesive and that pushouts are computed componentwise, thus if $g\colon F\to G$ is the pushout of $r\colon K\to R$, then $g_v\colon F(v)\to G(v)$ is mono for every $v$ with depth $0$.
\end{remark}

\begin{example}
\todo{esempi (no idea)}
\end{example}

\begin{restatable}{lemma}{lemVTame}
  All node-merging graphical rewriting systems are very tame.
\end{restatable}

[Proof in \Cref{lemVTame}]


\begin{example}[E-graphs]
  Consider the category $\cat{EGraphs}$, where objects are (directed)
  graphs endowed with an equivalence over nodes and arrows are graph
  morphisms which preserve the equialence as considered in~\cite{},
  closely related to the Egg graphs in~\cite{WNW:egg}. Formally,
  $\cat{EGraphs}$ can be seen as the subcategory of the presheaf
  $[E \rightrightarrows V \to Q, \cat{Set}]$ where objects are
  constrained to have the component $Q$ surjective. Explicitly, an
  e-graph $G = \langle s_G, t_G, q_G \rangle$ where
  $\langle s_G, t_G: E_G \rightrightarrows V_G$ provides the graphical
  structure, while the surjective function $q_G : V_G \to Q_G$ maps
  each node to the corresponding equivalence class.
  
  A morphims in $\cat{EGraph}$ is is mono if the components over $E$
  and $V$ are mono, i.e., if it is mono as a morphism in
  $\cat{Graph}$. It is \emph{regular mono} if also the component on
  $Q$ is mono, i.e., if it reflects equivalence classes besides
  preserving them.

  \todo[inline]{$\cat{EGraph}$ is quasi-adhesive and very tame}  
\end{example}

\subsection{Switch equivalence via permutation equivalence} \label{subsec:perm}


Given DPO rewriting system $(\X, \R)$,  a derivation $\der{D}$ determines a diagram
$\Delta(\der{D})$ in $\X$. Using $\mathcal{M}$-adhesivity, it can be shown  that such diagram has a colimit $(\tpro{D}, \{\iota_X\}_{X\in \Delta(\der{D})})$ and that, if $(\X, \R)$ is linear, then every coprojection is in $\mathcal{M}$ ( see \Cref{lem:colim} for details and a proof of these statements).

\begin{definition}[Consistent permutation]
	\label{def:permcon}
	Let $\X$ be an $\mathcal{M}$-adhesive category and consider a
	left-linear DPO rewriting system $(\X, \R)$ on it.  Take two
	non-empty decorated derivations $(\der{D}, \alpha, \omega)$ and
	$(\der{D}', \alpha', \omega')$ with the same length and with
	isomorphic sources and targets.
	
	A \emph{consistent} permutation between two derivations $\der{D}=\{\dder{D}_i\}_{i=0}^n$ and
	$\der{D}'=\{\dder{D}'_i\}_{i=0}^n$ with associated sequence of rules
	$r(\der{D})=\{\rho_i\}_{i=0}^n$ is a permutation
	$\sigma\colon [0,n]\to [0,n]$ such that, for every $i\in [0,n]$,
	$\rho_i=\rho'_{\sigma(i)}$ and, moreover, there exists a
	\emph{mediating isomorphism}
	$\xi_\sigma\colon \tpro{D} \to \lpro \der{D}' \rpro$ fitting in the
	following diagrams, where $m_i, m'_i, h_i$ and $h'_i$ are,
	respectively, the matches and back-matches of $\dder{D}_i$ and
	$\dder{D}'_i$.
	
	\[
	\xymatrix@C=30pt{L_i
		\ar[r]^{m_i} \ar[d]_{m'_{\sigma(i)}}& G_i \ar[r]^{\iota_{G_i}}
		&\tpro{D} \ar[d]^{\xi_\sigma} & R_i \ar[r]^{h_i}
		\ar[d]_{h'_{\sigma(i)}}& G_{i+1} \ar[r]^{\iota_{G_{i+1}}}
		&\tpro{D} \ar[d]^{\xi_\sigma} \\G'_{\sigma(i)}
		\ar[rr]_{\iota'_{G'_{\sigma(i)}}}&& \lpro \der{D}' \rpro&
		G'_{\sigma(i)+1} \ar[rr]_{\iota'_{G'_{\sigma(i)+1}}}&& \lpro
		\der{D}' \rpro}
	\]
\end{definition}


	
\begin{restatable}{proposition}{propcoswitch}\label{prop:coswitch}
	consistente tra cose equivalenti
\end{restatable}	
[Proof in \Cref{propcoswitch-proof}]

\begin{lemma}
	lineare => identità implica astrazione
\end{lemma}
\begin{proof}
	contenuto...
\end{proof}

\begin{restatable}{theorem}{thmcoswitch}
lineare iff
\end{restatable}


\subsection{The $3$-steps lemmas}

The characterisation of switch equivalence based on consistent
permutation for linear rules allows to derive some simple but
fundamental results for the theory, namely
\begin{itemize}
\item the result of switching does not depends on the order, but only
  on the permutation;
\item independence is global, meaning that if two independent steps are
  moved forward or backward in a derivation, they remain
  independent.
\end{itemize}

The failure of the characterisation for left-linear rules poses the
question asking whether the results above also fail.
%
We observe that
\begin{itemize}

\item the first can be proved, yet a different proof technique is needed,
  taking advantage from the axiomatic characterisation of switching;

\item Independence is global partially fails: one part holds, the
  other in a weak form, but this is conceptually fini (related to or
  causality).
\end{itemize}

As a beginning step in our analysis of switch equivalence, we will
establish three lemmas dealing with derivations of length
$3$. \todo{We will probably limit tho this}


\begin{restatable}[First $3$-steps Lemma]{lemma}{lemThreeSteps}
  \label{lem:three-steps}
  Let $\X$ be an $\mathcal{M}$-adhesive category and
  $(\X,\R)$ a left-linear DPO rewriting system.
%
  Consider a
  derivation $\der{D}=\{\dder{D}_i\}_{i=0}^2$ and suppose that
  $(i_0,i_1)$ is a good independence pair between $\dder{D}_0$ and
  $\dder{D}_1$, $(a_0,a_1)$ one between $\dder{D}_1$ and $\dder{D}_2$
  and $(e_0, e_1)$ one between $\dder{D}_0$ and
  $S_{a_0,a_1}(\dder{D}_2)$.
\end{restatable}


[Proof~\ref{lemThreeSteps-proof}]

\rem{remove}{
\begin{lemma}[Second $3$-steps Lemma]\label{lem:secondo}
	contenuto...
\end{lemma}
\begin{proof}
	contenuto...
\end{proof}



\begin{lemma}[Third $3$-steps Lemma]\label{lem:terzo}
	contenuto...
\end{lemma}
\begin{proof}
	contenuto...
\end{proof}
}

\todo{dividere in due il lemma dopo. devono venire tre lemmi}
\begin{restatable}[Three steps Lemma]{lemma}{lemIndepGlobalLeft}
  \label{lem:indep-global-left}
  Let $\X$ be an $\mathcal{M}$-adhesive category and
  $(\X,\R)$ a left-linear DPO rewriting system.
  %
  Consider a derivation
  $\der{D}=\{\dder{D}_i\}_{i=0}^2$ and suppose that $(i_0,i_1)$ is a
  good independence pair between $\dder{D}_0$ and $\dder{D}_1$,
  $(a_0,a_1)$ one between $\dder{D}_1$ and $\dder{D}_2$ and
  $(e_0, e_1)$ one between $\dder{D}_0$ and
  $S_{a_0,a_1}(\dder{D}_2)$. Then the following properties hold true.
  \begin{enumerate}
  \item $S_{e_0,e_1}(\dder{D}_0)$ and $S_{a_0,a_1}(\dder{D}_1)$ are
     sequentially independent.
  \item If $S_{i_0, i_1}(\dder{D}_0)\updownarrow_! \dder{D}_2$ with a
    good independence pair $(\alpha_0, \alpha_1)$, then
    $S_{i_0,i_1}(\dder{D}_1)$ and $S_{\alpha_0, \alpha_1}(\dder{D}_2)$
    are  sequentially independent.
  \end{enumerate}
\end{restatable}

[Proof~\ref{lemIndepGlobalLeft-proof}]


\rem{eliminare??? Non so come sia legato col resto}{ 

  \begin{lemma}\label{lem:consperm} Let $(\X, \R)$ be a left-linear
    DPO rewriting system $(\X, \R)$. Then the following hold true
    \begin{enumerate}
    \item If $\dder{D}$ and $\dder{D'}$ are two direct derivations
      such that $\dder{D}\updownarrow \dder{D'}$, then for every
      filler $(u,u')$ between them, the function
      \[\tau\colon 2\to2 \qquad x \mapsto \begin{cases}
          1 & x=0 \\
          0 & x=1
        \end{cases}\] defines a consistent permutation between
      $\dder{D}\cdot \dder{D}'$ and $\sder{D}{D'}$;
    \item if $\der{D}$ and $\der{D}'$ are two switch equivalent
      derivation, then there exists a consistent permutation between
      them.
    \end{enumerate}
  \end{lemma}
  \begin{proof}
    \begin{enumerate}
    \item
    \item \qedhere
    \end{enumerate}
  \end{proof}
  This, together with ???
  \begin{corollary}
    \todo{unicità}
  \end{corollary}

  \begin{example}\label{ex:contro}\todo{permutazione consistente non
      implica scambiabilità}
  \end{example}




  \subsubsection{No needs for useless switches}
  \todo{richiamare cosa sono le inversioni}

  \begin{lemma}[Fundamental Lemma] Let $(\der{D}, \alpha,
    \omega)$ and $(\der{D'}, \alpha, \omega)$ be two decorated
    derivations and suppose that there exists a non-empty
    sequence $\{\der{D}_i\}_{i=0}^n$ a of derivations witnessing
    $\der{D}\equiv^s \der{D}'$. For every $i\in [0,n-1]$, let
    $\nu_i\colon [0, \lgh(\der{D})-1]\to [0, \lgh(\der{D})-1]$
    be the $2$-cycle associated to the switch between
    $\der{D}_i$ and $\der{D}_{i+1}$, let also $\sigma$ be the
    associated consistent permutation between
    $(\der{D}, \alpha, \omega)$ and
    $(\der{D}', \alpha', \omega')$. Define $k$ as the maximum of
    the set
    \[M_\sigma:=\{j\in [0, \lgh(\der{D}-1] \mid \sigma(j+1) <
      \sigma(j) \}\] If every $\nu_i$ is an inversion for
    $\sigma$, then there exists a family $\{\nu'_i\}_{i=0}^n$ of
    inversions for $\sigma$ such that jj\todo{capire come
      scriverlo meglio, l} $\nu'_0=(k, k+1)$.
  \end{lemma}
  \begin{remark}
    Since $\{\der{D}_i\}_{i=0}^n$ is non-empty, then $\sigma$
    must be different from $\id{[0, \lgh(\der{D}-1)]}$ and
    $M_{\sigma}\neq \emptyset$.
  \end{remark}
  \begin{proof}
    We proceed by induction on $n$.

    \smallskip\noindent $n=1$. In this case there is nothing to
    prove.

    \smallskip \noindent $n>1$. Then...
  \end{proof}

  \begin{corollary}
    contenuto...
  \end{corollary}
  \begin{proof}
    contenuto...
  \end{proof}


  \begin{corollary}
    contenuto...
  \end{corollary}
  \begin{proof}
    contenuto...
  \end{proof}



  \begin{corollary}
    contenuto...
  \end{corollary}
  \begin{proof}
    contenuto...
  \end{proof}


  \begin{corollary}[No need for useless switches]
    contenuto...
  \end{corollary}
  \begin{proof}
    contenuto...
  \end{proof}
}
\newpage


\newpage
\iffalse
	\begin{corollary} Given a tame  left-linear DPO rewriting system  $(\X,\R)$ with $\X$ an $\mathcal{M}$-adhesive category and a decorated derivation $(\der{D}, \alpha, \omega)$ with $\der{D}=\{\dder{D}_i\}_{i=0}^2$. Then the following are true:
		\begin{enumerate}
			\item suppose that $(a_1,a_2)$ and $(e_0,e_1)$ are independence pairs witnessing  $\dder{D}_1\updownarrow_! \dder{D}_2$ and $\dder{D}_0\updownarrow_! S_{a_1,a_2}(\dder{D}_2)$ respectively, if $\dder{D}_0\updownarrow_!\dder{D}_1$, then $S_{e_0,e_1}(\dder{D}_0)\updownarrow_!S_{a_2,a_2}(\dder{D}_1)$;
			\item
		\end{enumerate}
	\end{corollary}

	\todo{ripulire le ipotesi: serve scambiabilità propria tra 1 e 2}
	\todo{sistemare il fatto che qua utilizziamo derivazioni di 3 passi}
	\begin{proof}
		\begin{enumerate}
			\item Let $(i_1,i_2)$ be the unique independence pair between $\dder{D}_0$ and $\dder{D}_1$.  By tameness and \Cref{lem:indep-global-left} we know that $S_{e_0,e_1}(\dder{D}_0)\updownarrow S_{a_1,a_2}(\dder{D}_1)$. Let $(\alpha_1, \alpha_2)$ be the independence pair constructed in the proof of \Cref{lem:indep-global-left}. Take another independence pair $(\beta_1, \beta_2)$,  by \Cref{rem:weak} we already know that $\beta_1=\alpha_1$. Let also $\der{E}$ be the derivation $S_{e_0, e_1}(S_{a_1, a_2}(\der{D}_2))\cdot S_{e_0, e_1}(\dder{D}_0) \cdot S_{a_1, a_2}(\der{D}_1)$. Let $\sigma\colon [0,2]\to [0,2]$ be the cycle $(0,1,2)$. By \Cref{lem:consperm} we know that $\sigma$ is a consistent permutation between $(\der{D}, \alpha, \omega)$ and $(\der{E}, \alpha, \omega)$. In particular, we have
			      \begin{align*}
				      \iota_{\der{E}, G_0}\circ f_0\circ  i_1 & =\xi_\sigma \circ 	\iota_{\der{D}, G_0}\circ f_0\circ  i_1 \\&= \xi_\sigma \circ 	\iota_{\der{D}, D_0}\circ  i_1\\ &= \xi_\sigma \circ \iota_{\der{D}, G_1} \circ g_0\circ  i_1 \\&=\xi_\sigma \circ \iota_{\der{D}, G_1}\circ m_1\\ &=  \iota_{\der{E}, G'_2}\circ y_1\circ b_1\\&= \iota_{\der{E}, G'_2}\circ z'_1\circ \beta_2\\&= \iota_{\der{E}, Q_5}\circ \beta_2
			      \end{align*}
			      Using	\Cref{rem:pbsalva} we can deduce the existence of a unique arrow $\beta'_2\colon L_1\to P_3$ fitting in the diagram below
			      \[\xymatrix@C=30pt{L_1  \ar@/^.3cm/[drr]^{\beta_2} \ar@/_.3cm/[ddr]_{i_1}\ar@{.>}[dr]^{\beta'_2}\\& P_3 \ar[r]^{n_1}  \ar[d]_{u_0}& Q_5 \ar[d]^{\iota_{\dder{E}, Q_5}}\\
				      &D_0 \ar[r]_{\iota_{\dder{E}, G_0}}& \tpro{E}}\]
			      If we further compute, we get
			      \begin{align*}
				      y_1\circ u_1\circ \beta'_2 & =z'_1\circ n_1\circ \beta'_2= \\&= z'_1\circ \beta_2 \\&=y_1\circ b_1
			      \end{align*}
			      Thus $(u_1\circ \beta'_2, b_2)$ is an independence pair between $S_{a_1, a_2}(\der{D}_2)$ and $S_{a_1, a_2}(\der{D}_1)$ therefore, by hypothesis $u_1\circ \beta'_2=b_1$. But this implies that $\beta'_2$ and $\alpha'_2$ are equal, giving us the thesis.

			      \todo{ripulire le ipotesi: serve scambiabilità propria tra 1 e 2}
			      \todo{sistemare il fatto che qua utilizziamo derivazioni di 3 passi}

			\item \qedhere
		\end{enumerate}
	\end{proof}


	\todo{mettere esempio del perché fallisce nell'altra direziome}
	\todo{anche una breve intro}

	\begin{lemma}\label{lem:iig2}Let $(\X,\R)$ be a left-linear DPO rewriting system with $\X$ an $\mathcal{M}$-adhesive category. For every derivation $\der{D}=\{\dder{D}_i\}_{i=0}^2$, if $(i_0,i_1)$ is a good independence pair between $\dder{D}_0$ and $\dder{D}_1$, $(a_1,a_2)$ one between $\dder{D}_1$ and $\dder{D}_2$ and $\dder{D}_0\updownarrow S_{a_1,a_2}(\dder{D}_2)$ with good independence pair $(e_0,e_1)$, then $S_{e_0,e_1}(\dder{D}_0)$ and $S_{a_1,a_2}(\dder{D}_1)$ are weakly sequentially independent.
	\end{lemma}
	\begin{proof} As in the proof of \Cref{lem:indep-global-left}, \Cref{def:filler,def:switch} yield to us some diagrams.  First of all, let $(v,v')$ be the filler between $\dder{D}_0$ and $\dder{D}_1$ associated to $(i_1, i_2)$, then we have
		\[\xymatrix{&&R_0 \ar@/_1cm/[ddrr]_(.35){i_0}|(.7)\hole \ar[dr]^{h_0}&& L_1\ar@/^1cm/[ddll]^(.35){i_1}  \ar[dl]_{m_1}\\&K_0\ar[dr]^{k_0}\ar[dl]_{l_0} \ar@/_.8cm/[ddrr]|(.36)\hole^(.65){v}\ar[ur]^{r_0}&& G_1 && K_1 \ar@/^.8cm/[ddll]|(.36)\hole_(.65){v'}\ar[dl]_{k_1}\ar[lu]_{l_1} \ar[dr]^{r_1}\\L_0 \ar@/_.8cm/[ddrr]_(.2){j_0}|(.31)\hole|(.81)\hole \ar[dr]^(.4){m_0}|(.61)\hole && D_0 \ar[dl]|(.4)\hole_(.65){f_0}\ar[ur]|(.5)\hole^(.7){g_0}&&D_1 \ar[dr]|(.4)\hole^(.65){g_1} \ar[ul]|(.5)\hole_(.65){f_1}&&R_1\ar@/^.8cm/[ddll]^(.2){j_1}|(.31)\hole|(.81)\hole\ar[dl]_{h_1}\\&G_0 &&P_1\ar[dr]_{q_1} \ar[dl]^{q_0}\ar[ur]^(.4){p_1}\ar[ul]_(.4){p_0}&&G_2\\L_1 \ar@/^.8cm/[uurr]^(.2){i_1} \ar[ur]_(.35){f_0\circ i_1}|(.61)\hole&&Q_0 \ar[ul]|(.4)\hole_(.65){s_0}\ar[dr]|(.5)\hole^(.65){t_0} &&Q_1\ar[ur]|(.4)\hole^(.65){t_1} \ar[dl]|(.5)\hole_(.65){s_1} && R_0  \ar[ul]^(.35){g_1\circ i_0}|(.61)\hole\ar@/_.8cm/[uull]_(.2){i_0}\\&K_1 \ar[ur]_{q_0\circ v'} \ar[dr]_{r_1} \ar[ul]^{l_1}\ar@/^.8cm/[uurr]_(.65){v'}&&G'_1&& K_0 \ar[ur]_{r_0} \ar[ul]^{q_1\circ v} \ar[dl]^{l_0} \ar@/_.8cm/[uull]^(.65){v}\\&& R_1 \ar[ur]_{\hspace{-5pt}s_1\circ j_1}\ar@/^1cm/[uurr]^(.25){j_1}&& L_0 \ar[ul]^{t_0\circ j_0\hspace{-5pt}} \ar@/_1cm/[uull]_(.25){j_0} |(.69)\hole }\]

		Secondly, the filler $(u,u')$ induced by $(a_1, a_2)$ between $\dder{D}_1$ and $\dder{D}_2$ yields:
		\[
			\xymatrix{&&R_1 \ar@/_1cm/[ddrr]_(.35){a_1}|(.7)\hole \ar[dr]^{h_1}&& L_2\ar@/^1cm/[ddll]^(.35){a_2}  \ar[dl]_{m_2}\\&K_1\ar[dr]^{k_1}\ar[dl]_{l_1} \ar@/_.8cm/[ddrr]|(.36)\hole^(.65){u}\ar[ur]^{r_1}&& G_2 && K_2 \ar@/^.8cm/[ddll]|(.36)\hole_(.65){u'}\ar[dl]_{k_2}\ar[lu]_{l_2} \ar[dr]^{r_2}\\L_1 \ar@/_.8cm/[ddrr]_(.2){b_1}|(.31)\hole|(.81)\hole \ar[dr]^(.4){m_1}|(.61)\hole && D_1 \ar[dl]|(.4)\hole_(.65){f_1}\ar[ur]|(.5)\hole^(.7){g_1}&&D_2 \ar[dr]|(.4)\hole^(.65){g_2} \ar[ul]|(.5)\hole_(.65){f_2}&&R_2\ar@/^.8cm/[ddll]^(.2){b_2}|(.31)\hole|(.81)\hole\ar[dl]_{h_2}\\&G_1 &&P_2\ar[dr]_{d_2} \ar[dl]^{d_1}\ar[ur]^(.4){c_2}\ar[ul]_(.4){c_1}&&G_2\\L_2 \ar@/^.8cm/[uurr]^(.2){a_2} \ar[ur]_(.35){f_1\circ a_2}|(.61)\hole&&Q_2 \ar[ul]|(.4)\hole_(.65){x_1}\ar[dr]|(.5)\hole^(.65){y_1} &&Q_3\ar[ur]|(.4)\hole^(.65){y_2} \ar[dl]|(.5)\hole_(.65){x_2} && R_1  \ar[ul]^(.35){g_2\circ a_1}|(.61)\hole\ar@/_.8cm/[uull]_(.2){a_1}\\&K_2 \ar[ur]_{d_1\circ u'} \ar[dr]_{r_2} \ar[ul]^{l_2}\ar@/^.8cm/[uurr]_(.65){u'}&&G'_2&& K_1 \ar[ur]_{r_1} \ar[ul]^{d_2\circ u} \ar[dl]^{l_1} \ar@/_.8cm/[uull]^(.65){u}\\&& R_2 \ar[ur]_{\hspace{-5pt}x_2\circ b_2}\ar@/^1cm/[uurr]^(.25){b_2}&& L_1 \ar[ul]^{y_1\circ b_1\hspace{-5pt}} \ar@/_1cm/[uull]_(.25){b_1} |(.69)\hole }\]

		Finally, the filler $(w,w')$  between $\dder{D}_0$ and $S_{u,u'}(\dder{D}_2)$ given by $(e_0, e_1)$ provides us with:
		\[\xymatrix{&&R_0 \ar@/_1cm/[ddrr]_(.35){e_0}|(.7)\hole \ar[dr]^{h_0}&& L_2\ar@/^1cm/[ddll]^(.35){e_1}  \ar[dl]_{f_1\circ a_2}\\&K_0\ar[dr]^{k_0}\ar[dl]_{l_0} \ar@/_.8cm/[ddrr]|(.36)\hole^(.65){w}\ar[ur]^{r_0}&& G_1 && K_2 \ar@/^.8cm/[ddll]|(.36)\hole_(.65){w'}\ar[dl]_{d_1\circ u'\hspace{-5pt}}\ar[lu]_{l_2} \ar[dr]^{r_2}\\L_0 \ar@/_.8cm/[ddrr]_(.2){o_0}|(.31)\hole|(.81)\hole \ar[dr]^(.4){m_0}|(.61)\hole && D_0 \ar[dl]|(.4)\hole_(.65){f_0}\ar[ur]|(.5)\hole^(.7){g_0}&&Q_2 \ar[dr]|(.4)\hole^(.65){y_1} \ar[ul]|(.5)\hole_(.65){x_1}&&R_2\ar@/^.8cm/[ddll]^(.2){o_1}|(.31)\hole|(.81)\hole\ar[dl]_(.35){x_2\circ b_2}\\&G_0 &&P_3\ar[dr]_{n_1} \ar[dl]^{n_0}\ar[ur]^(.4){u_1}\ar[ul]_(.4){u_0}&&G'_2\\L_2 \ar@/^.8cm/[uurr]^(.2){e_1} \ar[ur]_(.35){f_0\circ e_1}|(.61)\hole&&Q_4 \ar[ul]|(.4)\hole_(.65){z_0}\ar[dr]|(.5)\hole^(.65){z'_0} &&Q_5\ar[ur]|(.4)\hole^(.65){z'_1} \ar[dl]|(.5)\hole_(.65){z_1} && R_0  \ar[ul]^(.35){y_1\circ e_0}|(.61)\hole\ar@/_.8cm/[uull]_(.2){e_0}\\&K_2 \ar[ur]_{n_0\circ w'} \ar[dr]_{r_2} \ar[ul]^{l_2}\ar@/^.8cm/[uurr]_(.65){w'}&&G'_1&& K_0 \ar[ur]_{r_0} \ar[ul]^{n_1\circ w} \ar[dl]^{l_0} \ar@/_.8cm/[uull]^(.65){w}\\&& R_2 \ar[ur]_{\hspace{-4pt}z_1\circ o_1}\ar@/^1cm/[uurr]^(.25){o_1}&& L_0 \ar[ul]^{z'_0\circ o_0\hspace{-5pt}} \ar@/_1cm/[uull]_(.25){o_0} |(.69)\hole }\]

		We have to construct the two dotted arrows in the diagram below.
		\[\xymatrix@C=15pt{L_0 \ar[d]_{z'_0\circ o_0}&& K_0 \ar[d]_{n_1\circ w}\ar[ll]_{l_0} \ar[r]^{r_0} & R_0 \ar@{.>}@/^.35cm/[drrr]_(.4){\alpha_1}|(.285)\hole \ar[dr]|(.28)\hole_{y_1\circ e_0} && L_1 \ar@{.>}@/_.35cm/[dlll]^(.4){\alpha_2} \ar[dl]|(.28)\hole^{y_1\circ b_1}& K_1 \ar[d]^{d_2\circ u}\ar[l]_{l_1} \ar[rr]^{r_1} && R_1 \ar[d]^{g_2\circ a_1}\\G'_1 && \ar[ll]^{z_1} Q_5 \ar[rr]_{z'_1}&& G'_2  && \ar[ll]^{x_2} Q_3 \ar[rr]_{y_2}&& G_2}\]

		Consider the arrows $i_0\colon R_0\to D_1$ and $e_0\colon R_0\to Q_2$. An easy computation shows that
		\begin{align*}
			f_1\circ i_0 & = h_0 \\&=x_1\circ e_0
		\end{align*}
		entailing the existence of  the dotted $\alpha'_1\colon R_0\to P_2$ in the diagram
		\[\xymatrix{R_0 \ar@{.>}[dr]^{\alpha'_1} \ar@/^.3cm/[drr]^{i_0} \ar@/_.3cm/[ddr]_{e_0}\\ &P_2 \ar[r]^{c_2} \ar[d]_{d_1}& D_1 \ar[d]^{f_1}\\ &Q_2\ar[r]_{x_1} & G_1}\]
		If we define $\alpha_1\colon R_0\to Q_3$ as $d_2\circ \alpha'_1$, then we easily get that
		\begin{align*}
			x_2\circ \alpha_1 & =x_2\circ d_2\circ \alpha'_1 \\&= y_1\circ d_1\circ \alpha'_1\\&=y_1\circ e_0
		\end{align*}

		For $\alpha_2$, we proceed similarly. First consider  $i_1\colon L_1\to D_0$ and $b_1\colon L_1 \to Q_2$ and notice that
		\begin{align*}
			g_0\circ i_1 & = m_1 \\&= x_1 \circ b_1
		\end{align*}
		implying the existence of $\alpha'_2\colon L_1\to P_3$ fitting in the diagram below.
		\[\xymatrix{L_1 \ar@{.>}[dr]^{\alpha'_2} \ar@/^.3cm/[drr]^{i_1} \ar@/_.3cm/[ddr]_{b_1}\\ &P_3 \ar[r]^{u_0} \ar[d]_{u_1}& D_0 \ar[d]^{g_0}\\ &Q_2\ar[r]_{x_1} & G_1}\]
		Let $\alpha_2\colon L_1\to Q_5$ be $n_1\circ \alpha'_2$, then
		\begin{align*}
			z'_1 \circ \alpha_2 & = z'_1 \circ n_1\circ \alpha'_2 \\&=y_1\circ u_1\circ \alpha'_2\\&=y_1\circ b_1
		\end{align*}
		The thesis now follows.
	\end{proof}
\fi





\section{Conclusions and further work}
\todo{VERY NICE CONCLUSIONS}

\bibliography{bibliog.bib}


\appendix
\section{Properties of $\mathcal{M}$-adhesive categories}\label{app:ade}

This first appendix is devoted to the proofs of some well-known results about $\mathcal{M}$-adhesive categories.

Observe that our notion of $\mathcal{M}$-adhesivity follows
\cite{ehrig2012,ehrig2014adhesive} and is different from the one of
\cite{azzi2019essence}. What is called $\mathcal{M}$-adhesivity in
that paper corresponds to our strict
$\mathcal{M}$-adhesivity. Moreover, in \cite{azzi2019essence} the
class $\mathcal{M}$ is assumed to be only stable under
pullbacks. However, if $\mathcal{M}$ contains all split monos, then
stability under pushouts can be deduced from the other
axioms~\cite[Prop.~$5.1.21$]{castelnovo2023thesis}.



\subsection{Some results on $\mathcal{A}$-stable and $\mathcal{A}$-Van Kampen squares}
We will start proving some general results regarding $\mathcal{A}$-Van Kampen and $\mathcal{A}$-stable squares. Let us begin recalling some classical results about  about pullbacks and pushouts

\begin{lemma}\label{lem:popb1} \label{lem:pb1}
	Let $\X$ be a category, and consider the following diagram 	in which the right square is a pullback.
	\[\xymatrix{X \ar[d]_{a} \ar[r]^{f}& \ar[r]^{g} Y \ar[d]^{b}& Z \ar[d]^{c}\\ A \ar[r]_{h}& B \ar[r]_{k}& C}\]
	Then the whole rectangle is a pullback if and only if the left square is one.
\end{lemma}

\begin{lemma}\label{lem:po1}
	Let $\X$ be a category, and consider the following diagram 	in which the left square is a pushout.
	\[\xymatrix{X \ar[d]_{a} \ar[r]^{f}& \ar[r]^{g} Y \ar[d]^{b}& Z \ar[d]^{c}\\ A \ar[r]_{h}& B \ar[r]_{k}& C}\]
	Then the whole rectangle is a pushout if and only if the right square is one.
\end{lemma}


The following proposition establishes a key property of $\mathcal{A}$-Van Kampen squares with a mono as a side: they are not only pushouts, but also pullbacks.

\begin{proposition}\label{prop:pbpo} Let $\mathcal{A}$ be a class of arrows stable under pushouts and containing all the isomorphisms.  If $m\colon A\to C$ is mono and belongs to $\mathcal{A}$, then every $\mathcal{A}$-Van Kampen square
	\[\xymatrix{A\ar[r]^{g} \ar[d]_{m} & B \ar[d]^{n} \\ C \ar[r]_{f}  & D}\]
	is also a pullback square and $n$ is a monomorphism.
\end{proposition}

\begin{proof}
	We can start considering the cube below and noticing that $n$, being the pushout of $m\in \mathcal{M}$, belongs to $\mathcal{A}$ too.
	\[\xymatrix@C=13pt@R=13pt{&A\ar[dd]|\hole_(.65){\id{A}}\ar[rr]^{g} \ar[dl]_{\id{A}} && B \ar[dd]^{\id{B}} \ar[dl]_{\id{B}} \\ A  \ar[dd]_{m}\ar[rr]^(.65){g} & & B \ar[dd]_(.3){n}\\&A\ar[rr]|\hole^(.65){g} \ar[dl]^{m} && B \ar[dl]^{n} \\C \ar[rr]_{f} & & D}\]
	By construction the top face of the cube is a pushout and the back one a pullback. The left face is a pullback because $m$ is mono. Thus the $\mathcal{A}$-Van Kampen property yields that the front and the right faces are pullbacks.
\end{proof}

The previous proposition, now, allows us to establish the following result.
\begin{lemma}\label{lem:varie}Let $\mathcal{A}$ be a class of arrows stable under pullbacks, pushouts and containing all isomorphisms.  Suppose that, the left square below is $\mathcal{A}$-Van Kampen, while the vertical faces in the right cube are pullbacks.
	\[\xymatrix@C=10pt@R=10pt{&&&&&A'\ar[dd]|\hole_(.65){a}\ar[rr]^{g'} \ar[dl]_{m'} && B' \ar[dd]^{b} \ar[dl]_{n'} \\ A \ar[dd]_{m}\ar[rr]^{g}&&B\ar[dd]^{n}&&C'  \ar[dd]_{c}\ar[rr]^(.7){f'} & & D' \ar[dd]_(.3){d}\\&&&&&A\ar[rr]|\hole^(.65){g} \ar[dl]^{m} && B \ar[dl]^{n} \\C\ar[rr]_{f} &&D&&C \ar[rr]_{f} & & D}\]
	Supppose, moreover that $m\colon A\to C$ and $d\colon D'\to D$ are mono and that $d$ belongs to $\mathcal{A}$. Then $d\leq n$ if and only if $c \leq m$.
\end{lemma}


\begin{remark}
	Recall that, given two monos $m:M\to X$ and $n:N\to X$ with the same codomain, $m\leq n$ means that there exists a, necessarily unique and necessarily mono, $k:M\to N$ fitting in the triangle below:
	\[\xymatrix{M\ar[rr]^{k}  \ar[dr]_{m}&& N \ar[dl]^{n}\\ & X}\]
	Notice, moreover, that  if $m\leq n$ and $n\leq m$, then the arrow $k:M\to N$ is an isomorphism.
\end{remark}

\begin{remark}
	Notice that, since $d$ is a mono and the vertical faces are pullbacks, then $a, b$ and $c$ are monomorphisms too. Moreover, $n$ is mono by \Cref{prop:pbpo}, so that even $m'$ and $n'$ are monos.
\end{remark}

\begin{proof}
	$(\Rightarrow)$ By hypothesis there exists $k:D'\to B$ such that $n\circ k = d$. By \Cref{prop:pbpo}, the bottom face of the cube is a pullback. Thus there exists a unique $h:C'\to A$ as in the diagram below, implying the thesis.
	\[\xymatrix@C=40pt{C'  \ar[d]^{h}\ar[r]^{f'} \ar@/_.5cm/[dd]_{c}& D' \ar@/^.5cm/[dd]^{d}\ar[d]_{k}\\ A \ar[d]^{m} \ar[r]^{g} & B \ar[d]_{n} \\  C \ar[r]_f & D}\]

	\smallskip \noindent
	$(\Leftarrow)$ Let $h:C\to A$ be such that $c=m\circ h$. By the $\mathcal{A}$-Van Kampen property the top face of the given cube is a pushout. Thus the dotted $k:D'\to B$ in the following diagram exists.
	\[\xymatrix@C=40pt{A' \ar@/_.5cm/[dd]_{a} \ar[d]^{m'} \ar[r]^{g'} & B' \ar[d]_{n'} \ar@/^.5cm/[dd]^{b} \\  C' \ar[d]^{h} \ar[r]_{f'} & D' \ar@{.>}[d]_{k}\\ A \ar[r]_g& B }\]
	Moreover, by construction we have
	\begin{align*}
		n\circ k \circ n' & = n\circ b=d\circ n'                                  \\
		n\circ k \circ f' & = n\circ g\circ h=f\circ m\circ h=f\circ c= d\circ f'
	\end{align*}
	We can therefore conclude that $n\circ k =d$.
\end{proof}

Finally, we can show that $\mathcal{A}$-stable pushouts enjoy a kind of \emph{pullback-pushout decomposition} property.

\begin{proposition}\label{prop:stab}Let $\X$ be a category and $\mathcal{A}$ a class of arrows stable under pullbacks. Suppose that, in the diagram below, the whole rectangle is an $\mathcal{A}$-stable pushout and the right square a pullback.
	\[\xymatrix{X \ar[d]_{a} \ar[r]^{f}& \ar[r]^{g} Y \ar[d]^{b}& Z \ar[d]^{c}\\ A \ar[r]_{h}& B \ar[r]_{k}& C}\]
	If the arrow $k$ is in $\mathcal{A}$ and it is a monomorphism,  then both squares are pushouts.
\end{proposition}

\begin{proof}
	We can begin noticing that $g$, being the pullback of $k$, is mono and in $\mathcal{A}$ too. Thus we can build the cube below, in which all the vertical faces are pullbacks, entailing that all the vertical arrows are in $\mathcal{A}$.
	\[\xymatrix@C=20pt@R=10pt{ & &X\ar[dl]_{f} \ar[ddd]^(.333333){\id{X}}|(.666666)\hole\ar[rrr]^{a} &&&A \ar[ddd]^{\id{A}} \ar[dl]^{h}\\& Y\ar[dl]_{\id{Y}} \ar[ddd]|(.333333)\hole_{\id{Y}} &&&B \ar[ddd]^{\id{B}} \ar[dl]_{\id{B}} \\Y \ar[ddd]_{g} \ar[rrr]^{b}&&& B\ar[ddd]^{k}\\&&X \ar[rrr]|(.34)\hole^{a}|(.67)\hole \ar[dl]_{f}&&& A \ar[dl]^{h}\\ & Y  \ar[dl]_{g}&&& B \ar[dl]^{k}\\ Z\ar[rrr]_{c} &&& C}\]
	By hypothesis the face is an $\mathcal{A}$-stable pushout and so its top one is a pushout. Using \Cref{lem:po1} we can conclude that the right half of the rectangle with which we have started is a pushout too.
\end{proof}

\subsection{Useful properties of $\mathcal{M}$-adhesive categories}

We can begin with an observation about decompo

\begin{proposition}
  \label{prop:deco}Let $\mathcal{A}$ be a class of arrows stable under
  pullbacks. For every arrow $f\colon X\to Y$ and monomorphism
  $m\colon Y\to Z$, if $m\circ f \in\mathcal{A}$ then
  $f\in \mathcal{A}$.
\end{proposition}
\begin{proof}
  Take the diagram
  \[\xymatrix{X \ar[d]_{\id{X}}\ar[r]^{f}& Y \ar[r]^{\id{Y}}  \ar[d]_{\id{Y}}& Y \ar[d]^{m}\\
		X \ar[r]_{f}& Y \ar[r]_{m} & Z}\]
	Since $m$ is mono the right square is a pullback, while the left square is a pullback by construction. By \Cref{lem:pb1} the whole rectangle is a pullback and the thesis follows.
\end{proof}
\begin{corollary}
  \label{cor:deco}
	In every $\mathcal{M}$-adhesive category $\X$, the class $\mathcal{M}$ is closed under decomposition.
\end{corollary}

Another result which can be immediately established, with the aid of \Cref{prop:pbpo}, is the following one.

\begin{proposition}
	\label{prop:pbpoad}
	Let $\X$ be an $\mathcal{M}$-adhesive category. Then
	$\mathcal{M}$-pushouts are pullbacks.
\end{proposition}

From \Cref{prop:pbpoad}, in turn, we can derive the following corollaries.
\begin{corollary}\label{cor:rego}
	In a $\mathcal{M}$-adhesive category $\X$, every $m\in\mathcal{M}$ is a regular mono.
\end{corollary}
\begin{proof}
	Let $m$ be an element of $\mathcal{M}$ and consider its pushout along itself.
	\[\xymatrix{X\ar@{>->}[r]^m \ar@{>->}[d]_m& Y\ar@{>->}[d]^{f}\\Y \ar@{>->}[r]_g & Z}\]
	By \Cref{prop:pbpoad} this square is a pullback, proving that $m$ is the equalizer of the arrows $f,g\colon Y\rightrightarrows Z$.
\end{proof}

The following result now follows at once noticing that a regular monomorphism which is also epic is automatically an isomorphism.

\begin{corollary}\label{prop:bal}
	If $\X$ is an $\mathcal{M}$-adhesive categories, then every epimorphisms in $\mathcal{M}$ is an isomorphisms. In particular, every adhesive category $\X$ is \emph{balanced}: if a morphism is monic and epic, then it is an isomorphism.
\end{corollary}

\begin{lemma}[$\mathcal{M}$-pushout-pullback decomposition]\label{lem:popb} Let $\X$ be an $\mathcal{M}$-adhesive category  and suppose that, in the diagram below, the whole rectangle is a pushout and the right square a pullback.
	\[\xymatrix{X \ar[d]_{a} \ar[r]^{f}& \ar[r]^{g} Y \ar[d]^{b}& Z \ar[d]^{c}\\ A \ar[r]_{h}& B \ar[r]_{k}& C}\]
	Then the following statements hold true:
	\begin{enumerate}
		\item if $a$ belongs to $\mathcal{M}$ and $k$ is a monomorphism,  then both squares are pushouts and pullbacks;
		\item if $f$ and $k $ are in  $\mathcal{M}$, then both squares are pushouts and pullbacks.
	\end{enumerate}
\end{lemma}
\begin{proof}		\begin{enumerate}
		\item By \Cref{prop:stab}, it follows that both squares are pushouts, thus the thesis follows from \Cref{prop:pbpoad}.
		\item By hypothesis, $g$ is the pullback of an arrow in $\mathcal{M}$, thus it belongs to it. But then $g\circ f\in \mathcal{M}$ too  and the whole rectangle is a $\mathcal{M}$-pushout. Therefore, by \Cref{prop:pbpoad} a pullback, so that its left half is a pullback too, by \Cref{prop:pbpo}. Moreover $k\circ h$ is in $\mathcal{M}$ as the pushout of $g\circ f$ and, by \Cref{cor:deco}, we also know that $h\in \mathcal{M}$.

		      Using \Cref{lem:po1}, it is enough to show that the left half of the original rectangle is a pushout. We can build the following cube:
		      \[\xymatrix@C=20pt@R=10pt{ & &X\ar@{>->}[dl]_{f} \ar[ddd]^(.333333){\id{X}}|(.666666)\hole\ar[rrr]^{a} &&&A \ar[ddd]^{\id{A}} \ar@{>->}[dl]^{h}\\& Y\ar[dl]_{\id{Y}} \ar[ddd]|(.333333)\hole_{\id{Y}} &&&B \ar[ddd]^{\id{B}} \ar[dl]_{\id{B}} \\Y \ar@{>->}[ddd]_{g} \ar[rrr]^{b}&&& B\ar@{>->}[ddd]^{k}\\&&X \ar[rrr]|(.34)\hole^{a}|(.67)\hole \ar@{>->}[dl]_{f}&&& A \ar@{>->}[dl]^{h}\\ & Y \ar[rrr]|(.67)\hole^{b} \ar@{>->}[dl]_{g}&&& B \ar@{>->}[dl]^{k}\\ Z\ar[rrr]_{c} &&& C}\]
		      Its vertical faces are all pullbacks and all the vertical arrows are in $\mathcal{M}$, hence the top face is a pushout and we can conclude. \qedhere
	\end{enumerate}
\end{proof}



Let us turn our attention to pushout complements.

\noindent
\parbox{11.7cm}{
	\begin{definition}[Pushout complement]
		Let $f\colon X\to Y$ and $g\colon Y\to Z$ be two composable arrows in a category $\X$. A \emph{pushout complement} for the pair $(f,g)$ is a pair $(h,k)$ with $h\colon X\to W$ and $k\colon W\to Z$ such that the square below commutes and it is a pushout.
	\end{definition}} \quad
\parbox{4cm}{\vspace{-.15cm}
	$\xymatrix{X \ar[r]^{f} \ar[d]_{h}& Y \ar[d]^{g} \\ W \ar[r]_{k}& Z}$}

\iffalse
	\begin{example}
		In a generic category $\X$, pushout complements may not exist: in $\Set$ the arrows $?_{2}\colon \emptyset \to 2$ and $!_2\colon 2\to 1$ do not have a pushout complement.

		Moreover, composable arrows $f\colon X\to Y$ and $g\colon Y\to Z$ may have  pushout complements which are non-isomorphic: for instance, in $\Set$ the two squares below are both pushouts.

		\[\xymatrix{2 \ar[r]^{!_2} \ar[d]_{\id{2}}& 1 \ar[d]^{\id{1}} & 2 \ar[r]^{!_2} \ar[d]_{!_2}& 1 \ar[d]^{\id{1}}\\ 2 \ar[r]_{!_2}& 1 & 1 \ar[r]_{\id{1}}& 1}\]
	\end{example}
\fi

Working in an $\mathcal{M}$-adhesive category we can amend the second defect.

\noindent
\parbox{9cm}{
	\begin{lemma}\label{lem:radj}
		Let $f\colon X \to Y$ be an arrow in an $\mathcal{M}$-adhesive category $\X$ and suppose that the left square aside is a pushout while the  right one is a pullback, with $ m\colon M \rightarrowtail X$ and $n\colon N \rightarrowtail Y$ in $\mathcal{M}$.
		Then $n\leq k$ if and only if $p_2\leq m$.
	\end{lemma}}
\parbox{3cm}{\vspace{-0ex}
	$\xymatrix{M \ar@{>->}[d]_{m}\ar[r]^{h}& Q \ar@{>->}[d]^{k} & P \ar[r]^{p_1} \ar@{>->}[d]_{p_2}& N \ar@{>->}[d]^{n}\\X\ar[r]_{f}&Y &X \ar[r]_f&Y}$}

\begin{proof}
	We start pulling back $n$ along $k$ and then pulling back the resulting arrow along $h$, getting the following two pullbacks square:
	\[\xymatrix{B \ar@{>->}[d]_{b}\ar@{>->}[r]^{s}& N \ar@{>->}[d]^{n} & A \ar[r]^{r} \ar@{>->}[d]_{a}& B \ar@{>->}[d]^{b}\\Q\ar@{>->}[r]_{k}&Y &M \ar[r]_h&Q\\}\]
	Notice that
	\[
		f\circ m\circ a=k\circ h\circ a=k\circ b\circ r=n\circ s\circ r\]

	Thus there exists a unique $t\colon A\to P$ as in the diagram
	\[\xymatrix{A \ar@{>->}[d]_{a} \ar@{.>}[dr]^{t}\ar[r]^{r} &B \ar@{>->}@/^.2cm/[dr]^{s} \\ M \ar@{>->}@/_.2cm/[dr]_{m}&P\ar[r]^{p_1}  \ar@{>->}[d]_{p_2}& N \ar@{>->}[d]^{n}\\ &X \ar[r]_{f} & Y}\]
	We can then consider the cube below, in which the front, right and back faces are pullbacks.
	\[\xymatrix@C=15pt@R=15pt{&A\ar@{>->}[dd]|\hole_(.65){a}\ar[rr]^{r} \ar[dl]_{t} && B \ar@{>->}[dd]^{b} \ar[dl]_{s} \\ P  \ar@{>->}[dd]_{p_2}\ar[rr]^(.7){p_1} & & N \ar@{>->}[dd]_(.3){n}\\&M\ar[rr]|\hole^(.65){h} \ar@{>->}[dl]^{m} && Q \ar@{>->}[dl]^{k} \\X \ar[rr]_{f} & & Y}\]
	We can notice that the left square is a pullback too. To see this it is enough to apply \Cref{lem:pb1} to the diagram
	\[\xymatrix{A  \ar@/^.4cm/[rr]^{s\circ r}\ar@{>->}[d]_{a} \ar[r]_{t}& \ar[r]_{p_1} P \ar@{>->}[d]^{p_2}& N \ar[d]^{c}\\ M \ar@/_.4cm/[rr]_{k\circ h}\ar@{>->}[r]^{m}& X \ar[r]^{f}& Y}\]

	The thesis now follows immediately from \Cref{lem:varie}. \qedhere
\end{proof}

\noindent
\parbox{10cm}{
	\begin{corollary}[Uniqueness of pushouts complements]\label{lem:pocomp}
		Let $\X$ be a $\mathcal{M}$-adhesive category. Given $m\colon X\mto Y$ in $\mathcal{M}$ and $f\colon Y\to Z$, let $(h_1, k_1)$ and $(h_2, k_2)$ be the pushout complements of $m$  and $f$ depicted aside. Then there exists a unique isomorphism $\phi\colon W_1\to W_2$ making the diagram on the right commutative.
	\end{corollary}}
\parbox{3cm}{$\xymatrix{&X \ar@{>->}[r]^{m} \ar[d]_{h_1} \ar@/_.3cm/[dl]_{h_2}& Y \ar[d]^{f} \\ W_2 \ar@{>->}@/_.4cm/[rr]_{k_2} &W_1 \ar@{.>}[l]_{\phi} \ar@{>->}[r]^{k_1} & Z }$}

\begin{proof}
	Consider the two pushout squares
	\[\xymatrix{M \ar[r]^{h_1}\ar@{>->}[d]_{m}& W_1 \ar@{>->}[d]^{k_1} & M \ar@{>->}[d]_{m} \ar[r]^{h_1}& W_2 \ar@{>->}[d]^{k_2}\\Y\ar[r]_{f}&Z &Y \ar[r]_{f}&Z}\]
	Since $\mathcal{M}$ is in $\mathcal{M}$, \cref{prop:pbpoad} guarantees that they are both pullbacks.
	Since $m\leq m$, by \Cref{lem:radj} we get that now entails that $k_1\leq k_2$ and $k_2\leq k_1$. Thus there exists an isomorphism $\phi\colon W_1\to W_2$ such that $k_1=k_2\circ \phi$. To see that $h_2=\phi\circ h_1$, we can compute:
	\[
		k_2\circ \phi \circ h_1  = k_1\circ h_1= n\circ m= k_2\circ h_2\]

	The claim now follows since $k_2$ is a monomorphism.
\end{proof}



\section{$\mathcal{M}$-adhesivity is not enough}\label{app:fill}
In \Cref{sec:ade} we introduced the notion of tame left-linear DPO rewriting system, a class that contains, for instance, all the linear rewriting systems. This result can be further refined: in a\cite{baldan2011adhesivity} a class  $\mathbb{B}$ of (quasi)adhesive category is defined for which the local Church-Rosser Theorem holds for left-linear DPO rewriting system. In our language, this means that every left-linear DPO rewriting system based on a category in $\mathbb{B}$ is tame. This section is devoted to repropose, and slightly generalize, the results of \cite{baldan2011adhesivity} to our context.

\noindent
\parbox{10.4cm}{
	\begin{definition}Let $\mathcal{M}$ be a class of monos in a category $\X$, closed under composition, containing all isomorphisms and stable under pullbacks and pushouts. Suppose that the  diagram on the right 	is given and the whole rectangle is a pushout. We say that $\X$ satisfies:
		\parbox{14cm}{\begin{itemize}
				\item the \emph{$\mathcal{M}$-mixed decomposition} property if, whenever $k$ belongs to $\mathcal{M}$ and the right half of the previous diagram is a pullback, then the left one is a pushout.
				\item the \emph{$\mathcal{M}$-pushout decomposition} property if whenever $a, b$ and $c$ belongs to $\mathcal{M}$ and the right half of the diagram above is a pushout, then its left half is a pushout too.
			\end{itemize}}
	\end{definition}} \parbox{3cm}{\vspace{-2cm}$\xymatrix{X \ar[d]_{a} \ar[r]^{f}& \ar[r]^{g} Y \ar[d]^{b}& Z \ar[d]^{c}\\ A \ar[r]_{h}& B \ar[r]_{k}& C}$}\\

The class of categories of type $\mathbb{B}$ is closed under the same constructions of \Cref{cor:slice,thm:functors}. This can be deduced at once from the fact that in such categories pullbacks and pushouts are computed componentwise.

\begin{lemma}\label{lem:closed} Let $\X$ be a category satisfying the $\mathcal{M}$-mixed and $\mathcal{M}$-pushout decomposition properties, then the following hold true:
	\begin{enumerate}
		\item  for every object $X$, $\X/X$ satisfies  the $\mathcal{M}/X$-mixed and the $\mathcal{M}/X$-pushout decomposition properties, while $X/\mathcal{M}$-adhesive satisfies their $X/\mathcal{M}$ variants, where
		\[\mathcal{M}/X:=\{m\in \mathcal{A}(\X/X) \mid m\in \mathcal{M} \}
		\qquad X/\mathcal{M}:=\{m\in \mathcal{A}(X/\X) \mid m\in
		\mathcal{M} \}
		\]
		\item  for every small
		category $\Y$, the category $\X^\Y$ satisfies the 
		$\mathcal{M}^{\Y}$-mixed and the $\mathcal{M}^{\Y}$-pushout decomposition properties , where
		\[\mathcal{M}^{\Y}:=\{\eta \in \mathcal{A}(\X^\Y) \mid \eta_Y \in
		\mathcal{M} \text{ for every } Y\in \Y\}\]
	\end{enumerate}
\end{lemma}

The following result shows that the mixed and pushout decomposition properties guarantee that every independence pair is good.

\begin{theorem}\label{thm:good1}Let $\X$ be an $\mathcal{M}$-adhesive category with all pushouts and satisfying the $\mathcal{M}$-mixed and the $\mathcal{M}$-pushout decomposition properties. Then every left-linear DPO rewriting system on $\X$ is tame.
\end{theorem}
\begin{proof}
	Consider the derivation $\der{D}=\{\dder{D}_i\}_{i=0}^1$, made by two weakly sequentially independent derivations, depicted below.
	\[\xymatrix@C=15pt{L_0 \ar[d]_{m_0}&& K_0 \ar[d]_{k_0}\ar@{>->}[ll]_{l_0} \ar[r]^{r_0} & R_0 \ar@/^.35cm/[drrr]|(.3)\hole_(.4){i_0} \ar[dr]|(.3)\hole_{h_0} && L_1 \ar@/_.35cm/[dlll]^(.4){i_1} \ar[dl]|(.3)\hole^{m_1}& K_1 \ar[d]^{k_1}\ar@{>->}[l]_{l_1} \ar[rr]^{r_1} && R_1 \ar[d]^{h_1} \\G_0 && \ar@{>->}[ll]^{f_0} D_0 \ar[rr]_{g_0}&& G_1  && \ar@{>->}[ll]^{f_1} D_1 \ar[rr]_{g_1}&& G_2 }\]

	We have to show that $(i_0, i_1)$ is a good independence pair. Now, by hypothesis $\X$ has all pushouts, thus the only thing to show is that the squares below are pushouts (the third one is the usual pullback of $f_1\colon D_1\mto G_0$ along $g_0\colon D_0\to G_1$).
	\[\xymatrix{K_0 \ar[r]^{r_0}  \ar[d]_{u_0}& R_0 \ar[d]^{i_0} & K_1 \ar@{>->}[r]^{l_1}  \ar[d]_{u_1}& L_1 \ar[d]^{i_1}&P \ar@{>->}[r]^{p_0} \ar[d]_{p_1}& D_0\ar[d]^{g_0} \\ P \ar[r]_{p_1} & D_1 &P \ar[r]_{p_0}  & D_0&D_1 \ar@{>->}[r]_{f_1} & G_1}\]

	To see this, consider the following two diagrams.
	\[\xymatrix{K_0 \ar@/^.4cm/[rr]^{k_0} \ar[d]_{r_0} \ar[r]_{u_0}& \ar@{>->}[r]_{p_0} P \ar[d]_{p_1}& D_0 \ar[d]^{g_0}&K_1 \ar@/^.4cm/[rr]^{k_1} \ar@{>->}[d]_{l_1} \ar[r]_{u_1}& \ar[r]_{p_1} P \ar@{>->}[d]_{p_0}& D_1 \ar@{>->}[d]^{f_1}\\ R_0 \ar@/_.4cm/[rr]_{h_0}  \ar[r]^{i_0}& D_1 \ar@{>->}[r]^{f_1}& G_1 &L_1 \ar@/_.4cm/[rr]_{m_1}  \ar[r]^{i_1}& D_0 \ar[r]^{g_0}& G_1}\]

	The thesis now follows from the $\mathcal{M}$-mixed and the $\mathcal{M}$-pushout decomposition property.
\end{proof}

Our next step is to identify sufficient conditions for a category $\X$ to satisfy the mixed and $\mathcal{M}$-pushout decomposition properties.

\begin{definition} 
	Let $\X$ be an  $\mathcal{M}$-adhesive category is of \emph{type $\mathbb{B}$} if:
	\begin{enumerate}
		\item every arrow in $\mathcal{M}$ is a coproduct coprojection;
		\item $\X$ has all pushouts;
		\item $\X$ has strict initial objects and, for every object $X$, the arrow $?_X\colon 0\to X $ belongs to $\mathcal{M}$;
		\item all pushouts are $\mathcal{M}$-stable.
	\end{enumerate}
\end{definition}

\begin{remark}
	It is worth to examine more closely conditions $1$ and $3$ of the above definition.
	\smallskip 
	\begin{itemize}
	\parbox{10.3cm}{\item Let $m_0\colon X_0 \rightarrowtail Y$ be an arrow in $\mathcal{M}$, the first condition means that there exists $m_1\colon X_1\to Y$ such that $(Y, \{m_i\}_{i=0}^1)$ is a coproduct. This, together with property $3$, entails that every coprojection in a coproduct is in $\mathcal{M}$. This follows since, any coproduct $(X_1+X_2, \{\iota_{X_i}\}_{i=0}^1)$ fits in a pushout diagram as the one aside.}
	  \parbox{3cm}{\vspace{-0em}$\xymatrix@C=25pt{0  \ar@{>->}[r]^-{?_{X_1}} \ar@{>->}[d]^{?_{X_2}}& X_1 \ar@{>->}[d]^{\iota_{X_1}}\\ X_2 \ar@{>->}[r]_-{\iota_{X_2}} & X_1+X_2}$}
		
	\smallskip 	\noindent 
	\parbox{13.5cm}{\item $\X$ has strict initial objects if it has initial objects and every arrow $f:X\to 0$ is an isomorphism. Notice that if initial objects are strict, then $?_X\colon 0\to X$ is mono for every $X$: indeed for every pair $f,g\colon Y\rightrightarrows 0$ then, by strictness, $Y$ is initial and so $f=g$.}
	\end{itemize} 
\end{remark}

\begin{example}
The category $\Set$ of sets and functions, with its class of monos, is of type $\mathbb{B}$. Similarly, the category $\textbf{Inj}$ of sets and injective functions is quasiadhesive and, with its class of regular monos, of type $\mathbb{B}$. 
\end{example}

Categories of type $\mathbb{B}$ satisfy a property resembling \emph{extensivity} \cite{carboni1993introduction}.

\noindent
\parbox{10cm}{\begin{proposition}\label{prop:ext}
Let $\X$ be an $\mathcal{M}$-adhesive category of type $\mathbb{B}$. Then for every diagram as the one aside, in which the bottom row is a coproduct cocone and the vertical arrows are in $\mathcal{M}$, the top row is a coproduct if and only if the two squares are pullbacks.
\end{proposition}}
\parbox{2cm}{\vspace{-.4em}$\xymatrix{A\ar[r]^-{f} \ar@{>->}[d]_{r}& C  \ar@{>->}[d]_{s}& B \ar@{>->}[d]^{t}\ar[l]_-{g}\\X  \ar[r]_-{\iota_X}& X+Y & Y \ar[l]^-{\iota_Y}}$}

\noindent 
\parbox{4cm}{$\xymatrix@C=10pt@R=10pt{&I\ar@{>->}[dd]|\hole_(.65){a}\ar[rr]^{k} \ar[dl]_{h} && B \ar@{>->}[dd]^{t} \ar[dl]_{g} \\ A  \ar@{>->}[dd]_{r}\ar[rr]^(.7){f} & & C \ar@{>->}[dd]_(.3){s}\\&0\ar@{>->}[rr]|\hole^(.65){?_Y} \ar@{>->}[dl]^{?_X} && Y \ar@{>->}[dl]^{\iota_Y} \\X \ar@{>->}[rr]_{\iota_X} & & X+Y}$}
 \parbox{10cm}{\begin{proof}
Consider the cube aside, in which the back faces are pullbacks. The bottom faces is an $\mathcal{M}$-Van Kampen pushout, thus the top face is a pushout if and only if the front faces are pullbacks. By strictness of $0$, $a\colon I\to 0$ is an isomorphism, so that $I$ is initial, therefore the $\mathcal{M}$-Van Kampen condition reduces to the request that $(C, \{f,g\})$ is a coproduct cocone if and only if the front faces are pullbacks, as claimed.
\end{proof}}

The previous result entails the following one, needed to show that  an $\mathcal{M}$-adhesive category of type $\mathbb{B}$ satisfy the $\mathcal{M}$-mixed and $\mathcal{M}$-pushout decomposition properties.

\noindent
\parbox{10.7cm}{\begin{proposition}\label{prop:po2}
		Let $\X$ be an $\mathcal{M}$-adhesive category of type $\mathbb{B}$, and suppose that the square on the right is an $\mathcal{M}$ pullback. Then there exists $E\in \X$, $e\colon E\mto C$ in $\mathcal{M}$, $\phi:E\to B_1$ such that $(C, \{m, e\})$ is a coproduct and $g=f+\phi$. Moreover, such a square is a pushout if and only if $\phi$ is an isomorphism.
\end{proposition}} \parbox{3cm}{\vspace{-.15cm}$\xymatrix{A  \ar[r]^{f} \ar@{>->}[d]_{m}& B_0 \ar@{>->}[d]^{\iota_{B_{0}}}\\ C\ar[r]_-{g} & B_0+B_1}$}
\iffalse 
\begin{proof}
		Pulling back $\iota_{B_0}\colon B_0\to B_0\to B_1$ and $\iota_{B_1}\colon B_1\to B_0 + B_1$ we get the diagram below and the thesis follows from \Cref{prop:ext}. The same diagram allows us to deduce $g=f+\phi$.	
	\parbox{5.2cm}{$\xymatrix{A\ar[r]^-{m} \ar@{>->}[d]_{r}& C  \ar@{>->}[d]_{g}& E \ar@{>->}[d]^{\phi}\ar[l]_-{e}\\B_0  \ar[r]_-{\iota_{B_0}}& B_0+B_1 & B_1 \ar[l]^-{\iota_{B_1}} \\ A & B_0 \\C & B_0+E\\ &&B_0+B_1}$}\parbox{8.8cm}{\vspace{-4.1cm}Let us show the second half of the thesis.
	
	$(\Rightarrow)$ It is immediate to show that the inner square in the second diagram on the left is a pushout. Thus we get the existence of the dotted isomorphism $\varphi\colon B_0+E\to B_0+B_1$.}
		
		\parbox{5cm}{aaaaaaaaaaaaaaaaaaa}
\end{proof}	



\noindent
\parbox{10.7cm}{\begin{corollary}\label{cor:po2}
Let $\X$ be an $\mathcal{M}$-adhesive category of type $\mathbb{B}$, and suppose that the square on the right is an $\mathcal{M}$ pullback. Then there exists $E\in \X$, $e\colon E\mto C$ in $\mathcal{M}$, $\phi:E\to B_1$ such that $(C, \{m, e\})$ is a coproduct and $g=f+\phi$. 
\end{corollary}} \parbox{3cm}{\vspace{-.5cm}$\xymatrix{A  \ar[r]^{f} \ar@{>->}[d]_{m}& B \ar@{>->}[d]^{\iota_{B_{0}}}\\ C\ar[r]_-{g} & B_0+B_1}$}

\noindent 
\parbox{9.5cm}{\begin{proof}	$(1\Rightarrow 2)$ 
	
	\smallskip \noindent $(2\Rightarrow 1)$
\end{proof}}

\fi 

\begin{lemma}\label{lem:prop} Let $\X$ be an $\mathcal{M}$-adhesive category of type $\mathbb{B}$, then $\X$ satisfies the $\mathcal{M}$-mixed and $\mathcal{M}$-pushout decomposition properties.
\end{lemma}
\begin{proof}$\mathcal{M}$-mixed decomposition property. This follows at once from \Cref{prop:stab}.
	
	\smallskip \noindent  $\mathcal{M}$-pushout-decomposition property.

\end{proof}


\begin{definition} Let $\X$
	\todo{classe B+}
\end{definition}

\begin{remark}\todo{adesività B+}
\end{remark}

From \Cref{lem:closed,lem:prop} we can now deduce at once the following.
\begin{corollary}
	\todo{due proprietà classe B+}
\end{corollary}

Using \Cref{thm:good1} we finally get the main result of our appendix.
\begin{corollary}
	\todo{filler e classe B+}
\end{corollary}



\section{Omitted proofs}
In this appendix we will provide the proofs of...



\subsection{Proofs for \Cref{sec:ade}}


\propUnique*

\label{propUnique-proof}
\begin{proof}
  Both pairs $(k, f)$ and $(k', f')$ are pushout complements of $l$
  and $n$, thus, \Cref{lem:pocomp} yields an isomorphism
  $\phi_D\colon D\to D'$. Computing we have
  \[
    g'\circ \phi_D \circ k= g' \circ k'=h'\circ r
  \]

  \noindent
  \parbox{3cm}{
    $\xymatrix{K \ar@/^.4cm/[rr]^{k'}\ar[d]_{r} \ar[r]_{k}&
      \ar[r]_{\phi_D} D \ar[d]^{g}& D' \ar[d]^{g'}\\ R
      \ar@/_.4cm/[rr]_{h'} \ar[r]^{h}& H \ar[r]^{\phi_H}& H'}$}
  \hfill
  \parbox{10cm}{ \hspace{15pt}Hence, we have another
    $\phi_H\colon H\to H'$. To see that $\phi_H$ is an isomorphism,
    consider the diagram aside. By hypothesis the whole rectangle and
    its left half are pushouts, therefore, by \Cref{lem:po1} its right
    square is a pushout too. The claim now follows from the fact that
    the pushout of an isomorphism is an isomorphism. \qedhere }
\end{proof}


\subsection{Proofs for \Cref{sec:equi}}

This section is devoted to provide further details and proof for the concepts introduced in \Cref{sec:equi}.

\subsubsection{Proofs for \Cref{subsec:switch}}


Consider a switch $\der{E}=\{\dder{E}_i\}_{i=0}^1$ for $\dder{D}_0$
and $\dder{D}_1$, then by definition $\dder{E}_0$ and $\dder{E}_1$ are
sequentially independent. We can then wonder if they are switchable.

\begin{proposition}
  \label{prop:switch}
  Let $\der{D}=\{\dder{D}_i\}_{i=0}^1$ be a derivation and suppose
  that $\der{E}=\{\dder{E}_i\}_{i=0}^1$ is a switch for it. Then
  $\der{D}$ is a switch for $\dder{E}_0$ and $\dder{E}_1$ along
  $(j_0, j_1)$.
\end{proposition}

\begin{proof}
    The first point of \Cref{def:switch} is obviously satisfied. The
    second one follows at once since
    \[m_{\der{E},0}= f_0\circ i_1 \qquad h_{\der{E},1}= g_{1}\circ i_0\]
    Finally, the third point holds because we already know that
    \[m_{\der{E},0}= f_0\circ i_1 \qquad h_{\der{E},1}= g_{1}\circ i_0\]
\end{proof}

\begin{lemma}[Uniqueness of switches]
  \label{thm:switch_uni}
  Let $\der{D}=\{\dder{D}_{i}\}_{i=0}^1$ be a derivation and suppose
  that both $\der{E}=\{\dder{E}_i\}_{i=0}^1$ and
  $\der{F}=\{\dder{F}_i\}_{i=0}^1$ are switches of $\dder{D}_0$ and
  $\dder{D}_1$. Then $\der{E}$ and $\der{F}$ are abstraction
  equivalent.
\end{lemma}

\begin{proof}
    By definition of switch we know that $\der{E}_0$ and $\der{F}_0$ have the same match $f_0\circ i_1$ and that $\der{E}_1$ and $\der{F}_1$ have the same comatch $g_1\circ i_0$. Thus we can build the solid part of the diagram below.
    \[\xymatrix@C=22pt{G_{0} && \ar@{>->}[ll]_{f_{\der{F},0}} D_{\der{F},0} \ar[rr]^{g_{\der{F},0}}&& G_{\der{F},1} && \ar@{>->}[ll]_{f_{\der{F},1}} D_{\der{F},1} \ar[rr]^{g_{\der{F},1}}&& G_{2}\\L_1 \ar[d]^{f_0\circ i_1} \ar[u]_{f_0\circ i_1}&& K_{1} \ar[d]^{k_{\der{E},0}} \ar[u]_{k_{\der{F},0}}\ar@{>->}[ll]_{l_1} \ar[r]^{r_1} & R_{1} \ar@/^.35cm/[drrr]|(.3)\hole_(.4){j_0} \ar[dr]|(.3)\hole_{h_{\der{E},0}} \ar@/_.35cm/[urrr]|(.3)\hole^(.4){a_0} \ar[ur]|(.3)\hole^{h_{\der{F},0}}& &L_{0} \ar@/_.35cm/[dlll]^(.4){j_1} \ar[dl]|(.3)\hole^{m_{\der{E},1}} \ar@/^.35cm/[ulll]_(.4){a_1} \ar[ul]|(.3)\hole_{m_{\der{F},1}}& K_{0} \ar[d]_{k_{\der{E},1}} \ar[u]^{k_{\der{F},1}}\ar@{>->}[l]_{l_{0}} \ar[rr]^{r_0} && R_{0} \ar[d]_{g_1\circ i_0} \ar[u]^{g_1\circ i_0}\\ \ar@/^.5cm/[uu]^{\id{G_0}} G_{0} && \ar[ll]^{f_{\der{E},0}} D_{\der{E},0} \ar@{.>}@/^.5cm/[uu]^(.33){\phi_{0}}|\hole \ar[rr]_{g_{\der{E},0}}&& G_{\der{E},1} \ar@{.>}[uu]^(.76){\psi_1}|(.45)\hole |(.55)\hole&& \ar@{>->}[ll]^{f_{\der{E},1}} D_{\der{E},1} \ar@{.>}@/_.5cm/[uu]_(.33){\phi_{1}}|\hole\ar[rr]_{g_{\der{E},1}}&& G_{2}\ar@{.>}@/_.5cm/[uu]_{\psi_2}}\]
    Now, by \Cref{prop:unique} we get $\phi_0\colon D_{\der{E},0}\to D_{\der{F},0}$ and $\psi_1 \colon G_{\der{E},1}\to G_{\der{F},1}$ such that
    \[\xymatrix{G_{0} & \ar@{>->}[l]_{f_{\der{F},0}} D_{\der{F},0} \ar[r]^{g_{\der{F},0}}& G_{\der{F},1} \\L_1 \ar[d]^{f_0\circ i_1} \ar[u]_{f_0\circ i_1}& K_{1} \ar[d]^{k_{\der{E},0}} \ar[u]_{k_{\der{F},0}}\ar[l]_{l_1} \ar[r]^{r_1} & R_{1}  \ar[d]_{h_{\der{E},0}}  \ar[u]^{h_{\der{F},0}}\\ \ar@/^.5cm/[uu]^{\id{G_0}} G_{0} & \ar@{>->}[l]^{f_{\der{E},0}} D_{\der{E},0} \ar@/^.5cm/[uu]^(.33){\phi_{0}}|\hole \ar[r]_{g_{\der{E},0}}& G_{\der{E},1} \ar@/_.5cm/[uu]_{\psi_1}}\]
    commutes. But then we have
    \[
      f_{\der{F},0}\circ \phi_0\circ j_1=f_{\der{E},0} \circ j_1=m_0=f_{\der{F},0}\circ a_1\]
    which entails $a_1= \phi_0\circ j_1$. This, in turn, gives us that
    \[\psi\circ m_{\der{E},1}=\psi \circ g_{\der{E},0}\circ j_1=g_{\der{F},0}\circ \phi_0\circ j_1=g_{\der{F},0}\circ a_1=m_{\der{F},1}\]
    
    Now, the square
    \[\xymatrix@C=30pt{K_0 \ar[d]_{k_{\der{E,1}}} \ar@{>->}[r]^{l_0}& L_0 \ar[d]^{\psi_1\circ m_{\der{E},1}} \\D_{\der{E},1} \ar@{>->}[r]_{\psi_1\circ f_{\der{E},1}}& G_{\der{F}, 1}} \]
    is a pushout and $\psi_1\circ m_{\der{E},1}=m_{\der{F},1}$, thus, by \Cref{prop:unique} there are $\phi_1\colon D_{\der{E},1}\to D_{\der{F},1}$ and $\psi_2\colon G_2\to G_2$ fitting in the diagram below
    \[\xymatrix{G_{\der{F},1} & \ar@{>->}[l]_{f_{\der{F},1}} D_{\der{F},1} \ar[r]^{g_{\der{F},1}}& G_2 \\L_0 \ar[d]^{m_{\der{E},1}} \ar[u]_{m_{\der{F},1}}& K_{1} \ar[d]^{m_{\der{E},1}} \ar[u]_{m_{\der{F},1}}\ar[l]_{l_1} \ar[r]^{r_1} & R_{1}  \ar[d]_{g_1\circ i_0}  \ar[u]^{g_1\circ i_0}\\ \ar@/^.5cm/[uu]^{\psi_1} G_{\der{E},1} & \ar@{>->}[l]^{f_{\der{E},1}} D_{\der{E},1} \ar@/^.5cm/[uu]^(.33){\phi_{1}}|\hole \ar[r]_{g_{\der{E},1}}& G_2 \ar@/_.5cm/[uu]_{\psi_2}}\]
    To conclude, it is now enough to notice that
    \[f_{\der{F},1} \circ \phi_1\circ j_0=\psi_1\circ f_{\der{E},1}\circ j_0=\psi_1\circ h_{\der{E,0}}=h_{\der{F},0}=f_{\der{F},1}\circ a_0\]
    allowing us to deduce $ \phi_1\circ j_0=a_0$.
    \qedhere
\end{proof}


\begin{proposition}
	\label{prop:tec}
	Let $(i_0, i_1)$ be an independence pair between two direct
	derivations $\dder{D}_0$ and $\dder{D}_1$ in a left-linear
	DPO rewriting system $(\X,\R)$. Consider the pullback of
	$f_1\colon D_1\mto G_1$ along $g_0\colon D_0\to G_1$ (first
	square below), then there exist the arrows
	$u_1\colon K_1\to P$ and $u_0\colon K_0\to P$ fitting in the
	central and right square. Moreover, the right one is always
	a pullback.
	\[\xymatrix{P \ar@{>->}[r]^{p_0} \ar[d]_{p_1} & D_0
		\ar[d]^{g_0}& K_0 \ar[r]^{r_0} \ar@{.>}[d]_{u_0}& R_0
		\ar[d]^{i_0} & K_1 \ar@{>->}[r]^{l_1} \ar@{.>}[d]_{u_1}&
		L_1 \ar[d]^{i_1}\\ D_1 \ar@{>->}[r]_{f_1} &G_1 &P
		\ar[r]_{p_1} & D_1 &P \ar[r]_{p_0} & D_0}
	\]
\end{proposition}

\begin{proof}
	We start noticing that
	\[
	f_1\circ i_0\circ r_0=h_0\circ r_0=g_0\circ k_0 \qquad
	g_0\circ i_1\circ l_1=m_1\circ l_1= f_1 \circ k_1
	\]
	Thus there exists the dotted $u_0\colon K_0\to P$,
	$u_1\colon K_1\to P$ in the diagram
	\[\xymatrix{K_0 \ar[d]_{r_0}
		\ar@{.>}[r]_{u_0}\ar@/^.4cm/[rr]^{k_0} &P
		\ar@{>->}[r]_{p_0} \ar[d]_{p_1}& D_0 \ar[d]^{g_0}&K_1
		\ar@{>->}[d]_{l_1}
		\ar@{.>}[r]_{u_1}\ar@/^.4cm/[rr]^{k_1} &P \ar[r]_{p_1}
		\ar@{>->}[d]_{p_0}& D_1 \ar@{>->}[d]^{f_1}\\R_0
		\ar@/_.4cm/[rr]_{h_0}\ar[r]^{i_0}& D_1
		\ar@{>->}[r]^{f_1}& G_1&L_1
		\ar@/_.4cm/[rr]_{m_1}\ar[r]^{i_1}& D_0 \ar[r]^{g_0}&
		G_1}\]
	By \Cref{prop:pbpoad,lem:pb1} we get that the left half of
	the second rectangle is a pullback.
	\qedhere
\end{proof}

There is another
useful properties of weakly sequentially independent direct
derivations which is worth mentioning.

\noindent
\parbox{9.5cm}
{\begin{proposition}\label{lem:cose}Let
    $\der{D}=\{\dder{D}_i\}_{i=0}^1$ be a derivation, and suppose
    that $(i_0, i_1)$ is an independence pair between $\dder{D}_0$
    and $\der{D}_1$.  If $\der{E}=\{\dder{E}_i\}_{i=0}^1$ is a
    switch of $\der{D}$ along $(i_0, i_1)$, then there exists
    $q_0\colon P\to D_{\der{E},0}$ in the diagram
    below \end{proposition}}
\parbox{4cm}{$\xymatrix{D_{\der{E},0} \ar@{>->}[d]_{f_{\der{E},0}}&P
    \ar@{.>}[l]_{q_0} \ar[r]^{p_1} \ar@{>->}[d]_{p_0}& D_1
    \ar@{>->}[d]^{f_1}\\ G_{0}&D_0 \ar@{>->}[l]^{f_0}\ar[r]_{g_0}&
    G_1}$}

\begin{proof}
  Consider the arrow $u_1\colon K_1\to P$ obtained using
  \Cref{prop:tec}. It fits in the diagram:
  \[\xymatrix{K_1 \ar[r]^{u_1} \ar@{>->}[d]_{l_1}&P
      \ar[r]^{\id{P}}\ar@{>->}[d]_{p_0}& P\ar@{>->}[d]^{p_0}\\L_1
      \ar[d]_{\id{L_1}}\ar[r]^{i_1}&D_0 \ar[r]^{\id{D_0}}
      \ar[d]_{\id{D_0}}& D_{0} \ar@{>->}[d]^{f_0}\\
      L_1\ar[r]^{i_1} \ar@/_.4cm/[rr]_{m_{\der{E},0}}&D_0
      \ar@{>->}[r]^{f_0}&G_0}\]

  Since $f_0$ is a monomorphism, every square in the above diagram
  is a pullback. Thus the whole right square is a pullback too.
  Applying \Cref{lem:radj} to the pair of squares below now gives
  us the wanted $q_0\colon P\to D_{\der{E},0}$.
  \[\xymatrix{K_1 \ar@{>->}[d]_{l_1}\ar[r]^{k_{\der{E},0}}&
      D_{\der{E},0} \ar@{>->}[d]^{f_{\der{E},0}} & K_1 \ar[r]^{u}
      \ar@{>->}[d]_{l_1}& P \ar@{>->}[d]^{f_0\circ
        p_0}\\L_1\ar[r]_{m_{\der{E},0}}&G_0 &L_1
      \ar[r]_{m_{\der{E},0}}&G_0} \]
\end{proof}

\subsubsection{Proofs for \Cref{subsec:CR}}

To prove the Local Church-Rosser Theorem, let us begin deducing some properties from the existence of a good independence pair. 

\begin{remark}\label{rem:deco} 
	Let $(i_0, i_1)$ be a good
	independence pair between the direct derivations $\dder{D}_0$ and
	$\dder{D}_1$. We can then build the solid part of the diagram
	below.
	\[\xymatrix{&&R_0 \ar@/_1cm/[ddrr]_(.35){i_0}|(.7)\hole
		\ar[dr]^{h_0}&& L_1\ar@/^1cm/[ddll]^(.35){i_1}
		\ar[dl]_{m_1}\\&K_0\ar[dr]^{k_0}\ar@{>->}[dl]_{l_0}
		\ar@/_.8cm/[ddrr]^(.65){u_0}\ar[ur]^{r_0}&& G_1 && K_1
		\ar@/^.8cm/[ddll]_(.65){u_1}\ar[dl]_{k_1}\ar@{>->}[lu]_{l_1}
		\ar[dr]^{r_1}\\L_0 \ar[dr]^{m_0}
		\ar@{.>}@/_.8cm/[ddrr]_{j_1}&& D_0
		\ar@{>->}[dl]|(.42)\hole_(.64){f_0}\ar[ur]|(.48)\hole^(.7){g_1}&&D_1
		\ar[dr]|(.42)\hole^(.65){g_1}
		\ar@{>->}[ul]|(.48)\hole_(.7){f_1}&&R_1\ar@/^.8cm/[ddll]^{j_0}\ar[dl]^{h_1}\\&G_0
		&&P\ar[dr]^{q_1}
		\ar@{>.>}[dl]_{q_0}\ar[ur]^(.4){p_1}\ar@{>->}[ul]_(.4){p_0}&&G_2\\&&Q_0
		\ar@{>->}@{>.>}[ul]_{a_0} &&Q_1\ar@{.>}[ur]^{b_1}}
	\]
	
	Let us complete this diagram defining the dotted arrows. To get
	$j_1\colon L_0\to Q_0$ and $q_0\colon P\to Q_0$ it is enough to
	take a pushout of $l_0$ along $u_0$, which exists since
	$l_0\in \mathcal{M}$. Moreover, the existence of the wanted
	$a_0\colon Q_0\to G_0$ and $b_1\colon Q_1\to G_2$ follows from
	\[
	f_0\circ p_0 \circ u_0 = f_0\circ k_0 = m_0\circ l_0 \qquad
	g_1\circ p_1\circ u_1 = g_1\circ k_1=h_1\circ r_1
	\]
	
	\noindent
	\parbox{5.5cm}{\hspace{15pt}We can prove some other properties of
		the arrows appearing in the diagram above. The four rectangles
		aside are pushouts and their left halves are pushouts
		too. Therefore, by \Cref{lem:po1}, also their right halves are
		pushouts.  Moreover, that $q_0$ and $a_0$ are pushouts of,
		respectively $l_0$ and $p_0$, thus they are elements of
		$\mathcal{M}$. By \Cref{prop:pbpoad} the left halves of the
		first and third rectangles are pullbacks too.}  \parbox{3.5cm}{
		$\xymatrix{K_0 \ar@/^.4cm/[rr]^{k_0}\ar[d]_{r_0}\ar[r]_{u_0}
			&P\ar[d]^{p_1} \ar@{>->}[r]_{p_0} & D_0 \ar[d]^{g_0}&K_1
			\ar@/^.4cm/[rr]^{k_1}\ar@{>->}[d]_{l_1}\ar[r]_{u_1}
			&P\ar@{>->}[d]^{p_0} \ar[r]_{p_1} & D_1 \ar@{>->}[d]^{f_1}\\
			R_0 \ar@/_.4cm/[rr]_{h_0} \ar[r]^{i_0}&D_1 \ar@{>->}[r]^{f_1}
			& G_1&L_1 \ar@/_.4cm/[rr]_{m_1} \ar[r]^{i_1}&D_0\ar[r]^{g_0} &
			G_1}$
		$\xymatrix{K_0
			\ar@/^.4cm/[rr]^{k_0}\ar@{>->}[d]_{l_0}\ar[r]_{u_0}
			&P\ar@{>->}[d]^{q_0} \ar@{>->}[r]_{p_0} & D_0
			\ar@{>->}[d]^{f_0}&K_1
			\ar@/^.4cm/[rr]^{k_1}\ar[d]_{r_1}\ar[r]_{u_1}
			&P\ar[d]^{q_1} \ar[r]_{p_1} & D_1 \ar[d]^{g_1}\\
			L_0 \ar@/_.4cm/[rr]_{m_0} \ar[r]^{j_1}&Q_0 \ar@{>->}[r]^{a_0}
			& G_0&R_1 \ar@/_.4cm/[rr]_{h_1} \ar[r]^{j_0}&Q_1 \ar[r]^{b_1}
			& G_2}$}
\end{remark}


\thmChurch*
\label{thmChurch-proof}

\begin{proof}

  Let us begin considering the diagram
  built in \Cref{rem:deco}:
  \[\xymatrix{&&R_0
      \ar@/_1cm/[ddrr]_(.35){i_0}|(.7)\hole \ar[dr]^{h_0}&&
      L_1\ar@/^1cm/[ddll]^(.35){i_1}
      \ar[dl]_{m_1}\\&K_0\ar[dr]^{k_0}\ar@{>->}[dl]_{l_0}
      \ar@/_.8cm/[ddrr]^(.65){u_0}\ar[ur]^{r_0}&& G_1 && K_1
      \ar@/^.8cm/[ddll]_(.65){u_1}\ar[dl]_{k_1}\ar@{>->}[lu]_{l_1}
      \ar[dr]^{r_1}\\L_0 \ar[dr]^{m_0} \ar@{>}@/_.8cm/[ddrr]_{j_1}&& D_0
      \ar@{>->}[dl]|(.42)\hole_(.64){f_0}\ar[ur]|(.48)\hole^(.7){g_1}&&D_1
      \ar[dr]|(.42)\hole^(.65){g_1}
      \ar@{>->}[ul]|(.48)\hole_(.7){f_1}&&R_1\ar@/^.8cm/[ddll]^{j_0}\ar[dl]^{h_1}\\&G_0
      &&P\ar[dr]^{q_1}
      \ar@{>->}[dl]_{q_0}\ar[ur]^(.4){p_1}\ar@{>->}[ul]_(.4){p_0}&&G_2\\&&Q_0
      \ar@{>->}@{>->}[ul]_{a_0} \ar@{.>}[dr]_{b_0} &&Q_1\ar[ur]^{b_1}
      \ar@{>.>}[dl]^{a_1} \\ &&&H_1 }\] Since $q_0\colon P\to Q_0$ is in
  $\mathcal{M}$, we can take its pushout along $q_1$ to get the dotted
  arrows $a_1\colon Q_1\to H_1$ and $q_0\colon Q_0\to H_1$. But then,
  since the composite of two pushout squares is a pushout, the diagram
  below is a derivation.
  \[\xymatrix@C=15pt{L_1 \ar[d]_{f_0 \circ i_1}&& K_1
      \ar[d]_{q_0\circ u_1}\ar@{>->}[ll]_{l_1} \ar[r]^{r_1} & R_1
      \ar@/^.35cm/[drrr]|(.3)\hole_(.4){j_0} \ar[dr]|(.3)\hole_{a_1\circ
        j_0} && L_0 \ar@/_.35cm/[dlll]^(.4){j_1} \ar[dl]|(.3)\hole^{b_0\circ
        j_1}& K_1 \ar[d]^{q_1\circ u_0}\ar@{>->}[l]_{l_0} \ar[rr]^{r_0} && R_0
      \ar[d]^{g_1\circ i_0} \\G_0 && \ar@{>->}[ll]^{a_0} D_0 \ar[rr]_{b_0}&&
      G_1 && \ar@{>->}[ll]^{a_1} D_1 \ar[rr]_{b_1}&& G_2}\] It is immediate
  now to see that it is a switch of $\dder{D}_0$ and $\dder{D}_1$ along
  $(i_0, i_1)$.
\end{proof}

\begin{remark}
	\label{rem:church}
	It is worth to point out that, by
	\Cref{prop:switch,thm:switch_uni}, the derivation we obtain from
	\Cref{thm:church} is made by two switchable derivations which will
	give back the starting derivation (up to abstraction equivalence),
	if switched again.
	
	Moreover, by \Cref{prop:pbpoad}, the first square below is a
	pullback, therefore, by \cref{rem:deco} the last three squares
	witness that $(j_0, j_1)$ is a good independence pair too.
	\[
	\xymatrix{P\ar[r]^{q_1} \ar@{>->}[d]_{q_0}&Q_1
		\ar@{>->}[d]^{a_1}&K_1 \ar[r]^{r_1} \ar[d]_{u_1}& R_1
		\ar[d]^{j_0} & K_0 \ar@{>->}[r]^{l_0} \ar[d]_{u_0}& L_0
		\ar[d]^{j_1}&K_0 \ar[r]^{r_0} \ar[d]_{u_0}& R_0\ar[d]^{i_0} \\
		Q_ 0\ar[r]_{b_0}&H_1&P \ar[r]_{q_1} & Q_1 &P \ar[r]_{q_0} &
		Q_0&P \ar[r]_{p_1} & D_1}
	\]
\end{remark}


\propEqui*
\label{propEqui-proof}

\begin{proof}
  Suppose that $\X$ is $\mathcal{M}$-adhesive and let $(i_0, i_1)$ be
  an independence pair as below.
  \[
    \xymatrix@C=15pt{L_0 \ar[d]_{m_0}&& K_0
      \ar[d]_{k_0}\ar@{>->}[ll]_{l_0} \ar@{>->}[r]^{r_0} & R_0
      \ar@/^.35cm/[drrr]|(.3)\hole_(.4){i_0} \ar[dr]|(.3)\hole_{h_0}
      && L_1 \ar@/_.35cm/[dlll]^(.4){i_1} \ar[dl]|(.3)\hole^{m_1}& K_1
      \ar[d]^{k_1}\ar@{>->}[l]_{l_1} \ar@{>->}[rr]^{r_1} && R_1
      \ar[d]^{h_1} \\G_0 && \ar@{>->}[ll]^{f_0} D_0
      \ar@{>->}[rr]_{g_0}&& G_1 && \ar@{>->}[ll]^{f_1} D_1
      \ar@{>->}[rr]_{g_1}&& G_2}\]

  Since $(\X, \R)$ is linear, then $r_1\colon K_1\to R_1$ belongs to
  $\mathcal{M}$, thus it admits a pushout along $u_1\colon K_1\to P$,
  as desired. Moreover, let us consider the two rectangles below.
  \[
    \xymatrix{K_0 \ar@/^.4cm/[rr]^{k_0}\ar@{>->}[d]_{r_0}\ar[r]_{u_0}
      &P\ar@{>->}[d]^{p_1} \ar@{>->}[r]_{p_0} & D_0
      \ar@{>->}[d]^{f_0}& K_1
      \ar@/^.4cm/[rr]^{k_1}\ar@{>->}[d]_{l_1}\ar[r]_{u_1}
      &P\ar@{>->}[d]^{p_0} \ar@{>->}[r]_{p_1} & D_1
      \ar@{>->}[d]^{f_1}\\ R_0 \ar@/_.4cm/[rr]_{h_0} \ar[r]^{i_0}&D_1
      \ar@{>->}[r]^{f_1} & G_1& L_1 \ar@/_.4cm/[rr]_{m_1}
      \ar[r]^{i_1}&D_0 \ar@{>->}[r]^{g_0} & G_1} \] By hypothesis
  $r_0$ and $l_1$ are in $\mathcal{M}$, thus $f_1$ and $g_0$ belong to
  it too. The first point of \Cref{lem:popb} yields the thesis.
\end{proof}

\begin{remark}\label{rem:fill} Let $\der{D}=\{\dder{D}_i\}_{i=0}^n$ be a derivation and $(i_0, i_1)$ a good independence pair between $\dder{D}_0$ and $\dder{D}_1$. Let also $\{\der{E}_{i}\}_{i=0}^1$ be a switch for them along $(i_0, i_1)$. Since, by \Cref{thm:switch_uni}, all switches are abstraction equivalent, we get the diagram below.
	
	\[\xymatrix{&&R_0 \ar@/_1cm/[ddrr]_(.35){i_0}|(.7)\hole
		\ar[dr]^{h_0}&& L_1\ar@/^1cm/[ddll]^(.35){i_1}
		\ar[dl]_{m_1}\\&K_0\ar[dr]^{k_0}\ar@{>->}[dl]_{l_0}
		\ar@/_.8cm/[ddrr]|(.36)\hole^(.65){u_0}\ar[ur]^{r_0}&& G_1 &&
		K_1
		\ar@/^.8cm/[ddll]|(.36)\hole_(.65){u_1}\ar[dl]_{k_1}\ar@{>->}[lu]_{l_1}
		\ar[dr]^{r_1}\\L_0
		\ar@/_.8cm/[ddrr]_(.2){i'_0}|(.31)\hole|(.81)\hole
		\ar[dr]^(.4){m_0}|(.61)\hole && D_0
		\ar@{>->}[dl]|(.4)\hole_(.65){f_0}\ar[ur]|(.5)\hole^(.7){g_0}&&D_1
		\ar[dr]|(.4)\hole^(.65){g_1}
		\ar@{>->}[ul]|(.5)\hole_(.65){f_1}&&R_1\ar@/^.8cm/[ddll]^(.2){i'_1}|(.31)\hole|(.81)\hole\ar[dl]_{h_1}|(.6)\hole\\&G_0
		&&P_1\ar[dr]_{q_1}	\ar@{>->}[dl]^{q_0}\ar[ur]^(.4){p_1}\ar@{>->}[ul]_(.4){p_0}&&G_2\\L_1	\ar@/^.8cm/[uurr]^(.2){i_1} \ar[ur]_(.35){m'_0}|(.61)\hole && D'_0	\ar@{>->}[ul]|(.4)\hole_(.65){f'_0}\ar[dr]|(.5)\hole^(.65){g'_0}	&&D'_1\ar[ur]|(.4)\hole^(.65){g'_1} \ar@{>->}[dl]|(.5)\hole_(.65){f'_1}	&& R_0 \ar[ul]^(.35){h'_1}|(.61)\hole\ar@/_.8cm/[uull]_(.2){i_0}\\ &K_1	\ar[ur]_{k'_0} \ar[dr]_{r_1}	\ar@{>->}[ul]^{l_1}\ar@/^.8cm/[uurr]_(.65){u_1}&&G'_1&& K_0\ar[ur]_{r_0} \ar[ul]^{k'_1} \ar@{>->}[dl]^{l_0} \ar@/_.8cm/[uull]^(.65){u_0}\\&& R_1	\ar[ur]_{h'_0}\ar@/^1cm/[uurr]^(.25){i'_1}&& L_0 \ar[ul]^{m'_1} \ar@/_1cm/[uull]_(.25){i'_0} |(.69)\hole }\] 

\end{remark}


\subsubsection{Proofs for \Cref{subsec:verytame}}

\begin{proposition}
	\label{pr:weak}
	Let $(\X, \R)$ be a linear DPO rewriting system. Then it is very tame.
\end{proposition}
\begin{proof}
	
	Let $(i_0, i_1)$ and $(i'_0, i'_1)$ be independence pairs for the
	direct derivations $\dder{D}$ and $\dder{D}'$. Notice that, by
	definition, we have
	\[
	f_1\circ i_0=h_0=f_1\circ i'_0 \qquad g_0\circ i_1=m_1= g_0\circ i'_1
	\]
	
	The arrow $f_1\colon D\to G_1$ is the pushout of
	$l_1\colon K_1\mto L_1$, thus it is in $\mathcal{M}$ implying
	$i_0=i'_0$. If, moreover, we suppose that the rule $\rho$ applied in
	$\dder{D}$ is linear, then $g_0\colon D_0\to G_1$ is in
	$\mathcal{M}$ too entailing $i'_1=i'_1$, too.
\end{proof}



\lemVTame*
\label{lemVTame}


\begin{proof}
	Every node-merging graphical rewriting system is tame because $\gph{G}$ is in the class $\mathbb{B}^+$ defined in \Cref{app:fill}.  Consider the following diagram 
	\[\xymatrix@R=25pt@C=30pt{L_0 \ar[d]_{m_0}&& K_0
		\ar[d]_{k_0}\ar@{>->}[ll]_{l_0} \ar[r]^{r_0} & R_0
		\ar@<-.5ex>@/^.35cm/[drrr]|(.23)\hole|(.34)\hole_(.4){i_0} 	\ar@<.5ex>@/^.35cm/[drrr]^(.2){j_0}|(.31)\hole |(.4)\hole
		\ar[dr]|(.3)\hole_{h_0} && L_1 \ar@<.5ex>@/_.35cm/[dlll]^(.4){i_1} \ar@<-.5ex>@/_.35cm/[dlll]_(.2){j_1}
		\ar[dl]|(.3)\hole^{m_1}& K_1 \ar[d]^{k_1}\ar@{>->}[l]_{l_1}
		\ar[rr]^{r_1} && R_1 \ar[d]^{h_1}\\G_0 && \ar@{>->}[ll]^{f_0}
		D_0 \ar[rr]_{g_0}&& G_1 && \ar@{>->}[ll]^{f_1} D_1
		\ar[rr]_{g_1}&& G_2}\] 
	
	Since $f_1$ is mono then $i_0=j_0$. For every $v$ with depth $0$ we have
	\[g_{0,v}\circ i_{1,v}=h_{0,v}=g_{0,v}\circ j_{1,v}\]
	which entails $i_{1,v}=j_{1,v}$. Let now $v$ be a vertex with depth greater than $0$, and take $x\in L_1(v)$. By definition, we have $p:w\to v$ and $y\in L_{1}(w)$  such that $\dph(w)=0$ and $L(p)(y)=x$. By naturality have:
	\[i_{1,v}(x)=i_{1,v}(L(p)(y))=L(p)(i_{1, w}(y))=L(p)(j_{1, w}(y))=j_{1,v}(L(p)(y))=j_{1,v}(x)\]
	From this we can deduce that $i_{1,v}=j_{1,v}$ as wanted.
\end{proof}

\subsection{Proofs for \Cref{subsec:perm}}

\begin{lemma}\label{lem:colim}
	Let $\X$ be an $\mathcal{M}$-adhesive category and $(\X, \R)$ a
	left-linear DPO rewriting system over it. The following properties
	hold true.
	\begin{enumerate}
		\item  If $\der{D}$ is a derivation from $G$ to $H$, then the diagram $\Delta(\der{D})$ has a colimit $(\tpro{D}, \{\iota_X\}_{X\in \Delta(\der{D})})$ such that $\iota_H$ belongs to $\mathcal{M}$. Moreover, if $(\X, \R)$ is linear, then every coprojection $\iota_X$ belongs to $\mathcal{M}$.
		
		\parbox{10cm}{
		\item Let $\der{D}$ be the concatenation of two derivations $\der{D}_1=\{\dder{D}_{1,i}\}_{i=0}^{n_1}$ between $G$ and $H$ and $\der{D}_2=\{\dder{D}_{2,j}\}_{j=0}^{n_2}$ between $H$ and $T$,  then the colimiting cocone $(\tpro{D}, \{\iota_X\}_{X\in \Delta(\der{D})})$ exists too and there is a pushout square as the one aside
		where $(\tproi{D}{1}, \{\iota_{1, X}\}_{X\in \Delta(\der{D}_1)})$ and $(\tproi{D}{2}, \{\iota_{2, X}\}_{X\in \Delta(\der{D}_2)})$ are the colimiting cocone for $\Delta(\der{D}_1)$ and $\Delta(\der{D}_2)$, respectively.} 
		\parbox{3cm}{\vspace{-.5cm}
		$\xymatrix{H\ar[r]^-{\iota_{2, H}} \ar@{>->}[d]_-{\iota_{1, H}} & \tproi{D}{2} \ar[d]^{p_2}\\  \tproi{D}{1} \ar[r]_{p_1}& \tpro{D}}$}
	\end{enumerate}
\end{lemma}
\begin{remark}\label{rem:cof}
	Let $I:\Deltamin(\dder{D})\to \Delta(\dder{D})$ be the inclusion functor. It is immediate to see that such functor is \emph{final} \cite{mac2013categories}. This means that for every functor $F\colon \Delta(\dder{D})\to \Y$ we have:
	\begin{enumerate}
		\item if  $(C, \{c_X\}_{X\in \Deltamin(\dder{D})})$ is colimiting for $F\circ I$, then there exists a colimiting cocone $(D, \{d_X\}_{X\in \Delta(\dder{D})})$ for $F$;
		\item $(C, \{c_X\}_{X\in \Deltamin(\dder{D})})$ and $(D, \{d_X\}_{X\in \Delta(\dder{D})})$ are colimiting for, repsectively, $F\circ I$ and $F$, then the canonical arrow $\phi\colon C\to D$ induced by $(D, \{d_X\}_{X\in \Deltamin(\dder{D})})$ is an isomorphism.
	\end{enumerate}
\end{remark}

\begin{proof}
	\begin{enumerate}
		\item Let us proceed by induction on the length of $\der{D}$.
		
		
		\smallskip \noindent
		$\lgh(\dder{D})=0$. then the
		$\tpro{\dder{D}}$ is simply $(G, \{\id{G}\})$ and
		$\id{G}\in \mathcal{M}$.¸
		
		\smallskip \noindent
		$\lgh(\dder{D})=1$. Suppose that $\dder{D}$
		has as its single component the derivation
		\[\xymatrix{L \ar[d]_{m}& K \ar[d]^{k}\ar@{>->}[l]_{l} \ar[r]^{r} & R \ar[d]^{h} \\G& \ar@{>->}[l]^{f} D \ar[r]_{g}& H  \\}\]
		The arrow $f$ is  the pushout of $l$ and so it is in in $\mathcal{M}$. We can thus consider the $\mathcal{M}$-pushout square
		\[\xymatrix{D \ar@{>->}[d]_{f} \ar[r]^{g} & H \ar@{>->}[d]^{p} \\G \ar[r]_{q}& P }\]
		Since $p\in \mathcal{M}$, the thesis follows immediately from \Cref{rem:cof}.
		
		\smallskip \noindent
		$\lgh(\dder{D})\geq 2$. Let $\der{D}$ be
		$\{\dder{D}_i\}_{i=0}^n$ with $n\geq 1$. Let also $\der{D}'$ be
		$\{\dder{D}_i\}^{n-1}_{i=0}$ and $\rho_n=(l_n, r_n)$ be the rule
		applied in $\dder{D}_n$. The pushout of $l_n$ is the arrow
		$f_n\colon D_n\to G_n$ is in $\mathcal{M}$ and, by inductive
		hypothesis, $\iota_{G_{n}}\colon G_{n}\to \lpro \der{D}'\rpro$ is
		in $\mathcal{M}$ too. Thus, we can consider the diagram below,
		having a pushout as its lower half.
		\[\xymatrix{L_n \ar[d]_{m_{n}}& K_{n} \ar[d]^{k_{n}}\ar@{>->}[l]_{l_{n}} \ar[r]^{r_{n}} & R_{n} \ar[d]^{h_n} \\G_{n} \ar@{>->}[d]_{\iota'_{G_{n}}}& \ar@{>->}[l]^{f_n} D_n \ar[r]_{g_n}& H  \ar@{>->}[d]^{q}\\ \lpro \der{D}' \ar[rr]_{p}\rpro && P}\]
		Notice that, as in the point above, the arrow $q\colon H\to P$ is the pushout of an element in $\mathcal{M}$, therefore it is enough to show that the diagram so constructed provides a colimiting cocone for $\Delta(\der{D})$.
		
		Let $(C, \{c_X\}_{X\in \Delta(\der{D})})$ be a cocone, since $\Delta(\der{D}')$ is a subdiagram of $\Delta(\der{D})$, we get another cocone $(c, \{c_X\}_{X\in \Delta(\der{D}')})$ which induces an arrow $c'\colon \lpro \der{D}' \rpro \to C$ such that
		\begin{align*}
			c'\circ \iota_{G_n} \circ f_n & =c_{G_n} \circ f_n \\&= c_{D_n}\\&= c_{H}\circ g_n
		\end{align*}
		Therefore the arrows $c'$ and $c_H$ induce a morphism $c\colon P\to C$ and the thesis now follows at once.
		
		\item As a first step, notice that
		$(\tpro{D}, \{\iota_X\}_{X\in \Delta(\der{D}_1)})$ and
		$(\tpro{D}, \{\iota_X\}_{X\in \Delta(\der{D}_2)})$ are cocone on,
		respectively, $\Delta(\der{D}_1)$ and $\Delta(\der{D}_2)$. Hence,
		there exist two arrows $p_1\colon \tproi{D}{1}\to \tpro{D}$,
		$p_2\colon \tproi{D}{2}\to \tpro{D}$ such that, for every
		$X\in \Delta(\der{D}_1)$ and $Y\in \Delta(\der{D}_2)$
		\[p_1\circ \iota_{1, X} = \iota_X \qquad p_2\circ \iota_{2, Y}=\iota_{2,Y}\]
		In particular, this entails the commutativity of the square
		\[\xymatrix{H \ar[dr]^{\iota_H} \ar[r]^-{\iota_{2, H}} \ar@{>->}[d]_-{\iota_{1, H}} & \tproi{D}{2} \ar[d]^{p_2}\\  \tproi{D}{1} \ar[r]_{p_1}& \tpro{D}}\]
		
		Let us now show that the square above is a pushout. Take two arrows $a\colon \tproi{D}{1}\to C$, $b\colon \tproi{D}{2}\to C$ such that
		\[a\circ \iota_{1, H}=b\circ \iota_{2, H}\]
		We can use the previous equality to define a cocone $(C, \{c_X\}_{X\in \Delta(\der{D})})$ putting:
		\[c_X:=\begin{cases}
			a\circ \iota_{1, X} & X\in \Delta(\der{D}_1) \\
			b\circ \iota_{2, X} & X\in \Delta(\der{D}_2)
		\end{cases}\]
		From this, we can deduce at once the existence of a unique $c\colon \tpro{D}\to C$ such that
		\[c\circ \iota_X = c_X\]
		By construction, for every $X\in  \Delta(\der{D}_1)$ and $Y\in  \Delta(\der{D}_2)$ we have
		\begin{align*}
			c\circ p_1 \circ \iota_{1,X}&=c\circ \iota_{X}=c_X =a\circ \iota_{1,X}=a\circ p_1\circ \iota_{1,X}\\
			c\circ p_2 \circ \iota_{2,Y}&=c\circ \iota_{Y}=c_Y =b\circ \iota_{2,Y}=b\circ p_2\circ \iota_{2,Y}
		\end{align*}
		Therefore $c\circ p_1=a$ and $c\circ p_2 = b$.
		
		For uniqueness, suppose that $c'\colon \tpro{D}\to C$ is such that $c'\circ p_1=a$ and $c'\circ p_2 = b$.
		Then, for every $X\in \Delta(\der{D})$ we have
		\[	c'\circ \iota_X  = \begin{cases}
			c'\circ p_1\circ \iota_{1, X} & X\in \Delta(\der{D}_1) \\
			c'\circ p_2\circ \iota_{2, X} & X\in \Delta(\der{D}_2)
		\end{cases} =\begin{cases}
			a\circ \iota_{1, X} & X\in \Delta(\der{D}_1) \\
			b\circ \iota_{2, X} & X\in \Delta(\der{D}_2)
		\end{cases}=c_X=c\circ \iota_X\]
		showing that $c'=c$ as wanted.	 \qedhere
	\end{enumerate}
\end{proof}

\noindent 
\parbox{8cm}{
\begin{corollary}
	\label{cor:colim}
	Let $\der{D}=\{\dder{D}_i\}_{i=0}^n$ a derivation of length $n+1$
	and fix an index $j\in[0,n]$. Define $\der{D}^j_1:=\{\dder{D}_i\}_{i=0}^{j-1}$, 
	$\der{D}^j_2=\{\dder{D}_j\}$ and 
	$\der{D}^j_3:=\{\dder{D}_i\}_{i=j+1}^n$, 	with the convention that
	$\der{D}^0_1$ and $\der{D}^n_3$ are the empty derivation on,
	respectively, $G_0$ and $G_n$. Then the square aside is a pushout
	and a pullback, where the two arrows $p_1\colon \lpro\der{D}^j_1 \rpro\to \tpro{D}$,
	$p_2\colon \lpro \der{D}^j_3\rpro \to \tpro{D}$ are induced by the
	cocones $(\tpro{D}, \{\iota_{X}\}_{X\in \Delta(\der{D}^j_1)})$ and
	$(\tpro{D},\{\iota_{X}\}_{X\in \Delta(\der{D}^j_3)})$, respectively.
\end{corollary}}
\parbox{4cm}{\vspace{-.5cm}$\xymatrix@C=30pt{D_{j} \ar[ddrr]^{\iota_{D_j}}
	\ar[r]^{g_j}\ar@{>->}[d]_{f_j}& G_{j+1}
	\ar[ddr]^{\iota_{G_{j+1}}}
	\ar[r]^-{\iota_{3, G_{j+1}}} & \lpro \der{D}^j_3\rpro \ar[dd]^{p_2} \\
	G_j\ar[drr]_{\iota_{G_{j}}} \ar@{>->}[d]_{\iota_{1,G_j}}\\ \lpro
	\der{D}^j_1 \rpro \ar[rr]_{p_1} &&\tpro{D} }$}


\begin{remark}
	If $\der{D}$ is empty then $\der{D}^j_1, \der{D}^j_2$ and
	$\der{D}^j_3$ are empty too.
\end{remark}

\begin{proof}
	We can notice that
	$\der{D}=\der{D}^j_1\cdot \der{D}^j_2 \cdot \der{D}^j_3$. By the
	first and the second point of \Cref{lem:colim} then we get the
	following diagram, in which all squares are $\mathcal{M}$-pushouts.
	
	\[
	\xymatrix@C=30pt{D_{j} \ar[r]^{g_j}\ar@{>->}[d]_{f_j}& G_{j+1}
		\ar[r]^-{\iota_{3,G_{j+1}}} \ar[d]_{\iota_{2, G_{j+1}}}
		\ar@/^.5cm/[dd]^{\iota_{1,2, G_{j+1}}}& \lpro \der{D}^j_3\rpro
		\ar[dd]^{p_2} \\ G_j \ar[r]^{\iota_{2,
				G_j}}\ar@{>->}[d]_{\iota_{1,G_j}} & \lpro \der{D}^j_2\rpro
		\ar[d]_a\\ \lpro \der{D}^j_1 \rpro \ar@/_.4cm/[rr]_{p_1}
		\ar[r]^b &\lpro \der{D}^j_1\cdot \der{D}^j_2 \rpro
		\ar[r]^c&\tpro{D} }
	\]
	Applying \Cref{lem:po1} twice we get that the whole square is an
	$\mathcal{M}$-pushout. Then the thesis follows from
	\Cref{prop:pbpoad}.
\end{proof}

\begin{corollary}
	\label{cor:ele}
	Let $\der{D}=\{\dder{D}_{i}\}_{i=0}^n$ be a derivation between $G$
	and $H$. Let $j$ and $k$ be two indexes less or equal than $n+1$ and
	suppose that $j< k$.  Consider two arrows $a\colon T\to G_j$,
	$b\colon T\to G_k$. If $\iota_{G_j}\circ a = \iota_{G_k}\circ b$,
	then there exist a unique arrow $c\colon T\to D_j $ such
	that \[f_j\circ c = a\qquad \iota_{D_j}\circ c =\iota_{G_k}\circ b\]
\end{corollary}


\begin{lemma}\label{lem:coswap}
Let $(\X, \R)$ be a tame rewriting system and $(i_0, i_1)$ an independence pair between $\dder{D}_{0}$ and $\dder{D}_1$, let also $\der{E}=\{\dder{E}_i\}_{i=0}^1$  be a switch of them along $(i_0, i_1)$. Then the swap $(1,2):[0,1]\to [0,1]$ is a consistent permutation.
\end{lemma}
\begin{proof}
	Let $(\tpro{D}, \{\iota_{X}\}_{X\in \delta(\der{D})})$ and $(\tpro{E}, \{\iota'_{X}\}_{X\in \delta(\der{E})})$ be the colimiting cocones of $\Delta(\der{D})$ and $\Delta(\der{E})$. Using the notation of \Cref{rem:fill} we have
	\[\begin{split}
&\iota'_{G_2}\circ g_1\circ p_1= \iota'_{G_2}\circ g'_1\circ q_1=\iota'_{D'_1}\circ q_1\\=&\iota'_{G'_1}\circ f'_1\circ 	q_1=\iota'_{G'_1}\circ g'_0\circ q_0=\iota'_{D'_0}\circ q_0\\=&\iota'_{G_0}\circ f'_0\circ q_0=\iota'_{G_0}\circ f_0\circ p_0
	\end{split} \qquad \begin{split}
	&\iota_{G_2}\circ g'_1\circ q_1= \iota_{G_2}\circ g_1\circ p_1=\iota_{D_1}\circ p_1\\=&\iota_{G_1}\circ f_1\circ 	q_1=\iota_{G_1}\circ g_0\circ p_0=\iota_{D_0}\circ p_0\\=&\iota_{G_0}\circ f_0\circ p_0=\iota_{G_0}\circ f'_0\circ q_0
	\end{split}\]

	Thus there exist $\phi_{G_1}\colon G_1\to \tpro{E}$ and $\psi_{G'_1}\colon G'_1\to \tpro{D}$ fitting in the diagrams below.
	
	\[\xymatrix{P \ar[d]_{p_1} \ar@{>->}[r]^{p_0}& D_0 \ar@{>->}[dr]^{f_0} \ar[d]^{g_1} && P \ar[d]_{q_1} \ar@{>->}[r]^{q_0}& D'_0 \ar@{>->}[dr]^{f'_0} \ar[d]^{g'_1} \\D_1 \ar[dr]_{g_1} \ar@{>->}[r]_{f_1}&G_1  \ar@{.>}[dr]_{\phi_{G_1}}& G_0 \ar[d]^{\iota'_{G_0}} & D'_1 \ar[dr]_{g'_1} \ar@{>->}[r]_{f'_1}&G'_1  \ar@{.>}[dr]_{\psi_{G'_1}}& G_0 \ar[d]^{\iota_{G_0}}\\ & G_2 \ar[r]_{\iota'_{G_2}} & \tpro{E} && G_2 \ar[r]_{\iota_{G_2}} & \tpro{D}}\]
	In this way we get two morphisms $\xi_{(1,2)}\colon \tpro{D}\to \tpro{E}$ and $\xi'_{(1,2)}\colon \tpro{E}\to \tpro{D}$ such that
	\[\begin{split}
		\xi_{(1,2)}\circ \iota_{G_0}&=\iota'_{G_0}\\
		\xi_{(1,2)}\circ \iota'_{G_0}&=\iota_{G_0}
	\end{split} \quad \begin{split}\xi_{(1,2)}\circ \iota_{G_1}&=\phi_{G_1}\\ \xi_{(1,2)}\circ \iota'_{G_1}&=\psi_{G_1}
	\end{split} \quad\begin{split}
		\xi_{(1,2)}\circ \iota_{G_2}&=\iota'_{G_2}\\
		\xi_{(1,2)}\circ \iota'_{G_2}&=\iota_{G_2}
	\end{split}  \]
	
	It is immediate to see that these two morphism are one the inverse of the other. We have
\begin{align*}		\xi_{(1,2)}\circ \iota_{G_0}\circ m_0&= \iota'_{G_0}\circ m_0=\iota'_{G'_0}\circ f'_0
	\circ i'_0=\iota'_{D'_0}\circ i'_0=	\iota'_{G'_1}\circ g'_0\circ i'_0=\iota'_{G'_1}\circ m'_1\\
			\xi_{(1,2)}\circ \iota_{G_0}\circ m_0&= \iota'_{G_0}\circ m_0=\iota'_{G'_0}\circ f'_0
	\circ i'_0=\iota'_{D'_0}\circ i'_0=	\iota'_{G'_1}\circ g'_0\circ i'_0=\iota'_{G'_1}\circ m'_1
\end{align*}
	These two computations gives us consistency.
\end{proof}



This, together with \Cref{lem:colim}, gives us at once the following two results.

\begin{corollary}\label{cor:coswap}
	contenuto...
\end{corollary}



\propcoswitch
\label{propcoswitch-proof}
\begin{proof}
	contenuto...
\end{proof}

\subsubsection{Proofs for \Cref{subsec:3steps}}

\lemThreeSteps*
\label{lemThreeSteps-proof}

\begin{proof}
NOTES ON THE FUNDAMENTAL LEMMA WE NEED

Everything is very tame, so that sequential independence coincides with switchability and we do not have ambiguity in the choice of an independence pair


We start with the following situation.
\[\xymatrix@C=15pt{L_0\ar[d]_{m_0}&& K_0 \ar[d]_{k_0}\ar[ll]_{l_0} \ar[r]^{r_0} & R_0 \ar@/^.35cm/[drrr]_(.4){i_0}|(.285)\hole \ar[dr]|(.28)\hole_{h_0} && L_1 \ar@/_.35cm/[dlll]^(.4){i_1} \ar[dl]|(.28)\hole^{m_1}& K_1 \ar[d]^{k_1}\ar[l]_{l_1} \ar[r]^{r_1} & R_1 \ar[dr]|(.28)\hole_{h_1} \ar@/^.35cm/[drrr]_(.4){j_0}|(.285)\hole  && L_2 \ar@/_.35cm/[dlll]^(.4){j_1} \ar[dl]|(.28)\hole^{m_2}& K_2 \ar[d]^{k_2}\ar[l]_{l_2} \ar[rr]^{r_2} && R_2 \ar[d]^{h_2} \\G_0 && \ar[ll]^{f_0} D_0 \ar[rr]_{g_0}&& G_1  && \ar[ll]^{f_1} D_1 \ar[rr]_{g_1}&& G_2 && \ar[ll]^{f_2} D_2 \ar[rr]_{g_2}&& G_3 }\]

We can do two things:

FIRST SEQUENCE OF SWITCHINGS:

Switch the first two


\[\xymatrix@C=15pt{L_1\ar[d]_{f_0\circ i_1}&& K_1 \ar[d]_{k'_1}\ar[ll]_{l_1} \ar[r]^{r_1} & R_1 \ar@/^.35cm/[drrr]_(.4){a_0}|(.285)\hole \ar[dr]|(.28)\hole_{h'_1} && L_0 \ar@/_.35cm/[dlll]^(.4){a_1} \ar[dl]|(.28)\hole^{m'_0}& K_0 \ar[d]^{k'_0}\ar[l]_{l_0} \ar[r]^{r_0} & R_0 \ar[dr]|(.28)\hole_{g_1\circ j_0} \ar@/^.35cm/[drrr]_(.4){b_0}|(.285)\hole  && L_2 \ar@/_.35cm/[dlll]^(.4){b_1} \ar[dl]|(.28)\hole^{m_2}& K_2 \ar[d]^{k_2}\ar[l]_{l_2} \ar[rr]^{r_2} && R_2 \ar[d]^{h_2} \\G_0 && \ar[ll]^{f'_0} D'_0 \ar[rr]_{g'_0}&& G'_1  && \ar[ll]^{f'_1} D'_1 \ar[rr]_{g'_1}&& G_2 && \ar[ll]^{f_2} D_2 \ar[rr]_{g_2}&& G_3 }\]

Moreover, we know that
\[m_0=f'_0\circ a_1 \qquad h_1=g'_1\circ a_0\]

Switch second and third
\[\xymatrix@C=15pt{L_1\ar[d]_{f_0\circ i_1}&& K_1 \ar[d]_{k'_1}\ar[ll]_{l_1} \ar[r]^{r_1} & R_1 \ar@/^.35cm/[drrr]_(.4){c_0}|(.285)\hole \ar[dr]|(.28)\hole_{h'_1} && L_2 \ar@/_.35cm/[dlll]^(.4){c_1} \ar[dl]|(.28)\hole^{f'_1\circ b_1}& K_2 \ar[d]^{k'_2}\ar[l]_{l_2} \ar[r]^{r_2} & R_2 \ar[dr]|(.28)\hole_{h'_2} \ar@/^.35cm/[drrr]_(.4){d_0}|(.285)\hole  && L_0 \ar@/_.35cm/[dlll]^(.4){d_1} \ar[dl]|(.28)\hole^{\hat{m}_0}& K_0 \ar[d]^{\hat{k}_0}\ar[l]_{l_0} \ar[rr]^{r_0} && R_0 \ar[d]^{g_2\circ b_0} \\G_0 && \ar[ll]^{f'_0} D'_0 \ar[rr]_{g'_0}&& G'_1  && \ar[ll]^{\hat{f}_1} \hat{D}_1 \ar[rr]_{\hat{g}_1}&& G'_2 && \ar[ll]^{f'_2} D'_2 \ar[rr]_{g'_2}&& G_3 }\]
And we know that
\[m'_0=\hat{f}_1\circ d_1 \qquad h_2=g'_2\circ d_0\]
\todo{qua abbiamo usato il secondo lemma dei tre passi (quello condizionale)}Since the rewriting system is very tame and using the three steps lemmas we can characterize $c_0\colon R_1\to \hat{D}_1$ and $c_1\colon L_2\to D'_0$ as the unique arrows fitting in the diagrams
\[\xymatrix@R=15pt@C=15pt{&&R_1 \ar@/_1.2cm/[dddl]_(.4){a_0}|(.63)\hole\ar[dl]^{j_0}\ar[dd]^{c'_0} \ar@{.>}[dr]^{c_0}\\&D_2 \ar[dl]^(.4){f_2}&&\hat{D}_1 \ar[dr]^{\hat{g}_1}\\G_2 && P\ar[dr]_{p_4} \ar[ur]_{p_3}\ar[dl]^{p_1} \ar[ul]^{p_2}&&G'_2\\& D'_1 \ar[ul]^{g'_1}&&D'_2 \ar[ur]_{f'_2}}\]

\[\xymatrix@R=15pt@C=15pt{&&L_2 \ar[dl]^{j_1} \ar@/_1.2cm/[dddl]_(.4){z_1}|(.63)\hole\ar[dd]^{c'_1} \ar@{.>}[dr]^{c_1}\\&D_1 \ar[dl]^(.4){f_1}&&D'_0 \ar[dr]^{g'_0}\\G_1 && Q\ar[dr]_{q_4} \ar[ur]_{q_3}\ar[dl]^{q_1} \ar[ul]_{q_2}&&G'_1\\& D_0 \ar[ul]^{g_0}&&D'_1 \ar[ur]_{f'_1}}\]


Switch first and second
\[\xymatrix@C=15pt{L_2\ar[d]_{f'_0\circ c_1}&& K_2 \ar[d]_{\hat{k}_2}\ar[ll]_{l_2} \ar[r]^{r_2} & R_2 \ar@/^.35cm/[drrr]_(.4){e_0}|(.285)\hole \ar[dr]|(.28)\hole_{\hat{h}_2} && L_1 \ar@/_.35cm/[dlll]^(.4){e_1} \ar[dl]|(.28)\hole^{\hat{m}_1}& K_1 \ar[d]^{\hat{k}_1}\ar[l]_{l_1} \ar[r]^{r_1} & R_1 \ar@/^.35cm/[drrr]_(.4){t_0} \ar[dr]|(.28)\hole_{\hat{g}_1\circ c_0}   && L_0  \ar[dl]|(.28)\hole^{\hat{m}_0} \ar@/_.35cm/[dlll]^(.4){t_1}& K_0 \ar[d]^{\hat{k}_0}\ar[l]_{l_0} \ar[rr]^{r_0} && R_0 \ar[d]^{g_2\circ b_0} \\G_0 && \ar[ll]^{\hat{f}_0} \hat{D}_0 \ar[rr]_{\hat{g}_0}&& \hat{G}_1  && \ar[ll]^{\tilde{f}_1} \tilde{D}_1 \ar[rr]_{\tilde{g}_1}&& G'_2 && \ar[ll]^{f'_2} D'_2 \ar[rr]_{g'_2}&& G_3 }\]
Moreover
\[f_0\circ i_1=\hat{f}_0\circ e_1 \qquad h'_2=\tilde{g}_1\circ e_0\]
and $(t_0, t_1)$ is characterized using again very tameness and the first $3$-steps Lemma as the unique arrows fitting in:
\[\xymatrix{&R_1 \ar[dl]_{a_0} \ar@{.>}[dr]^{t_0} \ar@/^.4cm/[dd]|\hole^(.6){c_0}\ar[d]_{t'_0}\\ D'_1 \ar[d]_{g'_1}&P\ar[d]_{p_3} \ar[r]^{p_4} \ar[l]_{p_1}&D'_2\ar[d]^{f'_2}\\G'_1&\hat{D}_1 \ar[l]^{\hat{f}_1}\ar[r]_{\hat{g}_1}& G'_2}\]



\[\xymatrix@R=15pt@C=15pt{&&L_ 0\ar[dl]^{a_1} \ar@/_1.2cm/[dddl]_(.4){d_1}|(.63)\hole\ar[dd]^{t'_1} \ar@{.>}[dr]^{t_1}\\&D'_0 \ar[dl]^(.4){g'_0}&&\tilde{D}_1 \ar[dr]^{\tilde{f}_1}\\G'_1 && S\ar[dr]_{s_4} \ar[ur]_{s_3}\ar[dl]^{s_1} \ar[ul]_{s_2}&&\hat{G}_1\\& \hat{D}_1 \ar[ul]^{\hat{f}_1}&&\hat{D}_0 \ar[ur]_{\hat{g}_0}}\]


SECOND SEQUENCE OF SWITCHINGS:

Switch third and second


\[\xymatrix@C=15pt{L_0\ar[d]_{m_0}&& K_0 \ar[d]_{k_0}\ar[ll]_{l_0} \ar[r]^{r_0} & R_0 \ar@/^.35cm/[drrr]_(.4){z_0}|(.285)\hole \ar[dr]|(.28)\hole_{h_0} && L_2 \ar@/_.35cm/[dlll]^(.4){z_1} \ar[dl]|(.28)\hole^{f_1\circ j_1}& K_2 \ar[d]^{\check{k}_2}\ar[l]_{l_2} \ar[r]^{r_2} & R_2 \ar[dr]|(.28)\hole_{\check{h}_2} \ar@/^.35cm/[drrr]_(.4){w_0}|(.285)\hole  && L_1 \ar@/_.35cm/[dlll]^(.4){w_1} \ar[dl]|(.28)\hole^{\check{m}_1}& K_1 \ar[d]^{\check{k}_1}\ar[l]_{l_1} \ar[rr]^{r_1} && R_1\ar[d]^{g_2\circ j_0} \\G_0 && \ar[ll]^{f_0} D_0 \ar[rr]_{g_0}&& G_1  && \ar[ll]^{\check{f}_1} \check{D}_1 \ar[rr]_{\check{g}_1}&& \check{G}_2 && \ar[ll]^{\check{f}_2} \check{D}_2 \ar[rr]_{\check{g}_2}&& G_3 }\]
Moreover
\[m_1=\check{f}_1\circ w_1 \qquad h_2=\check{g}_2\circ w_0\]

Switch second and first
\[\xymatrix@C=15pt{L_2\ar[d]_{f_0\circ z_1}&& K_2 \ar[d]_{\mathring{k}_2}\ar[ll]_{l_2} \ar[r]^{r_2} & R_2 \ar@/^.35cm/[drrr]_(.4){y_0}|(.285)\hole \ar[dr]|(.28)\hole_{\mathring{h}_2} && L_0 \ar@/_.35cm/[dlll]^(.4){y_1} \ar[dl]|(.28)\hole^{\check{m}_0}& K_0 \ar[d]^{\check{k}_0}\ar[l]_{l_0} \ar[r]^{r_0} & R_0 \ar[dr]|(.28)\hole_{\check{g}_1\circ z_0} \ar@/^.35cm/[drrr]_(.4){x_0}|(.285)\hole  && L_1 \ar@/_.35cm/[dlll]^(.4){x_1} \ar[dl]|(.28)\hole^{\check{m}_1}& K_1 \ar[d]^{\check{k}_1}\ar[l]_{l_1} \ar[rr]^{r_1} && R_1\ar[d]^{g_2\circ j_0} \\G_0 && \ar[ll]^{\check{f}_0} \check{D}_0 \ar[rr]_{\check{g}_0}&& \check{G}_1  && \ar[ll]^{\mathring{f}_1} \mathring{D}_1 \ar[rr]_{\mathring{g}_1}&& \check{G}_2 && \ar[ll]^{\check{f}_2} \check{D}_2 \ar[rr]_{\check{g}_2}&& G_3 }\]
As before we have
\[m_0=\check{f}_0\circ y_1 \qquad \check{h}_2=\mathring{g}_1\circ y_0\]
The really important thing to notice is that, by very tameness we have
\begin{align*}
	f_0'\circ c_1 & =f'_0\circ q_3\circ c'_1 \\&=f_0\circ q_1\circ c'_1\\&=f_0\circ z_1
\end{align*}

\todo{GOOD! i primi match sono uguali, ora giochiamoci il primo lemma dei tre passi}

Switch second and third
\[\xymatrix@C=15pt{L_2\ar[d]_{f_0\circ z_1}&& K_2 \ar[d]_{\mathring{k}_2}\ar[ll]_{l_2} \ar[r]^{r_2} & R_2 \ar@/^.35cm/[drrr]_(.4){v_0}|(.285)\hole \ar[dr]|(.28)\hole_{\mathring{h}_2} && L_1 \ar@/_.35cm/[dlll]^(.4){v_1} \ar[dl]|(.28)\hole^{\mathring{f}_1 \circ x_1}& K_1 \ar[d]^{\mathring{k}_1}\ar[l]_{l_1} \ar[r]^{r_1} & R_1 \ar@/^.35cm/[drrr]_(.4){u_0}|(.285)\hole\ar[dr]_{\check{h}_1}   && L_0 \ar@/_.35cm/[dlll]^(.4){u_1} \ar[dl]^{\mathring{m}_0}& K_0 \ar[d]^{\mathring{k}_0}\ar[l]_{l_0} \ar[rr]^{r_0} && R_0\ar[d]^{\check{g}_2\circ x_0} \\G_0 && \ar[ll]^{\check{f}_0} \check{D}_0 \ar[rr]_{\check{g}_0}&& \check{G}_1  && \ar[ll]^{\bar{f}_1} \bar{D}_1 \ar[rr]_{\bar{g}_1}&& \bar{G}_2 && \ar[ll]^{\bar{f}_2} \bar{D}_2 \ar[rr]_{\bar{g}_2}&& G_3 }\]

\[\xymatrix@R=15pt@C=15pt{&&R_1 \ar@/_1.2cm/[dddl]_(.4){a_0}|(.63)\hole\ar[dl]^{j_0}\ar[dd]^{c'_0} \ar@{.>}[dr]^{c_0}\\&D_2 \ar[dl]^(.4){f_2}&&\hat{D}_1 \ar[dr]^{\hat{g}_1}\\G_2 && P\ar[dr]_{p_4} \ar[ur]_{p_3}\ar[dl]^{p_1} \ar[ul]^{p_2}&&G'_2\\& D'_1 \ar[ul]^{g'_1}&&D'_2 \ar[ur]_{f'_2}}\]

\[\xymatrix@R=15pt@C=15pt{&&L_1 \ar[dl]^{w_1} \ar@/_1.2cm/[dddl]_(.4){i_1}|(.63)\hole\ar[dd]^{v'_1} \ar@{.>}[dr]^{v_1}\\&\check{D}_1 \ar[dl]^(.4){\check{f}_1}&&\check{D}_0 \ar[dr]^{g'_0}\\G_1 && \check{Q}\ar[dr]_{\check{q}_4} \ar[ur]_{\check{q}_3}\ar[dl]^{\check{q}_1} \ar[ul]_{\check{q}_2}&&\check{G}_1\\& D_0 \ar[ul]^{g_0}&&\mathring{D}_1 \ar[ur]_{\check{f}_1}}\]




HUGE DIAGRAM


\[\xymatrix@C=15pt{
	& K_1 \ar[d]^{\hat{k}_1}\ar[l]_{l_1} \ar[r]^{r_1} & R_1 \ar@/^.35cm/[drrr]_(.4){t_0} \ar[dr]|(.28)\hole_{\hat{g}_1\circ c_0}   && L_0  \ar[dl]|(.28)\hole^{\hat{m}_0} \ar@/_.35cm/[dlll]^(.4){t_1}& K_0 \ar[d]^{\hat{k}_0}\ar[l]_{l_0} \ar[rr]^{r_0} && R_0 \ar[d]^{g_2\circ b_0} \\G_0 && \ar[ll]^{\hat{f}_0} \hat{D}_0 \ar[rr]_{\hat{g}_0}&& \hat{G}_1  && \ar[ll]^{\tilde{f}_1} \tilde{D}_1 \ar[rr]_{\tilde{g}_1}&& G'_2 && \ar[ll]^{f'_2} D'_2 \ar[rr]_{g'_2}&& G_3 \\
	L_2\ar[d]_{f_0\circ z_1} \ar[u]^{f'_0\circ c_1}&& K_2 \ar[d]_{\mathring{k}_2}\ar[ll]_{l_2} \ar[r]^{r_2} & R_2 \ar@/_.35cm/[urrr]^(.4){e_0}|(.285)\hole \ar[ur]|(.28)\hole^{\hat{h}_2} \ar@/^.35cm/[drrr]_(.4){v_0}|(.285)\hole \ar[dr]|(.28)\hole_{\mathring{h}_2} && L_1 \ar@/^.35cm/[ulll]_(.4){e_1} \ar[ul]|(.28)\hole_{\hat{m}_1} \ar@/_.35cm/[dlll]^(.4){v_1} \ar[dl]|(.28)\hole^{\mathring{f}_1 \circ x_1}& K_1 \ar[d]^{\mathring{k}_1}\ar[l]_{l_1} \ar[r]^{r_1} & R_1 \ar@/^.35cm/[drrr]_(.4){u_0}|(.285)\hole\ar[dr]_{\check{h}_1}   && L_0 \ar@/_.35cm/[dlll]^(.4){u_1} \ar[dl]^{\mathring{m}_0}& K_0 \ar[d]^{\mathring{k}_0}\ar[l]_{l_0} \ar[rr]^{r_0} && R_0\ar[d]^{\check{g}_2\circ x_0} \\G_0 && \ar[ll]^{\check{f}_0} \check{D}_0 \ar[rr]_{\check{g}_0}&& \check{G}_1  && \ar[ll]^{\bar{f}_1} \bar{D}_1 \ar[rr]_{\bar{g}_1}&& \bar{G}_2 && \ar[ll]^{\bar{f}_2} \bar{D}_2 \ar[rr]_{\bar{g}_2}&& G_3 }\]

Now, by determinism we have isomorphisms $\phi_{\check{D}_0}\colon \check{D}_0\to \hat{D}_0$ $\phi_{\check{G}_1}\colon \check{G}_1\to \hat{G}_1$
Notice that
\begin{align*}
	\hat{f}_0\circ \phi_{\check{D}_0}\circ v_1 & = \check{f}_0\circ v_1 \\&=\check{f}_0\circ \check{q}_3\circ v'_1\\&=f_0\circ \check{q}_1\circ v'_1\\&=f_0\circ i_1\\&=\hat{f}_0\circ e_1
\end{align*}

GOOD: $\phi_{\check{D}_0}$ sends $v_1$ to $e_1$, thus it respects the match of $L_1$. Automatically it sends $v_0$ to $e_0$.

CHECK THAT $\check{g}_2\circ x_0=g_2\circ j_0$

USO IL PRIMO LEMMA DEI TRE PASSI:

\[\xymatrix{&R_0 \ar[dl]_{i_0} \ar@{.>}[dr]^{x_0} \ar@/^.4cm/[dd]|\hole^(.6){j_0}\ar[d]_{x'_0}\\ D_1 \ar[d]_{g_1}&P'\ar[d]_{p'_3} \ar[r]^{p'_4} \ar[l]_{p'_1}&\check{D}_2\ar[d]^{\check{g}_2}\\G_2&D_2 \ar[l]^{f_2}\ar[r]_{g_2}& G_3}\]



\[\xymatrix@R=15pt@C=15pt{&&L_ 0\ar[dl]^{I_1} \ar@/_1.2cm/[dddl]_(.4){d_1}|(.63)\hole\ar[dd]^{t'_1} \ar@{.>}[dr]^{t_1}\\&D'_0 \ar[dl]^(.4){g'_0}&&\tilde{D}_1 \ar[dr]^{\tilde{f}_1}\\G'_1 && S\ar[dr]_{s_4} \ar[ur]_{s_3}\ar[dl]^{s_1} \ar[ul]_{s_2}&&\hat{G}_1\\& \hat{D}_1 \ar[ul]^{\hat{f}_1}&&\hat{D}_0 \ar[ur]_{\hat{g}_0}}\]
\end{proof}


\lemIndepGlobalLeft*
\label{lemIndepGlobalLeft-proof}

\begin{proof}
  As a preliminary step, we are going to use
  \Cref{def:filler,def:switch} to get some diagrams.  First of all,
  let $(v,v')$ be the filler between $\dder{D}_0$ and $\dder{D}_1$
  associated to $(i_0, i_1)$, then we have
  \[\xymatrix{&&R_0 \ar@/_1cm/[ddrr]_(.35){i_0}|(.7)\hole
  	\ar[dr]^{h_0}&& L_1\ar@/^1cm/[ddll]^(.35){i_1}
  	\ar[dl]_{m_1}\\&K_0\ar[dr]^{k_0}\ar[dl]_{l_0}
  	\ar@/_.8cm/[ddrr]|(.36)\hole^(.65){v}\ar[ur]^{r_0}&& G_1 &&
  	K_1
  	\ar@/^.8cm/[ddll]|(.36)\hole_(.65){v'}\ar[dl]_{k_1}\ar[lu]_{l_1}
  	\ar[dr]^{r_1}\\L_0
  	\ar@/_.8cm/[ddrr]_(.2){j_0}|(.31)\hole|(.81)\hole
  	\ar[dr]^(.4){m_0}|(.61)\hole && D_0
  	\ar[dl]|(.4)\hole_(.65){f_0}\ar[ur]|(.5)\hole^(.7){g_0}&&D_1
  	\ar[dr]|(.4)\hole^(.65){g_1}
  	\ar[ul]|(.5)\hole_(.65){f_1}&&R_1\ar@/^.8cm/[ddll]^(.2){j_1}|(.31)\hole|(.81)\hole\ar[dl]_{h_1}|(.6)\hole\\&G_0
  	&&P_1\ar[dr]_{q_1}
  	\ar[dl]^{q_0}\ar[ur]^(.4){p_1}\ar[ul]_(.4){p_0}&&G_2\\L_1
  	\ar@/^.8cm/[uurr]^(.2){i_1} \ar[ur]_(.35){f_0\circ
  		i_1}|(.61)\hole&&Q_0
  	\ar[ul]|(.4)\hole_(.65){s_0}\ar[dr]|(.5)\hole^(.65){t_0}
  	&&Q_1\ar[ur]|(.4)\hole^(.65){t_1} \ar[dl]|(.5)\hole_(.65){s_1}
  	&& R_0 \ar[ul]^(.35){g_1\circ
  		i_0}|(.61)\hole\ar@/_.8cm/[uull]_(.2){i_0}\\&K_1
  	\ar[ur]_{q_0\circ v'} \ar[dr]_{r_1}
  	\ar[ul]^{l_1}\ar@/^.8cm/[uurr]_(.65){v'}&&H'_1&& K_0
  	\ar[ur]_{r_0} \ar[ul]^{q_1\circ v} \ar[dl]^{l_0}
  	\ar@/_.8cm/[uull]^(.65){v}\\&& R_1
  	\ar[ur]_{\hspace{-5pt}s_1\circ
  		j_1}\ar@/^1cm/[uurr]^(.25){j_1}&& L_0 \ar[ul]^{t_0\circ
  		j_0\hspace{-5pt}} \ar@/_1cm/[uull]_(.25){j_0} |(.69)\hole
  }\]

  Secondly, the filler $(u,u')$ induced by $(a_0, a_1)$ between
  $\dder{D}_1$ and $\dder{D}_2$ yields:
  \[
    \xymatrix{&&R_1 \ar@/_1cm/[ddrr]_(.35){a_0}|(.7)\hole
      \ar[dr]^{h_1}&& L_2\ar@/^1cm/[ddll]^(.35){a_1}
      \ar[dl]_{m_2}\\&K_1\ar[dr]^{k_1}\ar[dl]_{l_1}
      \ar@/_.8cm/[ddrr]|(.36)\hole^(.65){u}\ar[ur]^{r_1}&& G_2 &&
      K_2
      \ar@/^.8cm/[ddll]|(.36)\hole_(.65){u'}\ar[dl]_{k_2}\ar[lu]_{l_2}
      \ar[dr]^{r_2}\\L_1
      \ar@/_.8cm/[ddrr]_(.2){b_0}|(.31)\hole|(.81)\hole
      \ar[dr]^(.4){m_1}|(.61)\hole && D_1
      \ar[dl]|(.4)\hole_(.65){f_1}\ar[ur]|(.5)\hole^(.7){g_1}&&D_2
      \ar[dr]|(.4)\hole^(.65){g_2}
      \ar[ul]|(.5)\hole_(.65){f_2}&&R_2\ar@/^.8cm/[ddll]^(.2){b_1}|(.31)\hole|(.81)\hole\ar[dl]_{h_2}\\&G_1
      &&P_2\ar[dr]_{d_1}
      \ar[dl]^{d_0}\ar[ur]^(.4){c_1}\ar[ul]_(.4){c_0}&&G_3\\L_2
      \ar@/^.8cm/[uurr]^(.2){a_1} \ar[ur]_(.35){f_1\circ
        a_1}|(.61)\hole&&Q_2
      \ar[ul]|(.4)\hole_(.65){x_1}\ar[dr]|(.5)\hole^(.65){y_1}
      &&Q_3\ar[ur]|(.4)\hole^(.65){y_2} \ar[dl]|(.5)\hole_(.65){x_2}
      && R_1 \ar[ul]^(.35){g_2\circ
        a_0}|(.61)\hole\ar@/_.8cm/[uull]_(.2){a_0}\\&K_2
      \ar[ur]_{d_0\circ u'} \ar[dr]_{r_2}
      \ar[ul]^{l_2}\ar@/^.8cm/[uurr]_(.65){u'}&&G'_2&& K_1
      \ar[ur]_{r_1} \ar[ul]^{d_1\circ u} \ar[dl]^{l_1}
      \ar@/_.8cm/[uull]^(.65){u}\\&& R_2
      \ar[ur]_{\hspace{-5pt}x_2\circ
        b_1}\ar@/^1cm/[uurr]^(.25){b_1}&& L_1 \ar[ul]^{y_1\circ
        b_0\hspace{-5pt}} \ar@/_1cm/[uull]_(.25){b_0} |(.69)\hole
    }\]


  Finally, the filler $(w,w')$ between $\dder{D}_0$ and
  $S_{a_0,a_1}(\dder{D}_2)$ given by $(e_0, e_1)$ provides us with:
  \[\xymatrix{&&R_0 \ar@/_1cm/[ddrr]_(.35){e_0}|(.7)\hole
      \ar[dr]^{h_0}&& L_2\ar@/^1cm/[ddll]^(.35){e_1}
      \ar[dl]_{f_1\circ a_1}\\&K_0\ar[dr]^{k_0}\ar[dl]_{l_0}
      \ar@/_.8cm/[ddrr]|(.36)\hole^(.65){w}\ar[ur]^{r_0}&& G_1 &&
      K_2 \ar@/^.8cm/[ddll]|(.36)\hole_(.65){w'}\ar[dl]_{d_0\circ
        u'\hspace{-5pt}}\ar[lu]_{l_2} \ar[dr]^{r_2}\\L_0
      \ar@/_.8cm/[ddrr]_(.2){o_0}|(.31)\hole|(.81)\hole
      \ar[dr]^(.4){m_0}|(.61)\hole && D_0
      \ar[dl]|(.4)\hole_(.65){f_0}\ar[ur]|(.5)\hole^(.7){g_0}&&Q_2
      \ar[dr]|(.4)\hole^(.65){y_1}
      \ar[ul]|(.5)\hole_(.65){x_1}&&R_2\ar@/^.8cm/[ddll]^(.2){o_1}|(.31)\hole|(.81)\hole\ar[dl]_(.35){x_2\circ
        b_1}\\&G_0 &&P_3\ar[dr]_{n_1}
      \ar[dl]^{n_0}\ar[ur]^(.4){u_1}\ar[ul]_(.4){u_0}&&G'_2\\L_2
      \ar@/^.8cm/[uurr]^(.2){e_1} \ar[ur]_(.35){f_0\circ
        e_1}|(.61)\hole&&Q_4
      \ar[ul]|(.4)\hole_(.65){z_0}\ar[dr]|(.5)\hole^(.65){z'_0}
      &&Q_5\ar[ur]|(.4)\hole^(.65){z'_1}
      \ar[dl]|(.5)\hole_(.65){z_1} && R_0 \ar[ul]^(.35){y_1\circ
        e_0}|(.61)\hole\ar@/_.8cm/[uull]_(.2){e_0}\\&K_2
      \ar[ur]_{n_0\circ w'} \ar[dr]_{r_2}
      \ar[ul]^{l_2}\ar@/^.8cm/[uurr]_(.65){w'}&&G'_1&& K_0
      \ar[ur]_{r_0} \ar[ul]^{n_1\circ w} \ar[dl]^{l_0}
      \ar@/_.8cm/[uull]^(.65){w}\\&& R_2
      \ar[ur]_{\hspace{-4pt}z_1\circ
        o_1}\ar@/^1cm/[uurr]^(.25){o_1}&& L_0 \ar[ul]^{z'_0\circ
        o_0\hspace{-5pt}} \ar@/_1cm/[uull]_(.25){o_0} |(.69)\hole
    }\]
  So equipped we can turn to the prove of our claims.
  \begin{enumerate}
  \item We have to construct the two dotted arrows in the diagram
    below.
    \[\xymatrix@C=15pt{L_0 \ar[d]_{z'_0\circ o_0}&& K_0
        \ar[d]_{n_1\circ w}\ar[ll]_{l_0} \ar[r]^{r_0} & R_0
        \ar@{.>}@/^.35cm/[drrr]_(.4){\beta_0}|(.285)\hole
        \ar[dr]|(.28)\hole_{y_1\circ e_0} && L_1
        \ar@{.>}@/_.35cm/[dlll]^(.4){\beta_1}
        \ar[dl]|(.28)\hole^{y_1\circ b_0}& K_1 \ar[d]^{d_1\circ
          u}\ar[l]_{l_1} \ar[rr]^{r_1} && R_1 \ar[d]^{g_2\circ
          a_1}\\G'_1 && \ar[ll]^{z_1} Q_5 \ar[rr]_{z'_1}&& G'_2 &&
        \ar[ll]^{x_2} Q_3 \ar[rr]_{y_2}&& G_3}\]

    Consider the arrows $i_0\colon R_0\to D_1$ and
    $e_0\colon R_0\to Q_2$. An easy computation shows that
    \begin{align*}
      f_1\circ i_0 & = h_0 \\&=x_1\circ e_0
    \end{align*}
    entailing the existence of the dotted
    $\beta'_0\colon R_0\to P_2$ in the diagram
    \[\xymatrix{R_0 \ar@{.>}[dr]^{\beta'_0} \ar@/^.3cm/[drr]^{i_0}
        \ar@/_.3cm/[ddr]_{e_0}\\ &P_2 \ar[r]^{c_1} \ar[d]_{d_0}& D_1
        \ar[d]^{f_1}\\ &Q_2\ar[r]_{x_1} & G_1}\]
    If we define $\beta_0\colon R_0\to Q_3$ as $d_1\circ \beta'_1$,
    then we easily get that
    \begin{align*}
      x_2\circ \beta_0 & =x_2\circ d_2\circ \beta'_0 \\&= y_1\circ d_0\circ \beta'_0\\&=y_1\circ e_0
    \end{align*}

    To define $\beta_1$, we proceed similarly. First consider
    $i_1\colon L_1\to D_0$ and $b_0\colon L_1 \to Q_2$ and notice
    that
    \begin{align*}
      g_0\circ i_1 & = m_1 \\&= x_1 \circ b_0
    \end{align*}
    implying the existence of $\beta'_1\colon L_1\to P_3$ fitting in
    the diagram below.
    \[\xymatrix{L_1 \ar@{.>}[dr]^{\beta'_1} \ar@/^.3cm/[drr]^{i_1}
        \ar@/_.3cm/[ddr]_{b_0}\\ &P_3 \ar[r]^{u_0} \ar[d]_{u_1}& D_0
        \ar[d]^{g_0}\\ &Q_2\ar[r]_{x_1} & G_1}\]
    Let $\beta_1\colon L_1\to Q_5$ be $n_1\circ \beta'_1$, then
    \begin{align*}
      z'_1 \circ \beta_1 & = z'_1 \circ n_1\circ \beta'_1 \\&=y_1\circ u_1\circ \beta'_1\\&=y_1\circ b_0
    \end{align*}
    Therefore, $(\beta_0, \beta_1)$ is the wanted independence pair.

  \item In addition to the three diagram above, we have a fourth one
    given by the filler $(\varphi_0, \varphi_1)$ associated to
    $(\alpha_0, \alpha_1)$.
    \[\xymatrix{&&R_0 \ar@/_1cm/[ddrr]_(.35){\alpha_0}|(.7)\hole \ar[dr]^{g_1\circ i_0}&& L_2\ar@/^1cm/[ddll]^(.35){\alpha_1}  \ar[dl]_{m_2}\\
        &K_0\ar[dr]^{\hspace{-4pt}q_1\circ v}\ar[dl]_{l_0} \ar@/_.8cm/[ddrr]|(.36)\hole^(.65){\varphi}\ar[ur]^{r_0}&& G_2 && K_2 \ar@/^.8cm/[ddll]|(.36)\hole_(.65){\varphi'}\ar[dl]_{k_2}\ar[lu]_{l_2} \ar[dr]^{r_2}\\
        L_0 \ar@/_.8cm/[ddrr]_(.2){\gamma_0}|(.31)\hole|(.81)\hole \ar[dr]^(.4){\hspace{-4pt}t_0\circ j_0}|(.61)\hole && Q_1 \ar[dl]|(.4)\hole_(.65){s_1}\ar[ur]|(.5)\hole^(.7){t_1}&&D_2 \ar[dr]|(.4)\hole^(.65){g_2} \ar[ul]|(.5)\hole_(.65){f_2}&&R_2\ar@/^.8cm/[ddll]^(.2){\gamma_1}|(.31)\hole|(.81)\hole\ar[dl]_(.35){h_2}\\&H'_1 &&P_4\ar[dr]_{\lambda_1} \ar[dl]^{\lambda_0}\ar[ur]^(.4){\zeta_1}\ar[ul]_(.4){\zeta_0}&&G_3\\
        L_2 \ar@/^.8cm/[uurr]^(.2){\alpha_1} \ar[ur]_(.35){s_1\circ
          \alpha_1}|(.61)\hole&&Q_6
        \ar[ul]|(.4)\hole_(.65){\xi_0}\ar[dr]|(.5)\hole^(.65){\xi'_0}
        &&Q_7\ar[ur]|(.4)\hole^(.65){\xi'_1}
        \ar[dl]|(.5)\hole_(.65){\xi_1} && R_0 \ar[ul]^(.35){g_2\circ
          \alpha_0}|(.61)\hole\ar@/_.8cm/[uull]_(.2){\alpha_0}\\&K_2
        \ar[ur]_{\lambda_0\circ \phi'} \ar[dr]_{r_2}
        \ar[ul]^{l_2}\ar@/^.8cm/[uurr]_(.65){\phi'}&&H'_2&& K_0
        \ar[ur]_{r_0} \ar[ul]^{\lambda_1\circ \phi} \ar[dl]^{l_0}
        \ar@/_.8cm/[uull]^(.65){\phi}\\&& R_2
        \ar[ur]_{\hspace{-4pt}\xi_1\circ
          \gamma_1}\ar@/^1cm/[uurr]^(.25){\gamma_1}&& L_0
        \ar[ul]^{\xi'_0\circ \gamma_0\hspace{-5pt}}
        \ar@/_1cm/[uull]_(.25){\gamma_0} |(.69)\hole }\]

    Our aim is to construct the dotted arrow in the following
    diagram.
    \[\xymatrix@C=15pt{L_1 \ar[d]_{f_0\circ i_0}&& K_1
        \ar[d]_{q_0\circ v'}\ar[ll]_{l_1} \ar[r]^{r_1} & R_1
        \ar@{.>}@/^.35cm/[drrr]_(.4){\epsilon_0}|(.285)\hole
        \ar[dr]|(.28)\hole_{s_1\circ j_1} && L_2
        \ar@{.>}@/_.35cm/[dlll]^(.4){\epsilon_1}
        \ar[dl]|(.28)\hole^{s_1\circ \alpha_1}& K_2
        \ar[d]^{\lambda_0\circ \phi'}\ar[l]_{l_2} \ar[rr]^{r_2} &&
        R_2 \ar[d]^{\xi_1\circ \gamma_1}\\G_0 && \ar[ll]^{s_0} Q_0
        \ar[rr]_{t_0}&& H'_1 && \ar[ll]^{\xi_0} Q_6
        \ar[rr]_{\xi'_0}&& H'_2}\]

    Let us start considering $j_1\colon R_1\to Q_1$ and
    $a_0\colon R_1\to D_2$. We have
    \begin{align*}
      f_2\circ a_0 & =h_1 \\&=t_1\circ j_1
    \end{align*}
    and thus we get an arrow $\epsilon'_0\colon R_1\to P_4$ which
    makes the diagram below commutative.
    \[\xymatrix{R_1 \ar@{.>}[dr]^{\epsilon'_0}
        \ar@/^.3cm/[drr]^{j_1} \ar@/_.3cm/[ddr]_{a_0}\\ &P_4
        \ar[r]^{\zeta_0} \ar[d]_{\zeta_1}& Q_1 \ar[d]^{t_1}\\
        &D_2\ar[r]_{f_2} & G_2}\]
    We can then define $\epsilon_0\colon R_1\to Q_6$ to be
    $\lambda_0\circ \epsilon'_0$. For such an arrow we have a chain
    of identities:
    \begin{align*}
      \xi_0\circ \epsilon_0 & =\xi_0\circ \lambda_0\circ \epsilon'_0 \\&=s_1\circ \zeta_0 \circ \epsilon'_0\\&=s_1\circ j_1
    \end{align*}

    Next, to define $\epsilon_1\colon L_2\to Q_0$ we take
    $e_1\colon L_2 \to D_0$ and $a_1\colon L_2\to D_1$. By
    definition we have $ g_0\circ e_1=f_1\circ a_1$, giving us the
    dotted arrow below
    \[\xymatrix{L_2 \ar@{.>}[dr]^{\epsilon'_1}
        \ar@/^.3cm/[drr]^{a_1} \ar@/_.3cm/[ddr]_{e_0}\\ &P_1
        \ar[r]^{p_1} \ar[d]_{p_0}& D_1 \ar[d]^{f_1}\\
        &D_0\ar[r]_{g_0} & G_1}\]
    Now, notice that
    \begin{align*}
      t_1\circ q_1\circ \epsilon'_1 & =g_1\circ p_1\circ \epsilon'_1 & =g_1\circ a_1 \\&=m_2
    \end{align*}
    Hence $(\alpha_0, q_1\circ \epsilon'_1)$ is an independence pair
    for $S_{i_0, i_1}(\dder{D}_0)$ and $\dder{D}_2$. By hypothesis
    $S_{i_0, i_1}(\dder{D}_0)\updownarrow_! \dder{D}_2$ and thus
    $q_1\circ \epsilon'_1$ must coincide with $\alpha_1$. Now, let
    $\epsilon_1\colon L_2\to Q_0$ be $q_0\circ
    \epsilon'_1$. Computing we get
    \begin{align*}
      t_0\circ \epsilon_1 & = t_0\circ q_0\circ \epsilon'_1 \\&=s_1\circ q_1\circ \epsilon'_1\\&=s_1\circ \alpha_1
    \end{align*}
    Allowing us to conclude that
    $S_{i_0,i_1}(\dder{D}_1)\updownarrow S_{\alpha_0,
      \alpha_1}(\dder{D}_2)$.  \qedhere
  \end{enumerate}
\end{proof}


\end{document}

%%% Local Variables:
%%% mode: latex
%%% TeX-master: t
%%% End:
