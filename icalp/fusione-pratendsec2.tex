

\makeatletter\Hy@SaveLastskip\label{proofsection:prAtEndii}\ifdefined\pratend@current@sectionlike@label\immediate\write\@auxout{\string\gdef\string\pratend@section@for@proofii{\pratend@current@sectionlike@label}}\fi\Hy@RestoreLastskip\makeatother\begin{proof}[Proof of \pratendRef{thm:prAtEndii}]\phantomsection\label{proof:prAtEndii}Both pairs $(k, f)$ and $(k', f')$ are pushout complements of $l$ and $n$, thus, \Cref {lem:pocomp} yields an isomorphism $\phi _D\colon D\to D'$. Computing we have \[ g'\circ \phi _D \circ k= g' \circ k'=h'\circ r\] \par \noindent \parbox {3cm}{ $\xymatrix {K \ar @/^.4cm/[rr]^{k'}\ar [d]_{r} \ar [r]_{k}& \ar [r]_{\phi _D} D \ar [d]^{g}& D' \ar [d]^{g'}\\ R \ar @/_.4cm/[rr]_{h'} \ar [r]^{h}& H \ar [r]^{\phi _H}& H'}$} \hfill \parbox {10cm}{ \hspace {15pt}Hence, we have another $\phi _H\colon H\to H'$. To see that $\phi _H$ is an isomorphism, consider the diagram aside. By hypothesis the whole rectangle and its left half are pushouts, therefore, by \Cref {lem:po1} its right square is a pushout too. The claim now follows from the fact that the pushout of an isomorphism is an isomorphism. \qedhere }\end{proof}
