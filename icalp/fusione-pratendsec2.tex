

\makeatletter\Hy@SaveLastskip\label{proofsection:prAtEndii}\ifdefined\pratend@current@sectionlike@label\immediate\write\@auxout{\string\gdef\string\pratend@section@for@proofii{\pratend@current@sectionlike@label}}\fi\Hy@RestoreLastskip\makeatother\begin{proof}[Proof of \pratendRef{thm:prAtEndii}]\phantomsection\label{proof:prAtEndii}We can start considering the cube below and noticing that $n$, being the pushout of $m\in \mathcal {M}$, belongs to $\mathcal {A}$ too. \[\xymatrix @C=13pt@R=13pt{&A\ar [dd]|\hole _(.65){\id {A}}\ar [rr]^{g} \ar [dl]_{\id {A}} && B \ar [dd]^{\id {B}} \ar [dl]_{\id {B}} \\ A \ar [dd]_{m}\ar [rr]^(.65){g} & & B \ar [dd]_(.3){n}\\&A\ar [rr]|\hole ^(.65){g} \ar [dl]^{m} && B \ar [dl]^{n} \\C \ar [rr]_{f} & & D}\] By construction the top face of the cube is a pushout and the back one a pullback. The left face is a pullback because $m$ is mono. Thus the $\mathcal {A}$-Van Kampen property yields that the front and the right faces are pullbacks.\end{proof}

\makeatletter\Hy@SaveLastskip\label{proofsection:prAtEndiii}\ifdefined\pratend@current@sectionlike@label\immediate\write\@auxout{\string\gdef\string\pratend@section@for@proofiii{\pratend@current@sectionlike@label}}\fi\Hy@RestoreLastskip\makeatother\begin{proof}[Proof of \pratendRef{thm:prAtEndiii}]\phantomsection\label{proof:prAtEndiii}$(\Rightarrow )$ By hypothesis there exists $k:D'\to B$ such that $n\circ k = d$. By \Cref {prop:pbpo}, the bottom face of the cube is a pullback. Thus there exists a unique $h:C'\to A$ as in the diagram below, implying the thesis. \[\xymatrix @C=40pt{C' \ar [d]^{h}\ar [r]^{f'} \ar @/_.5cm/[dd]_{c}& D' \ar @/^.5cm/[dd]^{d}\ar [d]_{k}\\ A \ar [d]^{m} \ar [r]^{g} & B \ar [d]_{n} \\ C \ar [r]_f & D}\] \par \smallskip \noindent $(\Leftarrow )$ Let $h:C\to A$ be such that $c=m\circ h$. By the $\mathcal {A}$-Van Kampen property the top face of the given cube is a pushout. Thus the dotted $k:D'\to B$ in the following diagram exists. \[\xymatrix @C=40pt{A' \ar @/_.5cm/[dd]_{a} \ar [d]^{m'} \ar [r]^{g'} & B' \ar [d]_{n'} \ar @/^.5cm/[dd]^{b} \\ C' \ar [d]^{h} \ar [r]_{f'} & D' \ar @{.>}[d]_{k}\\ A \ar [r]_g& B }\] Moreover, by construction we have \begin {align*} n\circ k \circ n'&= n\circ b=d\circ n'\\ n\circ k \circ f' &= n\circ g\circ h=f\circ m\circ h=f\circ c= d\circ f' \end {align*} We can therefore conclude that $n\circ k =d$.\end{proof}

\makeatletter\Hy@SaveLastskip\label{proofsection:prAtEndiv}\ifdefined\pratend@current@sectionlike@label\immediate\write\@auxout{\string\gdef\string\pratend@section@for@proofiv{\pratend@current@sectionlike@label}}\fi\Hy@RestoreLastskip\makeatother\begin{proof}[Proof of \pratendRef{thm:prAtEndiv}]\phantomsection\label{proof:prAtEndiv}We can begin noticing that $g$, being the pullback of $k$, is mono and in $\mathcal {A}$ too. Thus we can build the cube below, in which all the vertical faces are pullbacks, entailing that all the vertical arrows are in $\mathcal {A}$. \[\xymatrix @C=20pt@R=10pt{ & &X\ar [dl]_{f} \ar [ddd]^(.333333){\id {X}}|(.666666)\hole \ar [rrr]^{a} &&&A \ar [ddd]^{\id {A}} \ar [dl]^{h}\\& Y\ar [dl]_{\id {Y}} \ar [ddd]|(.333333)\hole _{\id {Y}} &&&B \ar [ddd]^{\id {B}} \ar [dl]_{\id {B}} \\Y \ar [ddd]_{g} \ar [rrr]^{b}&&& B\ar [ddd]^{k}\\&&X \ar [rrr]|(.34)\hole ^{a}|(.67)\hole \ar [dl]_{f}&&& A \ar [dl]^{h}\\ & Y \ar [dl]_{g}&&& B \ar [dl]^{k}\\ Z\ar [rrr]_{c} &&& C}\] By hypothesis the face is an $\mathcal {A}$-stable pushout and so its top one is a pushout. Using \Cref {lem:po1} we can conclude that the right half of the rectangle with which we have started is a pushout too.\end{proof}

\makeatletter\Hy@SaveLastskip\label{proofsection:prAtEndv}\ifdefined\pratend@current@sectionlike@label\immediate\write\@auxout{\string\gdef\string\pratend@section@for@proofv{\pratend@current@sectionlike@label}}\fi\Hy@RestoreLastskip\makeatother\begin{proof}[Proof of \pratendRef{thm:prAtEndv}]\phantomsection\label{proof:prAtEndv}Take the diagram \[\xymatrix {X \ar [d]_{\id {X}}\ar [r]^{f}& Y \ar [r]^{\id {Y}} \ar [d]_{\id {Y}}& Y \ar [d]^{m}\\ X \ar [r]_{f}& Y \ar [r]_{m} & Z}\] Since $m$ is mono the right square is a pullback, while the left square is a pullback by construction. By \Cref {lem:pb1} the whole rectangle is a pullback and the thesis follows.\end{proof}

\makeatletter\Hy@SaveLastskip\label{proofsection:prAtEndvi}\ifdefined\pratend@current@sectionlike@label\immediate\write\@auxout{\string\gdef\string\pratend@section@for@proofvi{\pratend@current@sectionlike@label}}\fi\Hy@RestoreLastskip\makeatother\begin{proof}[Proof of \pratendRef{thm:prAtEndvi}]\phantomsection\label{proof:prAtEndvi}Let $m$ be an element of $\mathcal {M}$ and consider its pushout along itself. \[\xymatrix {X\ar @{>->}[r]^m \ar @{>->}[d]_m& Y\ar @{>->}[d]^{f}\\Y \ar @{>->}[r]_g & Z}\] By \Cref {prop:pbpoad} this square is a pullback, proving that $m$ is the equalizer of the arrows $f,g\colon Y\rightrightarrows Z$.\end{proof}

\makeatletter\Hy@SaveLastskip\label{proofsection:prAtEndvii}\ifdefined\pratend@current@sectionlike@label\immediate\write\@auxout{\string\gdef\string\pratend@section@for@proofvii{\pratend@current@sectionlike@label}}\fi\Hy@RestoreLastskip\makeatother\begin{proof}[Proof of \pratendRef{thm:prAtEndvii}]\phantomsection\label{proof:prAtEndvii}\begin {enumerate} \item By \Cref {prop:stab}, it follows that both squares are pushouts, thus the thesis follows from \Cref {prop:pbpoad}. \item By hypothesis, $g$ is the pullback of an arrow in $\mathcal {M}$, thus it belongs to it. But then $g\circ f\in \mathcal {M}$ too and the whole rectangle is a $\mathcal {M}$-pushout. Therefore, by \Cref {prop:pbpoad} a pullback, so that its left half is a pullback too, by \Cref {prop:pbpo}. Moreover $k\circ h$ is in $\mathcal {M}$ as the pushout of $g\circ f$ and, by \Cref {cor:deco}, we also know that $h\in \mathcal {M}$. \par Using \Cref {lem:po1}, it is enough to show that the left half of the original rectangle is a pushout. We can build the following cube: \[\xymatrix @C=20pt@R=10pt{ & &X\ar @{>->}[dl]_{f} \ar [ddd]^(.333333){\id {X}}|(.666666)\hole \ar [rrr]^{a} &&&A \ar [ddd]^{\id {A}} \ar @{>->}[dl]^{h}\\& Y\ar [dl]_{\id {Y}} \ar [ddd]|(.333333)\hole _{\id {Y}} &&&B \ar [ddd]^{\id {B}} \ar [dl]_{\id {B}} \\Y \ar @{>->}[ddd]_{g} \ar [rrr]^{b}&&& B\ar @{>->}[ddd]^{k}\\&&X \ar [rrr]|(.34)\hole ^{a}|(.67)\hole \ar @{>->}[dl]_{f}&&& A \ar @{>->}[dl]^{h}\\ & Y \ar [rrr]|(.67)\hole ^{b} \ar @{>->}[dl]_{g}&&& B \ar @{>->}[dl]^{k}\\ Z\ar [rrr]_{c} &&& C}\] Its vertical faces are all pullbacks and all the vertical arrows are in $\mathcal {M}$, hence the top face is a pushout and we can conclude. \qedhere \end {enumerate}\end{proof}

\makeatletter\Hy@SaveLastskip\label{proofsection:prAtEndviii}\ifdefined\pratend@current@sectionlike@label\immediate\write\@auxout{\string\gdef\string\pratend@section@for@proofviii{\pratend@current@sectionlike@label}}\fi\Hy@RestoreLastskip\makeatother\begin{proof}[Proof of \pratendRef{thm:prAtEndviii}]\phantomsection\label{proof:prAtEndviii}We start pulling back $n$ along $k$ and then pulling back the resulting arrow along $h$, getting the following two pullbacks square: \[\xymatrix {B \ar @{>->}[d]_{b}\ar @{>->}[r]^{s}& N \ar @{>->}[d]^{n} & A \ar [r]^{r} \ar @{>->}[d]_{a}& B \ar @{>->}[d]^{b}\\Q\ar @{>->}[r]_{k}&Y &M \ar [r]_h&Q\\}\] Notice that \[ f\circ m\circ a=k\circ h\circ a=k\circ b\circ r=n\circ s\circ r\] \par Thus there exists a unique $t\colon A\to P$ as in the diagram \[\xymatrix {A \ar @{>->}[d]_{a} \ar @{.>}[dr]^{t}\ar [r]^{r} &B \ar @{>->}@/^.2cm/[dr]^{s} \\ M \ar @{>->}@/_.2cm/[dr]_{m}&P\ar [r]^{p_1} \ar @{>->}[d]_{p_2}& N \ar @{>->}[d]^{n}\\ &X \ar [r]_{f} & Y}\] We can then consider the cube below, in which the front, right and back faces are pullbacks. \[\xymatrix @C=15pt@R=15pt{&A\ar @{>->}[dd]|\hole _(.65){a}\ar [rr]^{r} \ar [dl]_{t} && B \ar @{>->}[dd]^{b} \ar [dl]_{s} \\ P \ar @{>->}[dd]_{p_2}\ar [rr]^(.7){p_1} & & N \ar @{>->}[dd]_(.3){n}\\&M\ar [rr]|\hole ^(.65){h} \ar @{>->}[dl]^{m} && Q \ar @{>->}[dl]^{k} \\X \ar [rr]_{f} & & Y}\] We can notice that the left square is a pullback too. To see this it is enough to apply \Cref {lem:pb1} to the diagram \[\xymatrix {A \ar @/^.4cm/[rr]^{s\circ r}\ar @{>->}[d]_{a} \ar [r]_{t}& \ar [r]_{p_1} P \ar @{>->}[d]^{p_2}& N \ar [d]^{c}\\ M \ar @/_.4cm/[rr]_{k\circ h}\ar @{>->}[r]^{m}& X \ar [r]^{f}& Y}\] \par The thesis now follows immediately from \Cref {lem:varie}. \qedhere \end{proof}

\makeatletter\Hy@SaveLastskip\label{proofsection:prAtEndix}\ifdefined\pratend@current@sectionlike@label\immediate\write\@auxout{\string\gdef\string\pratend@section@for@proofix{\pratend@current@sectionlike@label}}\fi\Hy@RestoreLastskip\makeatother\begin{proof}[Proof of \pratendRef{thm:prAtEndix}]\phantomsection\label{proof:prAtEndix}Consider the two pushout squares \[\xymatrix {M \ar [r]^{h_1}\ar @{>->}[d]_{m}& W_1 \ar @{>->}[d]^{k_1} & M \ar @{>->}[d]_{m} \ar [r]^{h_1}& W_2 \ar @{>->}[d]^{k_2}\\Y\ar [r]_{f}&Z &Y \ar [r]_{f}&Z}\] Since $\mathcal {M}$ is in $\mathcal {M}$, \cref {prop:pbpoad} guarantees that they are both pullbacks. Since $m\leq m$, by \Cref {lem:radj} we get that now entails that $k_1\leq k_2$ and $k_2\leq k_1$. Thus there exists an isomorphism $\phi \colon W_1\to W_2$ such that $k_1=k_2\circ \phi $. To see that $h_2=\phi \circ h_1$, we can compute: \[ k_2\circ \phi \circ h_1 = k_1\circ h_1= n\circ m= k_2\circ h_2\] \par The claim now follows since $k_2$ is a monomorphism.\end{proof}

\makeatletter\Hy@SaveLastskip\label{proofsection:prAtEndx}\ifdefined\pratend@current@sectionlike@label\immediate\write\@auxout{\string\gdef\string\pratend@section@for@proofx{\pratend@current@sectionlike@label}}\fi\Hy@RestoreLastskip\makeatother\begin{proof}[Proof of \pratendRef{thm:prAtEndx}]\phantomsection\label{proof:prAtEndx}By construction, the pairs $(k, f)$ and $(k', f')$ are pushout complements of $l$ and $n$. Thus, the existence of the isomorphism $t\colon D\to D'$ follows from \Cref {lem:pocomp}. Now, computing we have \[ g'\circ t \circ k= g' \circ k'=h'\circ r\] Hence, we have the wanted $s\colon H\to H'$. To see that $s$ is an isomorphism, consider the diagram \[\xymatrix {K \ar @/^.4cm/[rr]^{k'}\ar [d]_{r} \ar [r]_{k}& \ar [r]_{t} D \ar [d]^{g}& D' \ar [d]^{g'}\\ R \ar @/_.4cm/[rr]_{h'} \ar [r]^{h}& H \ar [r]^{s}& H'}\] By hypothesis the whole rectangle and its left half are pushouts, therefore, by \Cref {lem:po1} its right square is a pushout too. The claim now follows from the fact that the pushout of an isomorphism is an isomorphism.\end{proof}
