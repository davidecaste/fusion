
\begin{proof}[Proof of \autoref{thm:prAtEndii}]\phantomsection\label{proof:prAtEndii}We can start considering the cube below and noticing that $n$, being the pushout of $m\in \mathcal {M}$, belongs to $\mathcal {A}$ too. \[\xymatrix @C=13pt@R=13pt{&A\ar [dd]|\hole _(.65){\id {A}}\ar [rr]^{g} \ar [dl]_{\id {A}} && B \ar [dd]^{\id {B}} \ar [dl]_{\id {B}} \\ A \ar [dd]_{m}\ar [rr]^(.65){g} & & B \ar [dd]_(.3){n}\\&A\ar [rr]|\hole ^(.65){g} \ar [dl]^{m} && B \ar [dl]^{n} \\C \ar [rr]_{f} & & D}\] By construction the top face of the cube is a pushout and the back one a pullback. The left face is a pullback because $m$ is mono. Thus the $\mathcal {A}$-Van Kampen property yields that the front and the right faces are pullbacks.\end{proof}
\begin{proof}[Proof of \autoref{thm:prAtEndiv}]\phantomsection\label{proof:prAtEndiv}We can begin noticing that $g$, being the pullback of $k$, is mono and in $\mathcal {A}$ too. Thus we can build the cube below, in which all the vertical faces are pullbacks, entailing that all the vertical arrows are in $\mathcal {A}$. \[\xymatrix @C=20pt@R=10pt{ & &X\ar [dl]_{f} \ar [ddd]^(.333333){\id {X}}|(.666666)\hole \ar [rrr]^{a} &&&A \ar [ddd]^{\id {A}} \ar [dl]^{h}\\& Y\ar [dl]_{\id {Y}} \ar [ddd]|(.333333)\hole _{\id {Y}} &&&B \ar [ddd]^{\id {B}} \ar [dl]_{\id {B}} \\Y \ar [ddd]_{g} \ar [rrr]^{b}&&& B\ar [ddd]^{k}\\&&X \ar [rrr]|(.34)\hole ^{a}|(.67)\hole \ar [dl]_{f}&&& A \ar [dl]^{h}\\ & Y \ar [dl]_{g}&&& B \ar [dl]^{k}\\ Z\ar [rrr]_{c} &&& C}\] By hypothesis the face is an $\mathcal {A}$-stable pushout and so its top one is a pushout. Using \Cref {lem:po1} we can conclude that the right half of the rectangle with which we have started is a pushout too.\end{proof}
\begin{proof}[Proof of \autoref{thm:prAtEndv}]\phantomsection\label{proof:prAtEndv}Take the diagram \[\xymatrix {X \ar [d]_{\id {X}}\ar [r]^{f}& Y \ar [r]^{\id {Y}} \ar [d]_{\id {Y}}& Y \ar [d]^{m}\\ X \ar [r]_{f}& Y \ar [r]_{m} & Z}\] Since $m$ is mono the right square is a pullback, while the left square is a pullback by construction. By \Cref {lem:pb1} the whole rectangle is a pullback and the thesis follows.\end{proof}
\begin{proof}[Proof of \autoref{thm:prAtEndvi}]\phantomsection\label{proof:prAtEndvi}Let $m$ be an element of $\mathcal {M}$ and consider its pushout along itself. \[\xymatrix {X\ar @{>->}[r]^m \ar @{>->}[d]_m& Y\ar @{>->}[d]^{f}\\Y \ar @{>->}[r]_g & Z}\] By \Cref {prop:pbpoad} this square is a pullback, proving that $m$ is the equalizer of the arrows $f,g\colon Y\rightrightarrows Z$.\end{proof}
\begin{proof}[Proof of \autoref{thm:prAtEndx}]\phantomsection\label{proof:prAtEndx}By construction, the pairs $(k, f)$ and $(k', f')$ are pushout complements of $l$ and $n$. Thus, the existence of the isomorphism $t\colon D\to D'$ follows from \Cref {lem:pocomp}. Now, computing we have \[ g'\circ t \circ k= g' \circ k'=h'\circ r\] Hence, we have the wanted $s\colon H\to H'$. To see that $s$ is an isomorphism, consider the diagram \[\xymatrix {K \ar @/^.4cm/[rr]^{k'}\ar [d]_{r} \ar [r]_{k}& \ar [r]_{t} D \ar [d]^{g}& D' \ar [d]^{g'}\\ R \ar @/_.4cm/[rr]_{h'} \ar [r]^{h}& H \ar [r]^{s}& H'}\] By hypothesis the whole rectangle and its left half are pushouts, therefore, by \Cref {lem:po1} its right square is a pushout too. The claim now follows from the fact that the pushout of an isomorphism is an isomorphism.\end{proof}
