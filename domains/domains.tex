\documentclass[a4paper,UKenglish,cleveref,pdftex,thm-restate,numberwithinsect]{lipics-v2021}
\newif\ifreport
%\reporttrue 
\reportfalse 
   
% full includes some additional material
\newif\iffull
%\fulltrue
\fullfalse

% additional text
\iffull
\newcommand{\full}[1]{{color{blue}#1}}
%\newcommand{\short}[1]{}
\else 
\newcommand{\full}[1]{}
%\newcommand{\short}[1]{#1}  xfl
\fi

\nolinenumbers %uncomment to disable line numbering

\ifreport
\nolinenumbers %uncomment to disable line numbering
\hideLIPIcs  %uncomment to remove references to LIPIcs series (logo, DOI, ...), e.g. when preparing a pre-final version to be uploaded to arXiv or another public repository
\else
\relatedversion{An extended version is available at \url{https://arxiv.org/abs/2407.06181}.} %optional, e.g. full version hosted on arXiv, HAL, or other respository/website
%\relatedversion{A full version of the paper is available at \url{...}.}
\fi

\bibliographystyle{abbrv}% the mandatory bibstyle



\usepackage{hyperref}
\usepackage[disable,%
textsize=tiny]{todonotes}
%\usepackage{etex}
\usepackage[all]{xy}
\SelectTips{cm}{}
% derivations and labels
\usepackage{proof}
\newcommand{\lab}[1]{\ensuremath{\mathsf{{#1}}}}
\newcommand{\slab}[1]{\ensuremath{\scriptstyle{\mathsf{{#1}}}}}
\usepackage{wrapfig}


%frecce
\newcommand{\mor}{\mathsf{Mor}}
\newcommand{\mon}{\mathsf{Mono}}
\newcommand{\reg}{\mathsf{Reg}}

% ``boxed'' \infer command
\newcommand{\binfer}[3][]{
  \mbox{\infer[#1]{#2}{#3}}}

% Command for labels on the left side of the rule
%       \inferL{<name>}{<post>}{<pre>}
% generates:
%             <pre>
%      <name> ------
%             <post>
%
\newlength{\myheight}
\newcommand{\inferL}[3]
  {\settoheight{\myheight}{\mbox{${#2}$}}
   \raisebox{\myheight}{{#1}}
   \makebox[1mm]{}
   \mbox{\infer{#2}{#3}}
}

\usepackage{amssymb,graphicx,epsfig,color}
%\usepackage[scriptsize]{subfigure}
\usepackage{subcaption}
\usepackage{wrapfig}

% re-stating 
%\usepackage{thm-restate}

% showing labels
%\usepackage[inline]{showlabels}
 
\usepackage{pgf}
\usepackage{tikz}
\usepackage{tikz-cd}
\usetikzlibrary{arrows,shapes,snakes,automata,backgrounds,petri,fit,positioning,calc}
\tikzstyle{node}=[circle, draw=black, minimum size=1mm, inner sep=1.5pt, font=\tiny]

\tikzstyle{trans}=[font=\scriptsize]
\tikzstyle{lab}=[font=\small]
\tikzset{
  coloredge/.style={
        ->,
        color=red
        %densely dotted
    }
}

\tikzset{
  colorloop/.style={
        loop,
        color=red
        %densely dotted
    }
}
  
\tikzcdset{
  perm/.style={
    shorten >=-1mm, shorten <=-1mm,
    -,
    % dotted
  }
}


\newcommand{\pgfBox}{
  \begin{pgfonlayer}{background} 
    \fill[blue!2,thick,draw=black!50,rounded corners,inner sep=3mm] ([xshift=-1.5pt,yshift=-1.5pt]current bounding box.south west) rectangle ([xshift=1.5pt,yshift=1.5pt]current bounding box.north east);
  \end{pgfonlayer}
}
\usepackage{scalerel}
\newcommand{\smallmin}{\scaleobj{0.6}{-}}
\newcommand{\Deltamin}{\Delta^{\hspace{-1pt}\downarrow\hspace{-1pt}}}
\newcommand{\Rrel}[1]   {\stackrel{{#1}}{\Longrightarrow}}
\newcommand{\oa}{\overline a}
\newcommand{\ob}{\overline b}
\newcommand{\oc}{\overline c}
\newcommand{\od}{\overline d}
\newcommand{\rec}{\emph{rec}}
\newcommand{\fn}[1]{{\mathtt{fn}}(#1)}
%\usepackage{latexsym}
\usepackage{stmaryrd}
\def\encodep#1{\llfloor#1\rrfloor}

\newcommand{\cat}[1]{\ensuremath{\mathbf{#1}}}

\newcommand{\dpo}{\textsc{dpo}}

% base classes of categories for adhesive and quasi adhesive case
\newcommand{\bAdh}{\ensuremath{\mathbb{B}}}
\newcommand{\bQAdh}{\ensuremath{\mathbb{QB}}}

%from pawel
\usepackage{amsmath}
\usepackage{amssymb}
\usepackage{amsthm}
\usepackage{enumerate}
\usepackage{xspace}
\usepackage{amsfonts}
\usepackage{mathrsfs}
\usepackage{cite}
\usepackage{float}
\usepackage{fancybox}
\usepackage{proof-at-the-end}
\usepackage{cleveref}


%\spnewtheorem*{notation}{Notation}{\bfseries}{\rmfamily}

%%%%%%%% MATHEMATICAL NOTATION %%%%%%%%%%%%%%%%%%%%%%%%%%%%%%%%%%%%%%%%%


\newcommand{\cori}{\mathbf{CS}}
\newcommand{\cla}{\mathbf{CL}}
\newcommand{\pos}{\mathbf{PO}}
\newcommand{\gori}{\mathbf{GS}}


%symbol for natural numbers
\newcommand{\nat}{\ensuremath{\mathbb{N}}}

% finite subset
\newcommand{\sfin}{\ensuremath{\subseteq_{\mathit{fin}}}}

% flattening of a multiset
\newcommand{\flt}[1]{\ensuremath{[\![{#1}]\!]}}

% compact elements
\newcommand{\compact}[1]{\ensuremath{\mathop{\mathsf{K}({#1})}}}

% principal ideal
\newcommand{\principal}[1]{\ensuremath{\mathop{\downarrow\!{#1}}}}

% ideal completion
\newcommand{\ideal}[1]{\ensuremath{\mathsf{Idl}({#1})}}

% complete prime elements
\newcommand{\pr}[1]{\ensuremath{\mathop{\mathit{pr}({#1})}}}
\newcommand{\wpr}[1]{\ensuremath{\mathop{\mathit{wpr}({#1})}}}

% irreducible elements
\newcommand{\ir}[1]{\ensuremath{\mathop{\mathit{ir}({#1})}}}

% difference of irreducible elements
\newcommand{\diff}[2]{\ensuremath{\delta({#1},{#2})}}


%arrows
\newcommand{\rightarrowdbl}{\rightarrow\mathrel{\mkern-14mu}\rightarrow}

\newcommand{\xrightarrowdbl}[2][]{%
	\xrightarrow[#1]{#2}\mathrel{\mkern-14mu}\rightarrow
}


\newcommand{\mini}[0]{\scalebox{.4}{$\bullet$}}
\newcommand{\rto}[0]{\xrightarrow{\mini}}
\newcommand{\fto}[0]{\xrightarrowdbl{\mini}}
\newcommand{\eto}[0]{\twoheadrightarrow}


% immediate precedence

% abbreviation for event structure
% \newcommand{\esabbr}{event structure}
\newcommand{\esabbr}{\textsc{es}}
\newcommand{\esnabbr}{\textsc{esnb}}
\newcommand{\esnmabbr}{\textsc{esn}}
\newcommand{\eseqabbr}{\textsc{epes}}

% predecessor of an irreducible
\newcommand{\pred}[1]{\ensuremath{\mathit{p}({#1})}}

% irreducible elements in es domains
\newcommand{\esir}[2]{\ensuremath{\langle{#1}, {#2}\rangle}}

% equivalence classes [of irreducibles]
\newcommand{\eqclass}[2][]{\ensuremath{[{#2}]_{\scriptscriptstyle {#1}}}}
% union of the equivalence classes of the elements in a set
\newcommand{\eqclasscup}[2]{\ensuremath{{#2}_{\scriptscriptstyle {#1}}}}

\newcommand{\eqclassir}[1]{\ensuremath{\eqclass[\leftrightarrow^*]{#1}}}

% quotient of set wrt a relation
\newcommand{\quotient}[2]{\ensuremath{{#1}_{\scriptscriptstyle {#2}}}}

% category of event structures 
\newcommand{\es}{\ensuremath{\mathsf{ES}}}
% category of stable event structures 
\newcommand{\ses}{\ensuremath{\mathsf{sES}}}
% category of prime event structures 
\newcommand{\pes}{\ensuremath{\mathsf{pES}}}
% category of prime event structures with equivalence
\newcommand{\epes}{\ensuremath{\mathsf{epES}}}

% category of connected event structures 
\newcommand{\ces}{\ensuremath{\mathsf{cES}}}

% category of weak prime algebraic domains domains 
\newcommand{\WDom}{\ensuremath{\mathsf{wDom}}}
% category of domains
\newcommand{\Dom}{\ensuremath{\mathsf{Dom}}}
% category of prime algebraic domains
\newcommand{\PDom}{\ensuremath{\mathsf{pDom}}}


%%%%% NON BINARY CONFLICT

% category of event structures 
\newcommand{\esn}{\ensuremath{\mathsf{ES_{nb}}}}
% category of stable event structures 
\newcommand{\sesn}{\ensuremath{\mathsf{sES_n}}}

% category of connected event structures 
\newcommand{\cesn}{\ensuremath{\mathsf{cES_{nb}}}}

% category of prime event structures 
\newcommand{\pesn}{\ensuremath{\mathsf{pES_n}}}

% category of fusion domains 
\newcommand{\WDomb}{\ensuremath{\mathsf{wDom_b}}}
% category of domains
\newcommand{\Domb}{\ensuremath{\mathsf{Dom_b}}}
% category of prime algebraic domains
\newcommand{\PDomb}{\ensuremath{\mathsf{pDom_b}}}

%%%%% END NON BINARY CONFLICT


% slice category
\newcommand{\slice}[2]{\ensuremath{({#1} \downarrow {#2})}}


% event structure for a domain
\newcommand{\zev}[0]{\ensuremath{\mathcal{E}}}
\newcommand{\ev}[1]{\ensuremath{\zev({#1})}}

% from general to connected event structures
\newcommand{\zconnes}[0]{\ensuremath{\mathcal{C}}}
\newcommand{\connes}[1]{\ensuremath{\zconnes({#1})}}
% and inclusion
\newcommand{\zinces}[0]{\ensuremath{\mathcal{I}}}
\newcommand{\inces}[1]{\ensuremath{\zinces({#1})}}


% stable version
\newcommand{\zsev}[0]{\ensuremath{\mathcal{E}_S}}
\newcommand{\sev}[1]{\ensuremath{\zsev({#1})}}

% with equivalence
\newcommand{\zeveq}[0]{\ensuremath{\mathcal{E}_{eq}}}
\newcommand{\eveq}[1]{\ensuremath{\zeveq({#1})}}

% es with equivalence to es and vice
\newcommand{\zfuse}[0]{\ensuremath{\mathcal{M}}}
\newcommand{\fuse}[1]{\ensuremath{\zfuse({#1})}}
\newcommand{\zunf}[0]{\ensuremath{\zunf}}
\newcommand{\unf}[1]{\ensuremath{\mathcal{U}({#1})}}



% Winskel/Droste version
\newcommand{\zevwd}[0]{\ensuremath{\mathcal{E}_{wd}}}
\newcommand{\evwd}[1]{\ensuremath{\zevwd({#1})}}




% configurations of an event structure
\newcommand{\conf}[1]{\ensuremath{\mathit{Conf}({#1})}}
% finite configurations
\newcommand{\conff}[1]{\ensuremath{\mathit{Conf_F}({#1})}}

% product of the sets of minimal enablinsg
\newcommand{\pmin}[1]{\ensuremath{U_{#1}}}

% connectectedness of minimal enablinsg
\newcommand{\conn}[1]{\ensuremath{\stackrel{#1}{\frown}}}


% domain for an event structure or graph grammar
\newcommand{\zdom}[0]{\ensuremath{\mathcal{D}}}
\newcommand{\dom}[1]{\ensuremath{\zdom({#1})}}


\newcommand{\zdomeq}[0]{\ensuremath{\mathcal{D}_{eq}}}
\newcommand{\domeq}[1]{\ensuremath{\zdomeq({#1})}}


% partial order for a graph grammar
\newcommand{\poset}[1]{\ensuremath{\mathcal{P}({#1})}}

% stable version
\newcommand{\pdom}[1]{\ensuremath{\mathcal{D}_S({#1})}}
\newcommand{\ppdom}[0]{\ensuremath{\mathcal{D}_S}}


% powerset
\newcommand{\Pow}[1]{\ensuremath{\mathbf{2}^{#1}}}

% powerset of finite subsets
\newcommand{\Powfin}[1]{\ensuremath{\mathbf{2}_\mathit{fin}^{#1}}}

% powerset of subsets of cardinality <= 1
\newcommand{\Powone}[1]{\ensuremath{\mathbf{2}_1^{#1}}}

% integer interval
\newcommand{\interval}[2][1]{\ensuremath{[{#1},{#2}]}}

% domain interval
\newcommand{\dint}[2]{\ensuremath{[{#1},{#2}]}}

% set of intervals
\newcommand{\IntSet}[1]{\ensuremath{\mathop{\mathit{Int}({#1})}}}

% intervals to irreducibles and vice
\newcommand{\inir}{\ensuremath{\mathop{\mathit{\zeta}}}}
\newcommand{\irin}{\ensuremath{\mathop{\mathit{\iota}}}}

% permutations
\newcommand{\perm}{\sigma}

% causes
\newcommand{\causes}[1]{\ensuremath{\lfloor {#1})}}

%%% GRAPH GRAMMARS


\newcommand{\Abs}[1]{\ensuremath{\mathsf{Abs}({#1})}}
\newcommand{\tr}[1]{\ensuremath{\mathsf{Tr}({#1})}}
% fusion safe traces
\newcommand{\trs}[1]{\ensuremath{\mathsf{Tr}_s({#1})}}
%\newcommand{\graph}{\ensuremath{\mathsf{Graph}}}
\newcommand{\tgraph}[1]{\ensuremath{\mathsf{Graph}_{#1}}}
\newcommand{\can}[1]{\ensuremath{\mathsf{C}({#1})}}
% source and target of a derivation
\newcommand{\source}[1]{\ensuremath{\mathsf{s}({#1})}}
\newcommand{\target}[1]{\ensuremath{\mathsf{t}({#1})}}
\newcommand{\col}[1]{\ensuremath{\mathsf{col}({#1})}}

% left decorated trace
\newcommand{\ltrace}[1]{\ensuremath{\langle {#1}\rangle_c}}

\newcommand{\bx}[1]{\phantom{\big(}#1{\phantom{\big)}}}
\newcommand{\bxx}[1]{\,#1\,}
\newcommand{\cycl}[1]{\ensuremath{\mbox{\textcircled{\scriptsize{$#1$}}}}}
\renewcommand{\iff}{\ensuremath{\Leftrightarrow}}

%%%%GENERAL CATEGORICAL NOTATION

\newcommand{\gph}[1]{\textbf{\textup{{#1}-Graph}}}

\newcommand{\mf}[1]{{#1}^\mathsf{m}}
\newcommand{\dph}{\mathsf{dph}}
%identità
\newcommand{\id}[1]{\mathsf{id}_{#1}}
%codominio
\newcommand{\cod}[1]{\mathsf{cod}({#1})}
	%Variabili categorie
\def\A{\textbf {\textup{A}}}
%mono
\newcommand{\mto}[0]{\rightarrowtail}


\newcommand{\sk}{\mathsf{sk}_{\X}}


\def\R{\mathsf{R}}
\def\B{\textbf {\textup{B}}}
\def\C{\textbf {\textup{C}}}
\def\D{\textbf {\textup{D}}}
\def\X{\textbf {\textup{X}}}
\def\Y{\textbf {\textup{Y}}}
\def\G{\textbf {\textup{G}}}
\def\Z{\textbf {\textup{Z}}}


\newcommand{\dm}[1]{\mathsf{dom}({#1})}


\newcommand{\sub}[3]{{\mathcal{#1}}\textrm{-}\mathsf{Sub}_{\textbf {\textup{#2}}}(#3)}

\newcommand{\msub}[2]{\mathsf{Sub}_{\textbf {\textup{#1}}}(#2)}


\newcommand{\ske}{\mathsf{sk}(\X)}
\renewcommand{\P}{\textbf {\textup{P}}}

%\derivazioni

\newcommand{\dder}[1]{\mathscr{#1}}
\newcommand{\sder}[2]{S_{i_1,i_2}(\mathscr{#1}, \mathscr{#2})}
\newcommand{\der}[1]{\underline{\dder{#1}}}
\def\dpo{\mathsf{C}^{\X}_R}
\def\gpo{\mathsf{G}^{\X}_R}
\def\dpi{[\mathsf{C}]^{\X}_R}
\def\gpi{[\mathsf{G}]^{\X}_R}

\newcommand{\ider}[1]{\mathscr{I}_{#1}}

%categorie
\def\Set{\textbf {\textup{Set}}}

%comma
\newcommand{\comma}[2]{#1\hspace{1pt} {\downarrow}\hspace{1pt} #2}
\newcommand{\cma}[2]{\mathcal{#1}\hspace{1pt} {\downarrow}\hspace{1pt} \mathcal{#2}}

%derivazioni
\newcommand{\lpro}{\langle \hspace{-1.85pt}[}
\newcommand{\rpro}{]\hspace{-1.85pt}\rangle}
\newcommand{\tpro}[1]{\lpro \der{#1}\rpro}
\newcommand{\tproi}[2]{\lpro \der{#1}_{#2}\rpro}
\newcommand{\lgh}[0]{\mathsf{lg}}

%%% NEW

\usepackage{xparse}

% inversions
\newcommand{\inv}[1]{\mathsf{inv}({#1})}
\newcommand{\tnv}[1]{\mathsf{tnv}({#1})}


%% sequential independence
%\newcommand{\seqind}{\ensuremath{\updownarrow}}


% direct shift
\newcommand{\shift}[1]{\ensuremath{\mathrel{{\leftrightsquigarrow}_{#1}}}}

% shift equivalence
\newcommand{\shifteq}[1][]{\ensuremath{\mathrel{{\equiv}^\mathit{sh}_{#1}}}}

% transp{source}[target]: if target not specified source+1
\NewDocumentCommand{\transp}{m o}{%
  \ensuremath{({#1},%
  \IfNoValueTF{#2}%
    {{#1}+1}%
    {#2}%
    )}
}

\NewDocumentCommand{\mycommand}{o}{%
  % <code>
  \IfNoValueTF{#1}
    {code when no optional argument is passed}
    {code when the optional argument #1 is present}%
  % <code>
}
% interchange
\newcommand{\IC}[1]{\ensuremath{\mathit{IC}({#1})}}

%%%%% Ambienti matematici  %%%%%%
%\newtheorem{theorem}{Theorem}[section]
%\newtheorem{proposition}[theorem]{Proposition}
%\newtheorem{lemma}[theorem]{Lemma}
%\newtheorem{corollary}[theorem]{Corollary}

\newtheorem*{setting}{Setting}
%\theoremstyle{definition}
%\newtheorem{definition}[theorem]{Definition}
\newtheorem*{notation}{Notation}
%\newtheorem{remark}[theorem]{Remark}
%\newtheorem{example}[theorem]{Example}


% commands for restructuring
\newcommand{\rem}[2]{{\color{blue}#1}{\color{red}#2}}
\renewcommand{\rem}[2]{}


% fake author for anonymous submission
%\author{Alan Turing} 
%{Department of Mathematics, Somewhere}
%{alan@turing.com}{}{}


\author{Paolo Baldan} 
{Department of Mathematics, University of Padua, Italy}
{baldan@math.unipd.it}
{https://orcid.org/0000-0001-9357-5599}{}

\author{Davide Castelnovo}
{Department of Mathematics, University of Padua, Italy}
{davide.castelnovo@math.unipd.it}
{https://orcid.org/0000-0002-5926-5615}{}

\author{Andrea Corradini}
{Department of Computer Science, University of Pisa, Italy}
{andrea.corradini@unipi.it}
{https://orcid.org/0000-0001-6123-4175}{}

\author{Fabio Gadducci}
{Department of Computer Science, University of Pisa, Italy}
{fabio.gadducci@unipi.it}
{https://orcid.org/0000-0003-0690-3051}{}


\authorrunning{P.~Baldan, D.~Castelnovo, A.~Corradini, F.~Gadducci}
%
\Copyright{Paolo Baldan, Davide Castelnovo, Andrea Corradini, and Fabio Gadducci}
%
\ccsdesc[500]{Theory of Computation~Models of computation}
\ccsdesc[500]{Theory of Computation~Semantics and reasoning}
%
\keywords{
Adhesive categories, double-pushout rewriting, left-linear rules, switch equivalence, local Church-Rosser property.
}

\funding{The research has been partially supported by the EuropeanUnion - NextGenerationEU under the National Recovery and Resilience Plan (NRRP) - Call PRIN 2022 PNRR - Project P2022HXNSC ``Resource Awareness in Programming: Algebra, Rewriting, and Analysis'', by the Italian MUR - Call PRIN 2022 - Project 20228KXFN2 
``Spatio-Temporal Enhancement of Neural nets for Deeply
Hierarchical Automatised Logic'' and by the University of Pisa - Call PRA 2022 - Project 2022\_99 ``Formal Methods for the Healthcare Domain based on Spatial Information''.}

\EventEditors{Rupak Majumdar and Alexandra Silva}
\EventNoEds{2}
\EventLongTitle{35th International Conference on Concurrency Theory (CONCUR 2024)}
\EventShortTitle{CONCUR 2024}
\EventAcronym{CONCUR}
\EventYear{2024}
\EventDate{Septembre 9--13, 2024}
\EventLocation{Calgary, Canada}
\EventLogo{}
\SeriesVolume{311}
\ArticleNo{6}

\title{Concurrent semantics for left-linear rewriting [in adhesive categories]}

\begin{document}
\maketitle


\begin{abstract}
  When moving from linear to left-linear rewriting, many properties
  cease to hold (see concur), and some require to overcome new
  obstacles to be proven. Here, we identify a very general class of
  left-linear rewriting systems on adhesive categories where a
  semantics based on event structures/domains with good properties can
  be defined. Two semantic structures introduced previously, namely
  connected event structures and weakly prime domains, introduced in
  the context of graph rewriting, emerge as canonical structures for
  rewriting with fusion in the context of adhesive categories.
\end{abstract}


%\tableofcontents

\section*{Summary}


\subsection*{1. Background}

CONCUR results: strong enforcing + well-switching => nice behaviour of independence.

\subsection*{2. A class of well behaved rewriting systems}
[note pag. 11-19]

Adhesive categories, with all POs, ...,
coreflection with a subcat from which we take lhs and monos as right legs

$\Rightarrow$

- strong enforcing + well-switching

- well-behaved with respect to quotients, e.g.,

\begin{enumerate}[a.]
\item  if we can rewrite $G$ with $L \leftarrow K \to R$ and we take a quotient
  of $G$ which factorises through the colimit, then we can still rewrite
  
\item if we can rewrite $G$ with $L \leftarrow K \to R$ and we take a $G'$ which
  has $G$ as a quotient and we can find a match in $G'$ (which factorises
  through the quotient) then we can still rewrite
  
  
\item existence of the ``largest'' quotient to which a rule applies
  
  
\item independence ``in the colimit'' [notes pag. 20-22 for ordinary
  derivations, later adapt to derivations up to fusion, which are ordinary
  derivations in the quotient]
\end{enumerate}

\subsection*{3. Domain and event structure semantics}

\subsubsection*{3.1 Domain of traces}

\begin{enumerate}[a.]
  
\item  finitary, coherent, irreducible algebraic

\item need of characterising the  irreducibles: traces with last step not switchable
\end{enumerate}

[notes page 1-9, this holds in general in the setting of concur]

\subsubsection*{3.2 Weak prime domain and connected event structures}

\begin{enumerate}[a.]
  
\item Interchargeability relation and its characterisation via the notion
  of derivation up to fusion

\item derivations up to fusion and properties [notes pag. 24-27]

\item characterising the interchangeability relation using
derivations up to fusion [some idea at pag. 29, notes]

\item  proving the interchangeability axioms to conclude that we
obtain a weak prime domain [pag. 30-32, notes]
\end{enumerate}

\section{Introduction}


\section{Background}
\label{sec:back}

\subsection{Consistent permutations}

\section{A class of well-behaved rewriting systems}
\label{sec:rewr}

This section is devoted to the introduction and study of a, quite general, setting in which the study of the concurrency properties of left-linear DPO-rewriting systems can be carried on.

\begin{setting}Let $(\X, \R)$ be a left-linear DPO-rewriting system on an $\mathcal{M}$-adhesive category $\X$. For the remaining of this section we assume the following properties:
	\begin{enumerate}
		\item $\X$ has all pushouts and pullbacks;
		\item there is a proper and stable factorization system $(\mathcal{E}, \mathcal{M})$ on $\X$;
		\item $(\X, \R)$ is strong-enforcing;
		\item there is a good coreflective subcategory $\Y$ of $\X$ containing that the left- hand side of every rule in $\R$;
		\item $ I_{\Y}\circ R_{\Y}$ sends arrows who are in  $\mathcal{R}_\Y\cap \mathcal{E}$ to arrows in $\mathcal{E}$;
		\item $\mathcal{R}_\Y $ contains the right leg of every rule in $\R$.
	\end{enumerate}
	
	Given an arrow $f\colon X\to Y$ of $\X$,  we will use the following notational conventions:
	\begin{itemize}
		\item $f\colon X\mto Y$ if $f$ belongs to $\mathcal{M}$;
		\item $f\colon X\eto Y$ if $f$ belongs to $\mathcal{E}$;
		\item $f\colon X \rto Y$ if $f$ belongs to $\mathcal{R}_\Y$ ;
		\item $f\colon X \fto Y$ if $f$ is a \emph{fusion}, i.e.~belongs to both $\mathcal{R}_\Y$ and $\mathcal{E}$. 
		\end{itemize}
\end{setting}

\begin{remark}\label{rem:cont}
	Notice that, by \Cref{prop:varie}, $\mathcal{R}_\Y$ contains all monos, therefore, in particular, it contains all arrows in $\mathcal{M}$.
\end{remark}

\begin{proposition}\label{prop:fusion}
	The class of fusions is stable under pullbacks, pushouts and closed under composition. Moreover, the image of a fusion through $I_\Y\circ R_{\Y}$ is again a fusion.
\end{proposition}
\begin{proof}
	By \Cref{prop:varie} we know that $\mathcal{R}_\Y$ is closed under composition and stable under pullbacks, while it is stable under pushout by definition. On yhe other hand, $\mathcal{E}$ is stable under pushout and closed under composition by \Cref{cor:iso2}. By hypothesis $(\mathcal{E}, \mathcal{M})$ is stable and so $\mathcal{E}$ is stable under pullbacks. Let now $e\colon X\fto Y$ be a fusion we have a diagram:
	\[\xymatrix@C=35pt{R_\Y(I_\Y(R_{\Y}(X))) \ar[r]^-{R_\Y(\epsilon_X)} \ar[d]_{R_\Y(I_\Y(R_\Y(e)))}&R_{\Y}(X) \ar[d]^{R_\Y(e)}\\R_\Y(I_\Y(R_{\Y}(Y))) \ar[r]_-{R_\Y(\epsilon_Y)}&R_\Y(Y)}\]
	
	By \Cref{cor:cou} the horizontal arrows are isomorphisms and $R_\Y(e)$ is mono because $e$ is a fusion, thus $I_\Y(R_\Y(e))$ is in $\mathcal{R}_\Y$. By hypothesis $I_\Y(R_\Y(e))$ is in $\mathcal{E}$ and we can conclude.
\end{proof}

\begin{example}
\todo{grafi}
\end{example}

\begin{remark}\label{rem:ws}
Under the conditions of our setting, the system $(\X, \R)$ is well switching. Indeed, suppose that the diagram below is given. Since $f_1$ is in $\mathcal{M}$ we have that $j_0=i_0$, on the other hand, $L_1$ belongs to $\Y$ and $g_0$ belongs to $\mathcal{R}_\Y$, and so $j_1$ is equal to $i_1$ by the fourth point of \Cref{prop:varie}.
	\[\xymatrix@C=15pt{L_0 \ar[d]_{m_0}&& K_0 \ar[d]_{k_0}\ar@{>->}[ll]_{l_0} \ar[r]_{r_0}^{\mini} & R_0 \ar@/^.3cm/@<-.5ex>[drrr]|(.22)\hole|(.32)\hole_(.4){i_0} \ar@<.5ex>@/^.3cm/[drrr]|(.3)\hole|(.39)\hole^(.7){j_0} \ar[dr]|(.29)\hole|(.35)\hole_{h_0} && L_1 \ar@<.5ex>@/_.3cm/[dlll]^(.4){i_1} \ar@<-.5ex>@/_.3cm/[dlll]_(.7){j_1} \ar[dl]|(.29)\hole|(.35)\hole^(.6){m_1}& K_1\ar[d]^{k_1}\ar@{>->}[l]_{l_1} \ar[rr]_{r_1}^{\mini} && R_1 \ar[d]^{h_1}\\G_0 && \ar@{>->}[ll]^{f_0} D \ar[rr]_{g_0}^{\mini}&& G_1  && \ar@{>->}[ll]^{f_1} D_1 \ar[rr]_{g_1}^{\mini}&& G_2}\]	
\end{remark}


\begin{lemma}Let $\der{D}=\{\dder{D}_{i}\}_{i=0}^{n}$ be a derivation, then $\iota_{\der{D}, G_{n+1}}\colon G_{n+1}\to \tpro{D}$ belongs to $\mathcal{M}$ and,  for every $i\in [0, n]$,  $\iota_{\der{D}, G_i}\colon G_i\to \tpro{D}$ is in $\mathcal{R}_\Y$.
\end{lemma}
\begin{remark}\label{rem:cont2}
	By \Cref{rem:cont}, $\mathcal{M}$ is a subclass of $\mathcal{R}_\Y$, so that the previous lemma entails that all the coprojections belong to $\mathcal{R}_\Y$.
\end{remark}
\begin{proof} 
We proceed by induction on the length of $\der{D}$.
	
		\smallskip \noindent $\lgh(\dder{D})=0$. then $\tpro{\dder{D}}$ is simply $(G_0, \{\id{G_0}\})$ and $\id{G_0}$ belongs to $\mathcal{M}$.
		
		\smallskip \noindent$\lgh(\dder{D})=1$. Suppose that $\dder{D}$ has as its single component the derivation
		\[\xymatrix{L_0 \ar[d]_{m_0}& K_0 \ar[d]^{k_0}\ar@{>->}[l]_{l_0} \ar[r]_{r_0}^{\mini} & R_0 \ar[d]^{h_0} \\G_0& \ar@{>->}[l]^{f_0} D_0 \ar[r]_{g_0}^{\mini}& G_1  \\}\]
	Then the thesis follows since we have the following pushout square.
		\[\xymatrix{D_0 \ar@{>->}[d]_{f_0} \ar[r]_{g_0}^{\mini} & G_1 \ar@{>->}[d]^{\iota_{\der{D}, G_1}} \\G_0 \ar[r]_-{\iota_{\der{D}, G_0}}^{\mini}& \tpro{D} }\]
		
		\smallskip \noindent$\lgh(\dder{D})\geq 2$. Let $\der{D}$ be $\{\dder{D}_i\}_{i=0}^n$ with $n\geq 1$. Let also $\der{D}'$ be $\{\dder{D}_i\}^{n-1}_{i=0}$ and $\rho_n=(l_n, r_n)$ be the rule applied in $\dder{D}_n$. We can then build the diagram below, in which the bootm half is a pushout.
		\[\xymatrix@C=30pt{L_n \ar[d]_{m_{n}}& K_{n} \ar[d]^{k_{n}}\ar@{>->}[l]_{l_{n}} \ar[r]_{r_{n}}^{\mini} & R_{n} \ar[d]^{h_n} \\ G_{n} \ar@{>->}[d]_{\iota_{\der{D}' , G_{n}}}& \ar@{>->}[l]^{f_n} D_n \ar[r]^{\mini}_{g_n}& G_{n+1}  \ar@{>->}[d]^{\iota_{\der{D}, G_{n+1}}}\\  \lpro \der{D}' \ar[rr]_{p}^{\mini} \rpro && \tpro{D}\\ G_i \ar[u]^-{\iota_{\der{D}', G_{i}}}_{\mini} \ar@/_.5cm/[urr]_{\iota_{\der{D}, G_i}}}\] 		
		
		The thesis follows from the inductive hypothesis and the fact that, by \Cref{prop:varie}, $\mathcal{R}_\Y$ is closed under composition.
\end{proof}
\begin{corollary}
	\todo{match}
\end{corollary}


Our next step is to show that, in our setting, there is a kind of ``minimum'' fusion of an object $X$ to which a rewriting rule can be applied.

\begin{definition} Let $\rho$ be a rule $(l, r)$ in $\R$ with left-hand side $L$. Let $e\colon X\fto Y$ and $e'\colon X\fto Y'$ be two fusions with the same domain and $m\colon L\to Y$ and $m'\colon L\to Y'$ be two arrows, we say that the pairs $(e,m)$ and $(e', m')$ are \emph{compatible with respect to $\rho$} if there exists $q\colon Y\fto Q$ and $q'\colon Y'\fto Q'$ making the following diagram commutative.

\[\xymatrix{X\ar@{>>}[rr]^\mini_{e} \ar@{>>}[dd]_{e'}^\mini & & Y \ar@{.>>}[dd]_{q}^\mini\\ &L  \ar[ur]_{m} \ar[dl]^{m'}\\Y' \ar@{.>>}[rr]_{q'}^{\mini} && Q}\]

Given a fusion $e\colon X\fto Y$ and an arrow $m\colon L\to Y$, we will denote by $\mathcal{F}_\rho(e, m)$ the class of pairs compatible with $(e, m)$ with respect to a given rule $\rho$.  
\end{definition}

\begin{example}
	tutti i sotto quozienti
\end{example}

\begin{proposition}
Let $\rho$ be a rule with left-hand side $L$ and $(e,m)$ a pair made by a fusion $e\colon X\fto Y$ and an arrow $m\colon L\to Y$. Then for every $(e', m')$ in $\mathcal{F}_\rho(e,m)$ \[\mathcal{F}_\rho(e,m)=\mathcal{F}_\rho(e',m')\] 
\end{proposition}
\begin{proof} Let $e'$ be an arrow $X\fto Y'$, by definition of compatibility, we know that there exist $q\colon Y'\to Q$ and $Q\colon Y\to Q$ fitting in the diagram below. 
\[\xymatrix{X\ar@{>>}[rr]^\mini_{e} \ar@{>>}[dd]_{e'}^\mini & & Y \ar@{.>>}[dd]_{q}^\mini\\ &L  \ar[ur]_{m} \ar[dl]^{m'}\\Y' \ar@{.>>}[rr]_{q'}^{\mini} && Q}\]

We are now going to show the two inclusions.
	
	\smallskip \noindent $(\subseteq)$. Let $(e'', m'')$ be another element of $\mathcal{F}_\rho(e,m)$, so that we have a square as the one below
	
	\[\xymatrix{X\ar@{>>}[rr]^\mini_{e} \ar@{>>}[dd]_{e''}^\mini & & Y \ar@{>>}[dd]_{\hat{q}}^\mini\\ &L  \ar[ur]_{m} \ar[dl]^{m''}\\Y'' \ar@{>>}[rr]_{q''}^{\mini} && \hat{Q}}\]
	
	Since $\X$ has all pushouts, and using \Cref{prop:fusion}, we can build the following diagram, where the bottom square is a pushout.
	\[\xymatrix{&& X \ar@{>>}[d]_{e}^{\mini} \ar@/^.3cm/@{>>}[drr]_{e'}^{\mini} \ar@/_.3cm/@{>>}[dll]^{e'''}_{\mini}\\Y'' \ar@{>>}[dr]_{q''}^{\mini} && Y \ar@{>>}[dr]_{q}^{\mini} \ar@{>>}[dl]^{\hat{q}}_{\mini}  && Y'\ar@{>>}[dl]^{q'}_{\mini}  \\ & \hat{Q} \ar@{>>}[dr]_{p_1}^{\mini} && Q\ar@{>>}[dl]^{p_2}_{\mini}\\ &&Q'}\] 
		
	To get the thesis, it is now enough to compute:
	\begin{align*}
		p_1\circ q''\circ m''&=p_1\circ \hat{q}\circ m\\&=p_2\circ q\circ m\\&=p_2\circ q'\circ m
	\end{align*}
	
		
	\smallskip \noindent $(\supseteq)$ This follows immediately from the previous point since $(e,m)\in \mathcal{F}_\rho(e,m)$.
\end{proof}


\begin{lemma}
	Let $e\colon X\fto Y$ be a fusion, $\rho$ a rule with left-hand side $L$ and $m\colon L\to Y$ an arrow.  Then there exists $(\mf{e},\mf{m})$, with $\mf{e}\colon X\fto \mf{Y}$,  in $\mathcal{F}_\rho(e, m)$ with the property that for every other $(e', m')$ in $\mathcal{F}_\rho(e, m)$, there exists an arrow $f_{(e', m')}\colon \mf{Y}\to Y'$, where $Y'$ is the codomain of $e'$, such that:
\[f_{(e', m')}\circ \mf{e} =e' \qquad f_{(e', m')}\circ \mf{m}=m'\]
\end{lemma}

\begin{proof} We start by taking the following pullback square. Since $e\colon X\fto Y$ is a fusion, by \Cref{prop:fusion} we know that $p_1\colon P\to L$ is a fusion too.
	\[\xymatrix{P \ar@{>>}[r]_{p_1}^{\mini} \ar[d]_{p_2}& L \ar[d]^{m} \\ X \ar@{>>}[r]_{e}^{\mini} & Y}\]
	
Now, applying $I_\Y\circ R_\Y$ we get the two squares in the diagram below, while the outer boundary of it is a pushout. Notice that, by hypothesis and \Cref{prop:fusion}, all the horizontal arrows and the coprojection $\mf{e}\colon X\to \mf{Y}$ are fusions. Moreover the universal property of pushouts yields the dotted $f_{(e,m)}\colon \mf{Y}\to Y$
	\[\xymatrix@C=55pt{I_\Y(R_\Y(P)) \ar[d]_{I_\Y(R_\Y(p_2)) } \ar@{>>}[r]_-{I_\Y(R_\Y(p_1))}^{\mini} & I_\Y(R_\Y(L)) \ar@/^.4cm/[dddr]^{n} \ar[d]_{I_\Y(R_\Y(m)) } \\ I_\Y(R_\Y(X)) \ar[d]_{\epsilon_X} \ar@{>>}[r]_-{I_\Y(R_\Y(e))}^{\mini}& I_\Y(R_\Y(Y)) \ar[d]_{\epsilon_X}  \\ X\ar@{>>}@/_.5cm/[drr]^{\mini}_{\mf{e}} \ar@{>>}[r]_{e}^{\mini}& Y \\ &&\mf{Y}\ar@{.>}[ul]_{f_{(e,m)}}}\]

Now, by hypothesis $L$ is in $\Y$, thus by \Cref{cor:counit} $\epsilon_L$ is an isomorphism and we can define $\mf{m}\colon L\to \mf{Y}$ as $n\circ \epsilon^{-1}_L$, so that 
\begin{align*}
f_{(e,m)}\circ \mf{m}&=f_{(e,m)}\circ n\circ \epsilon^{-1}_L\\&= \epsilon_X\circ I_\Y(R_\Y(m))\\&=m\circ \epsilon_L\circ \epsilon^{-1}_L\\&=m
\end{align*}

We have now to show that $(\mf{e}, \mf{m})$ belongs to $\mathcal{F}_\rho(e,m)$. But to do so it is enough to consider the pushout square below:
\[\xymatrix{X \ar@{>>}[r]^{\mf{e}}^\mini \ar@{>>}[d]_{e}^\mini& \mf{Y} \ar[d]_{\mf{q}}^\mini\\ Y \ar@{>>}[r]_{q}^\mini & Q}\]

Indeed, on one hand we have
\begin{align*}
	q\circ f_{(e,m)}\circ \mf{e}&=q\circ e\\&=\mf{q}\circ \mf{e}
\end{align*}
So that, since $\mf{e}$ is an epimorphism, we know that  $q\circ f_{(e,m)}=\mf{q}$. But then
\begin{align*}
	q\circ m&=q\circ f_{(e,m)}\circ \mf{m}\\&=
	\mf{q}\circ \mf{m}
\end{align*}

	Let now $(e', m')$ be another element of $\mathcal{F}_\rho(e,m)$, thus we have a square as the one below.
\[\xymatrix{X\ar@{>>}[rr]^\mini_{e} \ar@{>>}[dd]_{e'}^\mini & & Y \ar@{>>}[dd]_{q}^\mini\\ &L  \ar[ur]_{m} \ar[dl]^{m'}\\Y' \ar@{>>}[rr]_{q'}^{\mini} && Q}\]
 
 Now, computing we have:
 \begin{align*}
 q'\circ	m'\circ \epsilon_L\circ I_\Y(R_\Y(p_1))&=q\circ m\circ \epsilon_L\circ I_\Y(R_\Y(p_1))\\&=q\circ m\circ p_1\circ \epsilon_P\\&=q\circ e \circ p_2\circ \epsilon_P\\&=q'\circ e' \circ \epsilon_X\circ I_\Y(R_\Y(p_2))
 \end{align*}	
	
Since $q'$ is a fusion, by the fourth point of \Cref{prop:varie} the outer square in the diagram below commutes.

	\[\xymatrix@C=55pt{I_\Y(R_\Y(P)) \ar[d]_{I_\Y(R_\Y(p_2)) } \ar@{>>}[r]_-{I_\Y(R_\Y(p_1))}^{\mini} & I_\Y(R_\Y(L))  \ar[dd]_{n} \ar[dr]^-{\epsilon_L}\\ I_\Y(R_\Y(X)) \ar[d]_{\epsilon_X} && L \ar[dd]^{m'} \ar[dl]_{\mf{m}}   \\ X\ar@{>>}@/_.5cm/[drr]^{\mini}_{e'} \ar@{>>}[r]_{\mf{e}}^{\mini}& \mf{Y} \ar@{.>}[dr]^{f_{(e',m')}}\\ &&Y'}\]

Thus  we obtain the arrow $f_{(e', m')}\colon \mf{Y}\to Y'$ by the universal property of pushouts.	
\end{proof}
 
 \begin{definition}\todo{quoziente minimo}
 \end{definition}
 

Finally....

\todo{indipendenza}

\subsection{Derivations with fusions}

\section{Domain and event structure semantics}
\subsection{Domain of traces}
\subsection{Weak prime domain and connected event structures}



\bibliographystyle{plain}
\bibliography{bibliog.bib}

\appendix

\section{Preliminaries on subobjects, images, pullbacks and factorization systems}\label{app:fact}


This first section is devoted to recall some classic notions related to subobjects, pullbacks and ways to factorize an arrow. The main source for the presentation is \cite[Ch.~1]{dikranjan2013categorical}.
\subsection{The pullback functor and right $\mathcal{M}$-images}

Let us start defining the \emph{pullback functor} and show how the existence of a left adjoint to it yields a notion of image of an arrow.

\begin{definition}
	Let $\mathcal{M}$ be a class of arrows in a category $\X$ with $\mathcal{M}$-pullbacks. Suppose that $\mathcal{M}$ is stable under pullbacks, then given an arrow $f\colon X\to Y$, we define the \emph{pullback functor} $f^*\colon \mathcal{M}/Y\to \mathcal{M}/X$ sending an arrow $m:M\to Y$ in $\mathcal{M}$ to a chosen pullback of it along $f$. Given a morphism $g:m\to n$ in $\mathcal{M}/Y$, the arrow $f^*(g)$ is the unique one fitting in the diagram below.
	\[\xymatrix{f^*(M)  \ar[dd]_{f^*(m)} \ar@{.>}[dr]_{f^*(g)}\ar[rr]^{p_m}&& M \ar[dr]^{g}\ar[dd]^(.3){m} \\&f^*(N)  \ar[rr]^(.35){p_n}|(.55)\hole \ar[dl]_{f^*(n)}&& N \ar[dl]^{n}\\ X \ar[rr]_{f}&& Y }\]
\end{definition}

\begin{notation}
	If $m\colon M\to X$ is an arrow in $\mathcal{M}$ and $f\colon X\to Y$ is another morphism, we will denote the chosen domain of $f^*(m)$ by $f^*(M)$.
\end{notation}

\begin{proposition} Let $f\colon X\to Y$ be an arrow in a category $\X$ and $\mathcal{M}$ a class of arrows stable under pullbacks and such that $f$ has a pullback along any arrow in $\mathcal{M}$. Let also $F\colon\D\to \mathcal{M}/Y$ be a diagram with a limiting cone $(l, \{l_D\}_{D\in \D})$ and suppose that  $L$ is the domain of $l$. Denoting $F(D)$ by  $m_D\colon M_D\to Y$, if  $(L, \{l_D\}_{D\in \D}+\{l\})$ is limiting for the diagram in $\X$ determined by the family of arrows $\{m_D\}_{D\in \D}$, then for every $f\colon X\to Y$, $f^*$ preserves the limit of $F$.
\end{proposition}
\begin{proof}
	Let $(a, \{a_D\}_{D\in \D})$ be a cocone in $\{\mathcal{M}/X\}$ over $f^*\circ F$. Then we have the solid part of the diagram below.
	\[\xymatrix{A \ar@/_.8cm/[dd]_{a}\ar[d]^{a_D}  \ar@{.>}[r]_-{b} \ar@/^.5cm/@{.>}[rrr]^{c}&f^*(L)  \ar@/^.7cm/[ddl]|(.37)\hole^(.6){f^*(l)} \ar[rr]_{q} \ar[dl]^{f^*(l_D)}&&L \ar@/^.3cm/[ddl]^{l} \ar[dl]_{l_D}\\f^*(M_D)  \ar[d]^{f^*(m_D)} \ar[rr]^(.7){p_{F(D)}}&& M_D \ar[d]_{m_D} \\ X \ar[rr]_{f}&& Y }\]
	
	Now, notice that, for every $D\in \D$ we have
	\begin{align*}
		m_D\circ p_{F(D)} \circ a_D&=f\circ f^*(m_D)\circ a_D \\&=f\circ a
	\end{align*}
	Moreover, given an arrow $g\colon D\to D'$ in $\D$ then we have
	\begin{align*}
		F(g)\circ p_{F(D)}\circ a_D&=p_{F(D')}\circ f^*(F(g))\circ a_D\\&=p_{F(D')}\circ a_{D'}
	\end{align*}
	
	Thus $(A, \{p_{F(D)}\circ a_D\}_{D\in \D}+ \{f\circ a\})$ is a cone over the diagram in $\X$ determined by $\{m_D\}_{D\in \D}$, so that we get the dotted arrow $b\colon A\to L$ in the diagram above. This arrow,  in turn, yields also $a\colon A\to f^*(L)$. Moreover, we also have that
	\[
	\begin{split}
		p_{F(D)}\circ f^*(l_D)\circ b&= l_D\circ q \circ b\\&=l_D\circ c\\&=p_{F(D)}\circ a_D
	\end{split}\qquad 
	\begin{split}
		f^*(m_D) \circ f^*(l_D)\circ b&= f^*(l)\circ b\\&=a\\&=f^*(m_D)\circ a_D
	\end{split}
	\]
	So that $b$ defines an arrow $a\to f^*(l)$ in $\mathcal{M}/X$ auch that $f^*(l_D)\circ b =a_D$
	
	To see that $b$ is unique, let $b'\colon a\to f^*(l)$ be another arrow such that $f^*(l_D)\circ b' =a_D$, then
	\[\begin{split}
		l\circ q \circ b'&=f\circ f^*(l)\circ b'\\&=f\circ a\\&=l\circ c
	\end{split} \qquad \begin{split}
		l_D\circ q \circ b'&=p_{F(D)}\circ f^*(l_D)\circ b'\\&=p_{F(D)}\circ a_D\\&=l_D\circ c
		3\end{split}\]
	Thus $q\circ b'=c$, implying that $b'=b$.
\end{proof}

The previous theorem gives us at once the following result.

\begin{corollary} \label{cor:finlim}
	Let $f\colon X\to Y$ be an arrow in a category $\X$ and $\mathcal{M}$ a class of arrows stable under pullbacks and such that $f$ has a pullback along any arrow in $\mathcal{M}$. Then $f^*$ preserves binary products. If, moreover, $\id{Y}$ belongs to $\mathcal{M}$ then $f^*$ preserves finite products.
\end{corollary}

Having examined the continuity properties of the pullback functor we can now turn to the notion of \emph{right $\mathcal{M}$-images}.

\begin{definition}
	Let $\mathcal{M}$ be a class of arrows in a category $\X$. A \emph{right $\mathcal{M}$-factorization
	} of a morphism $f\colon X\to Y$ is a pair of arrows $m\colon M\to Y$ and $e\colon X\to M$ such that:
	\begin{enumerate}
		\item $m$ belongs to $\mathcal{M}$ and $f=m\circ e$;
		\item given an arrow $n\colon N\to Z$ in $\mathcal{M}$ and two morphisms $g\colon Y\to Z$, $h\colon X\to N$ makint the solid part of the diagram below commutative, then there exists a unique dotted $v\colon M\to N$ fitting in it.
		\[\xymatrix{X \ar[r]_-{e} \ar@/^.4cm/[rr]^{f} \ar[d]_{h}& M \ar[r]_-{m} \ar@{.>}[dl]^{v} & Y \ar[d]^{g}\\N \ar[rr]_{n} && Z}\] 
	\end{enumerate}  
	
	The arrow $m$ will be called the \emph{right $\mathcal{M}$-image} of $f$.
\end{definition}

\begin{remark}\label{rem:uniq}
	Right $\mathcal{M}$-factorization are essentially unique. To see this, let $e\colon X\to M$ and $m\colon M\to Y$ be a right $\mathcal{M}$-factorization of an arrow $f\colon X\to Y$ and suppose that  $e'\colon X\to M'$ and $m'\colon M'\to Y$ is another, then we  get the existence of the dotted arrows in the following two diagrams
	\[\xymatrix{X \ar[r]^-{e}  \ar[d]_-{e'}& M \ar[r]^-{m} \ar@{.>}[dl]^{v} & Y \ar[d]^{\id{Y}} & X \ar[r]^-{e'}  \ar[d]_{e}& M' \ar[r]_-{m'} \ar@{.>}[dl]^{w} & Y \ar[d]^{\id{Y}}\\M' \ar[rr]_{m'} && Y & M \ar[rr]_{m} && Y}\] 
	On the other hand, from the commutativity of the diagram below we get at once that $v\circ w=\id{M'}$ and $w\circ v=\id{M}$.
	\[\xymatrix{X \ar[rr]^-{e}  \ar[dr]_-{e'} \ar[dd]_{e}&& M \ar[r]^-{m} \ar[dl]^{v} & Y \ar[dd]^{\id{Y}} & X \ar[dr]^{e} \ar[rr]^-{e'}  \ar[dd]_{e'}&& M' \ar[r]_-{m'} \ar[dl]^{w} &Y \ar[dd]^{\id{Y}}\\ & M' \ar[dl]^{w}  \ar[drr]^{m'} &&&&M\ar[dl]^{v} \ar[drr]^{m}\\M \ar[rrr]_{m} &&& Y & M' \ar[rrr]_{m'} &&& Y}\] 
\end{remark}


\begin{proposition}\label{prop:im}
	Let $\mathcal{M}$ be a class of arrows in a category $\X$ and suppose that every morphism in $\X$ has a right $\mathcal{M}$-factorization. Then for every arrow $f\colon X\to Y$, there is a functor $f_*\colon \mathcal{M}/X\to \mathcal{M}/Y$ sending $m\colon M\to X$ to a chosen right $\mathcal{M}$-image of $f\circ m$.
\end{proposition}
\begin{notation}
	As in the case of the pullback functor, if $m$ is an arrow $M\to X$, then we will denote the chosen domain of $f_*(m)$ by $f_*(M)$.
\end{notation}
\begin{proof}
	We only have to define $f_*$ on arrows. Let thus $g$ be a morphism in $\mathcal{M}/X$ between $m\colon M\to X$ and $n\colon N\to X$. Now, since \[f\circ m = f\circ n\circ g\] the solid part of the diagram below commutes, allowing us to define $f_*(g)$ as the dotted arrow.
	\[\xymatrix@C=30pt{M \ar[dr]_(.35){m}|(.55)\hole\ar[r]^-{e_m} \ar[d]_{g}& f_*(M)  \ar@/_.3cm/@{.>}[ddl]^(.7){f_*(g)}\ar[r]^-{f_*(m)}& Y \ar[dd]^{\id{Y}}\\N \ar[d]_{e_n} \ar[r]|(.32)\hole^(.6){n} & X  \ar[dr]^{f}\\f_*(N) \ar[rr]_{f_*(n)}&& Y}\]
	
	It is clear that, with the above definition, $f_*(\id{m})=\id{f_*(m)}$. Let $h\colon n\to o$ be another morphism in $\mathcal{M}/X$, then we can build the following diagram.
	\[\xymatrix@C=30pt{M \ar[r]^-{e_m} \ar[d]_{g}& f_*(M)  \ar[d]_{f_*(g)}\ar[r]^-{f_*(m)}& Y \ar[dd]^{\id{Y}}\\N \ar[r]^-{e_n} \ar[d]_{h} & f_*(N) \ar[d]_{f_*(h)} \ar[dr]^{f_*(n)}\\O\ar[r]_-{e_o}& f_*(O) \ar[r]_{f_*(o)}& Y}\]
	From this it follows at once that $f_*(h)\circ f_*(g)=f_*(g\circ h)$.
\end{proof}
\begin{remark} If $\id{X}$ belongs to $\mathcal{M}$, then $f_*(\id{X})$ is a right $\mathcal{M}$-image of $f$.
\end{remark}

The relationship of right $\mathcal{M}$-images to the pullback functors is explained by the following two results.

\begin{proposition}\label{prop:part}
	Let $\mathcal{M}$ be a class of arrows  in a category $\X$, containing  all identities and stable under pullbacks. Suppose, moreover, that every morphism has a right $\mathcal{M}$-factorization. Then $f_*\dashv f^*$ for every arrow $f\colon X\to Y$.  
\end{proposition}
\begin{proof}
	First of all we can notice that for every $m\colon M\to X$ in $\mathcal{M}/X$ the commutativity of the solid part of the following diagram yields the dotted morphism $\eta_n\colon m\to f^*(f_*(m))$.
	\[\xymatrix@C=30pt{M \ar@{.>}[dr]^{\eta_m} \ar@/^.4cm/[drr]^{e_m} \ar@/_.4cm/[ddr]_{m}\\ & f^*(f_*(M)) \ar[d]^{f^*(f_*(m))}\ar[r]^-{p_{f_*(m)}}& f_*(M) \ar[d]^{f_*(m)}\\ & X\ar[r]_f & Y}\]
	
	To see that $\eta_n$ is the component of a unit $\eta\colon \id{\mathcal{M}/X}\to f^*\circ f_*$, let $g$ be a morphism $m\to f^*(n)$ for some $n\colon N\to Y$ in $\mathcal{M}/Y$. If $p_n:f^*(N)\to N$ is the canonical projection, then 
	\begin{align*}
		n\circ p_n\circ g&=f\circ f^*(n)\circ g\\&=f\circ m
	\end{align*}
	Therefore we can consider the diagram below to get $h\colon f_*(m)\to n$ in $\mathcal{M}/Y$. 
	\[\xymatrix@C=30pt{M \ar[r]^-{e_m} \ar[d]_{g}& f_*(M) \ar[r]^-{f_*(m)}\ar@{.>}@/^.3cm/[ddl]^{h} & Y \ar[dd]^{\id{Y}}\\f^*(N) \ar[d]_{p_n}\\N \ar[rr]_{n}&& Y}\]
	
	Finally, the equations
	\[\begin{split}
		f^*(n)\circ f^*(h)\circ \eta_m&=f^*(f_*(m))\circ \eta_m\\&=m\\&=f^*(n)\circ g
	\end{split} \qquad \begin{split}
		p_n\circ f^*(h)\circ \eta_m&=h\circ p_m\circ \eta_m\\&=h\circ e_m\\&=p_n\circ g
	\end{split}\]
	entail that $f^*(h)\circ \eta_m=g$ as wanted.
	
	Let now $h'\colon f_*(m)\to n$ be an arrow such that $f*(h')\circ \eta_m=g$, then
	\begin{align*}
		h'\circ e_m&=h'\circ p_{f_*(m)}\circ \eta_m\\&=p_n\circ f^*(h')\circ \eta_m\\&=p_n\circ g
	\end{align*}
	Thus $h'$ fits in the diagram
	\[\xymatrix@C=30pt{M \ar[r]^-{e_m} \ar[d]_{g}& f_*(M) \ar[r]^-{f_*(m)}\ar@/^.3cm/[ddl]^{h'} & Y \ar[dd]^{\id{Y}}\\f^*(N) \ar[d]_{p_n}\\N \ar[rr]_{n}&& Y}\]
	showing that $h=h'$.	
\end{proof}

\begin{lemma}\label{lem:rfact}
	Let $\mathcal{M}$ be a class of arrows  in a category $\X$, containing all identities then the following are equivalent:
	\begin{enumerate}
		\item $\mathcal{M}$ is stable under pullbacks, $\X$ has all $\mathcal{M}$-pullbacks and every morphism has a right $\mathcal{M}$-factorization;
		\item $\mathcal{M}$ is stable under pullbacks, $\X$ has all $\mathcal{M}$-pullbacks and for every morphism $f\colon X\to Y$, $f^*$ has a left adjoint;
		\item every morphism has a right $\mathcal{M}$-factorization and for every arrow $f\colon X\to Y$, $f_*$ has a right adjoint.
	\end{enumerate}
\end{lemma}
\begin{proof}
	$(1\Rightarrow 2)$ This follows at once from \Cref{prop:part}.
	
	\smallskip \noindent $(2\Rightarrow 1)$ Let $L_f\colon \mathcal{M}/X\to \mathcal{M}/Y$ be the left adjoint to $f^*$. Since $\id{X}$ belongs to $\mathcal{M}$ we can define $m\colon M\to Y$ in $\mathcal{M}$ as  $L_f(\id{X})$. Moreover, let $\eta_{\id{X}}$ be the component at $\id{X}$ of the unit of the adjunction $L_f \dashv f^*$, if $p_m$ is the projection $f^*(M)\to M$, we can also define $e$ as $p_m\circ \eta_{\id{X}}$. Notice that, by construction, $m\circ e=f$.
	\[\xymatrix@C=30pt{X \ar@/^.4cm/[rr]^e \ar@/_.4cm/[dr]_{\id{X}} \ar[r]_-{\eta_{\id{X}}} & f^*(M) \ar[r]_-{p_m} \ar[d]^{f^*(m)}& M \ar[d]^{m}\\ & X \ar[r]_f& Y}\]
	
	We have to show that $e$ and $m$ so defined really give us a right $\mathcal{M}$-factorization of $f$. We thus start with the solid part of the diagram below, with $n\in \mathcal{M}$.
	\[\xymatrix@C=35pt{f^*(M)\ar[dr]^-{p_m}\\X \ar[r]^-{e} \ar[u]^{\eta_{\id{X}}} \ar|(.36)\hole@/_.5cm/[rr]_{f} \ar[d]_{h}&   M \ar[r]^-{m} \ar@{.>}[dl]^(.6){v} & Y \ar[d]^{g}\\N \ar[rr]_{n} && Z}\] 
	
	We can apply the pullback property of the inner rectangle below to get the dotted arrow $w\colon M\to f^*(g^*(N))$, which, by construction, defines an arrow $m\to f^*(g^*(n))$.
	\[\xymatrix@C=30pt{X \ar@/^.5cm/[drrr]^{h} \ar@/_1cm/[ddr]_{\id{X}} \ar@{.>}[dr]^{w}\\ &f^*(g^*(N)) \ar[d]_{f^*(g^*(n))}\ar[r]^-{p_{g^*(n)}}& g^*(N) \ar[d]_{g^*(n)} \ar[r]^{p_n} & N \ar[d]^{n} \\ &X \ar[r]_{f}& Y \ar[r]_g & Z}\]
	
	By construction $w$ defines a morphism $\id{X}\to f^*(g^*(n))$, thus by adjointness, this implies the existence of a unique $\hat{w}\colon m\to g^*(n)$ such that $f^*(\hat{w})\circ \eta_{\id{X}} =w$. Let $v$ be $p_n\circ \hat{w}$, then
	\[\begin{split}
		v\circ e &= p_n\circ \hat{w} \circ p_m\circ \eta_{\id{X}} \\&=p_n\circ p_{g^*(n)}\circ f^*(\hat{w})\circ \eta_{\id{X}}\\&=p_n\circ p_{g^*(n)}\circ w \\&=h
	\end{split}\qquad  \begin{split}
		n\circ v &=n\circ p_n\circ \hat{w}\\&=g\circ g^*(n)\circ \hat{w}\\&=g\circ m\\&
	\end{split}\] 
	
	To see that such $v$ is unique, let $v'$ be another arrow $M\to N$ such that
	\[v'\circ e=h \qquad n\circ v'=g\circ m\]
	The second of these equations entail the existence of the dotted $q\colon M\to g^*(N)$.
	\[\xymatrix{M \ar@{.>}[dr]^{q}\ar@/^.3cm/[drr]^{v'} \ar@/_.3cm/[ddr]_{m}\\ &g^*(N) \ar[d]^{g^*(n)}\ar[r]^{p_n}& N\ar[d]^{n}\\ &Y\ar[r]_{g} & Z}\]
	Notice that, in particular, $q$ is a morphism $m\to g^*(n)$ in $\mathcal{M}/Y$. Now, we can also consider the following diagram, which entails that $f^*(q)\circ \eta_{\id{X}}=w$.
	\[\xymatrix@C=30pt{f^*(M) \ar@/_1.2cm/[ddd]_{f^*(m)}\ar[dd]_{f^*(q)}\ar[rr]^{p_m}&&M \ar[dd]_{q}\ar@/^.9cm/[ddd]_(.4){m}|(.67)\hole \ar@/^.6cm/[ddr]^{v'}\\&X\ar@/^.8cm/[ddl]|(.33)\hole_(.2){\id{X}} \ar[drr]_(.3){h}|\hole|(.74)\hole\ar[ul]^{\eta_{\id{X}}} \ar[ur]_{e}\\ f^*(g^*(N)) \ar[d]^{f^*(g^*(n))}\ar[rr]_(.67){p_{g^*(n)}}&& g^*(N) \ar[d]_{g^*(n)} \ar[r]_(.65){p_n} & N \ar[d]^{n} \\ X \ar[rr]_{f}&& Y \ar[r]_g & Z}\]
	Thus $q$ must coincide with $\hat{w}$ and therefore $v'=v$.
	
	\smallskip \noindent $(1\Rightarrow 3)$ This also follows from \Cref{prop:part}.
	
	\smallskip \noindent $(3\Rightarrow 1)$ Let $R_f\colon \mathcal{M}/Y\to \mathcal{M}/Y$ be the right adjoint to $f_*$. Given $n\colon N\to Y$ in $\mathcal{M}$, we can build the rectangle below, where $f(R_f(n))$ and $e$ is a (chosen) right $\mathcal{M}$-factorization of $f\circ R_f(n)$ and $\epsilon\colon f_*\circ R_f\to \id{\mathcal{M}/Y}$ is the counit of the adjunction $f_* \dashv f^*$. 
	\[\xymatrix{R_f(N) \ar[d]_{R_f(n)} \ar[r]^-{e}& f_*(R_f(N))  \ar[r]^-{\epsilon_n} \ar[dr]_{f_*(R_f(n)) \hspace{3pt}}& N \ar[d]^{n}\\
		X \ar[rr]_{f}&& Y}\]
	
	We have to show that such a rectangle is a pullback. To do so, let $h\colon Z\to N$ and $g\colon Z\to X$ be arrows such that $n\circ h=f\circ g$ and consider the right $\mathcal{M}$-factorization of $g$ given by arrows $i\colon Z\to K$ and $k\colon K\to X$, with $k\in \mathcal{M}$. Thus there exists the dotted $w\colon K\to N$ in the diagram below.
	\[\xymatrix{Z \ar[r]_{i}\ar@/^.4cm/[rr]^{g}\ar[d]_{h}& K \ar@{.>}[dl]^{w} \ar[r]_k & X \ar[d]^{f}\\
		N \ar[rr]_{n}&& Y}\]
	
	In turn, we can factor $f\circ k$ to get the following rectangle and, henceforth, the existence of the dotted $u\colon f_*(K)\to N$.
	\[\xymatrix{K \ar[r]_-{j}\ar@/^.4cm/[rr]^-{f\circ k}\ar[d]_{w}& f_*(K) \ar@{.>}[dl]^{u} \ar[r]_{f_*(k)} & Y \ar[d]^{\id{Y}}\\
		N \ar[rr]_{n}&& Y}\]
	
	By construction $u$ is a morphism $f_*(k)\to n$, we can thus consider the unique morphism $\hat{u}\colon k\to R_f(n)$ such that
	\[\epsilon_n\circ f_*(\hat{u})=u\]
	
	Summing up we have constructed the diagram that follows.
	\[\xymatrix{Z \ar@/^.4cm/[drrrr]^{h}\ar[dr]^{i} \ar@/_1cm/[ddrr]_{g}\\& K \ar[r]^-{\hat{u}} \ar@/_.2cm/[dr]_{k}& R_f(N) \ar[d]_{R_f(n)} \ar[r]^-{e}& f_*(R_f(N))  \ar[r]_-{\epsilon_n} \ar[dr]_{f_*(R_f(n)) \hspace{3pt}}& N \ar[d]^{n}\\ && 
		X \ar[rr]_{f}&& Y}\]
	
	To conclude it is now enough to show that $\hat{u}\circ i$ is the unique morphism $Z\to R_f(N)$ such that 
	\[\epsilon_n\circ e \circ \hat{u}\circ i=h\qquad R_f(n)\circ \hat{u}\circ i=g\]
	
	Let, therefore, $t$ be an arrow $Z\to R_f(N)$ such that
	\[\epsilon_n\circ e \circ t=h\qquad R_f(n)\circ t=g\]
	The second equation provides us with the commutative rectangle below, guaranteeing the existence of the dotted $s\colon k\to R_f(n)$.
	\[\xymatrix{Z \ar[r]_-{i}\ar@/^.4cm/[rr]^-{g}\ar[d]_{t}& K \ar@{.>}[dl]^{s} \ar[r]_{k} & X \ar[d]^{\id{X}}\\
		R_f(N) \ar[rr]_{R_f(n)}&& X}\]
	
	If we compute we get:
	\[\begin{split}
		n\circ \epsilon_n\circ e\circ s&=f_*(R_f(n))\circ e\circ s\\&=f\circ R_f(n)\circ s\\&=f\circ k
	\end{split} \qquad \begin{split}
		\epsilon_n\circ e\circ s\circ i&=\epsilon_n\circ e\circ t\\&=h\\&
	\end{split} \]
	
	On the other hand, the definition of $f_*$ gives us the inner rectangle of the diagram below.
	\[\xymatrix@C=30pt{K \ar@/_1.2cm/[ddd]_{w}\ar[dr]_(.35){k}|(.55)\hole\ar[r]^-{j} \ar[d]_{s}& f_*(K)  \ar@/_.3cm/[ddl]^(.75){f_*(s)}\ar[r]^-{f_*(k)}& Y \ar[dd]^{\id{Y}}\\R_f(N) \ar[d]_{e} \ar[r]|(.34)\hole_(.6){R_f(n)} & X  \ar[dr]^{f}\\f_*(R_f(N)) \ar[d]_{\epsilon_n} \ar[rr]_{f_*(R_f(n))}&& Y\\N\ar@/_.8cm/[urr]_{n}}\]
	
	From the commutativity of the whole diagram it follows that $\epsilon_n\circ f_*(s)=u$ and thus $s=\hat{u}$, entailing the thesis.
\end{proof}

\subsection{Factorization systems}

We move now to the notion of \emph{factorization system} \cite{adamek2009abstract,bousfield1977constructions,kelly1991note,rosicky2007factorization,tholen1983factorizations}. Let us start by recalling its definition.

\begin{definition}\label{def:fs}
	Let $\X$ be a category and $\mathcal{E}$, $\mathcal{M}$ two classes of arrows, we will say that $(\mathcal{E},\mathcal{M})$ is a \emph{factorization system} if:
	\begin{enumerate}
		\item $\mathcal{E}$ and $\mathcal{M}$ are closed under composition with isomorphisms: if $f\colon X\to Y$ belongs to $\mathcal{E}$ (to $\mathcal{M}$) and $h\colon Y\to Z$ is an isomorphism then $h\circ f$ belongs to $\mathcal{E}$ (to $\mathcal{M}$);
		\item  every arrow $f\colon X\to Y$ of $\X$ admits a \emph{$(\mathcal{E}, \mathcal{M})$-factorization}, i.e.~there are arrows $e\in \mathcal{E}$ and $m\in \mathcal{M}$ with the property that $f=m\circ e$;
		\item every $e\in \mathcal{E}$ has the \emph{left lifting property} with respect to every $m\in \mathcal{M}$: for every commutative square as the one below, with $e\in \mathcal{E}$ and $m\in \mathcal{M}$ there exists a unique dotted $k\colon Y\rightarrow Z$ which fits in it.
		\[\xymatrix{X \ar[r]^g \ar[d]_{e}& Z \ar[d]^{m}\\ Y \ar[r]_{f} \ar@{.>}[ur]^{k}& V}\] 
	\end{enumerate}
	
	A factorization system is \emph{proper} if every $e\in \mathcal{E}$ is epi and every $m\in \mathcal{M}$ is mono; it is \emph{stable} if $\mathcal{E}$ is stable under pullbacks.
\end{definition}

\begin{remark}\label{rem:dual}It is immediate to notice that, given a factorization system $(\mathcal{E}, \mathcal{M})$ on $\X$, then $(\mathcal{M}, \mathcal{E})$ is a factorization system on $\X^{op}$. Such factorization system on $\X^{op}$ is proper if and only if $(\mathcal{E}, \mathcal{M})$ is so, it is stable if and only if $\mathcal{M}$ is stable under pushouts in $\X$.  
\end{remark}

We can now begin to prove some properties of factorization systems.

\begin{proposition}\label{prop:el}
	Let $(\mathcal{E}, \mathcal{M})$ be a factorization system on a category $\X$, then the following hold true:
	\begin{enumerate}
		\item $\mathcal{E}$ is the class of all arrows with the left lifting properties with respect to every $m\in \mathcal{M}$;
		\item every isomorphism belongs to $\mathcal{E}$;
		\item $\mathcal{E}$ is closed under composition.
	\end{enumerate}
\end{proposition}
\begin{proof}
	\begin{enumerate}
		\item By definition, every arrow in $\mathcal{E}$ has the left lifting property with respect to every arrow in $\mathcal{M}$. To prove the converse, let $f\colon X\to Y$ be an arrow with such a property with respect to every element of $\mathcal{M}$.  We can factorize it as $m\circ e$ with $m\colon M\to Y$ in $\mathcal{M}$ and $e\colon X\to M$ in $\mathcal{E}$. Thus, in the diagram below, the outer square commutes and so we get the dotted $k\colon Y\to M$.
		\[\xymatrix{X\ar[d]_f\ar[r]^e & M\ar[d]^{m}\\Y\ar[r]_{\id{Y}} \ar@{.>}[ur]^{k} & Y}\]
		
		On the other hand, $k\circ m$ fits in the diagram below, showing that $k$ and $m$ are mutually inverses. Thus $f=m\circ e$ belongs to $\mathcal{E}$.
		\[\xymatrix{X\ar[rr]^{e} \ar[dd]_{e} \ar[dr]^{f}&& M\ar[dd]^{m} \ar@{<-}[dl]^{m}\\& Y\ar@{<-}[dl]^{k}\\M \ar[rr]_{m}&&Y}\]
		\item By the previous point, it is enough to show that every isomorphism has the left lifting property. But this follows at once: indeed given a square
		\[\xymatrix{X\ar[d]_e\ar[r]^f & M\ar[d]^{m}\\E\ar[r]_{g} \ar@{.>}[ur]^{k} & Y}\]
		the unique $k\colon E\to M$ filling the diagonal in the diagram above is $f\circ e^{-1}$.
		
		\item Let $e\colon X\to Y$ and $e'\colon Y\to Z$ be two arrows in $\mathcal{E}$. Suppose that the outer rectangle below is given, with $m\in \mathcal{M}$.
		\[\xymatrix{X  \ar[r]^f \ar[d]_{e}& M \ar[dd]^{m}\\Y \ar@{.>}[ur]^{k} \ar[d]_{e'}\\Z \ar@{.>}[uur]_{h}\ar[r]_{g}& W}\]
		
		Then the first dotted arrow $k\colon Y\to M$ exists since $e$ has the left lifting property with respecto to $m$. Its existence, in turn yields the existence of the other dotted arrow $h\colon Z\to M$. 
		
		Let $h'\colon Z\to M$ be another arrow such that 
		\[m\circ h'=g \qquad h'\circ e'\circ e=f\]
		Then $h'\circ e'$ must coincide with $k$ by the uniqueness clause of the left lifting property and thus, by the same clause, $h'$ must be equal to $h$. By the first point $e'\circ e$ belongs to $\mathcal{E}$. \qedhere 
	\end{enumerate}
\end{proof}

Taking into account \Cref{rem:dual}, the previous proposition gives us the following results.

\begin{corollary}\label{cor:m}Let $(\mathcal{E}, \mathcal{M})$ be a factorization system on a category $\X$, then the following hold true:
	\begin{enumerate}
		\item every isomorphism belongs to $\mathcal{M}$;
		\item $\mathcal{M}$ is closed under composition.
	\end{enumerate}
\end{corollary}

In the presence of a factorization system, we also have right $\mathcal{M}$-factorizations.

\begin{proposition}\label{prop:rfact}
	Let $(\mathcal{E}, \mathcal{M})$ be a factorization system on a category $\X$. Let $f\colon X\to Y$ be a morphism with $e\colon X\to M$ and $m\colon M\to Y$ as an $(\mathcal{E}, \mathcal{M})$-factorization of $f$, then the pair $(e, m)$ is also a right $\mathcal{M}$-factorization of $f$.
\end{proposition}
\begin{proof}
	Suppose that the solid part of the rectangle below is given, with $n\colon N\to Z$ in $\mathcal{M}$. 
	\[\xymatrix{X \ar[d]_{h} \ar@/^.4cm/[rr]^{f} \ar[r]_{e}& M \ar@{.>}[dl]^{k} \ar[r]_{m} & Y \ar[d]^{g}\\N \ar[rr]_{n}&& Z}\]
	Bu the existence of the dotted $k\colon M\to N$ is guaranteed by the left lifting property of $e$ with respect to $n$.
\end{proof}

The previous proposition, together with \Cref{rem:uniq} gives us the following.

\begin{corollary}\label{prop:iso}
	Let $(\mathcal{E}, \mathcal{M})$ be a factorization system on a category $\X$. If $e\colon Y\to E$, $e'\colon Y\to E'$ and $m\colon Y\to X$, $m'\colon Y'\to X $ are arrows, respectively, in $\mathcal{E}$ and $\mathcal{M}$ such that
	$e'\circ m'=e\circ m$,  then  there exist a unique isomorphism $\phi\colon E \rightarrow E'$ such that the diagram below commutes.
	\[\xymatrix@C=40pt{Y \ar[r]^{e'} \ar[d]_{e}& E' \ar[d]^{m'}\\ E \ar[r]_{m} \ar@{.>}[ur]^{\phi}& X}\]
\end{corollary}

The previous corollary, in turn, allows us to deduce other useful properties of factorization systems.

\begin{proposition}\label{cor:iso} Given a factorization system $(\mathcal{E}, \mathcal{M})$ on a category $\X$, the following hold:
	\begin{enumerate}
		\item an arrow $f\colon X\to Y$ with $(\mathcal{E}, \mathcal{M})$-decomposition given by $e$ in $\mathcal{E}$ and $m\in \mathcal{M}$ is in $\mathcal{M}$ if and only if $e$ is an isomorphism;
		\item $f\in \mathcal{E}$ and $f\in \mathcal{M}$ if and only if $f$ is an isomorphism;
		\item $\mathcal{M}$ is stable under pullback.
		\item  if $(\mathcal{E}, \mathcal{M})$ is proper, then $g\circ f$ is in $\mathcal{M}$ implies $f\in \mathcal{M}$.
	\end{enumerate}
\end{proposition}
\begin{proof}
	\begin{enumerate}
		\item  ($\Rightarrow$) By hypothesis $f=f\circ \id{X}$ is a factorization with $\id{X}\in \mathcal{E}$ and $f\in \mathcal{M}$, thus the thesis follows from \Cref{prop:iso}.
		
		\smallskip \noindent
		($\Leftarrow$) Let $f=m\circ e$ be an $(\mathcal{E}, \mathcal{M})$-factorization. The thesis  follows from \Cref{cor:m} since, if $e$ is an isomorphism, then $f$ is the composition of two arrows in $\mathcal{M}$.
		
		\item This follows from the previous point, \Cref{cor:m} and \Cref{prop:el}.
		
		\item 	Suppose that the square below is a pullback, with $m\in \mathcal{M}$
		\[\xymatrix{X\ar[r]^{g} \ar[d]_{n} & M \ar[d]^{m} \\ Y \ar[r]_{f}  & Z}\] 
		We can factor  $n$ as $t\circ e$ for $e\colon X\to M$ in  $\mathcal{E}$ and $t\colon T\to Y$ in $ \mathcal{M}$.  By the left lifting property we get the dotted $l\colon T\to M$ in the diagram below.
		\[\xymatrix{X \ar[d]_{e}\ar[rr]^{g} & &V\ar[d]^{m}\\  T  \ar@{.>}[urr]^{l} \ar[r]_{t}&Y\ar[r]_{f} & Z}\]
		Therefore we get another diagram
		\[\xymatrix{ T \ar@/^.5cm/[drr]^{l}\ar@{.>}[dr]^{k} \ar@/_.5cm/[ddr]_{t}\\ & X \ar[d]_{n}\ar[r]^{g}& M  \ar[d]^{m} \\ &Y \ar[r]_{f} & Z }\]
		and thus we can deduce the existence of the dotted $k\colon E\to V$. On the one hand, computing we get
		\[\begin{split}
			n\circ 	k\circ e&=t  \circ  e\\ &=n
		\end{split}\qquad \begin{split}
			g\circ k\circ e &=l\circ e\\ &=g \end{split}\]
		and so $k\circ e= \id{X}$.  On the other hand 
		\[\begin{split}
			t\circ k \circ t &= \id{Y} \circ t \\&= t
		\end{split} \qquad 
		\begin{split}
			k \circ t \circ e &= k\circ n \\&= e
		\end{split}
		\]	
		so the following diagram commutes.
		\[
		\xymatrix@C=50pt{X \ar[d]_{e}\ar[r]^{e}& T \ar[d]^{t} \\
			T\ar[ur]^{e\circ k}\ar[r]_{t} & Y}\]
		Hence, by the uniqueness clause of the left lifting property, $e\circ k=\id{E}$. Therefore $e$ is an isomorphism and the thesis follows from  point $1$. 
		
		\item 
		Factor $f$ and $g$ as $m_f\circ e_f$ and $m_g\circ e_g$, respectively. Let also $h$ be $e_g\circ m_f$ and factor it as $m_h\circ e_h$ so that we get 
		\[\xymatrix@C=45pt{& C \ar[dr]^{m_h}&\\
			A\ar[ur]^{e_h} \ar[rr]^{h}  \ar[dr]^{m_f}& &B \ar[d]^{m_g}\\
			X \ar[u]^{e_f} \ar[r]_f& Y \ar[r]_g \ar[ur]^{e_g} & Z}\]
		B \Cref{prop:el,cor:m}, $\mathcal{E}$ and $\mathcal{M}$ are closed under composition, therefore $e_h\circ e_f\in \mathcal{E}$ and $m_g \circ m_h\in \mathcal{M}$. Thus these two arrows gives a $(\mathcal{E}, \mathcal{M})$-factorization of $g\circ f$. On the other hand $g\circ f\in \mathcal{M}$, thus point $1$ above  implies that  $e_h\circ e_f$ is an isomorphism, so $e_f$ is a epic split mono which is also an epimorphism. \qedhere
	\end{enumerate}
\end{proof}

With the help of \Cref{rem:dual} we can easy dualize the previous proposition.
\begin{corollary}\label{cor:iso2}
	Given a factorization system $(\mathcal{E}, \mathcal{M})$ on a category $\X$, the following hold:
	\begin{enumerate}
		\item an arrow $f\colon X\to Y$ is in $\mathcal{E}$ if and only if $m_f$ is an isomorphism;
		\item $\mathcal{E}$ is stable under pushouts;
		\item  if $(\mathcal{E}, \mathcal{M})$ is proper, then $g\circ f$ is in $\mathcal{E}$ implies $g\in \mathcal{E}$.
	\end{enumerate}
\end{corollary}

We conclude this section with a result linking closely factorization systems with the existence of right $\mathcal{M}$-factorizations.

\begin{lemma}\label{lem:fact}Let $\mathcal{M}$ be a class of arrows in a category $\X$ closed under composition with isomorphisms, then the following are equivalent:
	\begin{enumerate}
		\item every arrow $f\colon X\to Y$ has a right $\mathcal{M}$-factorization and $\mathcal{M}$ is closed under composition;
		\item there exists a class $\mathcal{E}$ such that $(\mathcal{E}, \mathcal{M})$ is a factorization system on $\X$.
	\end{enumerate}
\end{lemma}
\begin{proof} $(1\Rightarrow 2)$ Let $\mathcal{E}$ be the class of arrows which have the left lifting property with respect to every element of $\mathcal{M}$. Let us show that the three properties of \Cref{def:fs} hold.
	
	\begin{enumerate}
		\item $\mathcal{M}$ is closed under composition with isomorphisms by hypothesis. On the other hand, if $e\colon X\to Y$ is in $\mathcal{E}$ and $h\colon Y \to Z$ is an isomorphism, then $h\circ e$ still belongs to $\mathcal{E}$. To see this, consider the square below, with $m\in \mathcal{M}$.
		\[\xymatrix{X\ar[r]^{f} \ar[d]_{e}& M\ar[dd]^{m}  \\Y \ar@{.>}[ur]_{k}\ar[d]_h \\Z \ar[r]_{g}& V}\]
		
		The dotted $k\colon Y\to M$ exists, and it is unique, because $e\in \mathcal{E}$. The thesis now follows taking $k\circ h^{-1}$ as the lifting of $g$. 
		
		\item Given $f\colon X\to Y$, let $e\colon X\to M$ and $m\colon M\to Y$ be a right $\mathcal{M}$-factorization of it. We have to show that $e$ belongs to $\mathcal{E}$. We can further factor $e$ as $n\circ d$, with $n\in \mathcal{M}$. Then we can build the rectangle below.
		\[\xymatrix{X  \ar[dr]_(.3){e}\ar[d]_{d} \ar@/^.4cm/[rr]^{f}\ar[r]_{e} & M \ar@{.>}[dl]^(.3){k} \ar[r]_{m} & Y \ar[d]^{\id{Y}}\\
			D \ar[r]_{n} & M \ar[r]_{m}& Y}\]
		
		By hypothesis $m\circ n$ is in $\mathcal{M}$, so we can deduce the existence of the dotted $k\colon M\to D$.   Now, from the diagram below, we can deduce that $n\circ k=\id{M}$.
		\[\xymatrix{X \ar[rr]^{e} \ar[dr]^{d}\ar[dd]_{e}& &M \ar[r]^{m}\ar[dl]^{k}& Y \ar[dd]^{\id{Y}}\\ & D\ar[dl]^{n}\\M\ar[rrr]_{m}&&&Y}\]
		
		On the other hand, the previous result yields also the commutativity of the following diagram, entailing $k\circ n=\id{D}$.
		\[\xymatrix{X \ar[rr]^{d} \ar[dr]^{e}\ar[dd]_{d}& &D \ar[r]^{n}\ar[dl]^{n}& M \ar[dd]^{\id{M}}\\ & M\ar[dl]^{k} \ar[drr]^{\id{M}}\\D\ar[rrr]_{n}&&&M}\]
		
		Now, suppose that the outer square in the diagram below is given, with $v\in \mathcal{M}$.
		\[\xymatrix{X \ar[dd]_{e} \ar[dr]_{d}\ar[rr]^{t}&& V \ar[dd]^{v}\\ & D \ar[dl]^{n} \ar@{.>}[ur]_{h}\\M \ar[rr]_{s}&& Z}\]
		
		Then the dotted $h\colon N\to V$ exists and it is unique since $n$ is a right $\mathcal{M}$-image of  $e$. Thus $h\circ k$ is the unique lifting for $s$ along $v$, proving that $e\in \mathcal{E}$.
		\item This holds by construction.
	\end{enumerate}
	
	\smallskip \noindent $(2\Rightarrow 1)$  This follows from the second point of \Cref{cor:m} and from \Cref{prop:rfact}. \qedhere  
\end{proof}

Putting together \Cref{lem:fact,lem:rfact} we get the following result.

\begin{corollary}\label{cor:fact}
	Let $\mathcal{M}$ be a class of arrows in a category $\X$ closed under composition stable under pullbacks and containing the identities. If $\X$ has $\mathcal{M}$-pullbacks and for every arrow $f\colon X\to Y$ the pullback functor $f^*$ has a left adjoint, then there exists a, unique, class $\mathcal{E}$ such that $(\mathcal{E}, \mathcal{M})$ is a factorization system on $\X$.
\end{corollary}

\subsection{$\mathcal{M}$-subobjects and factorization systems}

In this section we focus on studying the category $\mathcal{M}/X$ when $\mathcal{M}$ is a class of monomorphism.

\begin{remark}
	If  $\mathcal{M}$ is a class of monos in a category $\X$, then $\mathcal{M}/X(f,g)$ has cardinality at most one for every two arrows $f$ and $g$ with codomain $X$. Thus it is a preordered class, in which $f\leq g$ if and only if  $\mathcal{M}/X(f,g)$ is non-empty. Now, it is immediate to notice that $f\leq g$ and $g\leq f$, then there exists a (unique) isomorphism fitting in the triangle below.
	\[\xymatrix{A\ar@{.>}[rr]^{\phi} \ar[dr]_{f} && B \ar[dl]^{g}\\ & X}\]
	In such a case we will write $f\equiv g$. It is immediate to see that in this way we get an equivalence relation on the class of objects of $\mathcal{M}/X$.
\end{remark}

Given the previous remark, we are led to the following definition.

\begin{definition} Let $\mathcal{M}$ be a class of monos in a category $\X$. For every object $X$ a \emph{$\mathcal{M}$-subobject} is an equivalence class $[f]$ of an object $f$ in $\mathcal{M}/X$ with respect to $\equiv$. Given two \emph{$\mathcal{M}$-subobjects} $[f]$ and $[g]$, represented by $f$ and $g$, respectively, we will define $[f]\leq [g]$ if and only if $f\leq g$. It is immediate to see that this defines a partially ordered class $(\sub{M}{X}{X}, \leq)$. 
	
	When $\mathcal{M}$ coincides with the class of all monos we will suppress the prefix, so that a \emph{subobject} is just an equivalence class of a mono and $(\msub{X}{X}, \leq)$ will denote the ordered class of such subobjects of $X$.
\end{definition}

\begin{remark}\label{rem:pbo} Let $f\colon X\to Y$ be an arrows admitting a pullback along arrows in $\mathcal{M}/Y$ and suppose that $\mathcal{M}$ is stable under pullbacks. The categories associated to $(\sub{M}{X}{X}, \leq)$ and $(\sub{M}{Y}{Y}, \leq)$ are equivalent to $\mathcal{M}/X$ and $\mathcal{M}/Y$, respectively, thus we get a monotone function $f^{-1}\colon \sub{M}{X}{Y}\to \sub{M}{X}{X}$ sending $[m]$ to $[f^*(m)]$.
	
	Similarly, if $f\circ m$ has a right $\mathcal{M}$-factorization for every $m\in \mathcal{M}$, then the functor $f_*$ induces a monotone function $\exists_f\colon \sub{M}{X}{X}\to \sub{M}{X}{Y}$.
	
	Notice that, whenever $\mathcal{M}$ is stable unr pullbacks, $\X$ has $\mathcal{M}$-pullbacks and  right $\mathcal{M}$-factorizations, then $\exists_f \dashv f^{-1}$ by \Cref{lem:rfact}.
\end{remark}

\Cref{cor:finlim} gives us immediately the following.

\begin{proposition}\label{prop:finlim}
	Let $f\colon X\to Y$ be an arrow in a category $\X$ and $\mathcal{M}$ a class of arrows stable under pullbacks and such that $f$ has a pullback along any arrow in $\mathcal{M}$. Then $f^{-1}$ preserves binary infima. If, moreover, $\id{Y}$ belongs to $\mathcal{M}$ then $f^{-1}$ preserves finite infima.
\end{proposition}

Now, let us recall the following lemma, which is the application of the Adjoint Functor Theorem to the partial ordered case.

\begin{lemma}\label{lem:adj}
	Let $f\colon (P,\leq)\to (Q, \leq)$ be a monotone function. If $f$ preserves all infima then it has a left adjoint.
\end{lemma}
\begin{proof}
	For every $q\in Q$ define
	\[P_q:=\{p\in P \mid q \leq f(p)\}\]
	Clearly if $q'\leq q$ then $P_{q}\subseteq P_{q'}$. Let us define $g(q)$ as the infimum of $P_q$, the previous observation entails that in this way we get a monotone function $g\colon (Q, \leq)\to (P, \leq)$.
	We can also notice that, since $f$ preserves infima:
	\begin{align*}
		f(g(q))&=f(\inf(P_q))\\&=\inf(f(P_q))
	\end{align*}
	Now, by definition, $q$ is less or equal than every element in $f(P_q)$, thus $q\leq f(g(q))$, so that $g(q)$ is actually the infimum of $P_q$. 
	
	So equpped, we can now prove that $g$ is the right adjoint to $f$.
	\begin{itemize}
		\item Suppose that $g(q)\leq p$.  Then $f(g(q))\leq p$, but we already know that $q\leq f(g(q))$, therefore we can conclude that $q\leq f(p)$.
		\item  Suppose that $q\leq f(p)$. Thus $p$ belongs to $P_q$ and so $g(q)\leq p$. \qedhere 
	\end{itemize}
\end{proof}

The previous result, together with \Cref{cor:fact} and \Cref{prop:finlim}, allows us to deduce the following result, which has been used in \cite{baldan2011lattice,braatz2010finitary} to show that certain $\mathcal{M}$-adhesive categories have a factorization system.

\begin{corollary}\label{cor:fin}
	Let $\mathcal{M}$ be a class of monomorphism in a category $\X$ stable under pullbacks, containing all identities and closed under composition. If $\sub{M}{X}{X}$ is finite for every object $X$ of $\X$, then there exists a factorization system $(\mathcal{E},\mathcal{M})$ on $\X$
\end{corollary}


\section{Coreflective subcategories}

In order to identify a class of well-behaved DPO-rewriting system we need to recall some other general notion of category theory.

\begin{definition}
	Let $\X$ be a category, a full subcategory $\Y$ of $\X$ is \emph{coreflective} if the inclusion functor $I_\Y\colon \Y\to \X$ has a right adjoint $R_\Y$, called \emph{coreflector}.  A morphism between two coreflective subcategories $\Y$ and $\Z$ is a functor $F\colon \Y \to \Z$ such that $I_\Z \circ F=I_\Y$ and $F\circ R_\Y$ is isomorphic to $R_\Z$.
	
	We will denote by $\cori(\X)$ the category obtained in this way.
\end{definition}

\begin{remark}
	Notice that, since adjoints are unique up to isomorphism, being a morphism $\Y\to \Z$ of $\cori(\X)$ does not depends on the choice of coreflectors for $\Y$ and $\Z$.
\end{remark}

\begin{remark}\label{rem:ff}
	Let $F\colon \Y \to \Z$ be a morphism in $\cori(\X)$. Since $I_\Z\circ F=I_\Y$, then $\Y$ must be a full subcategory of $\Z$ and $F$ must coincide with its inclusion. In particular, $F$ is full and faithful and $\cori(\X)$ is a ordered class.
\end{remark}

\begin{example}\label{ex:1}
	\todo{connessi}
\end{example}

\begin{example}\label{ex:2}
	\todo{grafi con equivalenza}
\end{example}
\begin{example}
	\todo{Ce ne sono altri? sarebbe bello usare i calcoli ma credo rientrino nel primo}
\end{example}


Let us recall the following well known fact.

\begin{proposition}\label{prop:counit}
	Let $L\colon \X\to \Y$ and $R\colon \Y \to \X$ be a pair of functors such that $L\dashv R$. Then the following are equivalent:
	\begin{enumerate}
		\item $L$ is full and faithful;
		\item the unit $\eta \colon \id{\X}\to R\circ L$ is an isomorphism.
	\end{enumerate} 
\end{proposition}
\begin{proof}
	We can start noticing that, for every pair of objects $X$ and $Y$ in $\X$, we have two functions:
	\[F_{X, Y}\colon \X(X, Y)\to \Y(L(X), L(Y)) \quad 	 \Phi_{X, Y}\colon \Y(L(X), L(Y))\to \X(X, R(L(Y)))\]
	where $F_L$ sends $f\colon X\to Y$ to $L(f)$ and $\Phi_{X, Y}$ is the bijection $\Y(L(X), L(Y))\to \X(X, R(L(Y)))$ given by $L\dashv R$. Now, notice that, for every $f\colon X\to Y$ the diagram below commutes.
	\[\xymatrix@C=35pt{X \ar[r]^{f} \ar[d]_{\eta_X}& Y \ar[d]^{\eta_Y}\\R(L(X)) \ar[r]_-{R(L(f))} &R(L(Y)) }\]
	
	This, in turn, entail that the composit $\Psi_{X,Y} \circ F_{X, Y}$ coincides with $\eta_Y\circ (-)$.
	
	\smallskip \noindent 
	$(1\Rightarrow 2)$ Let $X$ and $Y$ be two objects of $\X$, by hypothesis $\eta_Y$ is an isomorphism and so $\eta_Y\circ (-)$ is a bijection. But then $F_{X, Y}$ must be a bijection too by what we have observed above.
	
	\smallskip \noindent 
	$(2\Rightarrow 1)$ Let $X$ be an object of $\X$, if $L$ is full and faithful then $\eta_Y\circ (-)$ is the composition of the two bijections $\Psi_{Y,Y}\circ F_{Y,Y}$. By the Yoneda Lemma this entails that $\eta_Y$ is an isomorphism.
\end{proof}


\begin{corollary}\label{cor:cou}
Let $L\colon \X\to \Y$ and $R\colon \Y \to \X$ be a pair of functors such that $L\dashv R$ and suppose that $L$ is full and faithful. Let $\epsilon\colon L\circ R\to \id{\Y}$ be the counit of the adjunction, then the following are true for every object $X$:
\begin{enumerate}
	\item the arrow $R(\epsilon_X)$ is an isomorphism;
	\item $L(R(\epsilon_X))$ is equal to $\epsilon_{L(R(X))}$.
\end{enumerate}
\end{corollary}
\begin{proof}
	\begin{enumerate}
		\item Let $\eta\colon \id{X}\circ R\circ L$ be the unit of the adjunction. By the triangular identity we know that $R(\epsilon_X)\circ \eta_{R(X)}=\id{R(X)}$. The thesis now follows from \Cref{prop:counit}.
		\item Using again the triangular identities and the previous point we have
		\begin{align*}
			\id{L(R(X))}&=\epsilon_{L(R(X))}\circ L(\eta_{R(X)})\\&= \epsilon_{L(R(X))}\circ L(R(\epsilon_X))^{-1}
		\end{align*}
		Which yields the theis. \qedhere 
	\end{enumerate}
\end{proof}

\begin{corollary}\label{cor:counit}
	Let $\Y$ be a coreflective subcategory of $\X$ and let $\epsilon \colon I_\Y\circ R_\Y \to  \id{\X}$ be the counit of $I_\Y\dashv R_\Y$. Given an object $X\in \X$, the following are equivalent:
	\begin{enumerate}
		\item $X$ is in the essential image of the inclusion $I_\Y$;
		\item $\epsilon_X$ is an isomorphism.
	\end{enumerate} 
\end{corollary}
\begin{proof}
	$(1\Rightarrow 2)$ By hypothesis there is an object $Y$ of $\Y$ and an isomorphism $\phi\colon I_\Y(Y)\to X$. This, in turn, yields the commutative square below.
	\[\xymatrix@C=40pt{I_\Y(Y) \ar[r]^-{\phi} \ar[d]_{I_\Y(\eta_Y)}&  X\\ I_\Y(R_\Y(I_\Y(Y)))  \ar[r]_-{I_\Y(R_\Y(\phi))}& I_\Y(R_\Y(X)) \ar[u]_{\epsilon_X}}\]
	
	By hypotesis $\phi$ is an isomorphism and $\eta_Y$ is so by \Cref{prop:counit}, thus $I_\Y(\eta_Y)$ and 
	$I_\Y(R_\Y(\phi))$ are isomorphisms too and the thesis follows.
	
	\smallskip \noindent
	$(2\Rightarrow 1)$ This is a tautology.
\end{proof}

Associated to a coreflective subcategory $\Y$ of $\X$ we have a class of arrows which enjoys some useful properties.

\begin{definition}
	Let $\Y$ be a coreflective subcategory of $\X$ with coreflector $R_\Y$, we define the class $\mathcal{R}_\Y$ as the class of all arrows in $\X$ whose image through $R_\Y$ is a mono in $\Y$.
\end{definition}

\begin{remark}
	Let $I_\Y\colon \Y \to \X$ be the inclusion functor of a coreflective subcategory $\Y$. If $R_\Y$ and $R'_\Y$ are coreflectors, then we have a natural isomorphism $\phi\colon R_\Y\to R'_\Y$ and so, for every $f\colon X\to Y$ there is a square
	\[\xymatrix{ \ar[d]_{\phi_X} R_\Y(X) \ar[r]^{R_\Y(f)}& R_\Y(Y) \ar[d]^{\phi_Y}\\R'_\Y(X) \ar[r]_{R_\Y(f)} & R'_\Y(Y)}\]
	
	In particular, $R_\Y(f)$ is mono if and only if $R'_\Y(f)$ is so, thus the definition of $\mathcal{R}_\Y$ is independent from the choice of coreflector.
\end{remark}

\begin{proposition}\label{prop:varie}
	Let $\Y$ be a coreflective subcategory of $\X$. Then the following hold true:
	\begin{enumerate}
		\item $\mathcal{R}_\Y$ contains all monos;
		\item $\mathcal{R}_\Y$ is stable under pullbacks;
		\item given $f\colon X\to Y$ and $g\colon Y\to Z$, if $g\circ f$ is in $\mathcal{R}_\Y$ then $f$ is in $\mathcal{R}_\Y$ too;
		\item given $f\colon X\to Y$ in $\mathcal{R}_\Y$ and $g,h\colon I_\Y(Z)\rightrightarrows X$, if $f\circ g = f\circ h$, then $g=h$.
	\end{enumerate}
\end{proposition}
\begin{proof}\begin{enumerate}
		\item This follows at once from the fact that $R_\Y$, being a right adjoint, preserves monos.
		\item Consider the following two diagrams, with $f\in \mathcal{R}_\Y$, in which the left one is a pullback. $R_\Y$ preserves limits, thus the right square is a pullback too.
		\[\xymatrix@C=30pt{P \ar[r]^{p_1} \ar[d]_{p_2}& X \ar[d]^{f} & R_\Y(P) \ar[r]^{R_\Y(p_1)} \ar[d]_{R_\Y(p_2)} & R_\Y(X) \ar[d]^{R_\Y(f)}\\Z \ar[r]_{g}& Y & R_\Y(Z) \ar[r]_{R_\Y(g)} & R_\Y(Y)}\]
		By definition, $R_\Y(f)$ is a mono in $\Y$, but this implies that $R_\Y(g)$ is a mono too, proving $g\in \mathcal{R}_\Y$.
		\item We know that $R_\Y(g\circ f)$ is mono, thus $R_\Y(f)$ is mono too.
		\item 	Since $f\circ g=f\circ h$ and $f$ is in $\mathcal{R}_\Y$ we can conclude that $R_\Y(g)=R_\Y(h)$. Thus we have:
		\begin{align*}
			g\circ \epsilon_{I_\Y(Z)}&=\epsilon_X\circ I_\Y(R_\Y(g))\\&=\epsilon_X\circ I_\Y(R_\Y(h))\\&=h\circ \epsilon_{I_\Y(Z)}
		\end{align*}
		By \Cref{cor:counit},  $\epsilon_{I_\Y(Z)}$ is an isomorphism and so $g=h$. \qedhere
	\end{enumerate}
\end{proof}

In general, the class $\mathcal{R}_{\Y}$ is not stable under pushout. Nonetheless such property is desirable. We are thus lead to the following definition.

\begin{definition}
	A coreflective subcategory is \emph{good} if  $\mathcal{R}_\Y$ is stable under pushout. $\gori(\X)$ will denote the subcategory of $\cori(\X)$ given by good coreflective subcategories. 
\end{definition}

\iffalse 
\begin{remark}\label{rem:good}
	Let $\Y$ be a full subcategory of $\X$ with coreflector $R_\Y$.  Let also $f\colon X\to Y$ be in $\mathcal{R}_\Y$ and consider the following two diagrams, where the left one is a pushout. Then $\Y$ is a good subcategory if and only if $R_{\Y}()$
\end{remark}
\fi 

We end this section extending the construction of $\mathcal{R}_\Y$ to a functor on $\cori(\X)$.

\begin{proposition}\label{prop:funct}
	Let $F\colon \Y \to \Z$ be a morphism of $\cori(\X)$, then $\mathcal{R}_\Z \subseteq \mathcal{R}_{\Y}$.
\end{proposition}
\begin{proof}
	Let $f$ be an arrow in $\mathcal{R}_\Z$. Since $F$ is an arrow in $\cori(\X)$ we know that there exists a natural isomorphism $\phi\colon F\circ R_\Y\to R_\Z$, therefore $F(R_\Y(f))$ is a monomorphism. By \Cref{rem:ff} we know that $F$ is full and faithful and so reflects mono, implying that $f$ belongs to $\mathcal{R}_\Y$ too.
\end{proof}

\begin{corollary}\label{cor:funct}
	Let $\cla(\X)$ be the category associated to the collection of all classes of a category $\X$, ordered by the inclusion. Then there exists a functor $\mathcal{R}_{(-)}\colon \cori(\X)^{op}\to \cla(\X) $.
	
	Let, moreover, $\pos(\X),$ be the subcategory of $\cla(\X)$  given by classes of arrows stable under pushouts, Then  $\mathcal{R}^G_{(-)}\colon \gori(\X)^{op}\to \pos(\X) $ fitting in the rectangle below, where the vertical arrows are inclusions.
	\[\xymatrix{\cori(\X)^{op} \ar[r]^{\mathcal{R}_{(-)}}& \cla(\X)\\ \gori(\X)^{op} \ar[u]^{J} \ar[r]_{\mathcal{R}^{G}_{(-)}}& \pos(\X) \ar[u]_{I}}\]
\end{corollary}

\iffalse 
\begin{definition}
	A \emph{comonad} $\m{C}$ on a category $\X$ is a triple $(C, \delta, \epsilon)$ given by a functor $C\colon \X\to \X$ and two natural transformations $\delta\colon C\to C\circ C $, $\epsilon\colon C\to \id{\X}$, called, respectively \emph{comultiplication} and \emph{counit} such that the following diagram commute
	\[\xymatrix@C=30pt{C \ar[d]_{\delta}\ar[r]^-{\delta}& C\circ C  \ar[d]^{\delta*C}&  C \ar[d]_{\delta}\ar[r]^{\delta} \ar[dr]^-{\id{C}} & C\circ C \ar[d]^{\epsilon *C}\\ C\circ C \ar[r]_-{C*\delta}& C\circ C\circ C & C\circ C \ar[r]_-{C*\epsilon} & C}\]
	
	A comonad $\m{C}$ is \emph{idempotent} if $\delta$ is an isomorphism.  A \emph{morphism} between two comonad $\m{C}=(C, \delta, \epsilon)$ and $\m{D}=(D, \mu, \eta)$ is a natural transformation $\chi\colon C\to D$ such that the following diagrams commute
	\[\xymatrix@C=30pt{ C \ar[r]^-{\delta} \ar[d]_{\chi} & C\circ C \ar[d]^{\chi*\chi}& C \ar[r]^{\epsilon} \ar[d]_{\chi}& 
		\id{\X}  \\ D \ar[r]_-{\mu} & D\circ D
		& D \ar[ur]_{\eta} }\]	
	
	We will denote by $\cmd(\X)$ the (possibly not locally small) category whose objects are idempotent comonads and whose arrows are morphisms of monads.
\end{definition}

\begin{remark} Let us recall that, given two functors $F, G\colon \X\rightrightarrows \Y$ and a natural transformation $\chi$ between them, then for every $X\in \X$ we have a naturality square
	\[\xymatrix@C=40pt{F(F(X)) \ar[r]^{F(\chi_X)} \ar[d]_{\chi_{F(X)}}& F(G(X)) \ar[d]^{\chi_{G(X)}}\\G(F(X)) \ar[r]_{G(\chi_X)} & G(G(X))}\]
	
	This entails that $(\chi * G)\circ (F*\chi)$ and $(G*\chi) \circ (\chi* F)$ are the same natural transformation, which we denote by $\chi*\chi$.
\end{remark}


\begin{proposition}
	Let $L\colon \X \to \Y$ and $R\colon \Y \to \X$ be two functors such that $L\dashv R$. If $\eta \colon \id{\X}\to R\circ L$ and $\epsilon\colon L\circ R \to \id{Y}$ are the unit and the counit of this adjunction, then $(L\circ R, L*\eta *R ,\epsilon)$ is a comonad on $\Y$.
\end{proposition}
\begin{proof}
	contenuto...
\end{proof}

The next definition shows that every comonad come from an adjunction.

\begin{definition}
	contenuto...
\end{definition}

\begin{proposition}
	contenuto...
\end{proposition}
\begin{proof}
	contenuto...
\end{proof}


We recall the following lemma relating idempotent comonads and coreflective subcategories.

\begin{lemma}\label{lem:comsub}
	Let $L\colon \Y\to \X$ and $R\colon \X \to \Y$ be two functors such that $L\dashv R$
\end{lemma}
\begin{proof}
	contenuto...
\end{proof}

Associated to any idempotent comonad we have a class of arrows.

\begin{definition}
	Let $\m{T}$ be an idempotent comonad on a category $\X$. We say that an arrow $f:X\to Y$ i
\end{definition}
\begin{remark}
	contenuto...
\end{remark}
\fi 


\subsection{A left adjoint to $\mathcal{R}_{(-)}$}

In this section we will show that, under suitable hypotheses, the functors $\mathcal{R}_{(-)}$ and $\mathcal{R}^G_{(-)}$ built in the previous sections, both have left adjoints. To do so, we need some preliminary work to be done.


\begin{definition}
	Let $\mathcal{M}$ be a class of arrows in a category $\X$ and let $\{m_i\}_{i\in I}$ be a family of objects in $\mathcal{M}/X$. We say that $\{m_i\}_{i\in I}$ has the \emph{epic $\mathcal{M}$-sum property} if:
	\begin{enumerate}
		\item it has a  coproduct $m\colon M\to X$;
		\item the family  of coprojection $\{k_i\}_{i\in I}$ is jointly epic in $\X$.
	\end{enumerate}
	
	We say that $\X$ has \emph{epic $\mathcal{M}$-sums} if every family in $\sub{M}{X}{X}$ has the epic $\mathcal{M}$-sum property.
\end{definition}

Clearly the family $\{k_i\}_{i\in I}$ is always epic in $\mathcal{M}/X$, but from this it does not follow that it is so in $\X$, as shown from the following example.

\begin{example}Let $\cat{Top}$ be the category of topological spaces and $\mathcal{C}$ the category of closed embeddings, and  consider the family $\{c_i\}_{i\in \mathbb{N}}$, where $c_i$ is the inclusion of $[\frac{1}{i+1}, +\infty)$ into $\mathbb{R}_{\geq 0}$. In $\mathcal{C}/\mathbb{R}_{\geq 0}$ their coproduct is the  identity $\id{\mathbb{R}_{\geq 0}}$. To see that the family of coprojections $\{k_i\}_{i\in \mathbb{N}}$ is not jointly epic in $\cat{Top}$, we can consider the space $X$ given by the quotient of $\mathbb{R}_{\geq 0}+\mathbb{R}_{\geq 0}$ by the equivalence relation generated from
	\[R:=\{(x, y)\in X\times X \mid x=\iota_1(r), y=\iota_2(r) \text{ for some } r \in \mathbb{R}_{\geq 0}\smallsetminus\{0\}\}\]
	Now, let $\pi$ be the quotient map and define $f_1, f_2\colon \mathbb{R}_{\geq 0}\rightrightarrows X$ as $\pi\circ \iota_1$ and $\pi \circ \iota_2$, respectively. Then, for every $i\in \mathbb{N}$ we have that $f_1\circ c_i=f_2\circ c_i$ because, given $r\in [\frac{1}{i+1}, +\infty)$ we have
	\begin{align*}
		f_1(c_i(r))&=\pi(\iota_1(c_i(r)))\\&=\pi(\iota_1(r))\\&=\pi(\iota_2(r))\\&=\pi(\iota_1(c_i(r)))\\&=f_1(c_i(r))
	\end{align*}	
	
On the other hand, $f_1(0)\neq f_2(0)$, thus $\cat{Top}$ does not have epic $\mathcal{C}$-sums.
\end{example}
\begin{example}
	\todo{insiemi}
\end{example}

\begin{example}
	\todo{adesive finitare}
\end{example}
\begin{example}
	\todo{simple graph}
\end{example}

We are now ready to construct a left adjoint to $\mathcal{R}_{(-)}$.

\begin{definition} 
	Let $\X$ be a category and $\mathcal{A}$ a class of arrows of it. We define the category $\C(\mathcal{A})$ as the full subcategory of $\X$ given by the objects $Z\in \X$ such that, for every  pair of arrows $g,h\colon Z\rightrightarrows X$, if $f\circ g=f\circ h$ for some $f\colon X\to Y$ belonging to $\mathcal{A}$, then $g=h$.
\end{definition}



A first thing that we can notice about $\C(\mathcal{A})$ is that, in the presence of a factorization system, is its closure under epimorphisms, as shown by the following proposition.

\begin{proposition}\label{prop:ima}
	$\mathcal{A}$ be a class of arrows in a category $\X$ If $e\colon X\to Y$ is an epimorphism with domain in $\C(\mathcal{A})$, then $Y$ belongs to $\C(\mathcal{A})$ too.
\end{proposition}
\begin{proof}
	Let $f\colon Z\to V$ be an arrow of $\mathcal{A}$ and take two other arrows $g,h\colon Y\to Z$ such that $f\circ g=f\circ h$. Then we also know that
	\[f\circ g\circ e= f\circ h\circ e\]
	By hypothesis $X$ is in $\C(\mathcal{A})$, so that $g\circ e=f\circ e$. The thesis now follows since $e$ is an epimorphism.
\end{proof}
\begin{example}
	\todo{spazi topologici puntati - controesempio}
\end{example}


\begin{lemma}\label{lem:cor}
	Let $\X$ be a category with a proper factorization system $(\mathcal{E}, \mathcal{M})$. Suppose, moreover, that $\X$ has the epic $\mathcal{M}$-sum property.  Then for every $\mathcal{A}\in \cla(\X)$, $\C(\mathcal{A})$ is a coreflective subcategory of $\X$.
\end{lemma}

\begin{proof}
	For every object $X$ of $\X$, let us define
	\[\mathcal{F}_X:=\{m\in \mathcal{M}/X \mid \dm{m} \text{ belongs to } \C(\mathcal{A})\}\] 
	We can then define $R(X)$ as the domain of the sum over $\mathcal{F}_X$ in $\mathcal{M}/X$. In particular, we have an arrow $\epsilon_X\colon R(X)\to X$, belonging to $\mathcal{M}$.
	
	First of all, we have to prove that $R(X)$ so defined is an object of $\C(\mathcal{A})$. Let thus $f\colon A\to Y$ be an arrow in $\mathcal{A}$ and suppose that there are $g,h\colon R(X)\to A$ such that  $f\circ h= f \circ g$. Now, for every $m\colon M\to X$ in $\mathcal{F}_X$ we have a coprojection $k_m\colon m\to \epsilon_X$. Precomposing with it we get
	\[f\circ h \circ k_m=f\circ g\circ k_m\]
	By definition $M$ is in $\C(\mathcal{A})$, so we can conclude that $h\circ k_m=g\circ k_m$. Since  this holds for every $m\in \mathcal{F}_Z$, the epic $\mathcal{M}$-sum property of $\X$ entails $f=g$ as wanted.
	
	To conclude, is it now enough to show that $\epsilon_X$ has the universal property of a counit. Let $Y$ be an object of $\C(\mathcal{A})$ with an arrow $f\colon Y\to X$. We can factor $f$ it as $m\circ e$ with $e\colon Y\to E$ and $m\colon E\to X$ in $\mathcal{M}$. By \Cref{prop:ima} we know that $E$ is in $\C(\mathcal{A})$, thus there exists a coprojection $k_m\colon m\to \epsilon_X$.  In particular we have
	\begin{align*}
		\epsilon_X\circ k_m\circ e&=m\circ e = f
	\end{align*}

On the other hand, if $g\colon Y\to R(X)$ is such that $\epsilon_X\circ g=f$, then $g$ must coincide with $k_m\circ e$ because $\epsilon_X$, being in $\mathcal{M}$ is monic.
\end{proof}

\begin{example}
	\todo{spazi topologici puntati - controesempio 2}
\end{example}


\begin{proposition}Let $(\mathcal{E}, \mathcal{M})$ be a proper factorization system on a acategory $\X$ with $\mathcal{M}$-sums. Then the following hold true:
	\begin{enumerate}
		\item for every class $\mathcal{A}$ in $\cla(\X)$, $\mathcal{A}\subseteq \mathcal{R}_{\C(\mathcal{A})}$;
		\item the functor $\mathcal{R}_{(-)}\colon \cori(\X )^{op}\to \cla(\X)$ has a left adjoint.
	\end{enumerate} 
\end{proposition}
\begin{proof}\begin{enumerate}
		\item Let $f\colon X\to Y$ be an arrow in $\mathcal{A}$ and suppose that $g,h\colon Z\to R(X)$ are arrows in $\C(\A)$ such that $R(f)\circ g=R(f)\circ h$. Then
		\begin{align*}
			f\circ \epsilon_X\circ I(g)&=\epsilon_Y\circ I(R(f))\circ I(g)\\&=
			\epsilon_Y\circ I(R(f))\circ I(h)\\&=f\circ \epsilon_X\circ I(h)
		\end{align*}
		Since $f$ is in $\mathcal{A}$, we can deduce that $\epsilon_X\circ I(g)=\epsilon_X\circ I(h)$. The thesis now follows from the fact that $\epsilon_X$ is mono.
		
		\item Let $\Y$ be a coreflective subcategory of $\X$ such that $\mathcal{A}\subseteq \mathcal{R}_\Y$. We have to show that $\Y\subseteq \C(\mathcal{A})$. Let then $Y$ be an object of $Y$, $f\colon X\to Z$ and arrow in $\mathcal{A}$ and $g, h\colon I_\Y(Y)\rightrightarrows X$ two arrows in $\X$ such that $f\circ g=f\circ h$. Applying $R_\Y$ and since, by hypothesis, $R_\Y(f)$ is mono, we get that $R_\Y(g)=R_\Y(h)$. Now, by naturality of the counit we have
	\begin{align*}g\circ \epsilon_{I_\Y(Z)}&=\epsilon_X\circ R_\Y(g)\\&=\epsilon_X\circ R_\Y(h)\\&=h\circ \epsilon_{I_\Y(Z)}
	\end{align*}
	By \Cref{cor:counit}  we know that $\epsilon_{I_\Y(Z)}$ is an isomorphism and we can conclude. \qedhere 
	\end{enumerate}
\end{proof}


\newpage
\begin{definition}Let $\X$ be an $\mathcal{M}$-adhesive category and consider a left-linear DPO-rewriting system $(\X, \R)$ on it.  Take two non-empty decorated derivations $(\der{D}, \alpha, \omega)$ and  $(\der{D}', \alpha', \omega')$ with the same length and with isomorphic sources and targets.
	
	If $\der{D}=\{\dder{D}_i\}_{i=0}^n$ and $\der{D}'=\{\dder{D}'_i\}_{i=0}^n$ with assciated sequence of rules $r(\der{D})=\{\rho_i\}_{i=0}^n$ and $r(\der{D}')=\{\rho'_i\}_{i=0}^n$, a \emph{semi-consistent permutation} between  $(\der{D}, \alpha, \omega)$ and $(\der{D}', \alpha', \omega')$ is a permutation $\sigma\colon [0,n]\to [0,n]$  such that, for every $i\in [0,n]$, $\rho_i=\rho'_{\sigma(i)}$ and, moreover, there exists a \emph{mediating isomorphism} $\xi_\sigma\colon \tpro{D} \to \lpro \der{D}' \rpro$ fitting in the following diagrams, where $m_i, m'_i, h_i$ and $h'_i$ are, respectively, the matches and back-matches of $\dder{D}_i$ and $\dder{D}'_i$.
	\[\xymatrix@C=30pt{\pi(G_0)\ar[r]^{\alpha} \ar[d]_{\alpha'} & G_0 \ar[r]^{\iota_{G_0}} &\tpro{D} \ar[d]^{\xi_\sigma}\\ G'_0 \ar[rr]_{\iota'_{G'_0}} & &\lpro \der{D}' \rpro\\L_i \ar[r]^{m_i} \ar[d]_{m'_{\sigma(i)}}& G_i \ar[r]^{\iota_{G_i}} &\tpro{D} \ar[d]^{\xi_\sigma} & R_i \ar[r]^{h_i} \ar[d]_{h'_{\sigma(i)}}& G_{i+1} \ar[r]^{\iota_{G_{i+1}}} &\tpro{D} \ar[d]^{\xi_\sigma} \\G'_{\sigma(i)} \ar[rr]_{\iota'_{G'_{\sigma(i)}}}&& \lpro \der{D}' \rpro& G'_{\sigma(i)+1} \ar[rr]_{\iota'_{G'_{\sigma(i)+1}}}&& \lpro \der{D}' \rpro}\]
\end{definition}

\begin{remark}\label{rem:coproj}
	Notice that, in particular, the previous diagram entails
	\[\xi_\sigma \circ \iota_{L_i}=\iota'_{L_{\sigma(i)}} \quad \xi_\sigma \circ \iota_{K_i}=\iota'_{K_{\sigma(i)}} \quad \xi_\sigma \circ \iota_{R_i}=\iota'_{R_{\sigma(i)}} \]
\end{remark}


\begin{proposition}\label{prop:isouno} For every semi-consistent permutation $\sigma$ between  $(\der{D}, \alpha, \omega)$ and $(\der{D}', \alpha', \omega')$, the mediating isomorphism $\xi_\sigma\colon \tpro{D}\to \lpro \der{D}'\rpro$ is unique.
\end{proposition}
\begin{proof} Let $\xi'_\sigma$ be another mediating isomorphism, then by \Cref{rem:coproj} we have
	\[\begin{split}
		\xi_\sigma \circ \iota_{L_i}&=\iota'_{L_{\sigma(i)}}\\&=\xi'_\sigma\circ \iota_{L_i}
	\end{split} \qquad \begin{split}
		\xi_\sigma \circ \iota_{K_i}&=\iota'_{K_{\sigma(i)}}\\&=\xi'_\sigma\circ \iota_{K_i}
	\end{split} \qquad \begin{split}
		\xi_\sigma \circ \iota_{R_i}&=\iota'_{R_{\sigma(i)}}\\&=\xi'_\sigma\circ \iota_{R_i}
	\end{split}\]
	
	Now, notice that 
	\begin{align*}
		\xi_\sigma \circ \iota_{G_0}&=\iota'_{G'_0}\circ \alpha'\circ \alpha^{-1}\\&=\xi'_\sigma \circ \iota_{G_0}
	\end{align*}
	
	If $\lgh(\der{D})=0$ this is enough to conclude, otherwise we are going to prove by induction that, for every $i\in [0, \lgh(\der{D})-1]$
	\[
	\xi_\sigma \circ \iota_{G_i}=\xi'_\sigma\circ \iota_{G_i}
	\]	
	
	\smallskip \noindent $i=0$. This is simply the result obtained before.
	
	\smallskip \noindent $i >0$. If $i>0$, we know that there is a pushout square
	\[\xymatrix{K_{i-1}\ar[r]^{r_{i-1}} \ar[d]_{k_{i-1}}& R_{i-1} \ar[d]^{h_{i-1}}\\ D_{i-1} \ar[r]_{g_{i-1}}& G_i}\] 
	By \Cref{rem:coproj} and the induction hypothesis we know that
	\[\begin{split}
		\xi_\sigma \circ \iota_{G_i}\circ h_{i-1}&=  \xi_\sigma\circ \iota_{R_{i-1}}\\&=\iota'_{R_{\sigma(i-1)}}\\&=\xi'_{\sigma}\circ \iota_{R_{i-1}}\\&=\xi'_{\sigma} \circ \iota_{G_i}\circ h_{i-1}\\&
	\end{split} \qquad
	\begin{split}
		\xi_\sigma \circ \iota_{G_i}\circ g_{i-1}&=\xi_{\sigma}\circ \iota_{D_{i-1}}\\&=\xi_{\sigma}\circ \iota_{G_{i-1}} \circ f_{i-1}\\&=\xi'_{\sigma}\circ \iota_{G_{i-1}} \circ f_{i-1}\\&=\xi'_{\sigma}\circ \iota_{D_{i-1}} \\&=\xi'_{\sigma}\circ \iota_{G_{i}} \circ g_{i-1}
	\end{split}
	\]
	
	Since $\xi_\sigma \circ \iota_{D_i}$ must be $\xi_\sigma \circ \iota_{G_i}\circ f_i$, we also have
	\[\xi_\sigma \circ \iota_{D_i}=\xi'_\sigma \circ \iota_{D_i}\] 
	and the thesis follows.
\end{proof}

\begin{proposition}\label{prop:abstoswitch}
	Let $(\der{D, \alpha, \omega})$ and $(\der{D}', \alpha, \omega')$ be two derivation with an abstract equivalence among them with the identity as first component. Then $\id{n}$ is a semi-consistent permutation among them.
\end{proposition}
\begin{proof}
Let $\xi\colon \tpro{D}\to \lpro \der{D}'\rpro$ be the unique arrow $\tpro{D}\to \lpro \der{D}'\rpro$ satisfying 
\[\xi \circ \iota_{G_i}:=\iota'_{G'_i}\circ \phi_{G_i}\] 

Then the square below commute.
	\[\xymatrix@C=30pt{\pi(G_0)\ar[r]^{\alpha} \ar[d]_{\alpha} & G_0 \ar[dl]^{\id{G_0}} \ar[r]^{\iota_{G_0}} &\tpro{D} \ar[d]^{\xi}\\ G_0 \ar[rr]_{\iota'_{G_0}} & &\lpro \der{D}' \rpro\\L_i \ar[r]^{m_i} \ar[d]_{m'_{i}}& G_i \ar[dl]^{\phi_i} \ar[r]^{\iota_{G_i}} &\tpro{D} \ar[d]^{\xi} & R_i \ar[r]^{h_i} \ar[d]_{h'_{i}}& G_{i+1} \ar[dl]_{\phi_{i+1}} \ar[r]^{\iota_{G_{i+1}}} &\tpro{D} \ar[d]^{\xi} \\G'_{\sigma(i)} \ar[rr]_{\iota'_{G'_{i}}}&& \lpro \der{D}' \rpro& G'_{i+1} \ar[rr]_{\iota'_{G'_{i+1}}}&& \lpro \der{D}' \rpro}\]
	But this is exactly the thesis.
\end{proof}


\begin{lemma}Let $\X$ be an $\mathcal{M}$-adhesive category and $(\X,\R)$ a well-switching rewriting system.
	Consider a derivation $\der{D}=\{\dder{D}_i\}_{i=0}^2$ and suppose that we have the following two switching sequences
	\[\der{D} \shift{(0,1)} \der{D}' \shift{(1,2)} \der{D}'' \shift{(0,1)}\der{D}'''\\
	\qquad \der{D}\shift{(1,2)}\der{E}'\shift{(0,1)} \der{E}'' \shift{(1,2)}\der{E}''' \]
	Then $\der{D}'''$ and $\der{E}'''$ are abstraction equivalent. The abstraction equivalence has the identity has a first and last isomorphism.
\end{lemma}
\begin{proof}
	To fix the notation, let us depict the derivation $\der{D}=\{\dder{D}_i\}_{i=0}^2$
	\[\xymatrix@C=15pt{L_0\ar[d]^{m_0}&& K_0 \ar[d]_{k_0}\ar@{>->}[ll]_{l_0} \ar[r]^{r_0} & R_0 \ar@/^.35cm/[drrr]_(.4){i_0}|(.29)\hole \ar[dr]|(.28)\hole_{h_0} && L_1 \ar@/_.35cm/[dlll]^(.4){i_1} \ar[dl]|(.28)\hole^{m_1}& K_1 \ar[d]^{k_1}\ar@{>->}[l]_{l_1} \ar[r]^{r_1} & R_1 \ar[dr]|(.28)\hole_{h_1} \ar@/^.35cm/[drrr]_(.4){j_0}|(.29)\hole  && L_2 \ar@/_.35cm/[dlll]^(.4){j_1} \ar[dl]|(.28)\hole^{m_2}& K_2 \ar[d]^{k_2}\ar@{>->}[l]_{l_2} \ar[rr]^{r_2} && R_2 \ar[d]_{h_2} \\G_0 && \ar@{>->}[ll]^{f_0} D_0 \ar[rr]_{g_0}&& G_1  && \ar@{>->}[ll]^{f_1} D_1 \ar[rr]_{g_1}&& G_2 && \ar@{>->}[ll]^{f_2} D_2 \ar[rr]_{g_2}&& G_3 }\]
	
	We are going to do the two sequences of switches and compare the results.
	
	\smallskip \noindent First sequence of switches.
	
	We begin switching $\dder{D}_0$ and $\dder{D}_1$ to get the following diagram
	\[\xymatrix@C=15pt{L_1\ar[d]^{f_0\circ i_1}&& K_1 \ar[d]_{k'_1}\ar@{>->}[ll]_{l_1} \ar[r]^{r_1} & R_1 \ar@/^.35cm/[drrr]_(.4){a_0}|(.29)\hole \ar[dr]|(.28)\hole_{h'_1} && L_0 \ar@/_.35cm/[dlll]^(.4){a_1} \ar[dl]|(.28)\hole^{m'_0}& K_0 \ar[d]^{k'_0}\ar@{>->}[l]_{l_0} \ar[r]^{r_0} & R_0 \ar[dr]|(.28)\hole_{g_1\circ i_0} \ar@/^.35cm/[drrr]_(.4){b_0}|(.29)\hole  && L_2 \ar@/_.35cm/[dlll]^(.4){b_1} \ar[dl]|(.28)\hole^{m_2}& K_2 \ar[d]^{k_2}\ar@{>->}[l]_{l_2} \ar[rr]^{r_2} && R_2 \ar[d]_{h_2} \\G_0 && \ar@{>->}[ll]^{f'_0} D'_0 \ar[rr]_{g'_0}&& G'_1  && \ar@{>->}[ll]^{f'_1} D'_1 \ar[rr]_{g'_1}&& G_2 && \ar@{>->}[ll]^{f_2} D_2 \ar[rr]_{g_2}&& G_3 }\]
	
	By \Cref{def:switch} we also know that $m_0=f'_0\circ a_1$ and $h_1=g'_1\circ a_0$. Now, let us switch $S_{i_0, i_1}(\dder{D}_0)$ and $\dder{D}_2$. In this case we obtain
	\[\xymatrix@C=15pt{L_1\ar[d]^{f_0\circ i_1}&& K_1 \ar[d]_{k'_1}\ar@{>->}[ll]_{l_1} \ar[r]^{r_1} & R_1 \ar@/^.35cm/[drrr]_(.4){c_0}|(.29)\hole \ar[dr]|(.28)\hole_{h'_1} && L_2 \ar@/_.35cm/[dlll]^(.4){c_1} \ar[dl]|(.28)\hole^{f'_1\circ b_1}& K_2 \ar[d]^{k'_2}\ar@{>->}[l]_{l_2} \ar[r]^{r_2} & R_2 \ar[dr]|(.28)\hole_{h'_2} \ar@/^.35cm/[drrr]_(.4){d_0}|(.29)\hole  && L_0 \ar@/_.35cm/[dlll]^(.4){d_1} \ar[dl]|(.28)\hole^{\hat{m}_0}& K_0 \ar[d]^{\hat{k}_0}\ar@{>->}[l]_{l_0} \ar[rr]^{r_0} && R_0 \ar[d]_{g_2\circ b_0} \\G_0 && \ar@{>->}[ll]^{f'_0} D'_0 \ar[rr]_{g'_0}&& G'_1  && \ar@{>->}[ll]^{\hat{f}_1} \hat{D}_1 \ar[rr]_{\hat{g}_1}&& G'_2 && \ar@{>->}[ll]^{f'_2} D'_2 \ar[rr]_{g'_2}&& G_3 }\]
	Using again the definition of switch, we know that $m'_0=\hat{f}_1\circ d_1$ and $h_2=g'_2\circ d_0$. Moreover, by hypothesis $(\X, \R)$ is well-switching and the hypothesis of the second point of \Cref{lem:indep-global-left} are satisfied by $\der{D}$. Thus we can characterise $c_0\colon R_1\to \hat{D}_1$ and $c_1\colon L_2\to D'_0$ as the unique arrows fitting in the diagrams below, where the bottom squares are pullbacks
	\[\xymatrix@R=15pt@C=15pt{&&R_1 \ar@/_1.2cm/[dddl]_(.4){a_0}|(.63)\hole\ar[dl]^{j_0}\ar[dd]^{c'_0} \ar@{>}[dr]^{c_0}&&&&&L_2 \ar[dl]^{j_1} \ar@/_1.2cm/[dddl]_(.4){z_1}|(.63)\hole\ar[dd]^{c'_1} \ar@{>}[dr]^{c_1}\\&D_2 \ar@{>->}[dl]^(.4){f_2}&&\hat{D}_1 \ar[dr]^{\hat{g}_1}  &&&D_1 \ar@{>->}[dl]^(.4){f_1}&&D'_0 \ar[dr]^{g'_0}\\G_2 && P\ar[dr]_{p_4} \ar@{>->}[ur]_{p_3}\ar@{>->}[dl]^{p_1} \ar[ul]^{p_2}&&G'_2&G_1 && Q\ar[dr]_{q_4} \ar@{>->}[ur]_{q_3}\ar@{>->}[dl]^{q_1} \ar[ul]^{q_2}&&G'_1\\& D'_1 \ar[ul]^{g'_1}&&D'_2 \ar@{>->}[ur]_{f'_2}&&& D_0 \ar[ul]^{g_0}&&D'_1 \ar@{>->}[ur]_{f'_1}}\]
	
	Finally, we switch the first two direct derivations
	\[\xymatrix@C=15pt{L_2\ar[d]^{f'_0\circ c_1}&& K_2 \ar[d]_{\hat{k}_2}\ar@{>->}[ll]_{l_2} \ar[r]^{r_2} & R_2 \ar@/^.35cm/[drrr]_(.4){e_0}|(.29)\hole \ar[dr]|(.28)\hole_{\hat{h}_2} && L_1 \ar@/_.35cm/[dlll]^(.4){e_1} \ar[dl]|(.28)\hole^{\hat{m}_1}& K_1 \ar[d]^{\hat{k}_1}\ar@{>->}[l]_{l_1} \ar[r]^{r_1} & R_1 \ar@/^.35cm/[drrr]_(.4){t_0}|(.29)\hole \ar[dr]|(.28)\hole_{\hat{g}_1\circ c_0}   && L_0  \ar[dl]|(.28)\hole^{\hat{m}_0} \ar@/_.35cm/[dlll]^(.4){t_1}& K_0 \ar[d]^{\hat{k}_0}\ar@{>->}[l]_{l_0} \ar[rr]^{r_0} && R_0 \ar[d]_{g_2\circ b_0} \\G_0 && \ar@{>->}[ll]^{\hat{f}_0} \hat{D}_0 \ar[rr]_{\hat{g}_0}&& \hat{G}_1  && \ar@{>->}[ll]^{\tilde{f}_1} \tilde{D}_1 \ar[rr]_{\tilde{g}_1}&& G'_2 && \ar@{>->}[ll]^{f'_2} D'_2 \ar[rr]_{g'_2}&& G_3 }\]
	As before, we have identities $f_0\circ i_1=\hat{f}_0\circ e_1$ and $h'_2=\tilde{g}_1\circ e_0$. Now, the derivation $S_{i_0, i_1}(\dder{D}_1)\cdot S_{i_0, i_1}(\dder{D}_0)\cdot \dder{D}_2$ satisfies the hypothesis of the first point of \Cref{lem:indep-global-left}. Thus, since there is at most one independence pair between two direct derivations, we know that  $t_0\colon R_1\to \tilde{D}_1$ and $t_1\colon L_0\to D'_2$  fit in
	\[\xymatrix{&R_1 \ar[dl]_{a_0} \ar@{>}[dr]^{t_0} \ar@/^.4cm/[dd]|\hole^(.6){c_0}\ar[d]_{t'_0}&&L_ 0\ar[d]_{d_1} \ar@/_.4cm/[ddrr]|(.41)\hole_(.7){a_1}\ar[drr]^{t'_1} \ar@{>}[rr]^{t_1}&&\tilde{D}_1 \ar@{>->}[r]^{\tilde{f}_1}&\hat{G}_1\\ D'_1 \ar@{>->}[d]_{f'_1}&P\ar@{>->}[d]_{p_3} \ar[r]^{p_4} \ar@{>->}[l]_{p_1}&D'_2\ar@{>->}[d]^{f'_2}&\hat{D}_1 \ar@{>->}[d]_{\hat{f}_1}&&S\ar@{>->}[r]_{s_4} \ar[u]_{s_3}\ar@{>->}[d]^{s_1} \ar[ll]_{s_2} &\hat{D}_0 \ar[u]_{\hat{g}_0} \ar@{>->}[d]^{\hat{f}_0}\\G'_1&\hat{D}_1 \ar@{>->}[l]^{\hat{f}_1}\ar[r]_{\hat{g}_1}& G'_2&G'_1 && D'_0 \ar[ll]^{g'_0}  \ar@{>->}[r]_{f'_0}& G_0 }\]
	
	\smallskip \noindent Second sequence of switches.
	
	This time we start switching $\dder{D}_1$ and $\dder{D}_2$.
	\[\xymatrix@C=15pt{L_0\ar[d]^{m_0}&& K_0 \ar[d]_{k_0}\ar@{>->}[ll]_{l_0} \ar[r]^{r_0} & R_0 \ar@/^.35cm/[drrr]_(.4){z_0}|(.29)\hole \ar[dr]|(.28)\hole_{h_0} && L_2 \ar@/_.35cm/[dlll]^(.4){z_1} \ar[dl]|(.28)\hole^{f_1\circ j_1}& K_2 \ar[d]^{\check{k}_2}\ar@{>->}[l]_{l_2} \ar[r]^{r_2} & R_2 \ar[dr]|(.28)\hole_{\check{h}_2} \ar@/^.35cm/[drrr]_(.4){w_0}|(.29)\hole  && L_1 \ar@/_.35cm/[dlll]^(.4){w_1} \ar[dl]|(.28)\hole^{\check{m}_1}& K_1 \ar[d]^{\check{k}_1}\ar@{>->}[l]_{l_1} \ar[rr]^{r_1} && R_1\ar[d]_{g_2\circ j_0} \\G_0 && \ar@{>->}[ll]^{f_0} D_0 \ar[rr]_{g_0}&& G_1  && \ar@{>->}[ll]^{\check{f}_1} \check{D}_1 \ar[rr]_{\check{g}_1}&& \check{G}_2 && \ar@{>->}[ll]^{\check{f}_2} \check{D}_2 \ar[rr]_{\check{g}_2}&& G_3 }\]
	As in the previous case, we already know that $m_1=\check{f}_1\circ w_1$ and $h_2=\check{g}_2\circ w_0$.
	
	Next, we switch $\dder{D}_0$ and $S_{j_0, j_1}(\dder{D}_2)$ getting
	\[\xymatrix@C=15pt{L_2\ar[d]^{f_0\circ z_1}&& K_2 \ar[d]_{\mathring{k}_2}\ar@{>->}[ll]_{l_2} \ar[r]^{r_2} & R_2 \ar@/^.35cm/[drrr]_(.4){y_0}|(.29)\hole \ar[dr]|(.28)\hole_{\mathring{h}_2} && L_0 \ar@/_.35cm/[dlll]^(.4){y_1} \ar[dl]|(.28)\hole^{\check{m}_0}& K_0 \ar[d]^{\check{k}_0}\ar@{>->}[l]_{l_0} \ar[r]^{r_0} & R_0 \ar[dr]|(.28)\hole_{\check{g}_1\circ z_0} \ar@/^.35cm/[drrr]_(.4){x_0}|(.29)\hole  && L_1 \ar@/_.35cm/[dlll]^(.4){x_1} \ar[dl]|(.28)\hole^{\check{m}_1}& K_1 \ar[d]^{\check{k}_1}\ar@{>->}[l]_{l_1} \ar[rr]^{r_1} && R_1\ar[d]_{g_2\circ j_0} \\G_0 && \ar@{>->}[ll]^{\check{f}_0} \check{D}_0 \ar[rr]_{\check{g}_0}&& \check{G}_1  && \ar@{>->}[ll]^{\mathring{f}_1} \mathring{D}_1 \ar[rr]_{\mathring{g}_1}&& \check{G}_2 && \ar@{>->}[ll]^{\check{f}_2} \check{D}_2 \ar[rr]_{\check{g}_2}&& G_3 }\]
	The usual identities $m_0=\check{f}_0\circ y_1$, $\check{h}_2=\mathring{g}_1\circ y_0$ hold. But the really important thing to notice is that the first match of these derivation coincide with $f'_0\circ c_1$
	\[	f_0'\circ c_1  =f'_0\circ q_3\circ c'_1 =f_0\circ q_1\circ c'_1=f_0\circ z_1\]
	
	Moreover, we can characterise the pair $(x_0, x_1)$. If we apply the first point of \Cref{lem:indep-global-left} to $\dder{D}$ we get an independence pair between the last two steps of the derivation above, which, since $(\X, \R)$ is well-switching, must coincide with $(x_0, x_1)$. Thus we have two diagrams
	
	\[\xymatrix{&R_0 \ar[dl]_{i_0} \ar@{>}[dr]^{x_0} \ar@/^.4cm/[dd]|\hole^(.6){z_0}\ar[d]_{x'_0}&&L_ 1\ar[d]_{w_1} \ar@/_.4cm/[ddrr]|(.41)\hole_(.7){i_1}\ar[drr]^{x'_1} \ar@{>}[rr]^{x_1}&&\mathring{D}_1 \ar@{>->}[r]^{\mathring{f}_1}&\check{G}_1\\ D_1 \ar@{>->}[d]_{f_1}&P'\ar@{>->}[d]_{p'_3} \ar[r]^{p'_4} \ar@{>->}[l]_{p'_1}&\check{D}_2\ar@{>->}[d]^{\check{f}_2}&\check{D}_1 \ar@{>->}[d]_{\check{f}_1}&&S'\ar@{>->}[r]_{s'_4} \ar[u]_{s'_3}\ar@{>->}[d]^{s'_1} \ar[ll]_{s'_2} &\check{D}_0 \ar[u]_{\check{g}_0} \ar@{>->}[d]^{\check{f}_0}\\G_1&\check{D}_1 \ar@{>->}[l]^{\check{f}_1}\ar[r]_{\check{g}_1}& \check{G}_2&G_1 && D_0 \ar[ll]^{g_0}  \ar@{>->}[r]_{f_0}& G_0 }\]
	From this diagram we can recover another identity. Notice that, if $p'_2\colon P'\to D_2$ is the last projection from $P'$, then
	\[f_2\circ p'_2\circ x'_0=g_1\circ p'_1\circ x'_0=g_1\circ i_0=f_2\circ b_0\]
	Implying, since $f_2$ is mono, that $p'_2\circ x'_0=b_0$. From this, we can also deduce that 
	\[\check{g}_2\circ x_0=\check{g}_2\circ p'_4\circ x'_0=g_2\circ p'_2\circ x'_0=g_2\circ b_0 \]
	
	
	Finally, switching the second and the third step we get
	\[\xymatrix@C=15pt{L_2\ar[d]^{f_0\circ z_1}&& K_2 \ar[d]_{\mathring{k}_2}\ar@{>->}[ll]_{l_2} \ar[r]^{r_2} & R_2 \ar@/^.35cm/[drrr]_(.4){v_0}|(.29)\hole \ar[dr]|(.28)\hole_{\mathring{h}_2} && L_1 \ar@/_.35cm/[dlll]^(.4){v_1} \ar[dl]|(.28)\hole^{\mathring{f}_1 \circ x_1}& K_1 \ar[d]^{\mathring{k}_1}\ar@{>->}[l]_{l_1} \ar[r]^{r_1} & R_1 \ar@/^.35cm/[drrr]_(.4){u_0}|(.29)\hole\ar[dr]_{\check{h}_1}|(.28)\hole   && L_0 \ar@/_.35cm/[dlll]^(.4){u_1} \ar[dl]^{\mathring{m}_0}|(.28)\hole& K_0 \ar[d]^{\mathring{k}_0}\ar@{>->}[l]_{l_0} \ar[rr]^{r_0} && R_0\ar[d]_{\check{g}_2\circ x_0} \\G_0 && \ar[ll]^{\check{f}_0} \check{D}_0 \ar[rr]_{\check{g}_0}&& \check{G}_1  && \ar@{>->}[ll]^{\bar{f}_1} \bar{D}_1 \ar[rr]_{\bar{g}_1}&& \bar{G}_2 && \ar@{>->}[ll]^{\bar{f}_2} \bar{D}_2 \ar[rr]_{\bar{g}_2}&& G_3 }\]
	As usual, this last switch brings identities $\check{m}_0=\bar{f}_1\circ u_1$ and $g_2\circ j_0=\bar{g}_2\circ u_0$. Applying the second point of \Cref{lem:indep-global-left} to $\dder{D}_0\cdot S_{j_0, j_1}(\dder{D}_2)\cdot S_{j_0, j_1}(\dder{D}_1)$ and the fact that $(\X, \R)$ is well-switching we get the following diagrams
	\[\xymatrix@R=15pt@C=15pt{&&R_2 \ar@/_1.2cm/[dddl]_(.4){y_0}|(.63)\hole\ar[dl]^{w_0}\ar[dd]^{v'_0} \ar@{>}[dr]^{v_0}&&&&&L_1 \ar[dl]^{w_1} \ar@/_1.2cm/[dddl]_(.4){i_1}|(.63)\hole\ar[dd]^{v'_1} \ar@{>}[dr]^{v_1}\\&\check{D}_2 \ar@{>->}[dl]^(.4){\check{f}_2}&&\bar{D}_1 \ar[dr]^{\bar{g}_1}  &&&\check{D}_1 \ar@{>->}[dl]^(.4){\check{f}_1}&&\check{D}_0 \ar[dr]^{\check{g}_0}\\\check{G}_2 && \bar{P}\ar[dr]_{\bar{p}_4} \ar@{>->}[ur]_{\bar{p}_3}\ar@{>->}[dl]^{\bar{p}_1} \ar[ul]^{\bar{p}_2}&&\bar{G}_2&G_1 && \bar{Q}\ar[dr]_{\bar{q}_4} \ar@{>->}[ur]_{\bar{q}_3}\ar@{>->}[dl]^{\bar{q}_1} \ar[ul]^{\bar{q}_2}&&\check{G}_1\\& \mathring{D}_1 \ar[ul]^{\mathring{g}_1}&&\bar{D}_2 \ar@{>->}[ur]_{\bar{f}_2}&&& D_0 \ar[ul]^{g_0}&&\mathring{D}_1 \ar@{>->}[ur]_{\mathring{f}_1}}\]
	
	We have to build the dotted arrows in the diagram below
	\[\xymatrix@C=15pt@R=30pt{G_0 && \ar@{>->}[ll]_{\hat{f}_0} \hat{D}_0 \ar[rr]^{\hat{g}_0}&& \hat{G}_1  && \ar@{>->}[ll]_{\tilde{f}_1} \tilde{D}_1 \ar[rr]^{\tilde{g}_1}&& G'_2 && \ar@{>->}[ll]_{f'_2} D'_2 \ar[rr]^{g'_2}&& G_3 \\
		L_2\ar[d]^{f_0\circ z_1} \ar[u]_{f'_0\circ c_1}&& K_2 \ar[u]^{\hat{k}_2}\ar[d]_{\mathring{k}_2}\ar@{>->}[ll]_{l_2} \ar[r]^{r_2} & R_2 \ar@/_.35cm/[urrr]^(.4){e_0}|(.29)\hole \ar[ur]|(.28)\hole^{\hat{h}_2} \ar@/^.35cm/[drrr]_(.4){v_0}|(.29)\hole \ar@/_.2cm/[dr]|(.28)\hole_{\mathring{h}_2} && L_1 \ar@/^.35cm/[ulll]_(.4){e_1} \ar[ul]|(.28)\hole_{\hat{m}_1} \ar@/_.35cm/[dlll]^(.4){v_1} \ar[dl]|(.28)\hole^{\mathring{f}_1 \circ x_1}& K_1  \ar[u]_{\hat{k}_1}\ar[d]^{\mathring{k}_1}\ar@{>->}[l]_{l_1} \ar[r]^{r_1} & R_1 \ar@/^.35cm/[drrr]_(.4){u_0}|(.29)\hole\ar@/_.2cm/[dr]_{\check{h}_1}|(.28)\hole \ar@/_.35cm/[urrr]^(.4){t_0}|(.29)\hole \ar[ur]|(.28)\hole^{\hat{g}_1\circ c_0}  && L_0\ar[ul]|(.28)\hole_{\hat{m}_0} \ar@/^.35cm/[ulll]_(.4){t_1} \ar@/_.35cm/[dlll]^(.4){u_1} \ar[dl]^{\mathring{m}_0}& K_0 \ar[d]_{\mathring{k}_0}\ar[u]^{\hat{k}_0}\ar@{>->}[l]_{l_0} \ar[rr]^{r_0} && R_0 \ar[u]^{g_2\circ b_0}\ar[d]_{\check{g}_2\circ x_0} \\G_0 \ar@/^.3cm/[uu]^{\id{G_0}}&& \ar@{>->}[ll]^{\check{f}_0} \check{D}_0 \ar@{.>}@/_.3cm/[uu]_(.3){\phi_{0}}|\hole  \ar[rr]_{\check{g}_0}&& \check{G}_1  \ar@{.>}@/_.0cm/[uu]^(.2){\psi_{1}}|(.43)\hole |(.57)\hole  && \ar@{>->}[ll]^{\bar{f}_1} \bar{D}_1 \ar@{.>}@/^.3cm/[uu]^(.3){\phi_{1}}|\hole \ar[rr]_{\bar{g}_1}&& \bar{G}_2 \ar@{.>}@/_.0cm/[uu]^(.2){\psi_{2}}|(.43)\hole |(.57)\hole && \ar@{>->}[ll]^{\bar{f}_2} \ar@{.>}@/_.4cm/[uu]_(.3){\phi_{2}}|\hole \bar{D}_2 \ar[rr]_{\bar{g}_2}&& G_3 \ar@{.>}@/_.3cm/[uu]_{\psi_{3}} }\]
	
	Now, we have already proved that $f_0\circ z_1=f'_0\circ c_1$, thus by \Cref{prop:unique} we have isomorphisms $\phi_{0}\colon \check{D}_0\to \hat{D}_0$ and $\psi_{1}\colon \check{G}_1\to \hat{G}_1$.
	Notice that
	\[
	\hat{f}_0\circ \phi_{0}\circ v_1  = \check{f}_0\circ v_1 =\check{f}_0\circ \check{q}_3\circ v'_1=f_0\circ \check{q}_1\circ v'_1=f_0\circ i_1=\hat{f}_0\circ e_1\]
	Thus, using the fact that $\hat{f}_0$ is mono, we can conclude that $\phi_0\circ v_1=e_1$. Thus we have
	\[\psi_1\circ \mathring{f}\circ x_1=\psi_1\circ \check{g}_0\circ v_1= \hat{g}_0\circ \phi_0\circ v_1=\hat{g}_0\circ e_1=\hat{m}_1 \]
	\Cref{prop:unique} now yields the $\phi_1\colon \bar{D}_1\to\tilde{D}_1$ and $\psi_2\colon \bar{G}_2\to G'_2$.  We can compute to get
	\[
	\tilde{f}_1\circ \phi_1\circ v_0=\psi_1\circ \bar{f}_1\circ v_0=\psi_1\circ \mathring{h}_2=\hat{h}_2=\tilde{f}_1\circ e_0\]
	Hence $\phi_1\circ v_0=e_0$. On the other hand, notice that
	\[\hat{f}_0\circ s_4\circ t'_1=f'_0\circ s_1\circ t'_1=f'_0\circ a_1=m_0=\check{f}\circ y_1=\hat{f}_0\circ \phi_0\circ y_1\]
	Yielding $s_4\circ t'_1=\phi_0\circ y_1$. But then
	\begin{align*}
		&\tilde{f}_1\circ \phi_1\circ u_1=\psi_1\circ \bar{f}_1\circ u_1=\psi_1\circ \check{m}_0= \psi_1\circ \check{g}_0\circ y_1\\=&\hat{g}_0\circ \phi_0\circ y_1=\hat{g}_0\circ s_4\circ t'_1=\tilde{f}_1\circ s_3\circ t'_1=\tilde{f}_1\circ t_1 
	\end{align*}
	
	In turn, the previous identity entails that
	\[\psi_2\circ \mathring{m}_0=\psi_2\circ \bar{g}_1\circ u_1 = \tilde{g}_1\circ \phi_1\circ u_1 = \tilde{g}_1\circ t_1=\hat{m}_0\]
	
	We can use \Cref{prop:unique} a third time to get $\phi_2\colon \bar{D}_2\to D'_2$ and $\psi_3\colon G_3\to G_3$.  Now, as a last computation, we have
	\[f'_2\circ t_0= \hat{g}_2\circ c_0=\psi_2\circ h_1=\psi_2\circ \bar{f}_2\circ u_0=f'_2\circ \phi_2\circ u_0\]
	Therefore $\phi_2\circ u_0=t_0$ and this concludes the proof.
	
Now, the sequence of switches between $\lpro \der{D}''\rpro$ and  $\lpro \der{E}'' \rpro$  witnesses that $\id{2}$ is a consistent permutation and so there is a mediating isomorphism $\xi_{\id{2}}$. On the other hand, by 
	
\end{proof}



\end{document}

%%% Local Variables:
%%% mode: latex
%%% TeX-master: t
%%% End:
