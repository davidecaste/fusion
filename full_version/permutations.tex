\chapter{Left-linear DPO-rewriting systems}

$\mathcal{M}$-adhesive categories are the right context in which to perform abstract rewriting using the so-called ``douple pushout approach'' (DPO). We will recall the basic definitions and properties of this approach to abstract rewriting. 
\section{First definitions and properties}
We are now going to study rewriting systems in $\mathcal{M}$-adhesive categories.

\begin{definition}[\cite{habel2012mathcal,heindel2009category}]
	Let $\X$ be an $\mathcal{M}$-adhesive category, a  \emph{left $\mathcal{M}$-linear} rule $\rho$ is a pair $(l,r)$ of arrows with the same domain, such that $l$ belongs to $\mathcal{M}$.  The rule $\rho$ is said to be \emph{$\mathcal{M}$-linear} if $r\in \mathcal{M}$ too. A rule $\rho$ is said to be \emph{consuming} if $l$ is not an isomorphism. We will represent a rule $\rho$ as a span 
	\[\xymatrix{L & K\ar@{>->}[l]_{l} \ar[r]^{r} & R}\]
	$L$ is the \emph{left-hand side}, $R$ is the \emph{right-hand side} and $K$ the \emph{glueing object}. 
	
	
	A \emph{left-linear DPO-rewriting system} is a pair $(\X, \R)$ where $\X$ is a $\mathcal{M}$-adhesive category and $R$ is a set of left $\mathcal{M}$-linear rules. $(\X, \R)$ is called \emph{linear} if every rule in $R$ is $\mathcal{M}$-linear. Similarly, a \emph{consuming} left-linear DPO-rewriting system is one in which every rule is consuming.
	
	Given  two objects $G$ and $H$ and a rule $\rho=(l,r)$ in $\R$, a \emph{direct derivation $\mathscr{D}$ from $G$ to $H$ applying the rule $\rho$} is a diagram as the one below, in which both squares are pushouts
	(adhesivity ensures that $f$ belongs to $\mathcal{M}$)
	  
	\[\xymatrix{L \ar[d]_{n}& K \ar[d]^{k}\ar@{>->}[l]_{l} \ar[r]^{r} & R \ar[d]^{h}\\G & \ar@{>->}[l]^{f} D \ar[r]_{g}& H}\]
	The arrow $n$ is called the \emph{match} of the derivation, while $h$ is its \emph{back-match}.
	We will denote a direct derivation $\dder{D}$ between $G$ and $H$ as $\dder{D}\colon G\Mapsto H$. 
\end{definition}


\begin{remark}\label{exa:conc} Let  $\dder{D}\colon G\Mapsto H$ be the direct derivation 
	\[\xymatrix{L \ar[d]_{n}& K \ar[d]^{k}\ar@{>->}[l]_{l} \ar[r]^{r} & R \ar[d]^{h}\\G & \ar@{>->}[l]^{f} D \ar[r]_{g}& H}\]
	If $\phi\colon G'\to G$ and $\psi\colon H\to H'$ are two isomorphisms, 	we can consider the direct derivation	$\phi * \dder{D}*\psi \colon G'\Mapsto H'$ given by the following diagram.
	\[\xymatrix{L \ar[d]_{\phi^{-1} \circ n}& K \ar[d]^{k}\ar@{>->}[l]_{l} \ar[r]^{r} & R \ar[d]^{\psi \circ h}\\G' & \ar@{>->}[l]^{\phi^{-1} \circ f} D \ar[r]_{\psi \circ g}& H'}\]
	
	In particular, we will use $\phi*\dder{D}$ and $\dder{D}*\psi$  to denote $\phi*\dder{D}*\id{H}$ and $\id{G}*\dder{D}*\psi$.
\end{remark}



$\mathcal{M}$-adhesivity of $\X$ guarantes the essential uniqueness of the result obtained rewriting an object, as shown by the next proposition.

\begin{proposition}\label{prop:unique} Let $\X$  be a $\mathcal{M}$-adhesive category. Suppose that the two direct derivations $\mathscr{D}$ and $\mathscr{D'}$ below, with the same match and applying the same left $\mathcal{M}$-linear rule $\rho$ are given.
	\[\xymatrix{L \ar[d]_{m}& K \ar[d]^{k}\ar@{>->}[l]_{l} \ar[r]^{r} & R \ar[d]^{h} & L \ar[d]_{m}& K \ar[d]^{k'}\ar@{>->}[l]_{l} \ar[r]^{r} & R \ar[d]^{h'}\\G & \ar@{>->}[l]^{f} D \ar[r]_{g}& H & G & \ar@{>->}[l]^{f'} D' \ar[r]_{g'}& H'}\]
	Then there exist isomorphisms $t\colon D\to D'$ and $s\colon H\to H'$ as in the following diagram.
	\[\xymatrix@C=40pt{&&D' \ar[r]^{g'} \ar@{>->}@/_.45cm/[dll]_{f'}&H'\\G & L \ar[l]_{m} & K \ar[u]_{k'} \ar[d]^{k}\ar[r]^{r} \ar@{>->}[l]_{l} &R\ar[u]^{h'} \ar[d]_{h}\\&&D\ar@{>->}@/^.45cm/[ull]^{f}\ar@/^.4cm/@{.>}[uu]^(.4){t}|\hole \ar[r]_{g}&H\ar@/_.4cm/@{.>}[uu]_{s}}\]
\end{proposition}
\begin{proof}
	By construction, the pairs $(k, f)$ and $(k', f')$ are pushout complements of $l$ and $n$. Thus, the existence of the isomorphism $t\colon D\to D'$ follows from the second point of \Cref{lem:radj}. Now, computing we have
	\begin{align*}
		g'\circ t \circ k &= g' \circ k'\\&=h'\circ r
	\end{align*}
	Hence, we have the wanted $s\colon H\to H'$. To see that $s$ is an isomorphism, consider the diagram 
	\[\xymatrix{K  \ar@/^.4cm/[rr]^{k'}\ar[d]_{r} \ar[r]_{k}& \ar[r]_{t} D \ar[d]^{g}& D' \ar[d]^{g'}\\ R \ar@/_.4cm/[rr]_{h'} \ar[r]^{h}& H \ar[r]^{s}& H'}\]
	By hypothesis the whole rectangle and its left half are pushouts, therefore, by \Cref{lem:po1} its right square is a pushout too. The claim now follows from the fact that the pushout of an isomorphism is an isomorphism.
\end{proof}

If we look to direct derivations as transitions, it is natural to consider them as edges in a direct graph. Taking objects as vertices objects led us to the following definition \cite{heindel2009category}.

\begin{definition}
	Let $(\X, \R)$ be a left-linear DPO-rewriting system, with $\X$ $\mathcal{M}$-adhesive. The \emph{DPO-derivation graph} of $(\X, \R)$ is the (large)  directed graph $\gpo$ having as vertices the objects of $\X$ and in which an edge between $G$ and $H$ is a direct derivation $\dder{D}\colon G\Mapsto H$.	A \emph{derivation} $\der{D}$ between two objects $G$ and $H$ is a path between them in $\gpo$. The \emph{source} and \emph{target} of $\der{D}$ are, respectively, $G$ and $H$.
\end{definition}

\begin{remark}
	We can spell out more explicitly what  a derivation $\dder{D}$ is.  An \emph{empty derivation} starting and ending in $G$ is just $G$ itself.  A \emph{non-empty derivation} $\dder{D}$ is a sequence $\{\dder{D}_i\}_{i=0}^n$ of direct derivations such that:
	\begin{enumerate}
		\item for every index $i$, $\dder{D}_i$ is a direct derivation $G_i \Mapsto G_{i+1}$;
		\item $G_0=G$ and $G_{n+1}=H$.
	\end{enumerate}
	
	We will call the number $n+1$ the \emph{length} of the derivation, denoted by $\lgh(\der{D})$. We will also say that an empty derivation has length $0$. 
	
	Moreover,  if every $\dder{D}_i$ applies the rule $\rho_i\in R$, then we can define an associated sequence of rules $r(\der{D})$ as $\{\rho_i\}_{i=0}^n$.
\end{remark}

\begin{remark}\label{rem:func}
	Consider a derivation $\der{D}$ in a left-linear DPO-rewriting system $(\X, \R)$. We can take the subcategory $\Delta(\der{D})$ of $\X$ given by the arrows appearing in $\der{D}$. This subcategory comes equipped with an inclusion functor $I(\der{D})\colon \Delta(\der{D})\to \X$. Moreover, we can further define $\Deltamin(\der{D})$ as the subcategory of $\Delta(\der{D})$ containing only the bottom row of the derivation.
\end{remark}

\begin{notation}Let $\der{D}=\{\dder{D}_i\}_{i=0}^n$ be a derivation. We will depict the $i^\text{th}$ element $\dder{D}_i$ of $\der{D}$ as in the following diagram.  
	\[\xymatrix{L_i \ar[d]_{m_i}& K_i \ar[d]^{k_i}\ar@{>->}[l]_{l_i} \ar[r]^{r_i} & R_i \ar[d]^{h_i} \\G_{i} & \ar@{>->}[l]^{f_{i}} D_{i} \ar[r]_{g_{i}}& G_{i+1} }\]
	Notice that, in particular, if $\der{D}\colon G\to H$, then $G_0=G$ and $G_{n+1}=H$. When $\der{D}$ has length $1$ we will suppress the indexes. In such case, we will also identify $\der{D}$ with its only element. 
\end{notation} 

\begin{example}\label{ex:1}
	Let us consider the category of $\cat{Graph}$ of graphs and graph morphisms, which is adhesive (see, for instance \cite{lack2005adhesive,lack2006toposes}).   We can endow $\cat{Graph}$ with a set of left-linear rules, for instance taking the following ones.  Here numbers are used to represent the
	morphisms from the interface to the left- and right-hand
	sides. Rules $\rho_0$ and $\rho_1$ are linear: $\rho_0$ generates a
	new node and a new edge, while $\rho_1$ creates a self-loop edge. Instead,
	$\rho_2$ is left-linear but not linear, as it ``merges'' two nodes.
	\begin{center}
		\begin{tikzpicture}[node distance=2mm, font=\small]
			%, baseline=(current bounding box.center), font=\small]
			\node (l) {
				\begin{tikzpicture}
					%
					\node at (0,0.53) {};
					\node at (0,0) [node, label=below:$1$] (1) {} ;        
					%
					\pgfBox
				\end{tikzpicture} 
			};
			\node[font=\scriptsize, above] at (l.north) {$L_0$};
			\node [right=of l] (k) {
				\begin{tikzpicture}
					%
					\node at (0,0.53) {}; 
					\node at (0,0) [node, label=below:$1$] (1) {};
					%
					\pgfBox
				\end{tikzpicture} 
			};
			\node[font=\scriptsize, above] at (k.north) {$K_0$};
			\node[below] at (k.south) {$\rho_0$};
			\node  [right=of k] (r) {
				\begin{tikzpicture}
					\node at (0,0.53) {}; 
					\node at (0,0) [node, label=below:$1$] (1) {};
					\node at (.5,0) [node, label=below:$2$] (2) {};
					\draw[->] (1) to[out=20, in=160] (2);
					%
					\pgfBox
				\end{tikzpicture}
			};
			\node[font=\scriptsize, above] at (r.north) {$R_0$};
			\path (k) edge[->] node[trans, above] {} (l);
			\path (k) edge[->] node[trans, above] {} (r);    
		\end{tikzpicture}
		%  }
	%
	\hspace{1cm}
	%  \hfill
	%
	%  \subcaptionbox*{}{
		% RULE P1
		%
		\begin{tikzpicture}[node distance=2mm, font=\small]
			\node (l) {
				\begin{tikzpicture}
					%
					\node at (0,0.53) {}; 
					\node at (0,0) [node, label=below:$1$] (1) {} ;
					% 
					\pgfBox
				\end{tikzpicture} 
			};
			\node[font=\scriptsize, above] at (l.north) {$L_1$};
			\node [right=of l] (k) {
				\begin{tikzpicture}
					%
					\node at (0,0.53) {}; 
					\node at (0,0) [node, label=below:$1$] (1) {};
					% 
					\pgfBox
				\end{tikzpicture} 
			};
			\node[font=\scriptsize, above] at (k.north) {$K_1$};
			\node[below] at (k.south) {$\rho_1$};
			\node  [right=of k] (r) {
				\begin{tikzpicture}
					\node at (0,0) [node, label=below:$1$] (1) {}
					edge [in=55, out=85, loop] ();
					%
					\pgfBox
				\end{tikzpicture}
			};
			\node[font=\scriptsize, above] at (r.north) {$R_1$};
			\path (k) edge[->] node[trans, above] {} (l);
			\path (k) edge[->] node[trans, above] {} (r);
		\end{tikzpicture}
		%  }
	% 
	\hspace{1cm}
	%\hfill
	%
	%    \subcaptionbox*{}{
		% RULE P2
		%
		\begin{tikzpicture}[node distance=2mm, font=\small]
			\node (l) {
				\begin{tikzpicture}
					%
					\node at (0,0.53) {}; 
					\node at (0,0) [node, label=below:$1$] (1) {} ;
					\node at (0.5,0) [node, label=below:$2$] (2) {} ;
					%
					\pgfBox
				\end{tikzpicture} 
			};
			\node[font=\scriptsize, above] at (l.north) {$L_2$};
			\node [right=of l] (k) {
				\begin{tikzpicture}
					%
					\node at (0,0.53) {}; 
					\node at (0,0) [node, label=below:$1$] (1) {} ;
					\node at (0.5,0) [node, label=below:$2$] (2) {} ;
					%
					\pgfBox
				\end{tikzpicture} 
			};
			\node[font=\scriptsize, above] at (k.north) {$K_2$};
			\node[below] at (k.south) {$\rho_2$};
			\node  [right=of k] (r) {
				\begin{tikzpicture}
					\node at (0,0.53) {}; 
					\node at (0,0) [node, label=below:$12$] (12) {};
					%
					\pgfBox
				\end{tikzpicture}
			};
			\node[font=\scriptsize, above] at (r.north) {$R_2$};
			\path (k) edge[->] node[trans, above] {} (l);
			\path (k) edge[->] node[trans, above] {} (r);
		\end{tikzpicture}
	\end{center}
	
	We can also give an example of derivation as follows, where the colors are used to avoid ambiguity in the edges. Note that the numbers in $\rho_1$ were changed consistently to represent the vertical morphisms.
	
	\begin{center}
		\begin{tikzpicture}[node distance=2mm, font=\small, baseline=(current bounding box.center)]      
			\node (L0) at (0,2) {
				\begin{tikzpicture}
					% 
					\node at (0,0.53) {};
					\node at (0,0) [node, label=below:$1$] (1) {} ;
					% 
					\pgfBox
				\end{tikzpicture} 
			};
			\node [right=of L0] (K0) {
				\begin{tikzpicture}
					% 
					\node at (0,0.53) {}; 
					\node at (0,0) [node, label=below:$1$] (1) {};
					% 
					\pgfBox
				\end{tikzpicture} 
			};
			\node [above] at (K0.north) {$\rho_0$};
			%     \node [above=of K1] {$\rho_0$};
			\node [right=of K0](R0) {
				\begin{tikzpicture}
					\node at (0,0.53) {}; 
					\node at (0,0) [node, label=below:$1$] (1) {};
					\node at (.5,0) [node, label=below:$2$] (2) {};
					\draw[coloredge] (1) to[out=20, in=160] (2);
					% 
					\pgfBox
				\end{tikzpicture}
			};
			\path (K0) edge[->] node[trans, above] {} (L0);
			\path (K0) edge[->] node[trans, above] {} (R0);
			
			\node at (4,2) (L1) {
				\begin{tikzpicture}
					% 
					\node at (0,0.53) {}; 
					\node at (0,0) [node, label=below:$2$] (2) {} ;
					% 
					\pgfBox
				\end{tikzpicture} 
			};
			\node [right=of L1] (K1) {
				\begin{tikzpicture}
					% 
					\node at (0,0.53) {}; 
					\node at (0,0) [node, label=below:$2$] (2) {};
					% 
					\pgfBox
				\end{tikzpicture} 
			};
			%     \node [above=of K1] {$\rho_1$};
			\node [above] at (K1.north) {$\rho_1$};
			\node [right=of K1] (R1) {
				\begin{tikzpicture}
					\node at (0,0) [node, label=below:$2$] (2) {}
					edge [in=55, out=85, loop] ();        
					% 
					\pgfBox
				\end{tikzpicture}
			};
			\path (K1) edge[->] node[trans, above] {} (L1);
			\path (K1) edge[->] node[trans, above] {} (R1);
			
			\node at (8,2) (L2) {
				\begin{tikzpicture}
					% 
					\node at (0,0.53) {}; 
					\node at (0,0) [node, label=below:$1$] (1) {} ;
					\node at (0.5,0) [node, label=below:$2$] (2) {} ;
					% 
					\pgfBox
				\end{tikzpicture} 
			};
			\node [right=of L2] (K2) {
				\begin{tikzpicture}
					% 
					\node at (0,0.53) {}; 
					\node at (0,0) [node, label=below:$1$] (1) {} ;
					\node at (0.5,0) [node, label=below:$2$] (2) {} ;
					% 
					\pgfBox
				\end{tikzpicture} 
			};
			%      \node [above=of K2] {$\rho_2$};
			\node [above] at (K2.north) {$\rho_2$};
			\node [right=of K2] (R2) {
				\begin{tikzpicture}
					\node at (0,0.53) {}; 
					\node at (0,0) [node, label=below:$12$] (12) {};
					% 
					\pgfBox
				\end{tikzpicture}
			};
			\path (K2) edge[->] node[trans, above] {} (L2);
			\path (K2) edge[->] node[trans, above] {} (R2);
			
			%%%%%% second row
			\node at (0,0) (G0) {
				\begin{tikzpicture}
					% 
					\node at (0,0.53) {};
					\node at (0,0) [node, label=below:$1$] (1) {} ;
					% 
					\pgfBox
				\end{tikzpicture} 
			};
			\node [right=of G0] (D0) {
				\begin{tikzpicture}
					% 
					\node at (0,0.53) {}; 
					\node at (0,0) [node, label=below:$1$] (1) {};
					% 
					\pgfBox
				\end{tikzpicture} 
			};
			\node at (3,0) (G1) {
				\begin{tikzpicture}
					% 
					\node at (0,0.53) {}; 
					\node at (0,0) [node, label=below:$1$] (1) {} ;
					\node at (0.5,0) [node, label=below:$2$] (2) {} ;
					\draw[coloredge] (1) to[out=20, in=160] (2);
					% 
					\pgfBox
				\end{tikzpicture} 
			};
			\path (D0) edge[->] node[trans, above] {} (G0);
			\path (D0) edge[->] node[trans, above] {} (G1);
			\path (L0) edge[->] node[trans, above] {} (G0);
			\path (K0) edge[->] node[trans, above] {} (D0);
			\path (R0) edge[->] node[trans, above] {} (G1);
			
			\node at (5,0) (D1) {
				\begin{tikzpicture}
					% 
					\node at (0,0.53) {};
					\node at (0,0) [node, label=below:$1$] (1) {} ;
					\node at (0.5,0) [node, label=below:$2$] (2) {} ;
					\draw[coloredge] (1) to[out=20, in=160] (2);
					% 
					\pgfBox
				\end{tikzpicture} 
			};
			\node at (7,0) (G2) {
				\begin{tikzpicture}
					% 
					\node at (0,0.53) {}; 
					\node at (0,0) [node, label=below:$1$] (1) {};
					\node at (0.5,0) [node, label=below:$2$] (2) {}
					edge [in=55, out=85, loop] (); 
					\draw[coloredge] (1) to[out=20, in=160] (2);
					% 
					\pgfBox
				\end{tikzpicture} 
			};
			
			\path (D1) edge[->] node[trans, above] {} (G1);
			\path (D1) edge[->] node[trans, above] {} (G2);
			\path (L1) edge[->] node[trans, above] {} (G1);
			\path (K1) edge[->] node[trans, above] {} (D1);
			\path (R1) edge[->] node[trans, above] {} (G2);
			
			\node at (9.46,0) (D2) {
				\begin{tikzpicture}
					% 
					\node at (0,0) [node, label=below:$1$] (1) {};
					\node at (0.5,0) [node, label=below:$2$] (2) {}
					edge [in=55, out=85, loop] ();
					\draw[coloredge] (1) to[out=20, in=160] (2);
					% 
					\pgfBox
				\end{tikzpicture} 
			};
			\node [right=of D2] (G3) {
				\begin{tikzpicture}
					% 
					\node at (0,0) [node, label=below:$12$] (12) {}
					edge [in=55, out=85, loop] ()
					edge [in=125, out=155, colorloop] ();
					% 
					\pgfBox
				\end{tikzpicture} 
			};
			\node[font=\scriptsize, below] at (G0.south) {$G_0$};
			\node[font=\scriptsize, below] at (D0.south) {$D_0$};      
			\node[font=\scriptsize, below] at (G1.south) {$G_1$};
			\node[font=\scriptsize, below] at (D1.south) {$D_1$};      
			\node[font=\scriptsize, below] at (G2.south) {$G_2$};
			\node[font=\scriptsize, below] at (D2.south) {$D_2$};      
			\node[font=\scriptsize, below] at (G3.south) {$G_3$};      
			\path (D2) edge[->] node[trans, above] {} (G2);
			\path (D2) edge[->] node[trans, above] {} (G3);
			\path (L2) edge[->] node[trans, above] {} (G2);
			\path (K2) edge[->] node[trans, above] {} (D2);
			\path (R2) edge[->] node[trans, above] {} (G3);
		\end{tikzpicture}
		%
	\end{center}
	
\end{example}

\begin{definition}
	The \emph{DPO-derivation category} $\dpo$ of a left-linear DPO-rewriting system $(\X, \R)$ is the category in which arrows between $G$ and $H$ are given by, possibly empty, derivations. Composition is concatenation of paths in $\gpo$ and identities are given by empty derivations.
\end{definition} 	
\begin{remark}
	More explicitly, given $\der{D}=\{\dder{D}\}_{i=0}^n$ between $G$ and $H$ and $\der{D}'=\{\dder{D}'_i\}_{i=0}^m$, their concatenation $\der{D}\cdot\der{D}'$ is the derivation $\{\dder{E}_i\}_{i=0}^{m+n+1}$ in which
	\[\dder{E}_i:=\begin{cases}
		\dder{D}_i & i \leq n\\
		\dder{D}'_{i-(n+1)} & n< i 
	\end{cases}\]	
	
	Notice, moreover that, $\der{D}\cdot \der{D'}$ is equal to $\der{D}'$ if $\der{D}$ is empty, while it coincides with $\der{D}$ if $\der{D}'$ has length zero.
\end{remark}

\Cref{exa:conc} allows us to compose derivations with isomorphisms.

\begin{definition} Let $(\X, \R)$ be a a left-linear DPO-rewriting system. Given a derivation $\der{D}=\{\dder{D}_{i}\}_{i=0}^n$ between $G$ and $H$ and isomorphisms $\phi\colon G'\to G$, $\psi\colon H\to H'$, the derivations  $\phi*\der{D}$ and $\der{D}*\psi$ are defined as
	\[\phi *\der{D} := \begin{cases}
		G' & \lgh(\der{D})=0\\ 
		\{\phi* \dder{D}_0\}\cdot \{\dder{D}_i\}_{i=1}^{n}  & \lgh(\der{D})\neq 0
	\end{cases} \qquad \der{D}*\psi := \begin{cases}
		H' & \lgh(\der{D})=0\\ 
		\{\dder{D}_i\}_{i=0}^{n-1} \cdot \{\dder{D}_n*\psi\} & \lgh(\der{D})\neq 0
	\end{cases}\] 
	
	Moreover, if $\lgh(\der{D})>0$,  we define the derivation $\phi *\der{D} * \psi$ as
	\[\phi *\der{D} * \psi = \{\phi* \dder{D}_0\}\cdot \{\dder{D}_i\}_{i=1}^{n-1} \cdot \{\dder{D}_n*\psi\}\] 
\end{definition}

\begin{remark}
	When $\der{D}$ consists only in the direct derivation $\dder{D}$, then $\phi*\der{D}*\psi$ is the derivation of length one whose unique element is $\phi*\dder{D}*\psi$.
\end{remark}

\subsection{Decorations and abstraction equivalence}
We are often interested in an object of $\X$ only up to isomorphism. It is therefore useful to consider a version of $\gpo$ in which vertices are classes of isomorphism of object of $\X$. In order to do so, some preliminary work is needed.

\begin{definition}\cite{mac2013categories}
	Let $\X$ be a category, we say that  $\X$ is \emph{skeletal} if, for every two objects $X$ and $Y$, the existence of an isomorphism $\phi\colon X\to Y$ entails $X=Y$. A \emph{skeleton} for a category $\X$ is a full subcategory $\ske$ which is skeletal and such that the inclusion functor $\ske\to \X$ is an equivalence. 
\end{definition}

\begin{remark}
	By definition the inclusion $\ske \to \X$ is an equivalence. In particular, for every objects $X$ of $\X$ there exists $\pi(X)$ in $\ske$ and an isomorphism $\phi_X\colon \pi(X) \to X$.
\end{remark}

\begin{proposition}\label{prop:ske}
	Every category $\X$ has a skeleton. 
\end{proposition}
\begin{proof}
	For every object $X\in \X$, pick a single representative $\pi(X)$ of its isomorphism class. Let $\ske$ be the full subcategory given by these objects. By definition $\ske$ is skeletal and the inclusion functor is full, faithful and essentially surjective.\qedhere 
\end{proof}
\begin{remark}
	The proof of \Cref{prop:ske} relies on the axiom of choice for classes.
\end{remark}
\begin{remark}
	It is possible to prove that every two skeleta of a given category $\X$ are isomorphic (not only equivalent). For the remaining of this paper we assume that a skeleton $\ske$ of $\X$ and a functor $\pi\colon \X\to \ske$ are chosen once and for all.
\end{remark}

\begin{definition}
	Let $(\X, \R)$ be a left-linear DPO-rewriting system, a \emph{decorated derivation} between two objects $G$ and $H$ is a triple $(\der{D}, \alpha, \omega)$, where $\der{D}$ is a derivation between $G$ and $H$, and $\alpha\colon \pi(G)\to G$ and $\omega\colon \pi(H)\to H$ are isomorphisms.
\end{definition}

\begin{notation}
	We will extend the use of the words length, source and target to decorated derivations in the obvious way, forgetting the decorations $\alpha$ and $\omega$.
\end{notation}

\begin{example}A decorated derivation $(\der{D}, \alpha, \omega)$ with $\der{D}$ empty is just a span
	\[\xymatrix{G & \pi(G) \ar[r]^-{\omega} \ar[l]_-{\alpha} & G}\]
	in which both $\omega$ and $\alpha$ are isomorphisms.
\end{example}

As we are interested in objects only up to isomorphism, so we are interested in (decorated) derivations only up to some notion of coherent isomorphism between them. This is done with the help of \Cref{rem:func}.

\begin{definition}Let $(\X, \R)$ be a left-linear DPO-rewriting system. An \emph{abstraction equivalence} between two derivations $\der{D}$ and $\der{D'}$ with the same length and such that $r(\der{D})=r(\der{D}')$ is a family of isomorphisms $\{\phi_X\}_{X\in \Deltamin(\der{D})}$ such that, for every $i\in [0, \lgh(\der{D})-1]$ the following diagram commutes
	\[\xymatrix@C=40pt{G'_i&D'_i \ar[r]^{g'_i} \ar@{>->}[l]_{f'_i}&G'_{i+1}\\  L_i \ar[u]^{m'_i} \ar[d]_{m_i}& K_i \ar[u]^{k'_i} \ar[d]_{k_i} \ar[r]^{r_i} \ar@{>->}[l]_{l_i} &R_i\ar[u]^{h'_i} \ar[d]_{h_i}\\G_i \ar@/_.45cm/[uu]_(.35){\phi_{G_i}}|\hole&D_i\ar@{>->}[l]^{f_i}\ar@/_.45cm/[uu]_(.35){\phi_{D_i}}|\hole \ar[r]_{g_i}&G_{i+1}\ar@/_.45cm/[uu]_{\phi_{H_i}}}\]
	
	If such an abstraction equivalence exists we will say that $\der{D}$ and $\der{D}'$ are \emph{abstraction equivalent} and we will use $\equiv^{abs}$ to denote the resulting relation-
	
Given two decorated derivations $(\der{D}, \alpha, \omega)$  and $(\der{D}', \alpha', \omega')$, with the same length $n$ and using the same rules, an abstraction equivalence $\{\phi_X\}_{X\in \Deltamin(\der{D})}$ is \emph{left consistent} if the left diagram below commutes, while it is \emph{right consistent} if the right one does so. A \emph{consistent} abstraction equivalence  is one which is both left and right consistent.
\[\xymatrix@C=15pt{&\pi(G_0) \ar[dr]^{\alpha'} \ar[dl]_{\alpha}&&& \pi(G_{n}) \ar[dr]^{\omega'} \ar[dl]_{\omega}\\ G_0 \ar[rr]_{\phi_{G_0}} && G'_0 &G_{n} \ar[rr]_{\phi_{G_{n}}} && G'_{n} } \]
	
We will say that $(\der{D}, \alpha, \omega)$  and $(\der{D}', \alpha', \omega')$ are \emph{left consistently}, respectively \emph{right consistently} or \emph{consistently}, abstraction equivalent, if there is a left consistent, respectively a right consistent or a consistent, abstraction equivalence between them. We will use $\equiv^{l}$, $\equiv^{r}$ and $\equiv^{a}$ to denote the resulting relations.
\end{definition}

\begin{proposition}\label{prop:equi}
The relations $\equiv^{abs}, \equiv^l, \equiv^r$ and $\equiv^a$ are equivalences.
\end{proposition}
\begin{proof}
	 
	 Let $\der{D}$ be a derivation, then $\{\id{X}\}_{X\in \Deltamin(\der{D})}$ is an abstraction equivalence between $\der{D}$ and itself. Moreover, starting with a decorated derivation $(\der{D}, \alpha, \omega)$, the same $\{\id{X}\}_{X\in \Deltamin(\der{D})}$  is also consistent, so that all the four relations are reflexive.
	 
	 For simmetry, let us suppose that $\{\phi_X\}_{X\in \Deltamin(\der{D})}$ is an abstraction equivalence between $\der{D}$ and $\der{D}'$, then $\{\phi^{-1}_X\}$ defines an abstraction equivalence  between $\der{D}'$ and $\der{D}$. Indeed, for every $i\in [0, \lgh(\der{D})-1]$ we have
	 \[\begin{split}
	 	m_i&=\id{G_i}\circ m_i\\&=\phi^{-1}_{G_i}\circ \phi_{G_i}\circ m_i\\&=\phi^{-1}_{G_i}\circ m'_i
	 \end{split} \qquad \begin{split}
 	k_i&=\id{D_i}\circ k_i\\&=\phi^{-1}_{D_i}\circ \phi_{D_i}\circ k_i\\&=\phi^{-1}_{D_i}\circ k'_i
	 \end{split}\qquad \begin{split}
 	h_i&=\id{G_{i+1}}\circ h_i\\&=\phi^{-1}_{G_{i+1}}\circ \phi_{G_{i+1}}\circ h_i\\&=\phi^{-1}_{G_{i+1}}\circ h'_i
	 \end{split} \]
	
	Starting with two decorated derivations $(\der{D}, \alpha, \omega)$ and $(\der{D}', \alpha', \omega')$, if $\{\phi_X\}_{X\in \Deltamin(\der{D})}$  is left, respectively right, consistent  then, putting $n$ as the length of $\der{D}$, we have:
	\[\begin{split}
		\alpha&=\id{G_0}\circ \alpha\\&=\phi^{-1}_{G_0}\circ \phi_{G_0}\circ \alpha\\&=\phi^{-1}_{G_0}\circ \alpha'
	\end{split}\qquad \begin{split}
		\omega&=\id{G_n}\circ \omega\\&=\phi^{-1}_{G_n}\circ \phi_{G_n}\circ \omega\\&=\phi^{-1}_{G_n}\circ \omega'
	\end{split}\]
	
	Finally, let $\{\phi_X\}_{X\in \Deltamin(\der{D})}$ and $\{\psi_X\}_{X\in \Deltamin(\der{D}')}$ be abstraction equivalence witnessing $\der{D} \equiv^{abs}\der{D}'$ and $\der{D}'\equiv^{abs}\der{D}''$. Then we can consider the family $\{\psi_X\circ \phi_X\}_{X\in \Deltamin(\der{D})}$  and check that, for every $i\in [0,\lgh(\der{D})]$ we have
	 \[\begin{split}
	\psi_{G_i}\circ \phi_{G_i}\circ m_i&= \psi_{G_i}\circ m'_i\\&=m''_i 
\end{split} \qquad \begin{split}
	\psi_{D_i}\circ \phi_{D_i}\circ k_i&= \psi_{D_i}\circ k'_i\\&=k''_i
\end{split}\qquad \begin{split}
\psi_{G_{i+1}}\circ \phi_{G_{i+1}}\circ h_i&= \psi_{G_{i+1}}\circ h'_i\\&=h''_i
\end{split} \]
	
	For decorations, if $\{\phi_X\}_{X\in \Deltamin(\der{D})}$ and $\{\psi_X\}_{X\in \Deltamin(\der{D}')}$ are left or right consistent we have
	 \[\begin{split}
		\psi_{G_0}\circ \phi_{G_0}\circ \alpha&= \psi_{G_0}\circ \alpha'\\&=\alpha'' 
	\end{split} \qquad \begin{split}
		\psi_{G_{\lgh(\der{D})}}\circ \phi_{G_{\lgh(\der{D})}}\circ \omega&= \psi_{G_{\lgh(\der{D})}}\circ \omega'\\&=\omega''
	\end{split}\]
	And we can conclude.
\end{proof}

\begin{notation}
	Since we are mainly interested into consistent abstraction equivalence, we will use  $[\der{D}, \alpha, \omega]_a$ to denote the equivalence class of  $(\der{D}, \alpha, \omega)$ with respect to $\equiv^a$.
	Such equivalence classes will be called  \emph{abstract decorated derivations}.  
\end{notation}

\begin{example}\label{rem:empty}
	Let $\{G\}$ and  $\{G'\}'$ be two empty derivations  An abstraction equivalence between $\{G\}$ and $\{G\}'$ is just an isomorphism $\phi\colon G\to G'$, so that $\pi(G)=\pi(G')$. If we are dealing with decorated derivations $(\{G\}, \alpha, \omega)$ and $(\{G'\}, \alpha, \omega)$,  left and right consistency of $\phi $ correspond to the commutativity of the diagrams below.
	\[\xymatrix@C=15pt{&\pi(G) \ar[dr]^{\alpha'} \ar[dl]_{\alpha}&&& \pi(G) \ar[dr]^{\omega'} \ar[dl]_{\omega}\\ G\ar[rr]_{\phi} && G' &G \ar[rr]_{\phi} && G' } \]
	
	In particular, this prove that two empty decorated derivations $(\{G\},\alpha, \omega) $ and $(\{G'\}',\alpha', \omega')$ are consistently abstraction equivalent if and only if
	\[	\alpha'\circ \alpha^{-1}=\omega'\circ \omega^{-1}\]
\end{example}

\begin{remark}\label{rem:res} \Cref{prop:unique} can be restated as saying that, given two direct derivations $\dder{D}$ and $\dder{D'}$ with the same match, there exists an abstract equivalence between them whose first component is the identity. 
\end{remark}

\begin{remark}\label{rem:absequi}
	Let $\der{D}$ be a derivation with source $G$ and target $H$. Let also $\phi\colon G'\to G$ and $\psi\colon H\to H'$ be two isomorphisms. Then for every $X\in \Deltamin(\der{D})$ we can define 
	\[\varphi_X:=\begin{cases}
		\phi^{-1} & X=G\\
		\id{X} & \text{otherwise}
	\end{cases} \qquad \varphi'_X:=\begin{cases}
		\psi & X=H\\
		\id{X} & \text{otherwise}
	\end{cases}\]
	It is immediate to see that the family $\{\varphi_X\}_{X\in \Deltamin(\der{D})}$ is an abstraction equivalence between $\der{D}$ and $\phi *\der{D}$, while $\{\varphi'_X\}_{X\in \Deltamin(\der{D})}$ is one between $\der{D}$ and $\der{D}*\psi$.  
	
	Taking in account the decorations, we also have that $\{\phi_X\}_{X\in \Deltamin(\der{D})}$ and  $\{\phi'_X\}_{X\in \Deltamin(\der{D})}$ witness, respectively, that 
	\[(\der{D}, \alpha, \omega) \equiv^a (\phi * \der{D}, \phi^{-1}\circ \alpha,   \varphi^{-1}_{H} \circ \omega) \qquad (\der{D}, \alpha, \omega) \equiv^a ( \der{D}*\psi, \varphi'_{G} \circ \alpha, \psi \circ \omega )\]

if $\der{D}$ is not empty, then the previous equations become
\[(\der{D}, \alpha, \omega) \equiv^a (\phi * \der{D}, \phi^{-1}\circ \alpha,   \omega) \qquad (\der{D}, \alpha, \omega) \equiv^a ( \der{D}*\psi, \alpha, \psi \circ \omega )\]

In particular, given an empty decorated derivation $(\{G\}, \alpha, \omega)$, we can apply what we have noticed to $\alpha \colon \pi(G)\to G$ and to $\omega^{-1}\colon G\to \pi(G) $ to get that 
\[(\{G\}, \alpha, \omega)\equiv^a (\{\pi(G)\}, \id{\pi(G)}, \alpha^{-1}\circ \omega) \qquad (\{G\}, \alpha, \omega)\equiv^a (\{\pi(G)\}, 	\omega^{-1}\circ \alpha, \id{\pi(G)})\]
\end{remark}


\begin{definition}\label{def:conc}
	Let $(\der{D}, \alpha, \omega)$ be a decorated derivation between $G$ and $H$ and $(\der{D}', \alpha', \omega')$ one between $H'$ and $K$. If $H$ and $H'$ are isomorphic, so that $\pi(H)=\pi(H')$, we define the  \emph{composite decorated derivation} putting
	\[(\der{D}, \alpha, \omega)\cdot (\der{D}', \alpha', \omega'):=\begin{cases}
		(\der{D}', \alpha'\circ \omega^{-1}\circ \alpha, \omega')	&\lgh(\der{D})=0 \\
		(\der{D}, \alpha, \omega \circ (\alpha')^{-1}\circ \omega')&\lgh(\der{D}')=0 \text{ and } \lgh(\der{D})\neq 0\\
		(\der{D}*\omega^{-1}\cdot \alpha'*\der{D'}, \alpha, \omega')	&\text{otherwise}
	\end{cases}\]
\end{definition}

\begin{remark}\label{rem:lgt}
	Let $(\der{D}, \alpha, \omega)$ and $(\der{D}', \alpha', \omega')$ two composable decorated derivations   such that $\lgh(\der{D})=n$ and $\lgh(\der{D}')=m$.	 Then $(\der{D}, \alpha, \omega)\cdot (\der{D}', \alpha', \omega')$ has length $n+m$.
\end{remark}

\begin{remark}\label{rem:id}
	Let  $(\{G\}, \alpha, \alpha)$ be a decoration of length $0$ with the same isomorphism as initial and final decoration. Then for every other derivations $(\der{D}, \beta, \omega)$ with source isomorphic to $G$, we have
	\[(\{G\}, \alpha, \alpha)\cdot (\der{D}, \beta, \omega) = (\der{D}, \beta, \omega)\]
	
	On the other hand, if  $(\der{E}, \gamma, \zeta)$  has a target isomorphic to $G$, we have
	\[(\der{E}, \gamma, \zeta)\cdot (\{G\}, \alpha, \alpha) = \begin{cases}
		(\{G\}, \alpha \circ \zeta^{-1}\circ \gamma, \alpha)	&\lgh(\der{E})=0 \\
		(\der{E}, \gamma, \zeta )& \lgh(\der{E})= 0
	\end{cases}\]
	
Now,  in the first case, we have that
\[\alpha \circ \zeta^{-1} \circ \gamma \circ \gamma ^{-1} = \alpha \circ \zeta^{-1}\]
therefore, by \Cref{rem:empty}, $(\der{E}, \gamma, \zeta)\cdot (\{G\}, \alpha, \alpha)\equiv^a  (\der{E}, \gamma, \zeta)$.
\end{remark}

\begin{remark}\label{ex:zero}
Let $(\{G\}, \alpha, \omega)$ and $(\{G'\}, \alpha', \omega')$ be two decorated derivations of length zero with $G$ and $G'$ isomorphic. Then we can always compose with another empty decorated derivations to get one from the other. In particular, using \Cref{rem:absequi} we have:
\begin{align*}
(\{G\}, \alpha, \omega)&\equiv^{a} (\pi(G), \id{\pi(G)}, \alpha^{-1}\circ \omega)\\&=(\{\pi(G)\}, (\alpha')^{-1}\circ \omega'\circ \omega'^{-1} \circ \alpha', \alpha^{-1}\circ \omega)\\&=(\{G'\}, \alpha', \omega')\cdot (\{\pi(G)\}, (\alpha')^{-1}\circ \omega', \alpha^{-1}\circ \omega)
\\(\{G\}, \alpha, \omega)&\equiv^{a} (\{\pi(G)\}, 	\omega^{-1}\circ \alpha, \id{\pi(G)})\\&=(\{\pi(G)\}, \omega^{-1}\circ \alpha, (\omega')^{-1}\circ\alpha'\circ (\alpha')^{-1} \circ \omega')\\&= (\{\pi(G)\}, \omega^{-1}\circ \alpha, (\omega')^{-1}\circ\alpha') \cdot (\{G'\}, \alpha', \omega')
\end{align*}
\end{remark}

We can generalize the previous remark to the following elementary, but useful, result about abstraction equivalence of composite decorated derivations. 
\begin{lemma}\label{rem:obv}
	Let  $(\der{D}_1, \alpha_1, \omega_1)$, $(\der{D}_2, \alpha_2, \omega_2)$, $(\der{D}'_1, \alpha'_1, \omega'_1)$ and $(\der{D}'_2, \alpha'_2, \omega'_2)$ be two pairs of composable derivations and define $(\der{E}, \xi, \theta)$ and $(\der{E}', \xi', \theta')$ as 
	\[(\der{E}, \xi, \theta):=(\der{D}_1, \alpha_1, \omega_1) \cdot (\der{D}_2, \alpha_2, \omega_2) \qquad (\der{E}', \xi', \theta'):=(\der{D}'_1, \alpha'_1, \omega'_1) \cdot (\der{D}'_2, \alpha'_2, \omega'_2)\]
	 If $\lgh(\der{D}_1)=\lgh(\der{D}'_1)$, $\lgh(\der{D}_2)=\lgh(\der{D}'_2)$ and $\der{E}\equiv^a \der{E}'$  then there exist empty derivations $(\{A\}, \gamma_1, \delta_1)$ and $(\{B\}, \gamma_2, \delta_2)$ such that
	 \[(\der{D}_1, \alpha_1, \omega_1) \equiv^a (\der{D}'_1, \alpha'_1, \omega'_1) \cdot  (\{A\}, \gamma_1, \delta_1) \qquad (\der{D}_2, \alpha_2, \omega_2) \equiv^a (\{B\}, \gamma_2, \delta_2)\cdot (\der{D}'_2, \alpha'_2, \omega'_2)  \]
\end{lemma}
\begin{proof} Let $\{\phi_X\}_{X\in \Deltamin(\der{E})}$ be an abstraction equivalent witnessing that $(\der{E}, \zeta, \theta)\equiv^{a}(\der{E}', \zeta', \theta')$.	We split the cases following \Cref{def:conc}.
	
	\smallskip \noindent $\lgh(\der{D}_1)=0$. In this case we have that
	\[(\der{E}, \zeta, \theta)=(\der{D}_2, \alpha_2\circ \omega^{-1}_1 \circ \alpha_1, \omega_2) \qquad (\der{E}', \zeta', \theta')=(\der{D}'_2, \alpha'_2\circ (\omega'_1)^{-1} \circ \alpha'_1, \omega'_2)\]
	By hypothesis $\der{D}_1$ and $\der{D}'_1$ are given by $\{G_{1,0}\}$ and $\{G'_{1,0}\}$. Moreover we know that $G_{1,0}$ is isomorphic to $G_{2,0}$ and  $G'_{1,0}$ to $G'_{2,0}$. Since $\phi_{G_{2,0}}$ is an isomorphism we conclude that $G_{1,0}$ and $G_{2,0}$ are isomorphic and we deduce the first half of the thesis from \Cref{ex:zero}. 
	
	For the second half, let us consider define $B$ as $\pi(G_{2,0})$ and
	\[\gamma_2:=\alpha^{-1}_1\circ \omega_1\qquad  \delta_2:=(\alpha'_1)^{-1}\circ \omega'_1  \]
	
	Hence we have
\[
(\{\pi(G_{2,0})\}, \alpha^{-1}_1\circ \omega_1,  (\alpha'_1)^{-1}\circ \omega'_1 )\cdot(\der{D}'_2, \alpha'_2, \omega'_2)  =(\der{D}_2, \alpha'_2\circ (\omega'_1)^{-1} \circ \alpha'_1\circ \alpha_1^{-1}\circ \omega_1, \omega'_2)\]
Now, by consistency of $\{\phi_{X}\}_{X\in \Deltamin(\der{D}_2)}$ we know that
\[\phi_{G_{2,0}}\circ \alpha_2\circ \omega^{-1}_1\circ \alpha_1=\alpha'_2\circ (\omega'_1)^{-1} \circ \alpha'_1\]
	therefore we also have the equality
	\[\phi_{G_{2,0}}\circ \alpha_2=\alpha'_2\circ (\omega'_1)^{-1} \circ \alpha'_1 \circ \alpha^{-1}_1\circ \omega_1 \]
	We conclude that $\{\phi_X\}_{X\in \der{D}_2}$ witnesses the thesis.
	
	\smallskip \noindent $\lgh(\der{D}_2)=0$ and $\lgh(\der{D}_1)\neq 0$. By definition we know that
	\[(\der{E}, \zeta, \theta)=(\der{D}_1, \alpha_1, \omega_1\circ \alpha_2^{-1}\circ \omega_2) \qquad (\der{E}', \zeta', \theta')=(\der{D}'_1, \alpha'_1, \omega'_1\circ (\alpha'_2)^{-1}\circ \omega'_2)\]
	
	Let $l$ be the length of $\der{D}_1$, and consider $(\{\pi(G'_{1, l})\}, (\omega'_2)^{-1}\circ \alpha'_2, \omega^{-1}_2\circ \alpha_2)$, so that
	\[ (\der{D}'_1, \alpha'_1, \omega'_1) \cdot  (\{\pi(G_{1, l})\}, (\omega'_2)^{-1}\circ \alpha'_2, \omega^{-1}_2\circ \alpha_2)= (\der{D}'_1, \alpha'_1, \omega'_1\circ (\alpha'_2)^{-1}\circ \omega'_2 \circ \omega^{-1}_2\circ \alpha_2)\]
	Now, let us consider the abstraction equivalence between  the decorated derivations $(\der{D}_1, \alpha_1, \omega_1)$ and  $(\der{D}'_1, \alpha'_1, \omega'_1) \cdot  (\{\pi(G_{1, l})\}, (\omega'_2)^{-1}\circ \alpha'_2, \omega^{-1}_2\circ \alpha_2)$ given by $\{\phi_X\}_{X\in \Deltamin(\der{D}_1)}$. On the one hand it is, by construction, left consistent.  On the other hand we know that 
	 
	\[\phi_{G_{1,l}}\circ \omega_1\circ \alpha_2^{-1}\circ \omega_2=\omega'_1\circ (\alpha'_2)^{-1}\circ \omega'_2 \]
	 
	 Composing on both sides with $\omega^{-1}_1\circ \alpha_2$ yields now the thesis.
	 
The second half of the thesis follows immediately from \Cref{ex:zero}.	
	
	\smallskip \noindent  $\lgh(\der{D}_1)$ and $\lgh(\der{D}_2)$ are both different from $0$. By \Cref{def:conc} in this case we have
	\[(\der{E}, \zeta, \theta)=(\der{D}_1*\omega^{-1}_1 \cdot \alpha_2*\der{D}_2, \alpha_1,\omega_2 ) \qquad (\der{E}', \zeta', \theta')=(\der{D}'_1*(\omega'_1)^{-1} \cdot \alpha'_2*\der{D}'_2, \alpha'_1,\omega'_2 )\]
	
	
	
\end{proof}

The next proposition justifies the use of decorations, guaranteeing that concatenation of abstract decorated derivations is well-defined.
\begin{lemma}\label{lem:conc}
	Given a decorated derivation $(\der{D}, \alpha, \omega)$  between $G$ and $H$ and  another one $(\der{E}, \beta, \xi)$ between $E$ and $K$ with $\pi(H)=\pi(E)$. If  $(\der{D}', \alpha', \omega')$ and $(\der{E}', \beta', \xi')$ are two other decorated derivations such that
	\[[\der{D}, \alpha, \omega]_a = [\der{D}', \alpha', \omega']_a \qquad [\der{D}, \beta, \xi]_a=[\der{E}', \beta', \xi']_a\]
	Then
	\[[(\der{D}, \alpha, \omega)\cdot (\der{E}, \beta, \xi)]_a=[(\der{D}', \alpha', \omega')\cdot (\der{E}', \beta', \xi')]_a\]
\end{lemma}

\begin{proof} Take two abstraction equivalences $\{\phi_X\}_{X\in \Deltamin(\der{D})}$ and $\{\varphi_X\}_{X\in \Deltamin(\der{D})}$ between $(\der{D}, \alpha, \omega)$ and $(\der{D}', \alpha', \omega')$ and between $(\der{E}, \beta, \xi)$ and $(\der{E}', \beta', \xi')$, respectively. To fix the notation, suppose that $(\der{D}', \alpha', \omega')$ goes from $G'$ to $H'$ and $(\der{E}', \beta', \xi')$ from $E'$ to $K'$. We have three cases.

	\smallskip \noindent  $\lgh(\der{D})=0$. Then, $\lgh(\der{D}')$ is $0$ too. By \Cref{def:conc} we have
		\begin{align*}
			(\der{D}, \alpha, \omega)\cdot (\der{E}, \beta, \xi)&=(\der{E}, \beta\circ \omega^{-1}\circ \alpha, \xi)\\
			(\der{D}', \alpha', \omega')\cdot (\der{E}', \beta', \xi')&=(\der{E}', \beta'\circ (\omega')^{-1}\circ \alpha', \xi')
		\end{align*}
		Now, notice that $G$ and $H$ must coincide. Moreover, by \Cref{rem:empty} we also know that $\pi(G)=\pi(G')$ too. The same \Cref{rem:empty} entails that the inner squares of the following diagram are commutative, so that the whole rectangle commutes too.
		\[\xymatrix@C=35pt{\pi(G) \ar[r]^{\alpha} \ar[d]_{\id{\pi(G)}}& G  \ar[d]_{\phi_G}\ar[r]^-{\omega^{-1}} & \pi(G) \ar[d]_{\id{\pi(G)}} \ar[r]^{\beta}& E \ar[d]^{\varphi_E}\\\pi(G') \ar[r]_{\alpha'}& G' \ar[r]_-{(\omega')^{-1}} & \pi(G') \ar[r]_{\beta'} & E'}\]
		
		We can  then conclude that $\{\varphi_X\}_{X\in \Deltamin(\der{D})}$ witnesses the fact that $(\der{E}, \beta\circ \omega^{-1}\circ \alpha, \xi)$ is abstraction equivalent to $(\der{E}', \beta'\circ (\omega')^{-1}\circ \alpha', \xi')$.
		
\smallskip \noindent  $\lgh(\der{D})\neq 0$ and $\lgh(\der{E})= 0$. As in the point above, we get that also $\der{E}'$  is an empty derivation, thus we have
		\begin{align*}
			(\der{D}, \alpha, \omega)\cdot (\der{E}, \beta, \xi)&=(\der{D},  \alpha, \omega \circ \beta^{-1}\circ \xi)\\
			(\der{D}', \alpha', \omega')\cdot (\der{E}', \beta', \xi')&=(\der{D}',  \alpha', \omega' \circ (\beta')^{-1}\circ \xi')
		\end{align*}
		In this case we have that $E=K$ and that $\pi(E)=\pi(E')$. From \Cref{rem:empty} we deduce that the diagram below commutes.
		\[\xymatrix@C=35pt{\pi(E) \ar[r]^{\xi} \ar[d]_{\id{\pi(E)}}& E  \ar[d]_{\varphi_E}\ar[r]^-{\beta^{-1}} & \pi(E) \ar[d]_{\id{\pi(E)}} \ar[r]^{\omega}& H \ar[d]^{\phi_H}\\\pi(E') \ar[r]_{\xi'}& E' \ar[r]_-{(\beta')^{-1}} & \pi(E') \ar[r]_{\omega'} & H'}\]
		The thesis now follows at once.
		
		\smallskip \noindent $\lgh(\der{D})\neq 0$ and $\lgh(\der{E})\neq 0$. In this case we have
		\begin{align*}
			(\der{D}, \alpha, \omega)\cdot (\der{E}, \beta, \xi)&=(\der{D}*\omega^{-1}\cdot \beta*\der{E}, \alpha, \xi)\\
			(\der{D}', \alpha', \omega')\cdot (\der{E}', \beta', \xi')&=(\der{D}'*(\omega')^{-1}\cdot \beta'*\der{E'}, \alpha', \xi')
		\end{align*}
		To fix the notation, suppose that $\der{D}$, $\der{D}'$, $\der{E}$ and $\der{E}'$ are given by
		\[\der{D}=\{\dder{D}_i\}_{i=0}^n \quad \der{D}'=\{\dder{D}'_i\}_{i=0}^n \quad \der{E}=\{\dder{E}_i\}_{i=0}^t \quad \der{E}'=\{\dder{E}'_i\}_{i=0}^t\]
		Moreover, noticing that the rule applied by $\dder{D}_i$ and the one applied in $\mathcal{E}_i$  must coincide with, respectively, the one applied in $\dder{D}'_i$ and the one applied $\dder{E}'_i$. We will also assume that $\dder{D}_i$, $\dder{D}'_i$, $\dder{E}_i$ and $\dder{E}'_i$ are given, respectively, by the following four diagrams. 
		\[\xymatrix{L_{\der{D},i} \ar[d]_{m_{\der{D}, i}}& K_{\der{D},i} \ar[d]_{k_{\der{D}, i}} \ar[r]^{r_{\der{D},i}} \ar@{>->}[l]_{l_{\der{D},i}} & R_{\der{D},i}\ar[d]^{h_{\der{D}, i}} &L_{\der{E},i} \ar[d]_{m_{\der{E}, i}}& K_{\der{E},i} \ar[d]_{k_{\der{E}, i}} \ar[r]^{r_{\der{E},i}} \ar@{>->}[l]_{l_{\der{E},i}} & R_{\der{E},i}\ar[d]^{h_{\der{E}, i}} \\G_i & D_i \ar[r]_{g_{\der{D},i}} \ar@{>->}[l]^{f_{\der{D},i}} & G_{i+1} & E_i & F_i\ar[r]_{g_{\der{E},i}} \ar@{>->}[l]^{f_{\der{E},i}}  & E_{i+1}}\]
		\[\xymatrix{L_{\der{D},i} \ar[d]_{m_{\der{D}', i}}& K_{\der{D},i} \ar[d]_{k_{\der{D}', i}} \ar[r]^{r_{\der{D},i}} \ar@{>->}[l]_{l_{\der{D},i}} & R_{\der{D},i}\ar[d]^{h_{\der{D}', i}} &L_{\der{E},i} \ar[d]_{m_{\der{E}', i}}& K_{\der{E},i} \ar[d]_{k_{\der{E}', i}} \ar[r]^{r_{\der{E},i}} \ar@{>->}[l]_{l_{\der{E},i}} & R_{\der{E},i}\ar[d]^{h_{\der{E}', i}} \\G'_i & D'_i \ar[r]_{g_{\der{D}',i}} \ar@{>->}[l]^{f_{\der{D}',i}} & G'_{i+1} & E'_i & F'_i\ar[r]_{g_{\der{E}',i}} \ar@{>->}[l]^{f_{\der{E}',i}}  & E'_{i+1}}\]
		
		Now, for every $X\in \Deltamin(\der{D}*\omega^{-1}\cdot \beta*\der{E})$ we can define
		\[\psi_X:=\begin{cases}
			\phi_X & X\in  \Deltamin(\der{D})\text{ and } X\neq H\\\varphi_X & X\in  \Deltamin(\der{E}) \text{ and } X\neq E\\\id{\pi(H)}& X=\pi(H)\end{cases}\]
		Notice that, since $\psi_G=\phi_G$ and $\psi_K=\varphi_K$ we have at once  the commutativity of the triangles
		\[\xymatrix@C=15pt{&\pi(G) \ar[dr]^{\alpha'} \ar[dl]_{\alpha}&&& \pi(K) \ar[dr]^{\xi'} \ar[dl]_{\xi}\\ G\ar[rr]_{\psi_G} && G' &K \ar[rr]_{\psi_K} && K' } \]
		
		
		To show that $\{\psi_X\}_{\Deltamin(\der{D}*\omega^{-1}\cdot \beta*\der{E})}$ is an abstraction equivalence, it is now enough to prove the commutativity of the diagrams below.
		\[\xymatrix@R=15pt@C=22pt{\pi(E')&&F'_0 \ar@{>->}[ll]_-{(\beta')^{-1}\circ f_{\der{E}',0}} \ar@{>->}@/^.2cm/[dll]^(.6){f_{\der{E}',0}} \ar[r]^{g_{\der{E}',0}}&E'_1 &G'_n&D'_n \ar[rr]^-{(\omega')^{-1}\circ g_{\der{D}',n}} \ar@/_.2cm/[drr]_(.6){g_{\der{D}',n}} \ar@{>->}[l]_{f_{\der{D}',n}}&&\pi(H')
			\\E' \ar[u]_{(\beta')^{-1}} &&&&&&&H' \ar[u]^{(\omega')^{-1}}
			\\  L_{\der{E}, 0} \ar[u]_{m_{\der{E}', 0}} \ar[d]^{m_{\der{E}, 0}}&& K_{\der{E}, 0} \ar[uu]_{k_{\der{E}', 0}} \ar[dd]^{k_{\der{E}, i}} \ar[r]^(.4){r_{\der{E},0}} \ar@{>->}[ll]_{l_{\der{E},0}} &R_{\der{E}, 0}\ar[uu]_{h_{\der{E}',0}} \ar[dd]^{h_{\der{E},0}} & L_{\der{D}, n} \ar[uu]^{m_{\der{D}',n}} \ar[dd]_{m_{\der{D},n}}& K_{\der{D}, n} \ar[uu]^{k_{\der{D}',n}} \ar[dd]_{k_{\der{D},n}} \ar[rr]^{r_{\der{D},n}} \ar@{>->}[l]_(.4){l_{\der{D},n}} &&R_{\der{D}, n}\ar[u]^{h_{\der{D}',n}} \ar[d]_{h_{\der{D},n}}
			\\E \ar[d]^{\beta^{-1}}&&&&&&&H \ar[d]_{\omega^{-1}}
			\\\pi(E) \ar@/^.7cm/[uuuu]^(.2){\id{\pi(E)}}&&F_0\ar@{>->}[ll]^{\beta^{-1}\circ f_{\der{E},0}}\ar@/^.7cm/[uuuu]^(.35){\varphi_{F_0}}|\hole \ar@{>->}@/_.2cm/[ull]_(.6){f_{\der{E},0}}\ar[r]_{g_{\der{E},0}}&E_1\ar@/^.7cm/[uuuu]^(.35){\varphi_{E_1}}|\hole&G_n \ar@/_.7cm/[uuuu]_(.35){\phi_{G_n}}|\hole&D_n\ar@{>->}[l]^{f_{\der{D},n}}\ar@/_.7cm/[uuuu]_(.35){\phi_{D_n}}|\hole \ar@/^.2cm/[urr]^(.6){g_{\der{D},n}}\ar[rr]_{\omega^{-1}\circ g_{\der{D},n}}&&\pi(H)\ar@/_.7cm/[uuuu]_(.2){\id{\pi(H)}}} \]
		
		To see this, in turn, it is enough to show that	the following squares are commutative.
		\[\xymatrix{R_{\der{D}, n} \ar[d]_{h_{\der{D}, n}}\ar[r]^{h_{\der{D}', n}} & H' \ar[d]_{(\omega')^{-1}} &L_{\der{E}, 0} \ar[d]^{m_{\der{E}, 0}}\ar[r]^{m_{\der{E}', 0}}& E' \ar[d]^{(\beta')^{-1}}\\H\ar[r]_{\omega^{-1}} &\pi(H)&E \ar[r]_{\beta^{-1}} & \pi(E)}\]
		\[\xymatrix{D_n \ar[rr]^{\phi_{D_n}} \ar[d]_{g_{\der{D}, n}}&&D'_n \ar[d]_{g_{\der{D}', n}}&F_0 \ar@{>->}[d]^{f_{\der{E}, 0}} \ar[rr]^{\varphi_{F_0}} && F_0' \ar@{>->}[d]^{f_{\der{E}', 0}}\\H \ar[r]_-{\omega^{-1}} &\pi(H)&H' \ar[l]^-{(\omega')^{-1}}  &E \ar[r]_-{\beta^{-1}} & \pi(E) & E' \ar[l]^-{(\beta')^{-1}}}\]
		
		For the first ones, we have
		\[
		\begin{split} 
			\omega^{-1} \circ h_{\der{D}, n}&=\id{\pi(H)}\circ \omega^{-1} \circ h_{\der{D}, n} \\&=(\omega')^{-1}\circ \phi_H \circ \omega \circ \omega^{-1} \circ h_{\der{D}, n}\\&=(\omega')^{-1}\circ \phi_H  \circ h_{\der{D}, n}\\&=(\omega')^{-1}  \circ h_{\der{D}', n}
		\end{split} \qquad  	\begin{split} 
			\beta^{-1} \circ m_{\der{E}, 0}&=\id{\pi(E)}\circ \beta^{-1} \circ m_{\der{E}, 0}\\&=(\beta')^{-1}\circ \varphi_{E} \circ \beta \circ \beta^{-1} \circ m_{\der{E}, 0}\\&=(\beta')^{-1}\circ \varphi_{E}  \circ m_{\der{E}, 0}\\&=(\beta')^{-1}  \circ m_{\der{E}', 0}
		\end{split}  \]
		
		Similarly, the commutativity of the second row of diagrams can be deduced from
		\[\begin{split}
			\omega^{-1} \circ g_{\der{D}, n}&=\id{\pi(H)}\circ \omega^{-1} \circ g_{\der{D}, n}\\&=(\omega')^{-1}\circ \phi_H \circ \omega\circ \omega^{-1} \circ g_{\der{D}, n}\\&=(\omega')^{-1}\circ \phi_H  \circ g_{\der{D}, n}\\&= (\omega')^{-1} \circ g_{\der{D}', n}\circ \phi_{D_n}
		\end{split}\qquad \begin{split}\beta^{-1} \circ f_{\der{E}, 0}&=\id{\pi(E)}\circ \beta^{-1} \circ f_{\der{E}, 0}\\&=(\beta')^{-1}\circ \varphi_E \circ \beta\circ \beta^{-1} \circ f_{\der{E}, 0}\\&=(\beta')^{-1}\circ \varphi_E  \circ f_{\der{E}, 0}\\&= (\beta')^{-1} \circ f_{\der{E}', n}\circ \varphi_{F_0}
		\end{split}\]
		
		The thesis now follows.	
\end{proof}

We are now ready to prove that the operatione defined by \Cref{def:conc} is associative up to decorated abstraction equivalence.

\begin{lemma}[Associativity Lemma]\label{lem:ass}  
Let $(\der{D}_1, \alpha_1, \omega_1)$ be a decorated derivation between $G$ and $H$, $(\der{D}_2, \alpha_2, \omega_2)$, one between $H$ and $K$ and $(\der{D}_3, \alpha_3, \omega_3)$ one between $K$ and $T$ in a left-linear DPO rewriting system. Then
\[((\der{D}_1, \alpha_1, \omega_1)\cdot (\der{D}_2, \alpha_2, \omega_2)) \cdot (\der{D}_3, \alpha_3, \omega_3) \equiv^a (\der{D}_1, \alpha_1, \omega_1)\cdot ( (\der{D}_2, \alpha_2, \omega_2)\cdot (\der{D}_3, \alpha_3, \omega_3))\]
\end{lemma}
\begin{proof} We have seven cases.
	
\smallskip \noindent  $\der{D}_1$, $\der{D}_2$ are both empty.
In this case computing we get:
\begin{align*}
	((\der{D}_1, \alpha_1, \omega_1)\cdot (\der{D}_2, \alpha_2, \omega_2)) \cdot (\der{D}_3, \alpha_3, \omega_3)&=(\der{D}_2, \alpha_2\circ \omega^{-1}_1 \circ \alpha_1, \omega_2)\cdot (\der{D}_3, \alpha_3,\omega_3)\\&=(\der{D}_3,\alpha_3\circ\omega^{-1}_2\circ \alpha_2\circ \omega^{-1}_1\circ \alpha_1 ,\omega_3)\\&=(\der{D}_1, \alpha_1, \omega_1) \cdot (\der{D}_3, \alpha_3\circ \omega^{-1}_2 \circ \alpha_2,\omega_3)\\&=(\der{D}_1, \alpha_1, \omega_1) \cdot ((\der{D}_2, \alpha_2, \omega_2)\cdot (\der{D}_3, \alpha_3, \omega_3))
\end{align*}


\smallskip \noindent $\der{D}_1$ and $\der{D}_3$ are empty while $\der{D}_2$ is not. Apply \Cref{def:conc} we get
\begin{align*}
((\der{D}_1, \alpha_1, \omega_1)\cdot (\der{D}_2, \alpha_2, \omega_2)) \cdot (\der{D}_3, \alpha_3, \omega_3)&=(\der{D}_2, \alpha_2\circ \omega^{-1}_1 \circ \alpha_1, \omega_2)\cdot (\der{D}_3, \alpha_3,\omega_3)\\&=(\der{D}_2, \alpha_2\circ \omega^{-1}_1 \circ \alpha_1, \omega_2\circ \alpha^{-1}_3 \circ \omega_3)\\&=(\der{D}_1, \alpha_1, \omega_1 )\cdot  (\der{D}_2, \alpha_2,\omega_2\circ \alpha^{-1}_3 \circ  \omega_3)\\&= (\der{D}_1, \alpha_1, \omega_1 )\cdot((\der{D}_2, \alpha_2, \omega_2)\cdot (\der{D}_3, \alpha_3, \omega_3) )
\end{align*}


\smallskip \noindent  $\der{D}_1$ is empty, $\der{D}_2$ and $\der{D}_3$ are not. As above we can compute to get
\begin{align*}
((\der{D}_1, \alpha_1, \omega_1)\cdot (\der{D}_2, \alpha_2, \omega_2)) \cdot (\der{D}_3, \alpha_3, \omega_3)&=(\der{D}_2, \alpha_2\circ \omega^{-1}_1 \circ \alpha_1, \omega_2)\cdot (\der{D}_3, \alpha_3, \omega_3)\\&=(\der{D}_2*\omega^{-1}_2\cdot \alpha_3*\der{D}_3, \alpha_2\circ \omega^{-1}_1 \circ \alpha_1, \omega_3 )\\&=(\der{D}_1, \alpha_1, \omega_1) \cdot (\der{D}_2*\omega^{-1}_2\cdot \alpha_3*\der{D}_3, \alpha_2, \omega_3 )\\&=(\der{D}_1, \alpha_1, \omega_1) \cdot ((\der{D}_2, \alpha_2, \omega_2) \cdot (\der{D}_3, \alpha_3, \omega_3)  )
\end{align*}

\smallskip \noindent $\der{D}_1$ is non-empty while $\der{D}_2$ and $\der{D}_3$ have length $0$. Therefore:
\begin{align*}
	((\der{D}_1, \alpha_1, \omega_1)\cdot (\der{D}_2, \alpha_2, \omega_2)) \cdot (\der{D}_3, \alpha_3, \omega_3)&=(\der{D}_1, \alpha_1, \omega_1\circ \alpha^{-1}_2\circ \omega_2)\cdot (\der{D}_3, \alpha_3, \omega_3)\\&=(\der{D}_1, \alpha_1, \omega_1\circ \alpha^{-1}_2\circ \omega_2\circ \alpha^{-1}_3 \circ \omega_3 )\\&=(\der{D}_1, \alpha_1, \omega_1) \cdot (\der{D}_3, \alpha_3\circ \omega^{-1}_2\circ \alpha_2, \omega_3)\\&=(\der{D}_1, \alpha_1, \omega_1) \cdot ((\der{D}_2, \alpha_2, \omega_2) \cdot (\der{D}_3, \alpha_3, \omega_3)  )
\end{align*}


\smallskip \noindent $\der{D}_1$ and $\der{D}_3$ are non-empty while $\der{D}_2$ is. As above we have 

\begin{align*}
	((\der{D}_1, \alpha_1, \omega_1)\cdot (\der{D}_2, \alpha_2, \omega_2)) \cdot (\der{D}_3, \alpha_3, \omega_3)&=(\der{D}_1, \alpha_1, \omega_1\circ \alpha^{-1}_2\circ \omega_2)\cdot (\der{D}_3, \alpha_3, \omega_3)\\&=(\der{D}_1*(\omega^{-1}_2\circ \alpha_2\circ \omega^{-1}_1)\cdot \alpha_3*\der{D}_3, \alpha_1, \omega_3 )\\(\der{D}_1, \alpha_1, \omega_1) \cdot ((\der{D}_2, \alpha_2, \omega_2) \cdot (\der{D}_3, \alpha_3, \omega_3)  )&=(\der{D}_1, \alpha_1, \omega_1) \cdot (\der{D}_3, \alpha_3\circ \omega^{-1}_2\circ \alpha_2, \omega_3)\\&=(\der{D}_1*\omega^{-1}_1\cdot (\alpha_3\circ \omega^{-1}_2 \circ \alpha_2)*\der{D}_3, \alpha_1, \omega_3)
\end{align*}

Let $n$ be the length of $\der{D}_1$, so that $\pi(G_{1,n})=\pi(G_{3,0})$. We can notice that the following two diagrams commute.
	\[\xymatrix@C=55pt{G_{1, n-1}&D_{1,n-1} \ar[r]^{g_{1,n}} \ar@{>->}[l]_{f_{1,n}}&G_{1,n}  \ar[r]^{\omega^{-1}_1}& \pi(G_{1,n})\\  L_{1,n-1} \ar[u]_{m_{1,n-1}} \ar[d]^{m_{1, n-1}}& K_{1, n-1} \ar[u]^{k_{1,n-1}} \ar[d]_{k_{1,n-1}} \ar[rr]^{r_{1, n-1}} \ar@{>->}[l]_{l_{1,n-1}} &&R_{1, n-1}\ar[u]^{\omega^{-1}_1\circ h_{1,n-1}} \ar[d]_{\omega^{-1}_2\circ \alpha_2\circ \omega^{-1}_1\circ h_{1, n-1}}\\G_{1, n-1} \ar@/^.80cm/[uu]^{\id{G_{1, n-1}}}&D_{1, n-1}\ar@{>->}[l]^{f_{1, n-1}}\ar@/_.80cm/[uu]_(.35){\id{D_{1, n-1}}}|\hole \ar[r]_{g_{1, n-1}}&G_{1, n}\ar[r]_-{\omega^{-1}_2\circ \alpha_2\circ \omega^{-1}_1}& \pi(G_{1, n})\ar@/_.80cm/[uu]_{\alpha^{-1}_2\circ \omega_2}}\]
		\[\xymatrix@C=55pt{\pi(G_{3, 0}) & G_{3,0}  \ar[l]_-{\alpha_3\circ \omega^{-1}_2\circ \alpha_2}&D_{3,0} \ar[r]^{g_{3,0}} \ar@{>->}[l]_{f_{3,0}}&G_{3,1}  \\  L_{3,0} \ar[u]_{\alpha_3\circ \omega^{-1}_2\circ \alpha_2\circ m_{3,0}} \ar[d]^{\alpha_3\circ  m_{3, 0}}& & K_{3, 0} \ar[u]^{k_{3,0}} \ar[d]_{k_{3,0}} \ar[r]^{r_{3, 0}} \ar@{>->}[ll]_{l_{3,0}} &R_{3, 0}\ar[u]^{ h_{3,0}} \ar[d]_{ h_{3, 0}}\\\pi(G_{3, 0})  \ar@/^.80cm/[uu]^{\alpha^{-1}_2\circ \omega_2}&G_{3,0} \ar[l]^{\alpha_3}&D_{3, 0}\ar@{>->}[l]^{f_{3, 0}}\ar@/_.80cm/[uu]_(.35){\id{D_{3, 0}}}|\hole \ar[r]_{g_{3, 0}}&G_{3, 1}\ar@/_.80cm/[uu]_{\id{G_{3,1}}}}\]
	
Thus the family $\{\phi_{X}\}_{X\in \Deltamin(\der{D}_1*(\omega^{-1}_2\circ \alpha_2\circ \omega^{-1}_1)\cdot \alpha_3*\der{D}_3)}$ defined as
\[\phi_X:=\begin{cases}
	\id{X} & X\neq \pi(G_{1,n})\\
	\alpha^{-1}_2 \circ \omega_2 & X= \pi(G_{1,n})
\end{cases}\]
is a consistent abstraction equivalence witnessing our thesis.



\smallskip \noindent  $\der{D}_1$, $\der{D}_2$ and $\der{D}_3$ are all non-empty. For this case we rely to the fact that concatenation of derivations, being simply concatenation of paths in a graph, is associative. Indeed we have
\begin{align*}
	((\der{D}_1, \alpha_1, \omega_1)\cdot (\der{D}_2, \alpha_2, \omega_2)) \cdot (\der{D}_3, \alpha_3, \omega_3)&=(\der{D}_1*\omega^{-1}_1\cdot \alpha_2*\der{D}_2, \alpha_1, \omega_2)\cdot (\der{D}_3, \alpha_3, \omega_3)\\&=((\der{D}_1*\omega^{-1}_1\cdot \alpha_2*\der{D}_2)*\omega^{-1}_2\cdot \alpha_3*\der{D}_3, \alpha_1, \omega_3 )\\&=(\der{D}_1*\omega^{-1}_1\cdot (\alpha_2*\der{D}_2*\omega^{-1}_2)\cdot \alpha_3*\der{D}_3, \alpha_1, \omega_3 )\\&=(\der{D}_1*\omega^{-1}_1\cdot \alpha_2*(\der{D}_2*\omega^{-1}_2 \cdot \alpha_3*\der{D}_3), \alpha_1, \omega_3)\\&=(\der{D}_1, \alpha_1, \omega_1) \cdot (\der{D}_2*\omega^{-1}_2 \cdot \alpha_3*\der{D}_3, \alpha_2, \omega_3)\\&=
	(\der{D}_1, \alpha_1, \omega_1) \cdot ((\der{D}_2, \alpha_2, \omega_2) \cdot (\der{D}_3, \alpha_3, \omega_3))
\end{align*}
And the thesis now follows.
\end{proof}

Given \Cref{rem:id}  and \Cref{lem:ass}, we are now ready to give the following definition.

\begin{definition}
	Let $(\X, \R)$ be a left-linear DPO-rewrityng system, with $\X$ an $\mathcal{M}$-adhesive category. The  category $\dpi$ is defined as follows:
	\begin{itemize}
		\item objects are isomorphism classes of objects of $\X$;
		\item an arrow $[G]\to [H]$ is an equivalence class $[\der{D}, \alpha, \omega]_a$ of a decorated derivation between $G'$ and $H'$ for some $G'$ and $H'$ such that $\pi(G')=G$ and $\pi(H')=H$;
		\item composition is concatenation of abstract decorated derivations;
		\item the identity on $[G]$ is $[\{G\}, \alpha, \alpha]_a$, where $\alpha$ is any isomorphism $\pi(G)\to G$.	\end{itemize}
\end{definition}


In the next proposition we fully characterize the isomorphism in $\dpi$.

\begin{proposition} Let $(\X, \R)$ be a left-linear DPO rewriting system, then the isomorphisms in $\dpi$ are exactly the classes of empty decorated derivations.
\end{proposition}
\begin{proof}
	Let $[\der{D}, \alpha, \omega]_a$ be an isomorphism, thus there exists $(\der{E}, \beta, \gamma)$ such that $(\der{D}, \alpha, \omega)\cdot (\der{E}, \beta, \gamma)\equiv^a(\{G_{\der{D}, 0}\}, \alpha, \alpha)$. Since the latter has length $0$, \Cref{rem:lgt} entails that $\der{D}$ is empty too.
	
	Vice versa, if $(\{G\}, \alpha, \omega)$ is a derivation of length $0$, then 
	\begin{align*}
	(\{G\}, \alpha, \omega) \cdot (\{G\}, \omega, \alpha)&=(\{G\}, \omega\circ \omega^{-1} \circ \alpha, \alpha )\\&=(\{G\}, \alpha, \alpha)\\
	(\{G\}, \omega, \alpha) \cdot (\{G\}, \alpha, \omega)&=(\{G\}, \alpha\circ \alpha^{-1} \circ \omega, \omega )\\&=(\{G\}, \omega, \omega)
	\end{align*}
	showing that  $[\{G\}, \alpha, \omega]_a$ is an isomorphism.
\end{proof}

\subsection{Colimits of derivations}\label{subsec:col}
Given a DPO-rewriting system $(\X, \R)$,  we have already noted in \Cref{rem:func} that a derivation $\der{D}$  determines a diagram $\Delta(\der{D})$ in $\X$. We can then wonder if such a diagram has a colimit. Clearly if $\der{D}$ is the empty derivation $G$ then a colimit for $\Delta(\der{D})$ is simply the object $G$. More generally, we have the following result.

\begin{lemma}\label{lem:colim}
	Let $\X$ be an $\mathcal{M}$-adhesive category and $(\X, \R)$ a left-linear DPO-rewriting system over it. The following properties hold true:
	\begin{enumerate}
		\item  if $\der{D}$ is a derivation from $G$ to $H$, then the diagram $\Delta(\der{D})$ has a colimit $(\tpro{D}, \{\iota_X\}_{X\in \Delta(\der{D})})$ such that $\iota_H$ belongs to $\mathcal{M}$.
		\item if $\der{D}$ is the concatenation $\der{D}_1\cdot \der{D}_2$ of two derivations $\der{D}_1=\{\dder{D}_{1,i}\}_{i=0}^{n_1}$ between $G$ and $H$ and $\der{D}_2=\{\dder{D}_{2,j}\}_{j=0}^{n_2}$ between $H$ and $T$,  then the colimiting cocone $(\tpro{D}, \{\iota_X\}_{X\in \Delta(\der{D})})$ exists too and there is a pushout square
		\[\xymatrix{H\ar[r]^-{\iota_{2, H}} \ar@{>->}[d]_-{\iota_{1, H}} & \tproi{D}{2} \ar[d]^{p_2}\\  \tproi{D}{1} \ar[r]_{p_1}& \tpro{D}}\]
		where $(\tproi{D}{1}, \{\iota_{1, X}\}_{X\in \Delta(\der{D}_1)})$ and $(\tproi{D}{2}, \{\iota_{2, X}\}_{X\in \Delta(\der{D}_2)})$ are the colimiting cocone for $\Delta(\der{D}_1)$ and $\Delta(\der{D}_2)$, respectively.
	\end{enumerate}
\end{lemma}
\begin{remark}\label{rem:cof}
Let $I:\Deltamin(\dder{D})\to \Delta(\dder{D})$ be the inclusion functor. It is immediate to see that such functor is \emph{final} \cite{mac2013categories}. This means that for every functor $F\colon \Delta(\dder{D})\to \Y$ we have:
\begin{enumerate}
	\item if  $(C, \{c_X\}_{X\in \Deltamin(\dder{D})})$ is colimiting for $F\circ I$, then there exists a colimiting cocone $(D, \{d_X\}_{X\in \Delta(\dder{D})})$ for $F$;
	\item $(C, \{c_X\}_{X\in \Deltamin(\dder{D})})$ and $(D, \{d_X\}_{X\in \Delta(\dder{D})})$ are colimiting for, respectively, $F\circ I$ and $F$, then the canonical arrow $\phi\colon C\to D$ induced by $(D, \{d_X\}_{X\in \Deltamin(\dder{D})})$ is an isomorphism.
\end{enumerate}
\end{remark}

\begin{proof}\begin{enumerate}
		\item Let us proceed by induction on the length of $\der{D}$.

	
	\smallskip \noindent $\lgh(\dder{D})=0$. then the $\tpro{\dder{D}}$ is simply $(G, \{\id{G}\})$ and $\id{G}\in \mathcal{M}$.
	
	\smallskip \noindent$\lgh(\dder{D})=1$. Suppose that $\dder{D}$ has as its single component the derivation
			\[\xymatrix{L \ar[d]_{m}& K \ar[d]^{k}\ar@{>->}[l]_{l} \ar[r]^{r} & R \ar[d]^{h} \\G& \ar@{>->}[l]^{f} D \ar[r]_{g}& H  \\}\]
			The arrow $f$ is  the pushout of $l$ and so it is in in $\mathcal{M}$. We can thus consider the following $\mathcal{M}$-pushout square.
			\[\xymatrix{D \ar@{>->}[d]_{f} \ar[r]^{g} & H \ar@{>->}[d]^{p} \\G \ar[r]_{q}& P }\]
			Since $p\in \mathcal{M}$, the thesis follows immediately from \Cref{rem:cof}. 
	
	\smallskip \noindent$\lgh(\dder{D})\geq 2$. Let $\der{D}$ be $\{\dder{D}_i\}_{i=0}^n$ with $n\geq 1$. Let also $\der{D}'$ be $\{\dder{D}_i\}^{n-1}_{i=0}$ and $\rho_n=(l_n, r_n)$ be the rule applied in $\dder{D}_n$. The arrow $f_n\colon D_n\to G_n$, being a pushout of $l_n$ is an element $\mathcal{M}$. By inductive hypothesis, $\iota_{G_{n}}\colon G_{n}\to \lpro \der{D}'\rpro$ is in $\mathcal{M}$ too, thus, we can consider the diagram below, having a pushout as its lower half.
			\[\xymatrix{L_n \ar[d]_{m_{n}}& K_{n} \ar[d]^{k_{n}}\ar@{>->}[l]_{l_{n}} \ar[r]^{r_{n}} & R_{n} \ar[d]^{h_n} \\G_{n} \ar@{>->}[d]_{\iota'_{G_{n}}}& \ar@{>->}[l]^{f_n} D_n \ar[r]_{g_n}& H  \ar@{>->}[d]^{q}\\ \lpro \der{D}' \ar[rr]_{p}\rpro && P}\] 
			Notice that, as in the point above, the arrow $q\colon H\to P$ is the pushout of an element in $\mathcal{M}$, therefore it is enough to show that the diagram so constructed provides a colimiting cocone for $\Delta(\der{D})$.
			
			Let $(C, \{c_X\}_{X\in \Delta(\der{D})})$ be a cocone. Since $\Delta(\der{D}')$ is a subdiagram of $\Delta(\der{D})$, we get another cocone $(C, \{c_X\}_{X\in \Delta(\der{D}')})$ which induces an arrow $c'\colon \lpro \der{D}' \rpro \to C$ such that
			\begin{align*}
				c'\circ \iota_{G_n} \circ f_n &=c_{G_n} \circ f_n\\&= c_{D_n}\\&= c_{H}\circ g_n
			\end{align*}
			Therefore the arrows $c'$ and $c_H$ induce a morphism $c\colon P\to C$ and the thesis now follows at once.
		
		\item  As a first step, notice that $(\tpro{D}, \{\iota_X\}_{X\in \Delta(\der{D}_1)})$ and $(\tpro{D}, \{\iota_X\}_{X\in \Delta(\der{D}_2)})$ are cocone on, respectively, $\Delta(\der{D}_1)$ and $\Delta(\der{D}_2)$. Hence, there exist two arrows $p_1\colon \tproi{D}{1}\to \tpro{D}$, $p_2\colon \tproi{D}{2}\to \tpro{D}$ such that, for every $X\in  \Delta(\der{D}_1)$ and $Y\in  \Delta(\der{D}_2)$
		\[p_1\circ \iota_{1, X} = \iota_X \qquad p_2\circ \iota_{2, Y}=\iota_{2,Y}\]
		In particular, this entails the commutativity of the square
				\[\xymatrix{H \ar[dr]^{\iota_H} \ar[r]^-{\iota_{2, H}} \ar@{>->}[d]_-{\iota_{1, H}} & \tproi{D}{2} \ar[d]^{p_2}\\  \tproi{D}{1} \ar[r]_{p_1}& \tpro{D}}\]
		
	Let us now show that the square above is a pushout. Take two arrows $a\colon \tproi{D}{1}\to C$, $b\colon \tproi{D}{2}\to C$ such that $a\circ \iota_{1, H}=b\circ \iota_{2, H}$. We can use the previous equality to define a cocone $(C, \{c_X\}_{X\in \Delta(\der{D})})$ putting:
	\[c_X:=\begin{cases}
		a\circ \iota_{1, X} & X\in \Delta(\der{D}_1)\\
		b\circ \iota_{2, X} & X\in \Delta(\der{D}_2)
	\end{cases}\]
From this, we can deduce at once the existence of a unique $c\colon \tpro{D}\to C$ such that $c\circ \iota_X = c_X$. By construction, for every $X\in  \Delta(\der{D}_1)$ and $Y\in  \Delta(\der{D}_2)$ we have
\[\begin{split}
	c\circ p_1 \circ \iota_{1,X}&=c\circ \iota_{X}\\&=c_X \\&=a\circ \iota_{1,X}\\&=a\circ p_1\circ \iota_{1,X}
\end{split}\qquad \begin{split}
	c\circ p_2 \circ \iota_{2,Y}&=c\circ \iota_{Y}\\&=c_Y \\&=b\circ \iota_{2,Y}\\&=b\circ p_2\circ \iota_{2,Y}
\end{split}\]
Therefore we get that $c\circ p_1=a$ and $c\circ p_2 = b$.

	For uniqueness, suppose that $c'\colon \tpro{D}\to C$ is such that $c'\circ p_1=a$ and  $c'\circ p_2 = b$. Then, for every $X\in \Delta(\der{D})$ we have
\begin{align*}
	c'\circ \iota_X &= \begin{cases}
	c'\circ p_1\circ \iota_{1, X} & X\in \Delta(\der{D}_1)\\
	c'\circ p_2\circ \iota_{2, X} & X\in \Delta(\der{D}_2)
	\end{cases}\\&=\begin{cases}
a\circ \iota_{1, X} & X\in \Delta(\der{D}_1)\\
b\circ \iota_{2, X} & X\in \Delta(\der{D}_2)
	\end{cases}\\&=c_X\\&=c\circ \iota_X
\end{align*}
showing that $c'=c$ as wanted.	 \qedhere 
	\end{enumerate}
\end{proof}

\begin{corollary}\label{cor:colim}
Let $\der{D}=\{\dder{D}_i\}_{i=0}^n$ a derivation of length $n+1$ and fix an index $j\in[0,n]$. Define
\[\der{D}^j_1:=\{\dder{D}_i\}_{i=0}^{j-1} \qquad  \der{D}^j_2=\{\dder{D}_j\} \qquad \der{D}^j_3:=\{\dder{D}_i\}_{i=j+1}^n\]
with the convention that $\der{D}^0_1$ and $\der{D}^n_3$ are the empty derivation on, respectively, $G_0$ and $G_n$. Then the square below is a pushout and a pullback
\[\xymatrix@C=30pt{D_{j} \ar[ddrr]^{\iota_{D_j}} \ar[r]^{g_j}\ar@{>->}[d]_{f_j}& G_{j+1} \ar[ddr]^{\iota_{G_{j+1}}} \ar[r]^-{\iota_{3, G_{j+1}}} & \lpro \der{D}^j_3\rpro \ar[dd]^{p_2} \\ G_j\ar[drr]_{\iota_{G_{j}}} \ar@{>->}[d]_{\iota_{1,G_j}}\\ \lpro \der{D}^j_1 \rpro \ar[rr]_{p_1}  &&\tpro{D} }\] 
Where the two arrows $p_1\colon \lpro\der{D}^j_1 \rpro\to \tpro{D}$, $p_2\colon \lpro \der{D}^j_3\rpro \to \tpro{D}$ are induced, respectively, by the cocones $(\tpro{D}, \{\iota_{X}\}_{X\in \Delta(\der{D}^j_1)})$ and $(\tpro{D}, \{\iota_{X}\}_{X\in \Delta(\der{D}^j_3)})$.
\end{corollary}
\begin{remark}
If $\der{D}$ is empty then $\der{D}^j_1, \der{D}^j_2$ and $\der{D}^j_3$ are empty too.
\end{remark}
\begin{proof}
We can notice that $\der{D}=\der{D}^j_1\cdot \der{D}^j_2 \cdot \der{D}^j_3$. By the first and the second point of \Cref{lem:colim} then we get the following diagram, in which all squares are $\mathcal{M}$-pushouts.

\[\xymatrix@C=30pt{D_{j}  \ar[r]^{g_j}\ar@{>->}[d]_{f_j}& G_{j+1}  \ar[r]^-{\iota_{3,G_{j+1}}} \ar[d]_{\iota_{2, G_{j+1}}}  \ar@/^.5cm/[dd]^{\iota_{1,2, G_{j+1}}}& \lpro \der{D}^j_3\rpro \ar[dd]^{p_2} \\ G_j \ar[r]^{\iota_{2, G_j}}\ar@{>->}[d]_{\iota_{1,G_j}} & \lpro \der{D}^j_2\rpro \ar[d]_a\\ \lpro \der{D}^j_1 \rpro \ar@/_.4cm/[rr]_{p_1} \ar[r]^b &\lpro \der{D}^j_1\cdot \der{D}^j_2 \rpro \ar[r]^c&\tpro{D} }\] 
Applying \Cref{lem:po1} twice we get that the whole square is an $\mathcal{M}$-pushout. Then the thesis follows from \Cref{prop:pbpoad}.
\end{proof}

\begin{remark}\label{rem:zero1} In particular, considering $j=0$ or $j=n$, we get that the following two squares are $\mathcal{M}$-pushouts and, thus, pullbacks.
	\[\xymatrix{D_0 \ar[r]^-{\iota_{3, D_0}} \ar@{>->}[d]_{f_0}& \lpro \der{D}^0_3 \rpro \ar@{>->}[d]^{p_2} &D_n \ar@{>->}[d]_{\iota_{1, D_n}}\ar[r]^{g_n}& G_n \ar@{>->}[d]^{\iota_{G_n}}\\ G \ar[r]_{\iota_G}& \tpro{D} & \lpro \der{D}^n_1 \rpro \ar[r]_{p_1}  & \tpro{D}}\]
\end{remark}

\begin{corollary}\label{cor:ele}
	Let $\der{D}=\{\dder{D}_{i}\}_{i=0}^n$ be a derivation between $G$ and $H$. Let $j$ and $k$ be two indexes less or equal than $n+1$ and suppose that $j< k$.  Consider two arrows $a\colon T\to G_j$, $b\colon T\to G_k$. If $\iota_{G_j}\circ a = \iota_{G_k}\circ b$, 
	then  there exist a unique arrow $c\colon T\to D_j $  such that \[f_j\circ c = a\qquad \iota_{D_j}\circ c =\iota_{G_k}\circ b\]
	\end{corollary}
\begin{proof} Consider the diagram
			\[\xymatrix@C=34pt{T  \ar@/_.6cm/[dd]_{a}\ar@{.>}[d]^{c}\ar[rr]^{b}&& G_k \ar[d]_{\iota_{3, G_k}} \ar@/^.6cm/[ddd]^{\iota_{G_k}}\\D_{j} \ar[r]^{g_j} \ar[ddrr]^{\iota_{D_j}}\ar@{>->}[d]^{f_j}& G_{j+1} \ar[r]^-{\iota_{3,G_{j+1}}} \ar[ddr]^{\iota_{G_{j+1}}} & \lpro \der{D}^j_3\rpro \ar[dd]_(.6){p_2} \\ G_j \ar@{>->}[d]_{\iota_{1,G_j}} \ar[drr]_{\iota_{G_j}}\\ \lpro \der{D}^j_1 \rpro \ar[rr]_{p_1}  &&\tpro{D} }\] 
			Thanks to \Cref{cor:colim} we know that the bottom right rectangle in the diagram above is a pullback and the thesis follows at once.
\end{proof}	

We want now to relate the colimit of a composite decorated derivation with the colimits of its components. The following two lemmas will be pivotal to develop the subsequent sections.

\begin{lemma}\label{rem:dett}
	Let $(\der{D}, \alpha, \omega)$ be the composite $(\der{D}_1, \alpha_1, \omega_1)\cdot (\der{D}_2, \alpha_2, \omega_2)$ of two derivations of length $l_1$ and $l_2$. Suppose that:
	\[r(\der{D}_1)=\{\rho_{1,i}\}_{i=0}^{l_1-1} \quad r(\der{D}_2)=\{\rho_{2,i}\}_{i=0}^{l_2-1} \quad  r(\der{D})=\{\rho_{i}\}_{i=0}^{l_1+l_2-1}\]
	and let  $G_{1,0}, G_{2,0}$ and $G_{0}$ be the sources of $\der{D}_1$, $\der{D}_2$,  and $\der{D}$. Similary, denote, by $G_{1,i+1}, G_{2,i+1}$ and $G_{i+1}$ the results of the application of, respectively, $\rho_{1,i}, \rho_{2,i}, \rho_i$ to $G_{1,i}, G_{2,i}$ and $G_{i}$  in the corresponding derivation. Then there are arrows $q_1\colon \lpro \der{D}_1\rpro \to \tpro{D}$, $q_2\colon \lpro \der{D}_2\rpro \to \tpro{D}$ such that
	\[ q_1\circ \iota_{G_{1,0}}\circ \alpha_1 = \iota_{G_{0}}\circ \alpha \quad  q_2\circ \iota_{G_{2,l_2}}\circ \omega_2 = \iota_{G_{l_1+l_2}}\circ \omega \quad 	q_2\circ \iota_{2, G_{2,0}} \circ \alpha_2=q_1\circ \iota_{1, G_{1, l_1}}\circ \omega_1\]
	and, for every $i\in [0, l_1-1]$, $j\in [l_1+1, l_1+l_2]$
	\[q_1\circ \iota_{1, G_{1,i}}=\iota_{G_{i}}\qquad q_2\circ \iota_{2, G_{2,j}}=\iota_{G_{j+l_1}} \]
\end{lemma}
\begin{proof}According to \Cref{def:conc}, we have three cases.
	\begin{itemize}
		\item $\lgh(\der{D}_1)=0$. Then  $(\der{D}, \alpha, \omega)$ is $(\der{D}_2, \alpha_2\circ \omega_1^{-1}\circ \alpha_1, \omega_2)$ thus we can take as $q_2$ the identity on $\lpro \der{D}_2\rpro$. Moreover, in this case $\iota_{1,G_{1,0}}$ is an isomorphism, so that we can take as $q_1$ the composition $\iota_{G_0}\circ \alpha_2\circ \omega^{-1}_1 \circ \iota^{-1}_{1,G_{1,0}}$.
		\item $\lgh(\der{D}_1)\neq 0$ and $\lgh(\der{D}_2)=0$ . Hence  $(\der{D}, \alpha, \omega)$ is $(\der{D}_1, \alpha_1, \omega_1 \circ \alpha^{-1}_2\circ \omega_2)$. We can then define $q_1$ as $\id{\lpro \der{D}_1\rpro}$.  Since $\iota_{2,G_{2,0}}$ is an isomorphism, $q_2$ can be taken as  $\iota_{G_{l_1}}\circ  \omega_1  \circ \alpha^{-1}_2 \circ \iota^{-1}_{2,G_{2,0}}$.
		\item  $\lgh(\der{D}_1)\neq 0$ and $\lgh(\der{D}_2)\neq 0$. In this case we have that $(\der{D}, \alpha, \omega)$  is $(\der{D}_1*\omega_1^{-1}\cdot \alpha_2*\der{D}_2, \alpha_1, \omega_2)$. By second point of \Cref{lem:colim} and by \Cref{rem:abscons2} we can build the following diagram.
		\[\xymatrix@C=40pt{&&G_{2,0}  \ar@/_.3cm/[dl]_{\alpha^{-1}_2} \ar@/^.3cm/[dr]^{\iota_{2, G_{2,0}}}\\&\pi(G_{2,0}) \ar[dr]^{\iota_{G_{l_1}}}\ar[r]^-{\iota'_{2, \pi(G_{2,0})}} \ar@{>->}[d]_-{\iota'_{1, \pi(G_{1,l_1})}} & \lpro \alpha_2*\der{D}_2 \rpro \ar@{>->}[d]^{p_2} &\tproi{D}{2} \ar[l]_-{\Gamma^{\alpha_2}} \ar@{>.>}@/^.2cm/[dl]^{q_2}\\ G_{1,l_1}  \ar@/^.3cm/[ur]^{\omega^{-1}_1}  \ar@{>->}@/_.3cm/[dr]_{\iota_{1, G_{1, l_1}}}& \lpro \der{D}_1*\omega_1^{-1} \rpro  \ar[r]_{p_1}& \tpro{D} \\ &\tproi{D}{1} \ar[u]^{\Gamma_{\omega^{-1}_1}}  \ar@{.>}@/_.2cm/[ur]_{q_1}}\]
		
		Let us define $q_1$ as $p_1\circ \Gamma_{\omega^{-1}_1}$ and $q_2$ as $p_2\circ \Gamma^{\alpha_2}$. Then, for every $i\in [0,l_1-1]$ and $j\in [l_1+1, l_1+l_2]$
		\[\begin{split}
			q_1\circ \iota_{1, G_{1,i}}&=p_1\circ \Gamma_{\omega^{-1}_1}\circ \iota_{1, G_{1,i}}\\&=p_1\circ \iota'_{1, G_{1,i}} \\&=\iota_{G_{i}}
		\end{split} \qquad \begin{split}
			q_2\circ \iota_{2, G_{2,j}}&=p_2\circ \Gamma^{\alpha_2}\circ \iota_{2, G_{2,j}}\\&=p_2\circ \iota'_{2, G_{2,j}} \\&=\iota_{G_{j+l_1}}
		\end{split}\]
		Moreover, we also have:
		\[
		\begin{split}
			q_1\circ \iota_{G_{1,0}}\circ \alpha_1&= p_1\circ \Gamma_{\omega^{-1}_1}\circ \iota_{G_{1,0}}\circ \alpha_1 \\&=p_1\circ \iota'_{1, G_{1,0}} \circ \alpha_1 \\&= \iota_{G_{0}}\circ \alpha
		\end{split}\qquad 
		\begin{split}
			q_2\circ \iota_{G_{2,l_2}}\circ \omega_2&= p_2\circ \Gamma^{\alpha_2}\circ \iota_{G_{2,l_2}}\circ \omega_2 \\&=p_2\circ \iota'_{1, G_{2,l_2}} \circ \omega_2 \\&= \iota_{G_{l_1+l_2}}\circ \omega
		\end{split} \] 
		
		This concludes the proof. \qedhere
	\end{itemize}	
\end{proof}


\begin{remark}\label{rem:dett2}
	It is worth noticing that, for every $i\in [0,l_1-1]$ and $j\in [l_1+1, l_1+l_2]$ the previous equalities also entail that,
	\[\begin{split}q_1\circ \iota_{1, L_{1,i}}&=\iota_{L_{i}}\\q_2\circ \iota_{2, L_{2,j}}&=\iota_{L_{j+l_1}} 	
	\end{split} \quad 
	\begin{split}    	q_1\circ \iota_{1, K_{1,i}}&=\iota_{K_{i}} \\q_2\circ \iota_{2, K_{2,j}}&=\iota_{K_{j+l_1}}\\ \end{split}\quad 
	\begin{split}    	q_1\circ \iota_{1, R_{1,i-1}}&=\iota_{R_{i-1}} \\q_2\circ \iota_{2, R_{2,j-1}}&=\iota_{R_{j+l_1-1}}\end{split}\quad 
	\begin{split}    	q_1\circ \iota_{1, D_{1,i}}&=\iota_{D_{i}} \\q_2\circ \iota_{2, D_{2,j}}&=\iota_{D_{j+l_1}}\end{split}\]
	
	Moreover, if $l_2\neq 0$, we also have that
	\begin{align*} 
		q_2\circ \iota_{2, D_{2, 0}}&=q_2\circ \iota_{2, G_{2, 1}}\circ g_{2, 0}\\&=\iota_{G_{l_1+1}}\circ g_{2,0}\\&=\iota_{G_{l_1+1}}\circ g_{l_1}\\&=\iota_{D_{l_1}}
	\end{align*}
	which, in turn, also entails that
	\begin{align*}
		q_2\circ \iota_{2, K_{2, 0}}&=q_2\circ \iota_{D_{2, 0}}\circ k_{2, 0}\\&=\iota_{D_{l_1}}\circ k_{l_1}\\&=\iota_{K_{l_1}}
	\end{align*}
	
	Finally, whenever $l_1$ and $l_2$ are different from $0$ we have other two equalities.
	\[q_1\circ \iota_{1, G_{1, l_1}}=\iota_{ G_{l_1}}\circ \omega^{-1}_1 \qquad q_2\circ \iota_{2, G_{2, 0}}=\iota_{ G_{l_1}}\circ \alpha^{-1}_2\]
	
\end{remark}


\section{Semi-consistent and consistent permutations}
	
Equipped with the theory developed so far, we can introduce the notions of \emph{semi-consistent} and \emph{consistent} permutation.

\begin{definition}[(Semi)-consistent permutation]\label{def:permcon}
	Let $\X$ be an $\mathcal{M}$-adhesive category and consider a left-linear DPO-rewriting system $(\X, \R)$ on it.  Consider two decorated derivations $(\der{D}, \alpha, \omega)$ and  $(\der{D}', \alpha', \omega')$ with the same length $l$ and with isomorphic sources and targets. Let also $r(\der{D})=\{\rho_i\}_{i=0}^{l-1}$ and $r(\der{D}')=\{\rho'_i\}_{i=0}^{l-1}$	be sequences of rules associated to the two derivations.
	
	A \emph{semi-consistent permutation} between  $(\der{D}, \alpha, \omega)$ and $(\der{D}', \alpha', \omega')$ is a permutation $\sigma\colon [0,l-1]\to [0,l-1]$  such that, for every $i\in [0,l-1]$, $\rho_i=\rho'_{\sigma(i)}$ and, moreover, there exists a \emph{mediating isomorphism} $\xi_\sigma\colon \tpro{D} \to \lpro \der{D}' \rpro$ fitting in the following diagrams, where $m_i, m'_i, h_i$ and $h'_i$ are, respectively, the matches and back-matches of $\dder{D}_i$ and $\dder{D}'_i$.
	\[\xymatrix@C=30pt{\pi(G_0)\ar[r]^{\alpha} \ar[d]_{\alpha'} & G_0 \ar[r]^{\iota_{G_0}} &\tpro{D} \ar[d]^{\xi_\sigma}\\ G'_0 \ar[rr]_{\iota'_{G'_0}} & &\lpro \der{D}' \rpro}\]
	\[\xymatrix@C=30pt{L_i \ar[r]^{m_i} \ar[d]_{m'_{\sigma(i)}}& G_i \ar[r]^{\iota_{G_i}} &\tpro{D} \ar[d]^{\xi_\sigma} & R_i \ar[r]^{h_i} \ar[d]_{h'_{\sigma(i)}}& G_{i+1} \ar[r]^{\iota_{G_{i+1}}} &\tpro{D} \ar[d]^{\xi_\sigma} \\G'_{\sigma(i)} \ar[rr]_{\iota'_{G'_{\sigma(i)}}}&& \lpro \der{D}' \rpro& G'_{\sigma(i)+1} \ar[rr]_{\iota'_{G'_{\sigma(i)+1}}}&& \lpro \der{D}' \rpro}\]
	
	A semi-consistent permutation is \emph{consistent} if the following square commutes too.
	\[\xymatrix@C=30pt{ \pi(G_{l}) \ar[r]^{\omega} \ar[d]_{\omega'} & G_{l} \ar@{>->}[r]^{\iota_{G_{l}}} &\tpro{D} \ar[d]^{\xi_\sigma} \\  G'_{l} \ar@{>->}[rr]_{\iota'_{G'_{l}}} & &\lpro \der{D}' \rpro}\]
\end{definition}

\begin{remark}\label{rem:inversa}
	Let  $(\der{D},\alpha, \omega)$ and $(\der{D}', \alpha', \omega')$ be two decorated derivations and $\sigma\colon [0,n]\to [0,n]$ a consistent permutation between them.  Then its inverse $\sigma^{-1}$ is a consistent permutation between $(\der{D}', \alpha', \omega')$ and $(\der{D},\alpha, \omega)$. Indeed, it is enough to consider, as mediating isomorphism, the inverse $\xi^{-1}_\sigma$ of $\xi_\sigma$.
\end{remark}

\begin{example}\label{rem:empty2}
	Let $(\{G\}, \alpha, \omega)$ and $(\{G'\}, \alpha', \omega')$ be two derivations of length $0$. Then $\id{\emptyset}$ is a consistent permutations between them if and only if there exists an isomorphism $\phi\colon G_0\to G'_0$ fitting in the diagrams below
	\[\xymatrix@C=15pt{&\pi(G) \ar[dr]^{\alpha'} \ar[dl]_{\alpha}&&& \pi(G) \ar[dr]^{\omega'} \ar[dl]_{\omega}\\ G\ar[rr]_{\phi} && G' &G \ar[rr]_{\phi} && G' } \]
	That is, by \Cref{rem:empty} if and only if they are abstract equivalent decorated derivations.
\end{example}


\begin{remark} \label{rem:coproj} Notice that if $\sigma$ is a semi-consistent permutation between two derivations  $\der{D}$ and $\der{D}'$ of length $l$, then for every  $i\in [0,l-1]$ the diagram below commutes.
	\[\xymatrix@C=40pt{G_i\ar@/^1cm/[rrr]^{\iota_{G_{i}}}&D_i\ar@/^.4cm/[rr]^(.35){\iota_{D_i}} \ar[r]_{g_i} \ar@{>->}[l]^{f_i}&G_{i+1} \ar[r]_{\iota_{G_{i+1}}}&\tpro{D} \ar[dd]^{\xi_\sigma}\\  L_i \ar[u]^{n_i} \ar[d]_{n'_{\sigma(i)}}& K_i \ar[d]^{k'_{\sigma(i)}} \ar[u]_{k_i} \ar[r]^{r_i} \ar@{>->}[l]_{l_i} &R\ar[u]^{h_i} \ar[d]_{h'_{\sigma(i)}}\\G'_{\sigma(i)}\ar@/_1cm/[rrr]_{\iota'_{G'_{\sigma(i)}}} &D'_{\sigma(i)}\ar@{>->}[l]_{f'_{\sigma(i)}} \ar[r]^{g'_i} \ar@/_.4cm/[rr]_(.35){\iota'_{D'_{\sigma(i)}}}&G'_{\sigma(i)+1} \ar[r]^{\iota'_{G'_{\sigma(i)+1}}}& \lpro\der{D}' \rpro }\]
	
	This follows at once from the following chain of identities:
	\begin{align*}
		\xi_\sigma \circ \iota_{D_i}\circ k_i&=\xi_\sigma \circ \iota_{G_{i}} \circ f_i\\&= \xi_\sigma \circ \iota_{G_{i}}\circ m_i \circ l_i\\&=\iota_{G'_{\sigma(i)}}\circ m'_i\circ l_i\\&=\iota_{G'_{\sigma(i)}}\circ f'_{\sigma(i)}\circ k'_i\\&=\iota_{D'_\sigma(i)}\circ k_i'
	\end{align*}
	
	In particular, the previous diagram entails the following identities.
	\[\xi_\sigma \circ \iota_{L_i}=\iota'_{L_{\sigma(i)}} \quad \xi_\sigma \circ \iota_{K_i}=\iota'_{K_{\sigma(i)}} \quad \xi_\sigma \circ \iota_{R_i}=\iota'_{R_{\sigma(i)}} \]
\end{remark}

The previous remark allows us to prove the following.

\begin{lemma}\label{prop:isouno} For every semi-consistent permutation $\sigma$ between  $(\der{D}, \alpha, \omega)$ and $(\der{D}', \alpha', \omega')$, the mediating isomorphism $\xi_\sigma\colon \tpro{D}\to \lpro \der{D}'\rpro$ is unique.
\end{lemma}
\begin{proof} Let $\xi'_\sigma$ be another mediating isomorphism, then by \Cref{rem:coproj} we have
	\[\begin{split}
		\xi_\sigma \circ \iota_{L_i}&=\iota'_{L_{\sigma(i)}}\\&=\xi'_\sigma\circ \iota_{L_i}
	\end{split} \qquad \begin{split}
		\xi_\sigma \circ \iota_{K_i}&=\iota'_{K_{\sigma(i)}}\\&=\xi'_\sigma\circ \iota_{K_i}
	\end{split} \qquad \begin{split}
		\xi_\sigma \circ \iota_{R_i}&=\iota'_{R_{\sigma(i)}}\\&=\xi'_\sigma\circ \iota_{R_i}
	\end{split}\]
	
	Now, notice that 
	\begin{align*}
		\xi_\sigma \circ \iota_{G_0}&=\iota'_{G'_0}\circ \alpha'\circ \alpha^{-1}\\&=\xi'_\sigma \circ \iota_{G_0}
	\end{align*}
	
	If $\lgh(\der{D})=0$ this is enough to conclude, otherwise we are going to prove by induction that, $\xi_\sigma \circ \iota_{G_i}=\xi'_\sigma\circ \iota_{G_i}$ for every $i\in [0, \lgh(\der{D})-1]$.
	
	\smallskip \noindent $i=0$. This is simply the result obtained before.
	
	\smallskip \noindent $i >0$. If $i>0$, we know that there is a pushout square
	\[\xymatrix{K_{i-1}\ar[r]^{r_{i-1}} \ar[d]_{k_{i-1}}& R_{i-1} \ar[d]^{h_{i-1}}\\ D_{i-1} \ar[r]_{g_{i-1}}& G_i}\] 
	By \Cref{rem:coproj} and the induction hypothesis we know that
	\[\begin{split}
		\xi_\sigma \circ \iota_{G_i}\circ h_{i-1}&=  \xi_\sigma\circ \iota_{R_{i-1}}\\&=\iota'_{R_{\sigma(i-1)}}\\&=\xi'_{\sigma}\circ \iota_{R_{i-1}}\\&=\xi'_{\sigma} \circ \iota_{G_i}\circ h_{i-1}\\&
	\end{split} \qquad
	\begin{split}
		\xi_\sigma \circ \iota_{G_i}\circ g_{i-1}&=\xi_{\sigma}\circ \iota_{D_{i-1}}\\&=\xi_{\sigma}\circ \iota_{G_{i-1}} \circ f_{i-1}\\&=\xi'_{\sigma}\circ \iota_{G_{i-1}} \circ f_{i-1}\\&=\xi'_{\sigma}\circ \iota_{D_{i-1}} \\&=\xi'_{\sigma}\circ \iota_{G_{i}} \circ g_{i-1}
	\end{split}
	\]
	
	Since $\xi_\sigma \circ \iota_{D_i}$ must be $\xi_\sigma \circ \iota_{G_i}\circ f_i$, we also have
	$\xi_\sigma \circ \iota_{D_i}=\xi'_\sigma \circ \iota_{D_i}$
	and the thesis follows.
\end{proof}


As a next step, we will show that the existence of abstraction equivalences satisfying certain coherence conditions is enough to guarantee the (semi-)consistency of the identical permutation.

\begin{proposition}\label{rem:abscons}
	Let $(\der{D}, \alpha, \omega)$ and $(\der{D}', \alpha', \omega')$ be two decorated derivations in a left-linear rewriting system $(\X, \R)$ of length $l$. Let also $\{\phi_X\}_{X\in \Deltamin(\der{D})}$ be an abstraction equivalence between $\der{D}$ and $\der{D}'$. If $\alpha'\circ \phi_{G_0}=\alpha$ then the the identity $\id{[0,l-1]}$ is a semi-consistent permutation. If, moreover, $\omega'\circ \phi_{G_{l}}=\omega$, then $\id{[0,l-1]}$ is a consistent one.
\end{proposition}
\begin{proof}
	For every $X\in \Deltamin(\der{D}) $, let $X'$ the codomain of $\phi_X$. Then we can define $\xi_{\id{[0,l-1]}}\colon \tpro{D}\to \lpro \der{D}'\rpro$ as the unique morphism such that, for every $X\in \Deltamin(\der{D})$, 
	\[\xi_{\id{[0,l-1]}} \circ \iota_{X}= \iota'_{X'}\circ \phi_X\]

As a first step we can notice that $\xi_{\id{[0,l-1]}}$ is an isomorphism: indeed we can define $\Theta\colon \lpro\der{D}'\rpro \to \tpro{D}$ as the unique arrow satisfying, for every $X'\in \Deltamin(\der{D}')$
\[\Theta \circ \iota'_{X'}= \iota_{X}\circ \phi^{-1}_X\]

From the following two computations we can derive that $\Theta$ is the inverse of $\xi_{\id{[0,l-1]}}$.
\[\begin{split}
	\Theta \circ \xi_{\id{[0,l-1]}}\circ \iota_{X}&=\Theta \circ\iota'_{X'}\circ \phi_X\\&=\iota_{X}\circ \phi^{-1}_X\circ \phi_X\\&=\iota_X 
\end{split}\qquad \begin{split}
\xi_{\id{[0,l-1]}}\circ \Theta \circ \iota'_{X'}&=\xi_{\id{[0,l-1]}}\circ \iota_X\circ \phi^{-1}_{X}\\&=\iota'_{X'}\circ \phi_X\circ \phi^{-1}_{X}\\&=\iota'_{X'}
\end{split}\]

To conclude, let us observe that all the internal subdiagrams of the following diagrams commutes, guaranteeing the commutativity of the full ones.
\[\xymatrix@C=30pt{\pi(G_0)\ar[r]^{\alpha} \ar[d]_{\alpha'} & G_0 \ar[dl]^{\phi_{G_0}} \ar[r]^{\iota_{G_0}} &\tpro{D} \ar[d]^{\xi_{\id{[0, l-1]}}}\\ G'_0 \ar[rr]_{\iota'_{G'_0}} & &\lpro \der{D}' \rpro}\]
\[\xymatrix@C=30pt{L_i \ar[r]^{m_i} \ar[d]_{m'_{\sigma(i)}}& G_i \ar[dl]^{\phi_{G_i}} \ar[r]^{\iota_{G_i}} &\tpro{D} \ar[d]^{\xi_{\id{[0, l-1]}}} & R_i \ar[r]^{h_i} \ar[d]_{h'_{i}}& G_{i+1} \ar[dl]^{\phi_{G_{i+1}}}\ar[r]^{\iota_{G_{i+1}}} &\tpro{D} \ar[d]^{\xi_{\id{[0, l-1]}}} \\G'_{i} \ar[rr]_{\iota'_{G'_{i}}}&& \lpro \der{D}' \rpro& G'_{i+1} \ar[rr]_{\iota'_{G'_{i+1}}}&& \lpro \der{D}' \rpro}\]

For the last half of the thesis, we can reason similarly. Indeed, the diagram below commutes.
\[\xymatrix@C=30pt{\pi(G_l)\ar[r]^{\omega} \ar[d]_{\omega'} & G_l \ar[dl]^{\phi_{G_l}} \ar@{>->}[r]^{\iota_{G_l}} &\tpro{D} \ar[d]^{\xi_{\id{[0, l-1]}}}\\ G'_l \ar@{>->}[rr]_{\iota'_{G'_l}} & &\lpro \der{D}' \rpro}\]
This clearly entails the thesis.
\end{proof}

The previous proposition, in turn, gives us a technical but quite useful result.

\begin{corollary}\label{cor:fromsemitocons}
			Let $(\der{D}, \alpha, \omega)$ and $(\der{D}', \alpha', \omega')$ be two decorated derivations in an left-linear rewriting system $(\X, \R)$ of length $l$. Let also $\{\phi_X\}_{X\in \Deltamin(\der{D})}$ be an abstraction equivalence between $\der{D}$ and $\der{D}'$ such that  $\alpha'\circ \phi_{G_0}=\alpha$. If  the identity $\id{[0,l-1]}$ is a consistent permutation, then $\omega'\circ \phi_{G_l}=\omega$, so that $(\der{D}, \alpha, \omega) \equiv^a (\der{D}', \alpha', \omega')$.
\end{corollary}
\begin{proof}
	Let $\xi_{\id{[0, l-1]}}$ be the isomorphism $\tpro{D}\to \lpro\der{D}'\rpro$ witnessing the consistency of $\id{[0, l-1]}$. By \Cref{prop:isouno} and \Cref{rem:abscons}, we know that $\xi_{\id{[0,l-1]}} \circ \iota_{X}= \iota'_{X'}\circ \phi_X$ for every object $X\in \Deltamin(\der{D})$. Then we have
	\begin{align*}
		\iota'_{ G'_{l}}\circ \phi_{G_l}\circ \omega&=\xi_{\id{[0, l-1]}}\circ \iota_{G_l}\circ \omega\\&=\iota'_{G'_l}\circ \omega'
	\end{align*}
	
	By the first point of \Cref{lem:colim}, $\iota'_{ G'_{l}}$ is mono and we can conclude.
\end{proof}


\begin{remark}\label{rem:abscons2}
	In particular, given a decorated derivation $(\der{D}, \alpha, \omega)$ with source $G$ and target $H$ and two isomorphisms $\phi\colon G'\to G$, $\psi\colon H\to H'$, we know, by \Cref{rem:absequi}, that 
	\[(\der{D}, \alpha, \omega) \equiv^a (\phi * \der{D}, \phi^{-1}\circ \alpha,   \varphi^{-1}_{H} \circ \omega) \qquad (\der{D}, \alpha, \omega) \equiv^a ( \der{D}*\psi, \varphi'_{G} \circ \alpha, \psi \circ \omega )\]
	
	Now, denoting by $(\tpro{D}, \{\iota_X\}_{X\in \Delta(\der{D})})$, $(\lpro \phi *\der{D} \rpro, \{\iota'_X\}_{X\in \Delta(\phi *\der{D})})$ and $(\lpro \der{D}*\psi \rpro, \{\hat{\iota}_X\}_{X\in \Delta(\der{D}*\psi)})$ the colimiting cocones, in this case we have that the (unique) mediating isomorphisms witnessing the consistency of $\id{[0, \lgh(\der{D})-1]}$ are given by the arrows $\Gamma^{\phi} \colon \tpro{D}\to \lpro \phi*\der{D} \rpro$ and $\Gamma_\psi\colon \tpro{D}\to \lpro \der{D}*\psi \rpro$  such that, for every $X\in \Deltamin(\der{D})$:
	\[\Gamma^\phi \circ \iota_X:=\begin{cases}
		\iota'_{G'}\circ \phi^{-1}  & X=G\\
		\iota'_X & \text{otherwise}
	\end{cases} \qquad \Gamma_\psi \circ \iota_X:=\begin{cases}
		\hat{\iota}_{H'}\circ \psi  & X=H\\
		\hat{\iota}_X & \text{otherwise}
	\end{cases}\]
\end{remark}


\begin{example}\todo{identita consistente non implica astrazione}Notice that, in general, there cannot, in general, be generalized to non-empty derivations. A counterexample is the following one.
\end{example}



We can also observe that (semi-)consistent permutations can be composed, as shown by the next proposition.

\begin{proposition}\label{rem:comp}  Let $(\der{D}, \alpha, \omega)$, $(\der{D}', \alpha', \omega')$ and $(\der{\hat{D}}, \hat{\alpha}, \hat{\omega})$ be three decorated derivations of length $l$. If $\sigma$ is a semi-consistent permutation between the first two and $\tau$ one between the second and the third, then $\tau \circ \sigma$ is a semi-consistent permutation between the first and the third. If, moreover, $\sigma$ and $\tau$ are consistent, then $\tau \circ \sigma$ is consistent too.
\end{proposition} 	

\begin{proof} Let $l$ be the length of the three given derivations.  By hypothesis each subdiagram of the following ones commutes, thus the whole diagrams commute too, proving that $\tau\circ \sigma$ is a semi-consistent permutation with mediating isomorphism given by $\xi_{\tau} \circ \xi_\sigma$.
		\[\xymatrix@C=18pt{ & G_0 \ar[rr]^{\iota_{G_0}} &&\tpro{D} \ar[d]^{\xi_\sigma} \\ \pi(G_0) \ar@/_.2cm/[dr]_{\hat{\alpha}}\ar@/^.2cm/[ur]^{\alpha} \ar[r]^{\alpha'} &G'_0 \ar[rr]^{\iota'_{G'_0}}  &&\lpro \der{D}'  \rpro \ar[d]^{\xi_{\tau}}\\ &\hat{G}_0  \ar[rr]_{\hat{\iota}_{\hat{G}_0}}&& \lpro \hat{\der{D}}\rpro }\]  
		\[\xymatrix@C=18pt{& G_i \ar[rr]^{\iota_{G_i}} &&\tpro{D} \ar[d]^{\xi_\sigma}&  &  G_{i+1} \ar[rr]^{\iota_{G_{i+1}}} &&\tpro{D} \ar[d]^{\xi_\sigma} \\ L_i \ar@/_.2cm/[dr]_{\hat{m}_{\tau(\sigma(i))}}\ar@/^.2cm/[ur]^{m_i} \ar[r]^{m'_{\sigma(i)}} &G'_{\sigma(i)} \ar[rr]^{\iota'_{G'_{\sigma(i)}}}  &&\lpro \der{D}'  \rpro \ar[d]^{\xi_{\tau}}&R_i \ar@/_.2cm/[dr]_{\hat{h}_{\tau(\sigma(i))}}\ar@/^.2cm/[ur]^{h_i} \ar[r]^{h'_{\sigma(i)}} & G'_{\sigma(i)+1} \ar[rr]^{\iota'_{G'_{\sigma(i)+1}}} && \lpro \der{D}' \rpro \ar[d]^{\xi_{\tau}}\\ &\hat{G}_{\tau(\sigma(i))}  \ar[rr]_-{\hat{\iota}_{\hat{G}_{\tau(\sigma(i))}}}&& \lpro \hat{\der{D}}\rpro  && \hat{G}_{\tau(\sigma(i))+1} \ar[rr]_-{\hat{\iota}_{\hat{G}_{\tau(\sigma(i))+1}}}&& \lpro \hat{\der{D}} \rpro	}\]
	
	If, in addition, $\sigma$ and $\tau$ are consistent then we also have the commutativity of the following additional diagram.
	\[\xymatrix@C=18pt{&   G_{l} \ar@{>->}[rr]^{\iota_{ G_{l}}} &&\tpro{D} \ar[d]^{\xi_\sigma}\\ \pi({ G_{l}})\ar@/_.2cm/[dr]_{\hat{\omega}}\ar@/^.2cm/[ur]^{\omega} \ar[r]^{\omega'} & { G'_{l}} \ar@{>->}[rr]^{\iota'_{{ G'_{l}}}} && \lpro \der{D}' \rpro \ar[d]^{\xi_{\tau}}\\& \hat{G}_l \ar@{>->}[rr]_{\hat{\iota}_{\hat{G}_l}}&& \lpro \hat{\der{D}} \rpro}\]
	
	Again this proves that $\xi_\tau \circ \xi_\sigma$ witnesses the constistency of $\tau\circ \sigma$.
\end{proof}
	
We have already proved in \Cref{rem:abscons} that the identity defines a consistent permutation between any two representatives of an equivalence class $[\der{D}, \alpha, \omega]_a$. This, together with \Cref{rem:comp}, allow us to immediately derive the following fact.

\begin{corollary}\label{prop:abs}
Let $(\der{D}_1, \alpha_1, \omega_1)$ and $(\der{D}_2, \alpha_2, \omega_2)$ be two decorated derivations and suppose that $\sigma$ is a consistent permutation between them. If  $(\der{D}'_1, \alpha'_1, \omega'_1)$ and $(\der{D}'_2, \alpha'_2, \omega'_2)$ are such that
\[ (\der{D}_1, \alpha_1, \omega_1)\equiv^a (\der{D}'_1, \alpha'_1, \omega'_1) \qquad (\der{D}_2, \alpha_2, \omega_2)\equiv^a (\der{D}'_2, \alpha'_2, \omega'_2) \] 
then $\sigma$ is a consistent permutation between $(\der{D}'_1, \alpha'_1, \omega'_1)$ and $(\der{D}'_2, \alpha'_2, \omega'_2)$ too.
\end{corollary}

\begin{proof} Let $l_1$ and $l_2$ be the lengths of $\der{D}_1$ and $\der{D}_2$. By \Cref{rem:abscons} we know that  $\id{[0, l_1-1]}$ is a consistent permutation between  $(\der{D}_1, \alpha_1, \omega_1)$ and $(\der{D}'_1, \alpha'_1, \omega'_1)$  and $\id{[0, l_2-1]}$ one between   $(\der{D}_2, \alpha_2, \omega_2)$  and $(\der{D}'_2, \alpha'_2, \omega'_2)$. Using \Cref{rem:comp} we get the thesis at once.
\end{proof}

The previous result suggests that the existence of a consistent permutations induces an equivalence relation on the arrows of $\dpi$. This is formalized by the next definition.

\begin{definition}Let $(\X, \R)$ be a left-linear DPO rewriting system. We say that two arrows
	$[\der{D}, \alpha, \omega]_a, [\der{D}', \alpha', \omega']_a\colon [G]\rightrightarrows [H]$ in $\dpi$ are \emph{permutation equivalent} if there exists a consistent permutation between $(\der{D}, \alpha, \omega)$  and $(\der{D}', \alpha', \omega')$. 
	
	We will use the notation $[\der{D}, \alpha, \omega]_a, \equiv^p [\der{D}', \alpha', \omega']_a$  to denote permutation equivalence, while we will use $[\der{D}, \alpha, \omega]_a, \equiv_\sigma^p [\der{D}', \alpha', \omega']_a$ to point out a witness $\sigma$ to this relation.
\end{definition}

\begin{remark}
	The relation $\equiv^p$ is an equivalence: \Cref{prop:abs} guarantees that it is well-defined, \Cref{rem:abscons} guarantees reflexivity, while symmetry and transitivity follow from, respectively, 
	\Cref{rem:inversa} and \Cref{rem:comp}.
\end{remark}

\begin{notation}
	We will denote by $[\der{D}, \alpha, \omega]_p$ the equivalence class of $[\der{D}, \alpha, \omega]_a$ with respect to the permutation equivalence.
\end{notation}

\begin{remark}\label{rem:ugu}
	It is worth noticing that, by \Cref{rem:empty2}, the permutation equivalence relation reduces to abstract equivalence in the case of empty derivations. This means that, given an object $G$, we have ~$[\{G\}, \alpha, \omega]_a=[\{G\}, \alpha, \omega]_p$.
\end{remark}

\section{Composing and restricting consistent permutations}

We want now to exploit the results proved at the end of \Cref{subsec:col} to study the behaviour of permutation equivalence with respect to composition of decorated derivations. As a first step we need to establish some technical points.

\begin{lemma}\label{prop:uniqu}
	Let $(\der{D}, \alpha, \omega)$ and $(\der{D}', \alpha', \omega')$ be two decorated derivations such that $[\der{D}, \alpha, \omega]_a\equiv^p_\sigma [\der{D}', \alpha', \omega']_a $. Suppose that 
	\[(\der{D}, \alpha, \omega)=(\der{D}_1, \alpha_1, \omega_1) \cdot (\der{D}_2, \alpha_2, \omega_2) \qquad (\der{D}', \alpha', \omega')=(\der{D}'_1, \alpha'_1, \omega'_1) \cdot (\der{D}'_2, \alpha'_2, \omega'_2) \]
	
	 Let $l_1$ and $l_2$ be the lengths of, respectively, $\der{D}_1$ and $\der{D}_2$ and suppose that  $[\der{D}_1, \alpha_1, \omega_1]_a\equiv^p_\tau [\der{D}'_1, \alpha'_1, \omega'_1]_a$ for some consistent permutation $\tau\colon [0, l_1-1]\to [0,l_1-1]$.  If $q_1\colon \lpro \der{D}_1\rpro \to \lpro \der{D}\rpro$  and $q'_1\colon \lpro \der{D}'_1\rpro \to \lpro \der{D}'\rpro$ are the canonical arrows defined in \Cref{rem:dett}, then the following hold true:
	\begin{enumerate}
	\item  if $l_1=0$, then the following diagram commutes:
		\[\xymatrix{\lpro \der{D}_1\rpro \ar[r]^{\xi_\tau} \ar[d]_{q_1}& \lpro \der{D}'_1 \ar[d]^{q'_1}\rpro\\ \lpro \der{D}\rpro \ar[r]_{\xi_\sigma} & \lpro \der{D}'\rpro } \]
	%\[\xi_\sigma \circ q_1 \circ \iota_{1, G_{1,0}} = q'_1\circ \xi_{\tau} \circ  \iota_{1, G_{1,0}}\]
	\item if $l_1\neq 0$, so that $G_{1,0}=G_0$, then $\xi_\sigma \circ \iota_{G_0} = q'_1\circ \xi_{\tau} \circ  \iota_{1, G_{1,0}}$.
	\end{enumerate}
	
	Moreover, define the set $E_{\sigma, \tau}$ as
	\[E_{\sigma, \tau}:=\{i\in  [0, l_1-1] \mid \sigma(j)=\tau(j) \text{ for every } j \leq i \}\]
	then also the following hold true:
	\begin{enumerate}
		\setcounter{enumi}{2}
	\item  if $E_{\sigma, \tau}\neq \emptyset$,  for every index $i\in [0, \max(E_{\sigma, \tau})]$ we have 
	$\xi_\sigma \circ \iota_{G_i}=q'_1 \circ \xi_{\tau} \circ \iota_{1, G_i}$;
	\item if $l_1-1\in E_{\sigma, \tau}$ and $l_2=0$ so that  $G_{l_1}=G_{1, l_1}$, then 
	$\xi_\sigma \circ \iota_{G_{l_1}}=q'_1 \circ \xi_{\tau} \circ \iota_{1, G_{l_1}}$;
	\item if $l_1-1\in E_{\sigma, \tau}$ and $l_2\neq 0$, so that $G_{l_1}=\pi(G_{2,0})$, then 
	$\xi_\sigma \circ \iota_{G_{l_1}}=q'_1 \circ \xi_{\tau} \circ \iota_{1, G_{l_1}}\circ \omega_1$;
	\item if $l_1-1\in E_{\sigma, \tau}$, then the following diagram commutes:
	\[\xymatrix{\lpro \der{D}_1\rpro \ar[r]^{\xi_\tau} \ar[d]_{q_1}& \lpro \der{D}'_1 \ar[d]^{q'_1}\rpro\\ \lpro \der{D}\rpro \ar[r]_{\xi_\sigma} & \lpro \der{D}'\rpro } \]
	\end{enumerate}
\end{lemma}
\begin{proof}\begin{enumerate}
		\item Recall that, in this case \[q_1=\iota_{G_0}\circ \alpha_2\circ \omega^{-1}_1 \circ \iota^{-1}_{1,G_{1,0}} \qquad q'_1=\iota'_{G'_0}\circ \alpha'_2\circ (\omega'_1)^{-1} \circ (\iota'_{1,G'_{1,0}})^{-1}\]
		
		If we compute we have
		\begin{align*}
	\xi_\sigma \circ q_1 &=\xi_\sigma \circ \iota_{G_0}\circ \alpha_2\circ \omega^{-1}_1 \circ \iota^{-1}_{1,G_{1,0}} \\&=\iota'_{G'_0}\circ \alpha'\circ \alpha^{-1}\circ \alpha_2\circ \omega^{-1}_1\circ \iota^{-1}_{1,G_{1,0}}
		\end{align*}
		Now, since $l_1=0$ we also have that 
		\[\alpha=\alpha_2\circ \omega_1^{-1}\circ \alpha_1 \qquad \alpha'=\alpha'_2\circ (\omega'_1)^{-1}\circ \alpha'_1\]
		
		Therefore, using \Cref{rem:empty}, we get:
		\begin{align*}
	\alpha'\circ \alpha^{-1}&=\alpha'_2\circ (\omega'_1)^{-1}\circ \alpha'_1\circ \alpha^{-1}_1\circ \omega_1\circ \alpha^{-1}_2\\&=\alpha'_2\circ (\omega'_1)^{-1}\circ \omega'_1\circ \omega^{-1}_1\circ \omega_1\circ \alpha^{-1}_2\\&=\alpha'_2\circ \alpha^{-1}_2
		\end{align*}
	Hence, again by \Cref{rem:empty} we obtain:
				\begin{align*}
			\xi_\sigma \circ q_1 &=\iota'_{G'_0}\circ \alpha'\circ \alpha^{-1}\circ \alpha_2\circ \omega^{-1}_1\circ \iota^{-1}_{1,G_{1,0}}\\&=\iota'_{G'_0}\circ \alpha'_2\circ \alpha^{-1}_2\circ \alpha_2\circ \omega^{-1}_1\circ \iota^{-1}_{1,G_{1,0}}\\&=\iota'_{G'_0}\circ \alpha'_2\circ \omega^{-1}_1\circ \iota^{-1}_{1,G_{1,0}}\\&=\iota'_{G'_0}\circ \alpha'_2\circ (\omega'_1)^{-1} \circ  \omega'_1\circ \omega^{-1}_1\circ \iota^{-1}_{1,G_{1,0}}\\&=\iota'_{G_0}\circ \alpha'_2\circ (\omega'_1)^{-1} \circ \alpha'_1\circ \alpha^{-1}_1\circ \iota^{-1}_{1,G_{1,0}}\\&=\iota'_{G'_0}\circ \alpha'_2\circ (\omega'_1)^{-1} \circ (\iota'_{1,G_{1,0}})^{-1}\circ \iota'_{1,G'_{1,0}}\circ \alpha'_1\circ \alpha^{-1}_1\circ \iota^{-1}_{1,G_{1,0}}\\&=q'_1\circ \iota'_{1,G'_{1,0}}\circ \alpha'_1\circ \alpha^{-1}_1\circ \iota^{-1}_{1,G_{1,0}}\\&=q'_1\circ \xi_{\tau}
		\end{align*}


		
		\item We can start noticing that
		\[\xi_\sigma \circ \iota_{G_0}=\iota'_{G'_0}\circ \alpha'\circ \alpha^{-1} \qquad \xi_{\tau} \circ \iota_{1,G_{1,0}}= \iota'_{1, G'_{1,0}} \circ \alpha'_1\circ \alpha^{-1}_1\] 
		Therefore:
		\begin{align*}
		q'_1\circ \xi_{\tau} \circ \iota_{1, G_{1,0}}&=q'_1 \circ \iota'_{1, G'_{1,0}} \circ \alpha'_1\circ \alpha^{-1}_1\\ &= \iota'_{G'_{0}}\circ \alpha' \circ \alpha^{-1}_1
		\end{align*}
		By hypothesis $\lgh(\der{D}_1) \neq 0$, thus $\alpha_1=\alpha$ and we can conclude.
		
		\item  From $E_{\sigma, \tau}\neq \emptyset$ we deduce that $l_1\neq 0$ and that $0\in E_{\sigma, \tau}$.   We proceed by induction on $i\in [0,\max(E_{\sigma, \tau})]$. 

\smallskip \noindent  If $i=0$ the thesis follows from the previous point.

\smallskip \noindent If $i>0$, we proceed as in the proof of \Cref{prop:isouno} considering the pushout square
			\[\xymatrix{K_{i-1}\ar[r]^{r_{i-1}} \ar[d]_{k_{i-1}}& R_{i-1} \ar[d]^{h_{i-1}}\\ D_{i-1} \ar[r]_{g_{i-1}}& G_i}\] 
		Since $i$ belongs to $E_{\sigma, \tau}$, then $i-1\in E_{\sigma, \tau}$ too. By definition, $i\leq l_1-1$, so that, using \Cref{rem:coproj}, \Cref{rem:dett} and the induction hypothesis we have
			\[\begin{split}
				\xi_\sigma \circ \iota_{G_i}\circ h_{i-1}&=  \xi_\sigma\circ \iota_{R_{i-1}}\\&=\iota'_{R'_{\sigma(i-1)}}\\&=\iota'_{R'_{\tau(i-1)}}\\&=q'_1\circ \iota'_{1, R'_{1,\tau(i-1)}}\\&=q'_1\circ \xi_{\tau}\circ \iota_{1, R_{i-1}}\\&=q'_1\circ \xi_{\tau} \circ \iota_{1, G_{1,i}}\circ h_{1, i-1}\\&=q'_1\circ \xi_\tau \circ \iota_{1, G_i} \circ h_{i-1}
			\end{split} \qquad  \begin{split}
			\xi_\sigma \circ \iota_{G_i}\circ g_{i-1}&=\xi_{\sigma}\circ \iota_{D_{i-1}}\\&=\xi_{\sigma}\circ \iota_{G_{i-1}} \circ f_{i-1}\\&=q'_1 \circ \xi_{\tau} \circ \iota_{1, G_{i-1}} \circ f_{i-1}\\&=q'_1 \circ \xi_{\tau} \circ \iota_{1, G_{i-1}} \circ f_{1, i-1}\\&=q'_1 \circ \xi_{\tau} \circ \iota_{1, D_{i-1}}\\&=q'_1 \circ \xi_{\tau} \circ \iota_{1, G_i}\circ g_{1, i-1}\\&=q'_1 \circ \xi_{\tau} \circ \iota_{1, G_i}\circ g_{i-1}
			\end{split}\]
	
	From which the thesis follows.
	
			\item Since $l_1-1\in E_{\sigma, \tau}$ we have $l_1\neq 0$ and $E_{\sigma, \tau}=[0,l_1-1]$.  We can start with the pushout
			\[\xymatrix{K_{l_1-1}\ar[r]^{r_{l_1-1}} \ar[d]_{k_{l_1-1}}& R_{i-1} \ar[d]^{h_{l_1-1}}\\ D_{l_1-1} \ar[r]_{g_{l_1-1}}& G_{l_1}}\] 
			By \Cref{def:conc}, we have $h_{l_1-1}=h_{1, l_1}$ and $g_{l_1-1}=g_{1,l_1}$. Since $q'_1=\id{\lpro \der{D}'_1\rpro}$ these two equalities imply that $	\iota'_{R_{\tau(l_1-1)}}=q'_1\circ \iota'_{1, R'_{1,\tau(l_1-1)}}$. 			
			
			Moreover, by hypothesis $l_1-1\in E_{\sigma, \tau}$, so that point $3$ of this lemma entails that
			\[\xi_\sigma \circ \iota_{G_{l_1-1}}=q'_1 \circ \xi_{\tau} \circ \iota_{1, G_{l_1-1}}\]
			
			We can thus repeat the same argument of the previous point:
			\[\begin{split}
				\xi_\sigma \circ \iota_{G_{l_1}}\circ h_{l_1-1}&=  \xi_\sigma\circ \iota_{R_{l_1-1}}\\&=\iota'_{R'_{\sigma(l_1-1)}}\\&=\iota'_{R'_{\tau(l_1-1)}}\\&=q'_1\circ \iota'_{1, R'_{1,\tau(l_1-1)}}\\&=q'_1\circ \xi_{\tau}\circ \iota_{1, R_{l_1-1}}\\&=q'_1\circ \xi_{\tau} \circ \iota_{1, G_{1,l_1}}\circ h_{1, l_1-1}\\&=q'_1\circ \xi_\tau \circ \iota_{1, G_{l_1}} \circ h_{l_1-1}			\end{split} 
			\quad  \begin{split}
				\xi_\sigma \circ \iota_{G_{l_1}}\circ g_{l_1-1}&=\xi_{\sigma}\circ \iota_{D_{l_1-1}}\\&=\xi_{\sigma}\circ \iota_{G_{l_1-1}} \circ f_{l_1-1}\\&=q'_1 \circ \xi_{\tau} \circ \iota_{1, G_{l_1-1}} \circ f_{l_1-1}\\&=q'_1 \circ \xi_{\tau} \circ \iota_{1, G_{l_1}-1} \circ f_{1, l_1-1}\\&=q'_1 \circ \xi_{\tau} \circ \iota_{1, D_{l_1-1}}\\&=q'_1 \circ \xi_{\tau} \circ \iota_{1, G_{l_1}}\circ g_{1, l_1-1}\\&=q'_1 \circ \xi_{\tau} \circ \iota_{1, G_{l_1}}\circ g_{l_1-1}
			\end{split}\]
		
		And again this implies the thesis.
			
			\item In this case, we can consider the following diagram, in which the inner rectangle and the outer border are pushouts
		\[\xymatrix@C=35pt{K_{1,l_1-1}\ar[r]^{r_{1,l_1-1}} \ar[d]_{k_{1,l_1-1}}& R_{i-1} \ar[d]_{h_{1,l_1-1}} \ar@/^.4cm/[ddr]^{h_{l_1-1}}\\ D_{1,l_1-1} \ar[r]_{g_{1,l_1-1}} \ar@/_.4cm/[drr]_{g_{l_1-1}}& G_{1,l_1} \ar[dr]^{\omega_1^{-1}}\\ &&\pi(G_{2,0})}\] 
		
		\iffalse 
	\[	\xi_\sigma \circ \iota_{G_{l_1}}=q'_1 \circ \xi_{\tau} \circ \iota_{1, G_{l_1}}\circ \omega_1\]
	\fi 
		We can start noticing that, by the proof of  the third point of \Cref{rem:dett}, we know that  $q'_1\circ \iota'_{1, G'_{1,l_1}}=\iota'_{G'_{l_1}} \circ \omega_1^{-1}$. Therefore
		\begin{align*}q'_1\circ \iota'_{1, R'_{1,l_1}}&=q'_1\circ \iota'_{1, G'_{1,l_1}}\circ  h'_{1, l_1}\\&=q'_1\circ \iota'_{1, G'_{1,l_1}}\circ \omega_1 \circ  h'_{ l_1}\\&= \iota'_{G'_{l_1}}\circ h'_{l_1}\\&=\iota'_{R'_{l_1-1}}
		\end{align*}
		
		The argument now proceeds almost verbatim like the one of the previous two point. Indeed, using the third point of this lemma we get:
		\[\begin{split}
			\xi_\sigma \circ \iota_{G_{l_1}}\circ h_{l_1-1}&=  \xi_\sigma\circ \iota_{R_{l_1-1}}\\&=\iota'_{R'_{\sigma(l_1-1)}}\\&=\iota'_{R'_{\tau(l_1-1)}}\\&=q'_1\circ \iota'_{1, R'_{1,\tau(l_1-1)}}\\&=q'_1\circ \xi_{\tau}\circ \iota_{1, R_{l_1-1}}\\&=q'_1\circ \xi_{\tau} \circ \iota_{1, G_{1,l_1}}\circ h_{1, l_1-1}\\&=q'_1\circ \xi_\tau \circ \iota_{1, G_{1, l_1}} \circ \omega_1 \circ  h_{l_1-1}			\end{split}   	\hspace{-5pt}	\begin{split}
			\xi_\sigma \circ \iota_{G_{l_1}}\circ g_{l_1-1}&=\xi_{\sigma}\circ \iota_{D_{l_1-1}}\\&=\xi_{\sigma}\circ \iota_{G_{l_1-1}} \circ f_{l_1-1}\\&=q'_1 \circ \xi_{\tau} \circ \iota_{1, G_{l_1}} \circ f_{l_1-1}\\&=q'_1 \circ \xi_{\tau} \circ \iota_{1, G_{l_1}} \circ f_{1, {l_1}-1}\\&=q'_1 \circ \xi_{\tau} \circ \iota_{1, D_{{l_1}-1}}\\&=q'_1 \circ \xi_{\tau} \circ \iota_{1, G_{l_1}}\circ g_{1, l_1-1}\\&=q'_1 \circ \xi_{\tau} \circ \iota_{1, G_{l_1}}\circ \omega_1\circ g_{l_1-1}
			\end{split}\]
			
			\item  The thesis follows from points $3$ and $4$ if $l_2=0$, from $3$ and $5$ if $l_2\neq 0$.	\qedhere
	\end{enumerate}
\end{proof}


We want now to show that permutation equivalence is a congruence with respect to composition in $\dpi$. This, in turn, will allow us to build  a category in which arrows are exactly the equivalence classes of permutation equivalent abstract decorated decorations.


\begin{remark}\label{rem:sum}
	Let $n$ and $m$ be two natural numbers and $\sigma\colon [0, n-1]\to [0,n-1]$ and $\tau\colon [0, m-1]\to [0,m-1]$ be two permutations. Then they induce a permutation $[0, n+m-1]\to [0, n+m-1]$. To see this, it is enough to define
		\[\sigma+\tau\colon[0, n+m-1]\to[0, n+m-1] \quad i \mapsto \begin{cases}
		\sigma(i) & i < n\\
		\tau(i-n) + n &  n\leq i 
	\end{cases}\]
\end{remark}



\begin{lemma}[Composition lemma]\label{lem:sum} Let $(\der{D}, \alpha, \omega)$ and $(\der{D}', \alpha', \omega')$ be two abstract decorated derivations such that:
	\[(\der{D}, \alpha, \omega)=(\der{D}_1, \alpha_1, \omega_1)\cdot (\der{D}_2, \alpha_2, \omega_2) \qquad (\der{D}', \alpha', \omega')=(\der{D}'_1, \alpha'_1, \omega'_1)\cdot (\der{D}'_2, \alpha'_2, \omega'_2)\]
	
Suppose that $[\der{D}_1, \alpha_1, \omega_1]_a \equiv^p_\sigma [\der{D}'_1, \alpha'_1, \omega'_1]_a$ and $[\der{D}_2, \alpha_2, \omega_2]_a \equiv^p_\tau [\der{D}'_2, \alpha'_2, \omega'_2]_a$, then:
\[[\der{D}, \alpha, \omega]_a\equiv^p_{\sigma+\tau} [\der{D}', \alpha', \omega']_a\]
\end{lemma}
\begin{proof}
By \Cref{prop:abs}	it is enough to show that $\sigma+\tau$	is a consistent permutation between $(\der{D}, \alpha, \omega)$ and $(\der{D}', \alpha', \omega')$.

We need to fix some notation.	Given $X_1\in \Delta(\der{D}_1)$, $X_2\in \Delta(\der{D}_2)$, $X'_1\in \Delta(\der{D}'_1)$, $X'_2\in \Delta(\der{D}'_2)$, we will denote their coprojections into, respectively, $\lpro \der{D}_1\rpro$, $\lpro \der{D}'_2\rpro$, $\lpro \der{D}'_1\rpro$, $\lpro \der{D}'_2\rpro$ by
	$\iota_{1, X_1}\colon X_1\to \lpro\der{D}_1\rpro$, $\iota_{1, X_2}\colon X_2\to \lpro\der{D}_1\rpro$, $\iota'_{1, X'_1}\colon X'_1\to \lpro\der{D}'_1\rpro$ and $\iota'_{2, X'_2}\colon X'_2\to \lpro\der{D}_1\rpro$. Similarly, given $X\in \Delta(\der{D})$ and $X'\in \Delta(\der{D}')$ we will use $\iota_X\colon X\to \tpro{D}$ and $\iota'_{X'}\colon X'\to \lpro \der{D}' \rpro$ for the coprojections. 
	
	We split the proof in three cases, according to \Cref{def:conc}.
	\begin{itemize}
		\item $\lgh(\der{D}_1)=0$. Thus $\lgh(\der{D}'_1)=0$ too  and we have
		\[
		(\der{D}, \alpha, \omega)=(\der{D}_2, \alpha_2\circ \omega_1^{-1}\circ \alpha_1, \omega_2) \qquad  (\der{D}', \alpha', \omega')=(\der{D}'_2, \alpha'_2\circ (\omega'_1)^{-1}\circ \alpha'_1, \omega'_2)\]
		
		Moreover, in this case $\sigma$ must be $\id{\emptyset}$ and $\sigma+\tau$ must be equal to $\tau$. The thesis now follows from the commutativity of the following diagram.
		\[\xymatrix@C=35pt{& G_{1,0}\ar[dd]^{\xi_{\id{\emptyset}}} \ar[dr]^{\omega^{-1}_1}&&G_{2,0}\ar[r]^-{\iota_{2, G_{2,0}}} & \lpro \der{D}_2\rpro \ar[dd]^{\xi_{\tau}}\\\pi(G_{1,0})  \ar[ur]^{\alpha_1} \ar[dr]_{\alpha'_1}&& \pi(G_{2,0}) \ar[ur]^{\alpha_2} \ar[dr]_{\alpha'_2}\\& G'_{1,0} \ar[ur]_{(\omega'_1)^{-1}}&&G'_{2,0} \ar[r]_-{\iota'_{2, G_{2,0}}}& \lpro \der{D}'_2\rpro}\]
		
		\item $\lgh(\der{D}_2)=0$.  As before this implies that also $\lgh(\der{D}'_2)$ is $0$.  Applying \Cref{def:conc} we get 
		\[
		(\der{D}, \alpha, \omega)=(\der{D}_1, \alpha_1, \omega_1 \circ (\alpha_2)^{-1}\circ \omega_2) \qquad (\der{D}', \alpha', \omega')=(\der{D}'_1, \alpha'_1, \omega'_1 \circ (\alpha'_2)^{-1}\circ \omega'_2)\]
		
		Since $\der{D}_2$ is empty, then $\tau=\id{\emptyset}$ and $\sigma+\tau=\sigma$. Let $n$ be $\lgh(\der{D}_1)$, as before the thesis follows from the diagram below.
		\[\xymatrix@C=35pt{& G_{2,0}\ar[dd]^{\xi_{\id{\emptyset}}} \ar[dr]^{\alpha^{-1}_2}&&G_{1,n}\ar@{>->}[r]^-{\iota_{1, G_{1,n}}} & \lpro \der{D}_2\rpro \ar[dd]^{\xi_{\sigma}}\\\pi(G_{2,0})  \ar[ur]^{\omega_2} \ar[dr]_{\omega'_2}&& \pi(G_{1,n}) \ar[ur]^{\omega_1} \ar[dr]_{\omega'_1}\\& G'_{2,0} \ar[ur]_{(\alpha'_1)^{-1}}&&G'_{1,n} \ar@{>->}[r]_-{\iota'_{1, G_{1,n}}}& \lpro \der{D}'_2\rpro}\]
		
		\item  $\lgh(\der{D}_1)\neq0$ and $\lgh(\der{D}_2)\neq 0$. Thus $\der{D}'_1$ and $\der{D}'_2$ are non-empty too. In this case we have
		\[	(\der{D}, \alpha, \omega)=(\der{D}_1*\omega_1^{-1}\cdot \alpha_2*\der{D}_2, \alpha_1, \omega_2)\qquad
		(\der{D}', \alpha', \omega')=(\der{D}'_1*(\omega'_1)^{-1}\cdot \alpha'_2*\der{D}'_2, \alpha'_1, \omega'_2)\]
		
		Let us assume that $\der{D}_1$, $\der{D}'_1$, $\der{D}_2$ and $\der{D}'_2$ are given by
		\[\der{D}_1=\{\dder{D}_{1,i}\}_{i=0}^n \quad \der{D}'_1=\{\dder{D}'_{1,i}\}_{i=0}^n \quad \der{D}_2=\{\dder{D}_{2,i}\}_{i=0}^t \quad \der{D}_2'=\{\dder{D}'_{2,i}\}_{i=0}^t\]
		By definition of consistent permutation, the rule applied by $\dder{D}_{1,i}$ and the one applied in $\dder{D}_{2,i}$  must coincide with, respectively, the one applied in $\dder{D}'_{1,i}$ and the one applied $\dder{D}'_{2,1}$. Let $\dder{D}_{1,i}$, $\dder{D}'_{1,i}$, $\dder{D}_{2,i}$ and $\dder{D}'_{2,i}$ be given, by the following four diagrams. 
		\[\xymatrix{L_{1,i} \ar[d]_{m_{1, i}}& K_{1,i} \ar[d]_{k_{1, i}} \ar[r]^{r_{1,i}} \ar@{>->}[l]_{l_{1,i}} & R_{1,i}\ar[d]^{h_{1, i}} &L_{1,i} \ar[d]_{m'_{1, i}}& K_{1,i} \ar[d]_{k'_{1, i}} \ar[r]^{r_{1,i}} \ar@{>->}[l]_{l_{1,i}} & R_{1,i}\ar[d]^{h'_{1, i}} \\G_{1,i} & D_{1,i} \ar[r]_{g_{1,i}} \ar@{>->}[l]^{f_{1,i}} & G_{1,i+1} & G'_{1,i} & D'_{1,i}\ar[r]_{g'_{1,i}} \ar@{>->}[l]^{f'_{1,i}}  & G'_{1,i+1}}\]		
		\[\xymatrix{L_{2,i} \ar[d]_{m_{2, i}}& K_{2,i} \ar[d]_{k_{2, i}} \ar[r]^{r_{2,i}} \ar@{>->}[l]_{l_{2,i}} & R_{2,i}\ar[d]^{h_{2, i}} &L_{2,i} \ar[d]_{m'_{2, i}}& K_{2,i} \ar[d]_{k'_{2, i}} \ar[r]^{r_{2,i}} \ar@{>->}[l]_{l_{2,i}} & R_{2,i}\ar[d]^{h'_{2, i}} \\G_{2,i} & D_{2,i} \ar[r]_{g_{2,i}} \ar@{>->}[l]^{f_{2,i}} & G_{2,i+1} & G'_{2,i} & D'_{2,i}\ar[r]_{g'_{2,i}} \ar@{>->}[l]^{f'_{2,i}}  & G'_{2,i+1}}\] 
		
		%Let $(\lpro \der{D}_1*(\omega_1)^{-1}\rpro, \{j_X\}_{X\in \Delta( \der{D}_1*(\omega_1)^{-1})})$, $(\lpro \der{D}'_1*(\omega'_1)^{-1}\rpro, \{j'_X\}_{X\in \Delta( \der{D}'_1*(\omega'_1)^{-1})})$, $(\lpro \der{D}_1*(\omega_1)^{-1}\rpro, \{j_X\}_{X\in \Delta( \der{D}_1*(\omega_1)^{-1})})$ and $(\lpro \der{D}'_1*(\omega'_1)^{-1}\rpro, \{j'_X\}_{X\in \Delta( \der{D}'_1*(\omega'_1)^{-1})})$ be colimiting cocones. 
		By \Cref{rem:dett} we get the following diagram, in which the two solid squares are pushouts:
		\[\xymatrix@C=45pt{\pi(G_{2,0}) \ar@/^.4cm/[rrr]^{\id{\pi(G_{2,0})}}\ar@{>->}[d]_{\iota_{1, G_{1,n+1}} \circ \omega_1}\ar[r]_{\iota_{2, G_{2,0}} \circ \alpha_2}& \lpro \der{D}_2 \rpro \ar@{>->}[d]^{q_2}\ar[r]_{\xi_\tau} & \lpro \der{D}'_2 \rpro \ar@{>->}[d]_{q'_2} &\pi(G'_{2,0}) \ar@{>->}[d]^{\iota'_{1, G'_{1,n+1}} \circ \omega'_1} \ar[l]^{\iota'_{2, G'_{2,0}} \circ \alpha'_2}\\ \lpro \der{D}_1 \rpro \ar@/_.4cm/[rrr]_{\xi_\sigma} \ar[r]^{q_1} & \tpro{D} \ar@{.>}[r]^{\xi_{\sigma+\tau}} & \lpro \der{D}'\rpro & \lpro \der{D}'_1\rpro  \ar[l]_{q'_1}}\]
		
		Moreover, we can point out that, for every index $i\in [0,n+t+2]$ we have
		
		\[\iota_{G_i}=\begin{cases}q_1\circ \iota_{1, G_{1,i}} & i \leq n\\
			q_1\circ \iota_{1, G_{1, i}}\circ \omega_1& i=n+1\\
			q_2\circ \iota_{2, G_{2,i-(n+1)}} & i > n+1 
		\end{cases} \qquad \iota'_{G'_i}=\begin{cases}q'_1\circ \iota'_{1, G'_{1,i}} & i \leq n\\
			q'_1\circ \iota'_{1, G'_{1, i}}\circ \omega'_1& i=n+1\\
			q'_2\circ \iota'_{2, G'_{2,i-(n+1)}} & i > n+1 
		\end{cases}\]
		
		Now, on the one hand, the existence of the dotted arrow $\xi_{\sigma+\tau}\colon \tpro{D}\to \lpro \der{D}'\rpro$ in the diagram above follows from the following computation:
		\begin{align*}
			q'_1\circ \xi_\sigma \circ \iota_{1, G_{1,n+1}} \circ \omega_1&=q'_1\circ \iota'_{1,G'_{1, n+1}} \circ \omega'_1\\&= q'_2 \circ \iota'_{2, G'_{2,0}} \circ \alpha'_2\\&=q'_2\circ \xi_\tau \circ \iota_{2, G_{2,0}} \circ \alpha_2
		\end{align*} 
		On the other hand, we can repeat the same computation for $\xi^{-1}_\sigma$ and $\xi^{-1}_\tau$ to get
		\begin{align*}
			q_1\circ \xi^{-1}_\sigma \circ \iota'_{1, G'_{1,n+1}} \circ \omega'_1&=q_1\circ \iota_{1,G_{1, n+1}} \circ \omega_1\\&= q_2 \circ \iota_{2, G_{2,0}} \circ \alpha_2\\&=q_2\circ \xi^{-1}_\tau \circ \iota'_{2, G'_{2,0}} \circ \alpha'_2
		\end{align*} 
		Hence there exists a $\psi\colon \lpro \der{D}'\rpro \to \tpro{D}$ such that:
		\[\psi \circ q'_1=q_1\circ \xi^{-1}_\sigma \qquad \psi \circ q'_2=q_2\circ \xi^{-1}_\tau\]
		
		Furthermore, the computations below show that $\psi$ is the inverse of $\xi_{\sigma+\tau}$.
		\[\begin{split}
			\psi \circ \xi_{\sigma+\tau}\circ  q_1&=\psi \circ q'_1\circ \xi_\sigma \\&=q_1\circ \xi^{-1}_{\sigma}\circ \xi_\sigma\\&=q_1\\\\
			\xi_{\sigma+\tau}\circ \psi \circ q'_1&=\xi_{\sigma+\tau}\circ q_1\circ \xi^{-1}_\sigma \\&=q'_1\circ \xi_{\sigma}\circ \xi^{-1}_\sigma\\&=q'_1
		\end{split}\qquad \begin{split}
			\psi \circ \xi_{\sigma+\tau}\circ  q_2&=\psi \circ q'_2\circ \xi_\tau \\&=q_2\circ \xi^{-1}_{\tau}\circ \xi_\tau\\&=q_2\\ \\ 	\xi_{\sigma+\tau}\circ \psi \circ q'_2&=\xi_{\sigma+\tau}\circ q_2\circ \xi^{-1}_\tau \\&=q'_2\circ \xi_{\tau}\circ \xi^{-1}_\tau\\&=q'_2
		\end{split}\]
		To conclude we have to show that $\xi_{\sigma+\tau}$ fits in the diagrams of \Cref{def:permcon}.  For the ones involving the decorations we have:
		\[\begin{split}
			\xi_{\sigma+\tau} \circ \iota_{G_{0}}\circ \alpha&=\xi_{\sigma+\tau} \circ q_1 \circ \iota_{1,G_{1,0}}\circ \alpha_1\\&=q_1'\circ \xi_\sigma \circ  \iota_{1,G_{1,0}} \alpha_1\\&=q'_1\circ \iota'_{1, G'_{1,0}}\circ \alpha'_1\\&=\iota'_{G'_0}\circ \alpha'
		\end{split}  \qquad 
		\begin{split}
			\xi_{\sigma+\tau} \circ \iota_{G_{n+t+2}}\circ \omega&=\xi_{\sigma+\tau} \circ q_2 \circ \iota_{2,G_{2,t+1}}\circ \omega_2\\&=q'_2\circ \xi_\tau \circ  \iota_{2,G_{2,t+1}} \omega_2\\&=q'_2\circ \iota'_{2, G'_{2,t+1}}\circ \omega'_2\\&=\iota'_{G'_{n+t+2}}\circ \omega'
		\end{split}  \]
		
		Let now $i$ be an index in $[0, n+t+1]$, we proceed splitting the cases.
		\begin{itemize}
			\item $i$ is in $[0,n]$. Then we have
			\begin{align*}
				\xi_{\sigma+\tau}\circ \iota_{G_i} \circ m_i &= \xi_{\sigma+\tau} \circ q_1\circ \iota_{1, G_{1,i}}\circ m_{1,i}\\&=q'_1 \circ \xi_\sigma \circ \iota_{1, G_{1,i}}\circ m_{1,i}\\&= q'_1\circ \iota'_{1, G'_{1,\sigma(i)}}\circ m'_{1,\sigma(i)}\\&=\iota'_{G'_{\sigma(i)}}\circ m'_{\sigma(i)}
			\end{align*}
			Now, suppose that $i\neq n$, then 
			\begin{align*}
				\xi_{\sigma+\tau}\circ \iota_{G_{i+1}} \circ h_i &= \xi_{\sigma+\tau} \circ q_1\circ \iota_{1, G_{1,{i+1}}}\circ h_{1,i}\\&=q'_1\circ \xi_\sigma \circ \iota_{1, G_{1,{i+1}}}\circ h_{1,i}\\&= q'_1\circ \iota'_{1, G'_{1,\sigma(i)+1}}\circ h'_{1,\sigma(i)}
			\end{align*}
			and we have two subcases. If $\sigma(i)\neq n$, then \[h'_{1, \sigma(i)}=h'_{\sigma(i)} \qquad q'_1\circ \iota'_{1, G'_{1,\sigma(i)+1}}=
			\iota'_{G'_{\sigma(i)+1}}\] 
			and we can conclude. Otherwise, we have
			\begin{align*}q'_1\circ \iota'_{1, G'_{1,n+1}}\circ h'_{1,n}&=q'_1 \circ \iota'_{1,G'_{1,n+1}} \circ \omega'_1\circ (\omega')^{-1}_1 \circ h'_{1,n}\\&=\iota'_{G'_{n+1}}\circ h'_n
			\end{align*}
			yielding again the wanted equality
			\[\xi_{\sigma+\tau}\circ \iota_{G_{i+1}} \circ h_i=\iota'_{G'_{\sigma(i)+1}}\circ h'_{\sigma(i)}\]
			Finally, if $i=n$ we get
			\begin{align*}
				\xi_{\sigma+\tau}\circ \iota_{G_{n+1}} \circ h_n &= \xi_{\sigma+\tau} \circ \iota_{\pi(G_{1,n+1})} \circ \omega^{-1}_1\circ h_{1,n}\\&= \xi_{\sigma+\tau} \circ q_1\circ \iota_{1, G_{1,n+1}} \circ \omega_1 \circ \omega^{-1}_1 \circ h_{1,n}\\&=q'_1 \circ \xi_\sigma \circ \iota_{1, G_{1,n+1}} \circ h_{1,n}\\&=q'_1\circ \iota'_{1, G_{1,\sigma(n)+1}} \circ h'_{1,\sigma(n)}
			\end{align*}
			We have again two cases. If $\sigma(n)\neq n$ then we can conclude at once since 
			\[h'_{1, \sigma(n)}=h'_{\sigma(n)} \qquad q'_1\circ \iota'_{1, G'_{1,\sigma(n)+1}}=
			\iota'_{G'_{\sigma(n)+1}}\] 
			While, if $\sigma(n)=n$, then we can use the identities
			\[(\omega'_1)^{-1}\circ h'_{1, n}=h'_{n} \qquad q'_1\circ \iota'_{1, G'_{1, n+1}}=
			\iota'_{G'_{n+1}}\circ (\omega'_1)^{-1}\] 
			to get
			\begin{align*}
				\xi_{\sigma+\tau}\circ \iota_{G_{n+1}} \circ h_n &= q'_1\circ \iota'_{1, G_{1,n+1}} \circ h'_{1,n}\\&=\iota'_{G'_{n+1}}\circ (\omega'_1)^{-1}\circ h'_{1,n}\\&=\iota'_{G'_{n+1}} h'_{n}
			\end{align*}
			\item $i$ is in $[n+1, n+t+1]$- We proceed  similarly to the point above. First of all we have
			\begin{align*}
				\xi_{\sigma+\tau}\circ \iota_{G_{i+1}} \circ h_i &= \xi_{\sigma+\tau} \circ q_2\circ \iota_{1, G_{2,i-n}}\circ h_{2,i-(n+1)}\\&=q'_2 \circ \xi_\tau \circ \iota_{2, G_{2,i-n}}\circ h_{2,i-(n+1)}\\&= q'_2\circ \iota'_{2, G'_{2,\tau(i-n)}}\circ h'_{2,\tau(i-(n+1))}\\&=\iota'_{G'_{\tau(i-n)}}\circ h'_{n+1+\tau(i-(n+1))}
			\end{align*}
			Next, supposing that $i\neq n+1$:	
			\begin{align*}
				\xi_{\sigma+\tau}\circ \iota_{G_{i}} \circ m_i &= \xi_{\sigma+\tau} \circ q_2\circ \iota_{2, G_{2,i-(n+1)}}\circ m_{2,i-(n+1)}\\&=q'_2\circ \xi_\tau \circ \iota_{2, G_{2,{i-(n+1)}}}\circ m_{2,i-(n+1)}\\&= q'_2\circ \iota'_{2, G'_{2,\tau(i-(n+1))}}\circ m'_{2,\tau(i-(n+1))}
			\end{align*}
			and we have again two possibilities. If $\tau(i-(n+1))\neq 0$, then the thesis follows from the equalities \[m'_{2, \tau(i-(n+1))}=m'_{(\sigma+\tau)(i)} \qquad q'_2\circ \iota'_{2, G'_{2,\tau(i-(n+1))}}=
			\iota'_{G'_{(\sigma+\tau)(i)}}\] 
			Suppose, instead, that $\tau(i-(n+1))= 0$, then we have
			\begin{align*} q'_2\circ \iota'_{2, G'_{2,0}}\circ m'_{2,0}&=q'_2 \circ \iota'_{2,G'_{2,0}} \circ \alpha'_2 \circ \alpha'^{-1}_2 \circ m'_{2,0}\\&=q'_1\circ \iota'_{1, G'_{1, n+1}}\circ \omega'_1\circ \alpha'^{-1}_2 \circ m'_{2,0}\\&=\iota'_{G'_{n+1}}\circ m'_{n+1}
			\end{align*}
			allowing us to conclude again.	
			
			We are left with the case $i=n+1$. Let us compute
			\begin{align*}
				\xi_{\sigma+\tau}\circ \iota_{G_{n+1}} \circ m_{n+1} &= \xi_{\sigma+\tau} \circ \iota_{\pi(G_{2,0})} \circ (\alpha'_2)^{-1}\circ m_{2,0}\\&=\xi_{\sigma+\tau} \circ q_1\circ \iota_{1, G_{1,n+1}} \circ \omega_1 \circ (\alpha'_2)^{-1}\circ m_{2,0}\\&=\xi_{\sigma+\tau} \circ q'_2 \circ \iota'_{2,G'_{2,0}} \circ \alpha'_2 \circ (\alpha'_2)^{-1}\circ m_{2,0} \\&=q'_2 \circ \xi_\tau \circ \iota_{2, G_{2,0}} \circ m_{2,0}\\&=q'_2\circ \iota'_{2, G_{2,\tau(0)}} \circ m'_{2,\tau(0)}
			\end{align*}
			We have again two cases. If $\tau(0)\neq 0$ then $(\sigma+\tau)(i)\neq n+1$ and we can conclude at once since 
			\[m'_{2, \tau(0)}=m'_{(\sigma+\tau)(n+1)} \qquad q'_2\circ \iota'_{2, G'_{2,\tau(0)}}=
			\iota'_{G'_{(\sigma+\tau)(0)}}\] 
			If $\tau(0)=0$, so that  $(\sigma+\tau)(0)= n+1$ we appeal to the equalities
			\[(\alpha')^{-1}_2\circ m'_{2, 0}=m'_{n+1} \qquad q'_2\circ \iota'_{2, G'_{2, 0}}=
			\iota'_{G'_{n+1}}\circ (\alpha'_2)^{-1}\] 
			to get
			\begin{align*}
				\xi_{\sigma+\tau}\circ \iota_{G_{n+1}} \circ m_{n+1} &= q'_2\circ \iota'_{2, G_{2,\tau(0)}} \circ m'_{2,\tau(0)}\\&=	\iota'_{G'_{n+1}}\circ (\alpha'_2)^{-1}\circ m'_{2,0}\\&=\iota'_{G'_{n+1}}\circ  m'_{n+1}
			\end{align*}
		\end{itemize}
		The thesis now follows at once.	 \qedhere 
\end{itemize} \end{proof}

\begin{definition}
Let $(\X, \R)$ be a left-linear DPO rewriting system. The category $\ppi$ is defined as follows: 
	\begin{itemize}
	\item objects are isomorphism classes of objects of $\X$;
	\item an arrow $[G]\to [H]$ is an equivalence class $[\der{D}, \alpha, \omega]_p$ of abstract decorated derivations between $[G]$ and $[H]$;
	\item given two arrows $[\der{D}_1, \alpha_1, \omega_1] _p\colon [G]\to [H]$ and $[\der{D}_2, \alpha_2, \omega_2] _p\colon [H]\to [K]$, their composition is defined as
	\[[\der{D}_2, \alpha_2, \omega_2]_p \circ [\der{D}_1, \alpha_1, \omega_1]_p:=[(\der{D}_1, \alpha_1, \omega_1) \cdot (\der{D}_2, \alpha_2, \omega_2)]_p\] 
	\item the identity on $[G]$ is $[\{G\}, \alpha, \alpha]_p$, where $\alpha$ is any isomorphism $\pi(G)\to G$.	\end{itemize}
\end{definition}
\begin{remark} We can point out two facts;
	\begin{itemize}
		\item the composition 	$[\der{D}_2, \alpha_2, \omega_2]_p \circ [\der{D}_1, \alpha_1, \omega_1]_p$ coincides with the permutation equivalence class of $	[\der{D}_2, \alpha_2, \omega_2]_a \circ [\der{D}_1, \alpha_1, \omega_1]_a$;
		\item by \Cref{rem:ugu}, $[\{G, \alpha, \alpha\}]_p=[\{G, \alpha, \alpha\}]_a$, guaranteeing that it is the neutral element with respect of the composition we have defined above.
	\end{itemize}
	
	In particular, there is a quotient functor $\Pi\colon \dpi\to \ppi$ which is the identity on objects and sends an abstraction equivalence class to its permutation equivalence one.
\end{remark}

\subsection{Uniqueness of consistent permutations}

We are now going to proceed towards the central result of this section: the uniqueness of a consistent permutation between two decorated derivations in a consuming DPO rewriting system.
 
\begin{lemma}\label{lem:impo}
Let  $(\der{D}_1, \alpha_1, \omega_1)$, $(\der{D}_2, \alpha_2, \omega_2)$ be two composable decorated derivations of length, respectively, $l_1$ and $l_2$  in a left-linear DPO-rewriting system $(\X,R)$. Let also  $(\der{D}'_1, \alpha'_1, \omega'_1)$ and $(\der{D}'_2, \alpha'_2, \omega'_2)$ be other two composable decorated derivations such that:
	\begin{align*}(\der{D}_1, \alpha_1, \omega_1)&\equiv^p_\tau(\der{D}'_1, \alpha'_1, \omega'_1)\\
	(\der{D}_1, \alpha_1, \omega_1)\cdot (\der{D}_2, \alpha_2, \omega_2)&\equiv^p_\sigma (\der{D}'_1, \alpha'_1, \omega'_1)\cdot (\der{D}'_2, \alpha'_2, \omega'_2)
	\end{align*}
	
Suppose, moreover, that the set
	\[D_{\sigma, \tau}:=\{i\in [0, l_1-1]\mid \sigma(i)\neq \tau(i)\}\]
	is non-empty and let $j$ be its minimum. If $r(\der{D}_1)$ is $\{\rho_i\}_{i=0}^{l_1-1}$ then the following hold true:
	\begin{enumerate}
		\item if $j=0$, then the rule $\rho_0$ is not consuming;
		\item if $j\neq 0$ then the rule $\rho_{j-1}$ is not consuming.
	\end{enumerate}
\end{lemma}
\begin{remark}\label{rem:minmax}It is worth to notice that the hypothesis $D_{\sigma, \tau} \neq \emptyset$ entails $l_1\neq 0$. Moreover, if $j\neq 0$, then $j-1$ is the maximum of $E_{\sigma, \tau}$.
\end{remark}
\begin{proof}
	\begin{enumerate}
		\item Let $k$ be $\sigma^{-1}(\tau(0))$ and notice that, since $\sigma(0)\neq \tau(0)$, then $0< k$.  By  the second point of  \Cref{prop:uniqu} we can consider the diagram
		\[\xymatrix@C=40pt{G_{0} \ar[drrr]_(.4){\iota_{G_0}}|(.675)\hole \ar[d]_{\iota_{1,G_{1,0}}}& L_0 \ar[r]_{m'_{\tau(0)}} \ar@/^.4cm/[rr]^{m_{k}} \ar[l]_{m_{0}} &G'_{\tau(0)} \ar[d]^(.45){\iota'_{G'_{\tau(0)}}} & G_{k} \ar[d]^{\iota_{G_k}}\\ \lpro\der{D}_1 \rpro \ar[r]_{\xi_{\tau}} &\lpro \der{D}'_1\rpro \ar[r]_{q'_1} & \lpro \der{D'}\rpro & \tpro{D} \ar[l]^{\xi_{\sigma}}}\]
		From \Cref{cor:ele}, we can conclude that there exists $c\colon L_0\to D_0$ such that $f_0\circ c=m_0$. We thus have the solid part of the commutative diagram below.
		\[\xymatrix{L_0 \ar@/^.3cm/[drr]^{\id{L_0}} \ar@{.>}[dr]_{t} \ar@/_.3cm/[ddr]_{c}\\ & K_0 \ar[d]_{k_0}\ar@{>->}[r]^{l_0}& L_0 \ar[d]^{m_0} \\& D_0 \ar@{>->}[r]_{f_0} & G_0} \]
		The internal square is an $\mathcal{M}$-pushout and thus a pullback, by \Cref{prop:pbpoad}, so that we have the existence of the dotted $t\colon L_0\to K_0$. Therefore $\id{L_0}=l_0\circ t$, proving that $l_0$ is an epimorphism. The thesis now follows from \Cref{cor:rego}. 
		\item  Let $k$ be $\sigma^{-1}(\tau(j-1))$ and notice that $\rho_{j-1}=\rho_k$. We have already noticed in \Cref{rem:minmax} that $j-1$ is the maximum of $E_{\sigma, \tau}$.  Hence, by the third point of \Cref{prop:uniqu} we have
		\[\xymatrix@C=40pt{G_{1,{j-1}} \ar[drrr]_(.4){\iota_{G_{j-1}}}|(.675)\hole \ar[d]_{\iota_{1,G_{1,j-1}}}& L_{j-1} \ar[r]_{m'_{\tau({j-1})}} \ar@/^.4cm/[rr]^{m_{k}} \ar[l]_{m_{{j-1}}} &G'_{\tau({j-1})} \ar[d]^{\iota'_{G'_{\tau({j-1})}}} & G_{k} \ar[d]^{\iota_{G_k}}\\ \lpro\der{D}_1 \rpro \ar[r]_{\xi_{\tau}} &\lpro \der{D}'_1\rpro \ar[r]_{q'_1} & \lpro \der{D'}\rpro & \tpro{D} \ar[l]^{\xi_{\sigma}}}\]
		Let $a$ be $\min(j-1, k)$, by \Cref{cor:ele},  there exists $c\colon L_{j-1}\to D_a$ such that $f_a\circ c=m_a$. As before this yields the solid part of the following diagram, in which the square is an $\mathcal{M}$-pushout.
		\[\xymatrix{L_{j-1} \ar@/^.3cm/[drr]^{\id{L_{j-1}}} \ar@{.>}[dr]_{t} \ar@/_.3cm/[ddr]_{c}\\ & K_{j-1} \ar[d]_{k_a}\ar@{>->}[r]^{l_{j-1}}& L_{j-1} \ar[d]^{m_a} \\& D_a \ar@{>->}[r]_{f_a} & G_a} \]

	From \Cref{prop:pbpoad} we can now deduce the existence of the dotted $t\colon L_{j-1}\to K_{j-1}$, from which the thesis  follows. \qedhere
	\end{enumerate}
\end{proof}


\begin{corollary}[Uniqueness of consistent permutation]\label{cor:unique}
Let $(\X,\R)$ be a consuming left-linear DPO-rewriting system. Then, for every two decorated derivations $(\der{D}, \alpha, \omega)$ and $(\der{D}', \alpha', \omega')$, there exists at most one consistent permutation between them.
\end{corollary}
\begin{proof}
Let $\sigma, \tau: [0, \lgh(\der{D})-1]\rightrightarrows [0, \lgh(\der{D})-1]$ be two consistent permutations. Let $H$  and $H'$ be, respectively, the target of $\der{D}$ and $\der{D}'$. Take two isomorphisms $\gamma:\pi(H)\to \gamma$, $\gamma':\pi(H')\to H'$, then, according to \Cref{def:conc}:
\[(\der{D}, \alpha, \omega)=(\der{D}, \alpha, \omega)\cdot (\der{D}_2, \gamma, \gamma) \qquad (\der{D}', \alpha', \omega')=(\der{D}', \alpha', \omega')\cdot (\der{D}'_2, \gamma', \gamma') \]
where $\der{D}_2$ and $\der{D}'_2$ are the empty derivation on $H$ and $H'$. Since $(\X, \R)$ is consuming, then \Cref{lem:impo} entails $D_{\sigma, \tau}=\emptyset$, from which the thesis follows.
\end{proof}

The previous results, in turn, allows us to deduce a property which is in some sense the converse of the Composition \Cref{lem:sum}.

\begin{lemma}[Restriction Lemma]\label{lem:presuffix} Let $(\X, \R)$ be a consuming left-linear DPO-rewriting system.  Let also $(\der{D}_1, \alpha_1, \omega_1)$, $(\der{D}_2, \alpha_2, \omega_2)$ be two composable decorated derivations of length, respectively, $l_1$ and $l_2$ such that there are  other two composable decorated derivations $(\der{D}_1, \alpha_1, \omega_1)$ and $(\der{D}_2, \alpha_2, \omega_2)$ with the property that
	\[[(\der{D}_1, \alpha_1, \omega_1)\cdot (\der{D}_2, \alpha_2, \omega_2)]_a\equiv^p_\sigma[(\der{D}'_1, \alpha'_1, \omega'_1)\cdot (\der{D}'_2, \alpha'_2, \omega'_2)]_a\] 

Suppose, finally, that $[\der{D_1}, \alpha_1, \omega_1]_a \equiv^p_\tau [\der{D}'_1, \alpha'_1, \omega'_1]_a$ for some $\tau$. Then the following hold true:
	\begin{enumerate}
		\item for every $i\in [0, l_2]$ there exist a unique $\zeta_i\colon G_{2,i}\to \lpro \der{D}'_2\rpro $ and a unique $\varsigma_{i}\colon G'_{2,i}\to \lpro \der{D}_2\rpro$ fitting in the diagrams below;
		\[\xymatrix@C=30pt{G_{2,i} \ar[r]^-{\iota_{2,G_{2,i}}} \ar@{.>}[d]_{\zeta_i}&\lpro \der{D}_2 \rpro \ar@{>->}[r]^{q_2}& \tpro{D} \ar[d]^{\xi_{\sigma}} &G'_{2,i} \ar[r]^-{\iota'_{2,G'_{2,i}}} \ar@{.>}[d]_{\varsigma_i}&\lpro \der{D}'_2 \rpro \ar@{>->}[r]^{q'_2}& \lpro D'\rpro \ar[d]^{\xi^{-1}_{\sigma}}\\
			\lpro \der{D}'_2 \rpro \ar@{>->}[rr]_{q'_2}&& \lpro \der{D}'\rpro & 			\lpro \der{D}_2 \rpro \ar@{>->}[rr]_{q_2}&& \lpro \der{D}\rpro }\]
	
	
		\item there exists an isomorphism $\psi\colon \lpro \der{D}_2\rpro \to \lpro \der{D}'_2\rpro$ fitting in the diagrams below; 
		\[\xymatrix@C=30pt{G_{2,i} \ar[r]^-{\iota_{2,G_{2,i}}} \ar@/_.3cm/[dr]_{\zeta_i}&\lpro \der{D}_2 \rpro \ar@{>->}[r]^{q_2} \ar@{.>}[d]_{\psi}& \tpro{D} \ar[d]^{\xi_{\sigma}} &G'_{2,i} \ar[r]^-{\iota'_{2,G'_{2,i}}} \ar@/_.3cm/[dr]_{\varsigma_i}&\lpro \der{D}'_2 \rpro\ar@{<.}[d]_{\psi}  \ar@{>->}[r]^{q'_2}& \lpro \der{D}'\rpro \ar[d]^{\xi^{-1}_{\sigma}}\\&
			\lpro \der{D}'_2 \rpro \ar@{>->}[r]_{q'_2}& \lpro \der{D}'\rpro & 		&	\lpro \der{D}_2 \rpro \ar@{>->}[r]_{q_2}& \lpro \der{D}\rpro }\]
		\item the permutation
		\[\varrho\colon [0,l_2-1]\to [0, l_2-1] \qquad i \mapsto \sigma(i+l_1)-l_1\]
		is a consistent one.
	\end{enumerate}
\end{lemma}

\begin{proof}
	To fix the notation, we will use $\der{D}=\{\dder{D}_i\}_{i=0}^{l_1+l_2-1}$ and $\der{D}'=\{\dder{D}_i\}_{i=0}^{l_1+l_2-1}$ to denote the composite derivations. The direct derivations $\dder{D}_i$ and $\dder{D}_i$ are given by the two diagrams below.
	\[\xymatrix{L_i \ar[d]_{m_i}& K_i \ar[r]^{r_i} \ar@{>->}[l]_{l_i} \ar[d]_{k_i}& R_i \ar[d]^{h_i}& L'_i \ar[d]_{m'_i} & K'_i \ar[d]_{k'_i} \ar[r]^{r'_i} \ar@{>->}[l]_{l'_i} & R'_i \ar[d]^{h'_i}\\ G_i & D_i \ar[r]_{g_i} \ar@{>->}[l]^{f_i} & G_{i+1} & G'_i & D'_i \ar[r]_{g'_i} \ar@{>->}[l]^{f'_i} & G'_{i+1}}\]
	
	We can also notice that,  by \Cref{lem:impo}, $\sigma(i)=\tau(i)$ for every $i\in  [0, l_1-1]$. In particular, this implies that the image  $[l_1, l_1+l_2-1]$ through $\sigma$ is again the interval $[l_1, l_1+l_2-1]$, that $\varrho$ is well-defined and $\sigma=\tau +\varrho$.  Moreover, if $l_1\neq 0$, then $l_1-1$ is the maximum of $E_{\sigma,\tau}$. 
	
	\begin{enumerate}
		\item 		To prove the existence of $\zeta_i$ we proceed by induction on $i\in [0, l_2]$.
		
		\smallskip \noindent $i=0$. We know that $G_{2,0}$ and $G'_{2,0}$ are isomorphic to, respectively, $G_{1, l_1}$ and $G'_{1, l_1}$. By \Cref{prop:abs}, $\tau$ is a consistent permutation between $(\der{D}_1, \alpha_1, \omega_1)$ and $(\der{D}'_1, \alpha'_1, \omega'_1)$, so that $G_{1,l_1}$ and $G'_{1, l_1}$ are isomorphic too. We can therefore consider as $\zeta_0$ the arrow $\iota'_{2, G'_{2,0}}\circ (\alpha'_2)\circ \alpha^{-1}_2$. Now, by \Cref{rem:dett,prop:uniqu}, all subdiagrams of the following diagram commute, so that the whole diagram is commutative and the thesis follows.
		\[\xymatrix@R=10pt@C=35pt{G_{2,0}\ar[r]^{\iota_{2, G_{2,0}}} \ar[dd]_{\alpha^{-1}_2}&\lpro \der{D}_2\rpro \ar[rr]^{q_2}&&\tpro{D}\ar[dddd]^{\xi_{\sigma}}\\& G_{1,l_1} \ar[r]^{\iota_{1, G_{1, l_1}}}& \lpro \der{D}_1\ar[dd]^{\xi_\tau} \ar[ur]_{q_1} \rpro \\\pi(G_{2,0}) \ar[ur]^{\omega_1}\ar[dr]_{\omega'_1} \\ & G'_{1,l_1} \ar[r]_{\iota'_{1, G'_{1, l_1}}}& \lpro \der{D}'_1 \rpro \ar[dr]^{q_1}\\G'_{2,0} \ar@{<-}[uu]^{\alpha'_2}\ar[r]_{\iota'_{2, G'_{2,0}}}&\lpro \der{D}'_2\rpro \ar[rr]_{q'_2}&& \lpro \der{D}'\rpro }\]
		
		\smallskip \noindent $i>0$.  Let $j$ be $\sigma(i-1+l_1)-l_1$. We have already observed that $\sigma(i-1+l_1)$ belongs to $[l_1, l_1+l_2-1]$, so that $j$ is an element of $[0, l_2-1]$.  By consistency of $\sigma$ we get that, if $\der{D}_1=\{\dder{D}_{2,s}\}_{s=0}^{l_2-1}$ and $\der{D}'_2=\{\dder{D}'_{1,s}\}_{s=0}^{l_2-1}$, then $\dder{D}_{2, i-1}$ and $\dder{D}'_{2, j}$ are given by the following two direct derivations.
		\[\xymatrix@C=30pt{L_{2,i-1} \ar[d]_{m_{2,i-1}} & K_{2,i-1} \ar[d]_{k_{2, i-1}} \ar[r]^{r_{2,i-1}} \ar@{>->}[l]_{l_{2,i-1}} & R_{2,i-1} \ar[d]^{h_{2,{i-1}}} & L_{2,i-1} \ar[d]_{m'_{2,j}} & K_{2,i-1}\ar[d]_{k'_{2, j}} \ar[r]^{r_{2,i-1}} \ar@{>->}[l]_{l_{2,i-1}}& R_{2,i-1} \ar[d]^{h'_{2, j}}\\ G_{2,i-1} & D_{2,i-1} \ar@{>->}[l]^{f_{2, i-1}} \ar[r]_{g_{2, i-1}} & G_{2,i} & G'_{2,j} & D'_{2,j}\ar@{>->}[l]^{f'_{2, j}} \ar[r]_{g'_{2, j}} & G'_{2,j+1} }\]
		
		Notice, moreover, that these two derivations coincide with $\dder{D}_{i-1+l_1}$ and  $\dder{D}'_{\sigma(i-1+l_1)}$.	Now, if we compute, from \Cref{rem:coproj,rem:dett2} we get:
		\begin{align*}
			q'_2\circ \iota'_{2, R'_{2, j}}\circ r_{2, j}&=q'_2\circ \iota'_{2, K'_{2,j}}\\&=q'_2\circ \iota'_{2,D'_{2,j}}\circ k'_{2,j}\\&= \iota'_{D'_{\sigma(i-1+l_1)}}\circ k'_{\sigma(i-1+l_1)}\\&=\xi_\sigma \circ \iota_{D_{i-1+l_1}} \circ k_{i-1+l_1}\\&=\xi_\sigma \circ q_2 \circ \iota_{2, D_{2,i-1}}\circ k_{2, i-1}\\&=\xi_\sigma \circ q_2 \circ \iota_{2, G_{2,i-1}}\circ f_{2,i-1}\circ k_{2,i-1}\\&=q'_2\circ \zeta_{i-1}\circ f_{2, i-1}\circ k_{2, i-1}
		\end{align*}
		
		Since $q'_2$ is in $\mathcal{M}$, then 
		\[\zeta_i\circ f_{2, i-1}\circ k_{2, i-1}=\iota'_{2, R'_{2, j}}\circ r_{2, j}\]
		Therefore there is a unique arrow $\zeta_i\colon G_{2, i}\to \lpro \der{D}'_2\rpro$ making the diagram below commutative.
		\[\xymatrix{K_{2, i-1} \ar[d]_{k_{2, i-1}} \ar[r]^{r_{2, i-1}} & R_{2,i-1} \ar[d]_{h_{2, i-1}} \ar@/^.3cm/[ddr]^{\iota'_{2, R'_{2,j}}}\\
		D_{2, i-1} \ar[r]_{g_{2, i-1}} \ar@/_.2cm/@{>->}[dr]_{f_{2, i-1}}& G_{2, i} \ar@{.>}[dr]^{\zeta_i} \\ &G_{2, i-1}\ar[r]_{\zeta_{i-1}}& \lpro \der{D}'_2\rpro}\] 
		
		Finally, we have to check that this $\zeta_i$ fits in the wanted rectangle. To see this, since $G_{2,i}$ is obtained as a pushout, it is enough to perform the following computations, again with the aid of \Cref{rem:coproj,rem:dett2}:
		\[\begin{split}
		q'_2\circ \zeta_i\circ h_{2, i-1}&=q'_2\circ \iota'_{2, R'_{2,j}}\\&=\iota'_{R'_{\sigma(i-1+l_1)}} \\&=\xi_\sigma \circ \iota'_{R_{i-1+l_1}}\\&=\xi_\sigma \circ q_2\circ \iota_{2, R_{2,i-1}}\\&=\xi_\sigma \circ q_2\circ \iota_{2, G_{2,i}}\circ h_{2,i-1}
		\end{split}\qquad \begin{split}
		q'_2\circ \zeta_i\circ g_{2, i-1}&=q'_2\circ \zeta_{i-1}\circ f_{2, i-1}\\&=\xi_\sigma\circ q_2\circ \iota_{2, G_{2,i-1}}\circ f_{2, i-1} \\&=\xi_\sigma \circ q_2\circ \iota_{2, D_{2, i-1}}\\&=\xi_\sigma \circ q_2\circ \iota_{2, G_{2, i-1}}\circ g_{2,i-1}\\&
		\end{split}\]		
	
		The uniqueness half of the thesis follows at once since $q'_2$ is in $\mathcal{M}$. 
		
		To prove the existence of $\varsigma_{i}$, it is enough to apply the previous point to $\sigma^{-1}$ and $\tau^{-1}$, noticing that, by \Cref{rem:inversa} and \Cref{prop:isouno}, $\xi^{-1}_\sigma$ must coincide with $\xi_{\sigma^{-1}}$.
		
		\item We can use the arrows $\zeta_i$  and $\varsigma_{i}$ constructed in point $1$ to define a cocone $(\lpro \der{D}'_2\rpro, \{c_X\}_{X\in \Deltamin(\der{D}_2)})$ on $\Deltamin(\der{D}_2)$ and another one $(\lpro \der{D}_2\rpro, \{d_Y\}_{Y\in \Deltamin(\der{D}'_2)})$ on $\Deltamin(\der{D}'_2)$  putting, for every $i\in [0, l_2]$
		\[c_X:=\begin{cases}
			\zeta_i & X=G_{2,i}\\
			\zeta_{i} \circ f_{2,i} & X=D_{2,i}
		\end{cases} \qquad d_Y:=\begin{cases}
		\varsigma_i & Y=G'_{2,i}\\
		\varsigma_{i} \circ f'_{2,i} & Y=D'_{2,i}
		\end{cases} \] 
		
		Notice that, by the construction provided in the previous point, for every $i\in [1, l_2]$ we have
		\[\zeta_i\circ g_{2, i-1}=\zeta_{i-1}\circ f_{2,i-1} \qquad \varsigma_i\circ g'_{2, i-1}=\varsigma_{i-1}\circ f'_{2,i-1} \]
		so that $(\lpro \der{D}'_2\rpro, \{c_X\}_{X\in \Deltamin(\der{D}_2)})$ and $(\lpro \der{D}_2\rpro, \{c_Y\}_{Y\in \Deltamin(\der{D}'_2)})$ are really cocones.
		
		Thus we can consider the induced arrows $\psi\colon \lpro \der{D}_2\rpro \to \lpro \der{D}'_2\rpro$ and $\phi\colon \lpro \der{D}'_2\rpro \to \lpro \der{D}_2\rpro$. Now, on the one hand given $X\in \Deltamin(\der{D}_2)$ we have:
		\begin{align*}
			q'_2\circ \psi \circ \iota_{2,X}&=q'_2\circ c_X\\&=\begin{cases}
				q'_2\circ  \zeta_i & X =G_{2,i}\\
				q'_2 \circ  \zeta_{i}\circ f_{2,i} & X={D_{2,i}}
			\end{cases}\\&=\begin{cases}
			\xi_\sigma \circ q_2\circ \iota_{2, G_{2,i}} & X =G_{2,i}\\
			\xi_\sigma \circ q_2\circ \iota_{2, G_{2,i}} \circ f_{2,i} & X={D_{2,i}}
			\end{cases}\\&=\begin{cases}
			\xi_\sigma \circ q_2\circ \iota_{2, G_{2,i}} & X =G_{2,i}\\
			\xi_\sigma \circ q_2\circ \iota_{2, D_{2,i}} & X={D_{2,i}}
			\end{cases}\\&= \xi_\sigma \circ q_2\circ \iota_{2, X}
		\end{align*}
		so that $q'_2\circ \psi = \xi_{\sigma} \circ q_2$. On the other hand, for every $Y\in \Deltamin(\der{D}'_2)$:
		\begin{align*}
	q_2\circ \phi \circ \iota'_{2,Y}&=q_2\circ d_Y\\&=\begin{cases}
		q_2\circ  \varsigma_i & Y =G'_{2,i}\\
		q_2 \circ \varsigma_{i}\circ f'_{2,i} & Y={D'_{2,i}}
	\end{cases}\\&=\begin{cases}
		\xi^{-1}_\sigma \circ q'_2\circ \iota'_{2, G'_{2,i}} & Y =G'_{2,i}\\
		\xi^{-1}_\sigma \circ q'_2\circ \iota'_{2, G'_{2,i}} \circ f'_{2,i} & Y={D'_{2,i}}
	\end{cases}\\&=\begin{cases}
		\xi^{-1}_\sigma \circ q'_2\circ \iota'_{2, G'_{2,i}} & Y =G'_{2,i}\\
		\xi^{-1}_\sigma \circ q'_2\circ \iota'_{2, D'_{2,i}} & Y={D'_{2,i}}
	\end{cases}\\&= \xi^{-1}_\sigma \circ q'_2\circ \iota_{2, Y}
		\end{align*}
	therefore we also have $q_2\circ \phi =\xi^{-1}_{\sigma}\circ q'_2$.  If we compute further, given $X\in \Deltamin(\der{D}_2)$ we get:
		\begin{align*}
			q_2\circ \phi \circ \psi \circ \iota_{2, X}&=q'_2\circ \phi \circ c_X\\&=\begin{cases}
				q_2\circ \phi \circ \zeta_i & X=G_{2,i}\\
				q_2\circ \phi \circ \zeta_{i} \circ f_{2,i} & X=D_{2,i}
			\end{cases}\\&=\begin{cases}
				\xi^{-1}_{\sigma}\circ q'_2 \circ \zeta_i & X=G_{2,i}\\
				\xi^{-1}_{\sigma}\circ q'_2 \circ \zeta_{i} \circ f_{2,i} & X=D_{2,i}
			\end{cases}\\&=\begin{cases}
				\xi^{-1}_{\sigma} \circ \xi_\sigma \circ q_2\circ \iota_{2, G_{2,i}} & X=G_{2,i}\\
				\xi^{-1}_{\sigma}\circ \xi_\sigma \circ q_2\circ \iota_{2, G_{2,i}}  \circ f_{2,i} & X=D_{2,i}
			\end{cases}\\&=\begin{cases}
				q_2\circ \iota_{2, G_{2,i}} & X=G_{2,i}\\
				q_2\circ \iota_{2, D_{2,i}} & X=D_{2,i}
			\end{cases}\\&=q_2\circ \iota_{2,X}
		\end{align*}
		Hence $q_2\circ \phi \circ \psi = q_2$ and, since $q_2$ is mono this entails $\phi \circ \psi =\id{\lpro \der{D}_2 \rpro}$.
		
		Similarly, if  $Y$ is an object of $\Deltamin(\der{D}_2)$ then:
				\begin{align*}
			q'_2\circ \psi \circ \phi \circ \iota'_{2, Y}&=q_2\circ \psi \circ d_Y\\&=\begin{cases}
				q'_2\circ \psi \circ \varsigma_i & Y=G'_{2,i}\\
				q'_2\circ \psi \circ \varsigma_{i} \circ f'_{2,i} & Y=D'_{2,i}
			\end{cases}\\&=\begin{cases}
				\xi_{\sigma} \circ q_2\circ \varsigma_i & Y=G'_{2,i}\\
				\xi_{\sigma} \circ q_2 \circ \varsigma_{i} \circ f'_{2,i} & Y=D'_{2,i}
			\end{cases}\\&=\begin{cases}
				\xi_{\sigma} \circ \xi^{-1}_\sigma \circ q'_2\circ \iota'_{2, G'_{2,i}} & Y=G'_{2,i}\\
				\xi_{\sigma}\circ \xi^{-1}_\sigma \circ  q'_2\circ \iota'_{2, G'_{2,i}}  \circ f'_{2,i} & Y=D'_{2,i}
			\end{cases}\\&=\begin{cases}
				q'_2\circ \iota'_{2, G'_{2,i}} & X=G'_{2,i}\\
				q'_2\circ \iota'_{2, D'_{2,i}} & X=D'_{2,i}
			\end{cases}\\&=q'_2\circ \iota'_{2,X}
		\end{align*}
		As before, this implies $q'_2\circ \psi \circ \phi= q'_2$, showing that $\phi$ is the inverse of $\psi$.
		
		\item Let us consider the isomorphism $\psi \colon \lpro \der{D}_2\rpro \to \lpro \der{D}'_2\rpro$ constructed in the previous point. To see that it witnesses the consistency we need to split the cases.
		
		\smallskip \noindent $l_1=0$. In this case we have that $\varrho=\sigma$, $\tau=\id{\emptyset}$ and 
		\begin{align*}(\{G_{1,0}\}, \alpha_1, \omega_1) \cdot (\der{D}_2, \alpha_2, \omega_2)&= (\der{D}_2, \alpha_2\circ \omega^{-1}_1\circ \alpha_1, \omega_2)\\(\{G'_{1,0}\}, \alpha'_1, \omega'_1) \cdot (\der{D}'_2, \alpha'_2, \omega'_2)&= (\der{D}'_2, \alpha'_2\circ (\omega'_1)^{-1}\circ \alpha'_1, \omega'_2)
		\end{align*}
		In particular, by \Cref{rem:dett} $q_2$  and $q'_2$ are identities, so that $\psi =\xi_\sigma$. To conclude it is thus enough to show that 
			\[\psi \circ \iota_{2, G_{2,0}} \circ \alpha_2=\iota'_{2, G'_{2,0}}\circ  \alpha'_2\]
		
		Now, since $\psi$ is equal to $\xi_\sigma$ then 
		\[\psi \circ \iota_{2, G_{2,0}} \circ \alpha_2\circ \omega^{-1}_1\circ \alpha_1=\iota'_{2, G'_{2,0}}\circ  \alpha'_2\circ (\omega'_1)^{-1}\circ \alpha'_1 \]
		
		Since by \Cref{rem:empty,rem:empty2}, we know that $ \omega^{-1}_1\circ \alpha_1= (\omega'_1)^{-1}\circ \alpha'_1$, the thesis now follows.
	

		\smallskip \noindent $l_2=0$ and $l_1\neq0$. Then $\varrho=\id{\emptyset}$, $\tau = \sigma$ and 
		\begin{align*}(\der{D}_1, \alpha_1, \omega_1) \cdot (\{G_{2,0}\}, \alpha_2, \omega_2)
			&=(\der{D}_1, \alpha_1, \omega_1 \circ \alpha_2^{-1}\circ \omega_2)\\(\der{D}'_1, \alpha'_1, \omega'_1) \cdot (\{G'_{2,0}\}, \alpha'_2, \omega'_2)&=(\der{D}'_1, \alpha'_1, \omega'_1 \circ (\alpha'_2)^{-1}\circ \omega'_2)
		\end{align*}
	
	Now, by the construction in point $1$, we can start computing to get:
	\begin{align*}
		 \psi \circ \iota_{2, G_{2,0}}\circ \alpha_2&=\zeta_0\circ \alpha_2\\&=  \alpha'_2\circ \alpha^{-1}_2\circ \alpha_2 \\&=\alpha'_2
	\end{align*}
	
		Now, \Cref{rem:dett} we have the following equalities:
		\[q_2=\iota_{G_{l_1}}\circ  \omega_1  \circ \alpha^{-1}_2 \circ \iota^{-1}_{2,G_{2,0}} \qquad q'_2=\iota'_{G_{l_1}}\circ  \omega'_1  \circ (\alpha'_2)^{-1} \circ (\iota'_{2,G_{2,0}})^{-1}\]
	
	Moreover, by hypothesis $\sigma$ is a consistent permutation, so that:
		\[\xi_\sigma\circ \iota_{G_{l_1}}\circ \omega_1 \circ \alpha_2^{-1}\circ \omega_2=\iota'_{G'_{l_1}}\circ \omega'_1 \circ (\alpha'_2)^{-1}\circ \omega'_2\]
		
		If we compute, now we get:
		\begin{align*}
			q'_2\circ \psi \circ \iota_{2, G_{2,0}}\circ \omega_2 & = \xi_\sigma \circ q_2\circ \iota_{2, G_{2,0}}\circ \omega_2 \\&=\xi_\sigma \circ \iota_{G_{l_1}}\circ  \omega_1  \circ \alpha^{-1}_2 \circ \iota^{-1}_{2,G_{2,0}}\circ \iota_{2, G_{2,0}}\circ \omega_2\\&=\xi_\sigma \circ \iota_{G_{l_1}}\circ  \omega_1  \circ \alpha^{-1}_2 \circ \omega_2\\&=\iota'_{G'_{l_1}}\circ \omega'_1 \circ (\alpha'_2)^{-1}\circ \omega'_2\\&=q'_2\circ \iota'_{2, G_{2,0}}\circ \omega_2 
		\end{align*}
		The thesis follows since $q_2$ is mono.
		
		\smallskip \noindent $l_1\neq0$ and $l_2\neq 0$.  We can start noticing that, for every $i\in [0, l_2-1]$ we have
		\[\begin{split}G_{l_1+i}&=\begin{cases}
		\pi(G_{2, 0}) & i=0\\
		G_{2,i} &i\neq 0
		\end{cases} \\G'_{l_1+i}&=\begin{cases}
			\pi(G'_{2, 0}) & i=0\\
			G'_{2,i} &i\neq 0
			\end{cases} \end{split}\qquad 
	\begin{split} m_{l_1+i}&=\begin{cases}
		\alpha^{-1}_{2}\circ m_{2,0} & i=0\\
		m_{2,i} & i\neq 0
		\end{cases} \\m'_{l_1+i}&=\begin{cases}
		(\alpha'_2)^{-1}\circ m'_{2,0} & i=0\\
		m'_{2,i} & i\neq 0
		\end{cases}
	\end{split} \qquad 
	\begin{split}
h_{l_1+i}&=h_{2,i}\\h'_{l_1+i}&=h'_{2,i}
\end{split}\]

We are now going to use the properties of $q_1$ and $q_2$ proven in the previous points, in \Cref{rem:dett} and \Cref{rem:dett2}. Let us start showing the consistency of $\varrho$ with respect to the decorations.  

On the one hand, computing we get:
		\begin{align*}
\psi \circ \iota_{2, G_{2,0}}\circ \alpha_2&=  \zeta_0\circ \alpha_2\\&=  \iota'_{2, G'_{2,0}}\circ \alpha'_2\circ \alpha^{-1}_2\circ \alpha_2\\&=\iota'_{2, G'_{2,0}}\circ \alpha'_2
\end{align*}
	On the other hand, by we have
		\begin{align*}
			q'_2\circ \psi \circ \iota_{2, G_{2,l_2}}\circ \omega_2&=\xi_\sigma\circ q_2\circ \iota_{2, G_{2,l_2}}\circ \omega_2\\&=\xi_\sigma \circ \iota_{G_{l_1+l_1}} \circ \omega_2\\&=\iota'_{G'_{l_1+l_2}}\circ \omega'_2\\&=q'_2\circ \iota'_{2, G'_{2, l_2}}\circ \omega'_{2}
		\end{align*}
	But $q'_2$ is mono and then  $\psi \circ \iota_{2, G_{2,l_2}}\circ \omega_2=\iota'_{2, G'_{l_2}}\circ \omega'_{2}$ as wanted. 
		
	We proceed similarly for the back-matches:
	\begin{align*}
		q'_2\circ \psi \circ \iota_{2, G_{2,i+1}}\circ h_{2,i}&=\xi_\sigma\circ q_2\circ \iota_{2, G_{2,i+1}}\circ h_{2,i}\\&=\xi_\sigma \circ \iota_{G_{i+l_1+1}} \circ h_{i+l_1}\\&=\iota'_{G'_{\sigma(i+l_1)+1}}\circ h'_{\sigma(i+l_1)}\\&=q'_2\circ \iota'_{2, G'_{\varrho(i)+1}}\circ h'_{2, \varrho(i)}	
	\end{align*}
	Again we conclude since $q'_2$ is a monomorphism.
	
	We turn now to consistency of $\varrho$ with respect to the matches. If $i$ is an index different from $0$ then we have:
		\begin{align*}	q_2\circ \psi \circ \iota_{2, G_{2,i}}\circ m_{2,i}&=\xi_\sigma\circ q_2\circ \iota_{2, G_{2,i}}\circ m_{2,i}\\&=\xi_\sigma \circ \iota_{G_{i+l_1}} \circ m_{i+l_1}\\&=\iota'_{G'_{\sigma(i+l_1)}}\circ m'_{\sigma(i+l_1)} \\&= 
			\begin{cases}
			\iota'_{G'_{l_1}}\circ (\alpha'_2)^{-1} \circ m'_{2, 0}  &\sigma(i+l_1) = l_1\\
				\iota'_{G'_{\sigma(i+l_1)}}\circ m'_{2,\varrho(i)} & \sigma(i+l_1)\neq l_1
			\end{cases}
			\\&=\begin{cases}
			q'_2\circ \iota_{2, G'_{2,0}} \circ \alpha'_2\circ  (\alpha'_1)^{-1} \circ m'_{2, 0}& \varrho(i)=0\\
			q'_2\circ \iota'_{G'_{2, \varrho(i)}} \circ m'_{2,\varrho(i)} & \varrho(i)\neq 0
			\end{cases}\\&=q_2\circ \iota'_{2, G'_{\varrho(i)}}\circ m'_{2, \varrho(i)}
		\end{align*}
		and the thesis follows since $q'_2$ belongs to $\mathcal{M}$.
	 
	 We are thus left with the case $i=0$. The argument is more or less the same as before:
	 \begin{align*}
	 	q_2\circ \psi \circ \iota_{2, G_{2,0}}\circ m_{2,0}&=\xi_\sigma \circ q_2\circ \iota_{2, G_{2,0}}\circ m_{2,0}\\&=\xi_\sigma \circ \iota_{G_{l_1}}\circ \alpha^{-1}_2\circ m_{2,0}\\&=\xi_\sigma \circ \iota_{ G_{l_1}}\circ m_{l_1}\\&=\iota'_{G_{\sigma(l_1)}}\circ m'_{\sigma(l_1)}\\&=\begin{cases}
	 		\iota'_{G'_{l_1}}\circ (\alpha'_2)^{-1} \circ m'_{2, 0}  &\sigma(l_1) = l_1\\
	 		\iota'_{G'_{\sigma(l_1)}}\circ m'_{2,\varrho(0)} & \sigma(l_1)\neq l_1
	 	\end{cases}
	 	\\&=\begin{cases}
	 		q'_2\circ \iota_{2, G'_{2,0}} \circ \alpha'_2\circ  (\alpha'_1)^{-1} \circ m'_{2, 0}& \varrho(i)=0\\
	 		q'_2\circ \iota'_{G'_{2, \varrho(i)}} \circ m'_{2,\varrho(i)} & \varrho(i)\neq 0
	 	\end{cases}\\&=q_2\circ \iota'_{2, G'_{\varrho(i)}}\circ m'_{2, \varrho(i)}
	 \end{align*}
	 But $q_2$ is a mono and so we can conclude.	  \qedhere 
	\end{enumerate}
\end{proof}

\begin{corollary}\label{cor:epi}
	Let $(\X, \R)$ be a consuming left-linear DPO rewriting, then in $\ppi$ every arrow is epi.
\end{corollary}
\begin{proof}
Let $[\der{D}, \alpha, \omega]_p$ be an arrow in $\ppi$ between $[G]\to [H]$ and suppose that two arrows $[\der{D}_1, \alpha_1, \omega_1]_p, [\der{D}_2, \alpha_2, \omega_2]_p\colon [H]\to [K]$  are given such that 
\[[\der{D}_1, \alpha_1, \omega_1]_p\circ [\der{D}, \alpha, \omega]_p=[\der{D}_2, \alpha_2, \omega_2]_p\circ [\der{D}, \alpha, \omega]_p\]

By construction and \Cref{prop:abs}, the previous equation entails the existence of a consistent permutation $\sigma$ between $(\der{D}, \alpha, \omega)\cdot (\der{D}_1, \alpha_1, \omega_1)$ and $(\der{D}, \alpha, \omega)\cdot (\der{D}_2, \alpha_2, \omega_2)$. Since the identity is a consistent permutation between $(\der{D}, \alpha, \omega)$ and itself, the thesis follows immediately from \Cref{lem:presuffix}.
\end{proof}




