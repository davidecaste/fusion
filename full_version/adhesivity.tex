

\chapter{$\mathcal{M}$-adhesive categories}

This first section is devoted to recall the definition and the basic theory of \emph{$\mathcal{M}$-adhesive categories} \cite{azzi2019essence,ehrig2012,ehrig2014adhesive,lack2005adhesive}. 

\begin{notation} 
We stipulate here some notational conventions which will be used throughout this paper. 
\begin{itemize}\item 
	Given a category $\X$ we will not distinguish notationally between $\X$ and its class of objects: so that ``$X\in \X$'' means that $X$ belongs to the class of objects of $\X$.  
	\item 
	If $1$ is a terminal object in a category $\X$,  the unique arrow $X\to 1$ from another object $X$ will be denoted by $!_X$. Similarly, if $0$ is initial in $\X$ then $?_X$ will denote the unique arrow $0\to X$. %When $\X$ is $\Set$ and $1$ is a singleton, $\delta_x$ will denote the arrow $1\to X$ with value $x\in X$.
	\item  $\mor(\X)$, $\mon(\X)$ and $\reg(\X)$ will denote the class of all arrows, monos and regular monos of $\X$, respectively.
	
	\item Given an integer $n\in \mathbb{Z}$, $[0,n]$ will denote the set
	\[[0,n]:=\{x\in \mathbb{N}\mid x\leq n\}\] 
	Indeed, if $n<0$, then $[0,n]=\emptyset$.
\end{itemize}
\end{notation}


\section{The Van Kampen property}
The key property that $\mathcal{M}$-adhesive categories enjoy is given by  the so-called \emph{Van Kampen condition} \cite{brown1997van,johnstone2007quasitoposes,lack2005adhesive}. We will recall it and examine some of its consequences. First of all we need to recall some terminology and facts regarding subclasses of $\mor(\X)$.

\begin{definition}
	Let $\X$ be a category and $\mathcal{A}$ a  subclass of $\mor(\X)$. We say that  $\mathcal{A}$ is
	\begin{itemize}
		\item 
		\emph{stable under pushouts (pullbacks)} if for every pushout (pullbacks) square 
		\[\xymatrix{A\ar[r]^f  \ar[d]_{m}& B \ar[d]^n \\ C \ar[r]_g & D}\]
		if $m \in \mathcal{A}$ ($n\in \mathcal{A}$) then $n \in \mathcal{A}$ ($m \in \mathcal{A}$);
		\item \emph{closed under composition} if $g, f\in \mathcal{A}$ implies $g\circ f\in \mathcal{A}$ whenever $g$ and $f$ are composable;
		\item \emph{closed under decomposition} if $g\circ f, g\in \mathcal{A}$ implies $f\in \mathcal{A}$.
	\end{itemize}
\end{definition}

\begin{remark}In the previous definition, ``decomposition'' corresponds to ``left cancellation'', but we prefer to stick to the name commonly used in literature (see e.g.~\cite{habel2012mathcal}).
\end{remark}


So equipped, we can introduce the notion of \emph{$\mathcal{A}$-Van Kampen square}.
\begin{definition}[Van Kampen property] Let $\X$ be a category  and consider the two diagrams below
	\[\xymatrix@C=10pt@R=10pt{&&&&&A'\ar[dd]|\hole_(.65){a}\ar[rr]^{f'} \ar[dl]_{m'} && B' \ar[dd]^{b} \ar[dl]_{n'} \\ A \ar[dd]_{m}\ar[rr]^{f}&&B\ar[dd]^{n}&&C'  \ar[dd]_{c}\ar[rr]^(.7){g'} & & D' \ar[dd]_(.3){d}\\&&&&&A\ar[rr]|\hole^(.65){f} \ar[dl]^{m} && B \ar[dl]^{n} \\C\ar[rr]_{g} &&D&&C \ar[rr]_{g} & & D}\]
	
	Given  a class of arrows $\mathcal{A}\subseteq \mor(\X)$, we say that the left square \emph{has the Van Kampen property relatively to $\mathcal{A}$}, or that it  is a \emph{$\mathcal{A}$-Van Kampen square} if:
	\begin{enumerate}
		\item is a pushout square;
		\item 	whenever the right cube has pullbacks as back and left faces and the vertical arrows belong to $\mathcal{A}$, then its top face is a pushout if and only if the front and right faces are pullbacks.
	\end{enumerate}
	Pushout squares which enjoy the ``if'' half of the second point of the condition above are called \emph{$\mathcal{A}$-stable}. 
	
We will call a $\mor(\X)$-Van Kampen ($\mor(\X)$-stable) square simply a \emph{Van Kampen} (\emph{stable}) square.
\end{definition}

Before proceeding further, we recall this classical result about pullbacks.

\begin{lemma}\label{lem:pb1}
	Let $\X$ be a category, and consider the following diagram 	in which the right square is a pullback.
	\[\xymatrix{X \ar[d]_{a} \ar[r]^{f}& \ar[r]^{g} Y \ar[d]^{b}& Z \ar[d]^{c}\\ A \ar[r]_{h}& B \ar[r]_{k}& C}\]
	Then the whole rectangle is a pullback if and only if the left square is one.
\end{lemma}

The previous result can be dualised to get an analogous lemma for pushouts.

\begin{lemma}\label{lem:po1}
	Let $\X$ be a category, and consider the following diagram 	in which the left square is a pushout.
	\[\xymatrix{X \ar[d]_{a} \ar[r]^{f}& \ar[r]^{g} Y \ar[d]^{b}& Z \ar[d]^{c}\\ A \ar[r]_{h}& B \ar[r]_{k}& C}\]
	Then the whole rectangle is a pushout if and only if the right square is one.
\end{lemma}

The following proposition establishes a key property of $\mathcal{A}$-Van Kampen squares with a mono as a side: they are not only pushouts, but also pullbacks.
\begin{proposition}\label{prop:pbpo} Let $\mathcal{A}$ be a class of arrows stable under pushouts and containing all the isomorphisms.  If $m\colon A\to C$ is mono and belongs to $\mathcal{A}$, then every $\mathcal{A}$-Van Kampen square
	\[\xymatrix{A\ar[r]^{g} \ar[d]_{m} & B \ar[d]^{n} \\ C \ar[r]_{f}  & D}\]
	is also a pullback square and $n$ is a monomorphism.
\end{proposition}
\begin{proof} We can start considering the cube below and noticing that $n$, being the pushout of $m\in \mathcal{M}$, belongs to $\mathcal{A}$ too.
	\[\xymatrix@C=13pt@R=13pt{&A\ar[dd]|\hole_(.65){\id{A}}\ar[rr]^{g} \ar[dl]_{\id{A}} && B \ar[dd]^{\id{B}} \ar[dl]_{\id{B}} \\ A  \ar[dd]_{m}\ar[rr]^(.65){g} & & B \ar[dd]_(.3){n}\\&A\ar[rr]|\hole^(.65){g} \ar[dl]^{m} && B \ar[dl]^{n} \\C \ar[rr]_{f} & & D}\]
	By construction the top face of the cube is a pushout and the back one a pullback. The left face is a pullback because $m$ is mono. Thus the $\mathcal{A}$-Van Kampen property yields that the front and the right faces are pullbacks. 
\end{proof}

The previous proposition, in turn, allows us to establish the following results.
\begin{lemma}\label{lem:varie}Let $\mathcal{A}$ be a class of arrows stable under pullbacks, pushouts and containing all isomorphisms.  Suppose that, the left square below is $\mathcal{A}$-Van Kampen, while the vertical faces in the right cube are pullbacks.
		\[\xymatrix@C=10pt@R=10pt{&&&&&A'\ar[dd]|\hole_(.65){a}\ar[rr]^{g'} \ar[dl]_{m'} && B' \ar[dd]^{b} \ar[dl]_{n'} \\ A \ar[dd]_{m}\ar[rr]^{g}&&B\ar[dd]^{n}&&C'  \ar@{.>}[dr]_{h} _{}\ar[dd]_{c}\ar[rr]^(.7){f'} & & D' \ar@{.>}[dr]^{k}\ar[dd]_(.3){d}\\&&&&&A\ar[rr]|\hole^(.65){g} \ar[dl]^{m} && B \ar[dl]^{n} \\C\ar[rr]_{f} &&D&&C \ar[rr]_{f} & & D}\]
Supppose, moreover that $m\colon A\to C$ and $d\colon D'\to D$ are mono and that $d$ belongs to $\mathcal{A}$. Then the dotted arrow  $k\colon D'\to B$ exists if and only if the other dotted arrow  $h\colon C'\to A$ exists too.
\end{lemma}

\begin{remark}
	Notice that, since $d$ is a mono and the vertical faces are pullbacks, then $a, b$ and $c$ are monomorphisms too. Moreover, $n$ is mono by \Cref{prop:pbpo}, so that even $m'$ and $n'$ are monos.
\end{remark}

\begin{proof}
	$(\Rightarrow)$ By hypothesis there exists $k:D'\to B$ such that $n\circ k = d$. By \Cref{prop:pbpo}, the bottom face of the cube is a pullback. Thus there exists a unique $h:C'\to A$ as in the diagram below, implying the thesis.
		\[\xymatrix@C=40pt{C'  \ar@{.>}[d]^{h}\ar[r]^{f'} \ar@/_.5cm/[dd]_{c}& D' \ar@/^.5cm/[dd]^{d}\ar[d]_{k}\\ A \ar[d]^{m} \ar[r]^{g} & B \ar[d]_{n} \\  C \ar[r]_f & D}\]
	
	\smallskip \noindent 
	$(\Leftarrow)$ Let $h:C\to A$ be such that $c=m\circ h$. By the $\mathcal{A}$-Van Kampen property the top face of the given cube is a pushout. Thus we get  the dotted $k:D'\to B$ in the following diagram exists.
		\[\xymatrix@C=40pt{A' \ar@/_.5cm/[dd]_{a} \ar[d]^{m'} \ar[r]^{g'} & B' \ar[d]_{n'} \ar@/^.5cm/[dd]^{b} \\  C' \ar[d]^{h} \ar[r]_{f'} & D' \ar@{.>}[d]_{k}\\ A \ar[r]_g& B }\]
Moreover, by construction we have
\[\begin{split}
n\circ k \circ n' &= n\circ b\\&=d\circ n'\\&\\&
\end{split} \qquad \begin{split}
n\circ k \circ f' &= n\circ g\circ h\\&=f\circ m\circ h\\&=f\circ c\\&= d\circ f' \end{split}\]
We can therefore conclude that $n\circ k =d$. \qedhere 
\end{proof}

Finally, we can show that $\mathcal{A}$-stable pushouts enjoy a kind of \emph{pullback-pushout decomposition} property.

\begin{proposition}\label{prop:stab}Let $\X$ be a category and $\mathcal{A}$ a class of arrows stable under pullbacks. Suppose that, in the diagram below, the whole rectangle is an $\mathcal{A}$-stable pushout and the right square a pullback.
	\[\xymatrix{X \ar[d]_{a} \ar[r]^{f}& \ar[r]^{g} Y \ar[d]^{b}& Z \ar[d]^{c}\\ A \ar[r]_{h}& B \ar[r]_{k}& C}\]
	If the arrow $k$ is in $\mathcal{A}$ and it is a monomorphism,  then both squares are pushouts.
\end{proposition}

\begin{proof} We can begin noticing that $g$, being the pullback of $k$, is mono and in $\mathcal{A}$ too. Thus we can build the cube below, in which all the vertical faces are pullbacks, entailing that all the vertical arrows are in $\mathcal{A}$.
	\[\xymatrix@C=20pt@R=10pt{ & &X\ar[dl]_{f} \ar[ddd]^(.333333){\id{X}}|(.666666)\hole\ar[rrr]^{a} &&&A \ar[ddd]^{\id{A}} \ar[dl]^{h}\\& Y\ar[dl]_{\id{Y}} \ar[ddd]|(.333333)\hole_{\id{Y}} &&&B \ar[ddd]^{\id{B}} \ar[dl]_{\id{B}} \\Y \ar[ddd]_{g} \ar[rrr]^{b}&&& B\ar[ddd]^{k}\\&&X \ar[rrr]|(.34)\hole^{a}|(.67)\hole \ar[dl]_{f}&&& A \ar[dl]^{h}\\ & Y  \ar[dl]_{g}&&& B \ar[dl]^{k}\\ Z\ar[rrr]_{c} &&& C}\]
	By hypothesis the face is an $\mathcal{A}$-stable pushout and so its top one is a pushout. Using \Cref{lem:po1} we can conclude that the right half of the rectangle with which we have started is a pushout too. \qedhere 
\end{proof}

\section{$\mathcal{M}$-adhesivity}

In this section we will define the notion of $\mathcal{M}$-adhesivity \cite{azzi2019essence,ehrig2012,ehrig2014adhesive,heindel2009category,lack2005adhesive} and explore some of the consequence of such a property. 

\begin{definition}[$\mathcal{M}$-adhesive category]
	Let $\X$ be a category and consider a subclass $\mathcal{M}$ of the class $\mon(\X)$ of monomorphisms such that:
	\begin{enumerate}
		\item $\mathcal{M}$ contains all isomorphisms and is closed under composition;
		\item $\mathcal{M}$ is stable under pullbacks and pushouts.
	\end{enumerate} 
	We will use $m\colon X\mto Y$ to denote that an arrow $m\colon X\to Y$ belongs to $\mathcal{M}$. $\X$ is said to be \emph{$\mathcal{M}$-adhesive} if
	\begin{enumerate}
		\item for every $m\colon X\mto Y$ in $\mathcal{M}$ and $g\colon Z\to Y$, a pullback square
		\[\xymatrix{P\ar[r]^p \ar@{>->}[d]_{n}& X \ar@{>->}[d]^{m}\\ Z \ar[r]_g& Y}\]
		exists, such pullbacks will be called \emph{$\mathcal{M}$-pullbacks};
		\item for every $m\colon X\mto Y$ in $\mathcal{M}$ and $f\colon X\to Z$, a pushout square
		\[\xymatrix{X \ar[r]^f \ar@{>->}[d]_{m}& Z \ar[d]^{q}\\ Y\ar[r]_p &Q}\]
		exists, such pushouts  will be called \emph{$\mathcal{M}$-pushouts}; 
		\item  $\mathcal{M}$-pushouts are $\mathcal{M}$-Van Kampen squares.
	\end{enumerate}
	
A category $\X$ is said to be \emph{strictly $\mathcal{M}$-adhesive} if $\mathcal{M}$-pushouts are Van Kampen squares.	
\end{definition}

\begin{remark}\label{rem:diff}Our notion of $\mathcal{M}$-adhesivity follows \cite{ehrig2012,ehrig2014adhesive} and is different from the one of \cite{azzi2019essence}. What is called $\mathcal{M}$-adhesivity in that paper corresponds to our strict $\mathcal{M}$-adhesivity. Moreover, in \cite{azzi2019essence} the class $\mathcal{M}$ is assumed to be only stable under pullbacks. However, if $\mathcal{M}$ contains all split monos, then stability under pushouts can be deduced from the other axioms \cite[Prop.~$5.1.21$]{castelnovo2023thesis}.
\end{remark}


\begin{remark}\label{rem:salva} 
	\emph{Adhesivity} and \emph{quasiadhesivity} as defined in \cite{lack2005adhesive,garner2012axioms}  coincide with  strict $\mon(\X) $-adhesivity and strict $\reg(\X)$-adhesivity, respectively. 
\end{remark}

\begin{remark}
	\label{rem:iso}
	Every category $\X$ is $\mathsf{I(\X)}$-adhesive, where $\mathsf{I(\X)}$ is the class of 
	isomorphisms.
\end{remark}


\begin{remark}\label{rem:pb}
	Let $X$ be an object of $\X$, with $\X$ an $\mathcal{M}$-adhesive. category We can consider  the full subcategory $\mathcal{M}/X$ of $\X/X$ made by arrows in $\mathcal{M}$. Since $\X$ has all $\mathcal{M}$-pullbacks, for every arrow $f\colon X\to Y$ we have a functor $f^*\colon \mathcal{M}/Y\to \mathcal{M}/X $ sending $m\colon M \mto X$ to a chosen pullback of it along $f$.
	\end{remark}

A first result we can prove regards closure under decomposition of $\mathcal{M}$.

\begin{proposition}\label{prop:deco}Let  $\mathcal{A}$ be a class of arrows stable under pullbacks. For every arrow $f\colon X\to Y$ and monomorphism $m\colon Y\to Z$, if $m\circ f \in\mathcal{A}$ then $f\in \mathcal{A}$.
\end{proposition}
\begin{proof}Take the diagram
	\[\xymatrix{X \ar[d]_{\id{X}}\ar[r]^{f}& Y \ar[r]^{\id{Y}}  \ar[d]_{\id{Y}}& Y \ar[d]^{m}\\
		X \ar[r]_{f}& Y \ar[r]_{m} & Z}\]
	Since $m$ is mono the right square is a pullback, while the left square is a pullback by construction. By \Cref{lem:pb1} the whole rectangle is a pullback and the thesis follows.
\end{proof}
\begin{corollary}\label{cor:deco}
	In every $\mathcal{M}$-adhesive category $\X$, the class $\mathcal{M}$ is closed under decomposition.
\end{corollary}

Another result which can be immediately established, with the aid of \Cref{prop:pbpo}, is the following one.
\begin{proposition}\label{prop:pbpoad}
	Let $\X$ be an $\mathcal{M}$-adhesive category. Then $\mathcal{M}$-pushouts are also pullback squares.
\end{proposition}

From \Cref{prop:pbpoad}, in turn, we can derive the following corollaries.
\begin{corollary}\label{cor:rego}
	In a $\mathcal{M}$-adhesive category $\X$, every $m\in\mathcal{M}$ is a regular mono.
\end{corollary}
\begin{proof}
Let $m$ be an element of $\mathcal{M}$ and consider its pushout along itself.
\[\xymatrix{X\ar@{>->}[r]^m \ar@{>->}[d]_m& Y\ar@{>->}[d]^{f}\\Y \ar@{>->}[r]_g & Z}\]
By \Cref{prop:pbpoad} this square is a pullback, proving that $m$ is the equalizer of the arrows $f,g\colon Y\rightrightarrows Z$. \qedhere 
\end{proof}

The following result now follows at once noticing that a regular monomorphism which is also epic is automatically an isomorphism.

\begin{corollary}\label{prop:bal}
If $\X$ is an $\mathcal{M}$-adhesive categories, then every epimorphisms in $\mathcal{M}$ is an isomorphisms. In particular, every adhesive category $\X$ is \emph{balanced}: if a morphism is monic and epic, then it is an isomorphism.
\end{corollary}


$\mathcal{M}$-adhesivity is well-behaved with respect to  the comma construction \cite{mac2013categories}, as shown by the following theorem.
\begin{theorem}[\cite{ehrig2006fundamentals,lack2005adhesive}]\label{lem:comma}
	Let $\A$ and $\B$ be respectively an $\mathcal{M}$-adhesive and an $\mathcal{M}'$-adhesive category. Let also $L:\A\rightarrow \C$ be a functor that preserves $\mathcal{M}$-pushouts, and  $R:\B\rightarrow \C$ be a functor which preserves pullbacks.Then $\comma{L}{R}$ is $\cma{M}{M'}$-adhesive, where 
	\[
	\cma{M}{M}':=\{(h,k)\in \mathcal{A}(\comma{L}{R}) \mid h\in \mathcal{M}, k\in \mathcal{M}'\}\]
\end{theorem}

In particular, we can apply this result to slices over and under a given object.

\begin{corollary}\label{cor:slice}
	Let  $X$ be an object of an $\mathcal{M}$-adhesive category $\X$. Then  $\X/X$ and $X/\X$ are, respectively, $\mathcal{M}_X$- and $\mathcal{M}^X$-adhesive, where
	\[\mathcal{M}_X:=\{m\in  \mathcal{A}(\X/X) \mid m\in \mathcal{M} \} \qquad \mathcal{M}^X:=\{m\in  \mathcal{A}(X/\X) \mid m\in \mathcal{M} \}\]
\end{corollary}


Another categorical construction which preserves $\mathcal{M}$-adhesivity property is the formation of the category of functors.

\begin{theorem}[\cite{ehrig2006fundamentals,lack2005adhesive}]\label{thm:functors}
If $\X$ is an $\mathcal{M}$-adhesive category, then for every small category $\Y$, the category $\X^\Y$  of functors $\Y\to \X$ is $\mathcal{M}^{\Y}$-adhesive, where
\[\mathcal{M}^{\Y}:=\{\eta \in \mathcal{A}(\X^\Y) \mid \eta_Y \in \mathcal{M} \text{ for every } Y\in \Y\}\]
\end{theorem}

We can list various examples of $\mathcal{M}$-adhesive categories (see \cite{castelnovo2023thesis,CastelnovoGM22,lack2006toposes}).

\begin{example}\todo{Topos, ipergrafi e grafi}
\end{example}

\begin{example}
	\todo{grafi semplici}
\end{example}


\begin{example}\todo{GRAFI GERARCHICI}
\end{example}

\begin{example}\todo{term graph}
\end{example}

We end this section proving a useful property of $\mathcal{M}$-adhesive categories:  $\mathcal{M}$-pushout-pullback decomposition.


\begin{lemma}[$\mathcal{M}$-pushout-pullback decomposition]\label{lem:popb} Let $\X$ be an $\mathcal{M}$-adhesive category  and suppose that, in the diagram below, the whole rectangle is a pushout and the right square a pullback.
\[\xymatrix{X \ar[d]_{a} \ar[r]^{f}& \ar[r]^{g} Y \ar[d]^{b}& Z \ar[d]^{c}\\ A \ar[r]_{h}& B \ar[r]_{k}& C}\]
	Then the following statements hold true:
	\begin{enumerate}
\item if $a$ belongs to $\mathcal{M}$ and $k$ is a monomorphism,  then both squares are pushouts and pullbacks;
\item if $f$ and $k $ are in  $\mathcal{M}$, then both squares are pushouts and pullbacks.
	\end{enumerate}
\end{lemma}
\begin{proof}
\begin{enumerate}
	\item By \Cref{prop:stab}, it follows that both squares are pushouts, thus the thesis follows from \Cref{prop:pbpoad}.
	\item By hypothesis, $g$ is the pullback of an arrow in $\mathcal{M}$, thus it belongs to it. But then $g\circ f\in \mathcal{M}$ too  and the whole rectangle is a $\mathcal{M}$-pushout. Therefore, by \Cref{prop:pbpoad} a pullback, so that its left half is a pullback too, by \Cref{prop:pbpo}. Moreover $k\circ h$ is in $\mathcal{M}$ as the pushout of $g\circ f$ and, by \Cref{cor:deco}, we also know that $h\in \mathcal{M}$.  
	
	Using \Cref{lem:po1}, it is enough to show that the left half of the original rectangle is a pushout. We can build the following cube:
	\[\xymatrix@C=20pt@R=10pt{ & &X\ar@{>->}[dl]_{f} \ar[ddd]^(.333333){\id{X}}|(.666666)\hole\ar[rrr]^{a} &&&A \ar[ddd]^{\id{A}} \ar@{>->}[dl]^{h}\\& Y\ar[dl]_{\id{Y}} \ar[ddd]|(.333333)\hole_{\id{Y}} &&&B \ar[ddd]^{\id{B}} \ar[dl]_{\id{B}} \\Y \ar@{>->}[ddd]_{g} \ar[rrr]^{b}&&& B\ar@{>->}[ddd]^{k}\\&&X \ar[rrr]|(.34)\hole^{a}|(.67)\hole \ar@{>->}[dl]_{f}&&& A \ar@{>->}[dl]^{h}\\ & Y \ar[rrr]|(.67)\hole^{b} \ar@{>->}[dl]_{g}&&& B \ar@{>->}[dl]^{k}\\ Z\ar[rrr]_{c} &&& C}\]
	Its vertical faces are all pullbacks and all the vertical arrows are in $\mathcal{M}$, hence the top face is a pushout and we can conclude. \qedhere 
\end{enumerate}
\end{proof}

\subsection{Pushout complements}

We turn now to the exam of the notion of \emph{pushout complement}. We will prove that, in an $\mathcal{M}$-adhesive category $\X$ certain pushout complements are unique and are \emph{final pullbacks complements} in the sense of \cite{dyckhoff1987exponentiable,corradini2006sesqui}.

\begin{definition}[Pushout complement]
Let $f\colon X\to Y$ and $g\colon Y\to Z$ be two composable arrows in a category $\X$. A \emph{pushout complement} for the composable pair $(f,g)$ is a pair $(h,k)$ with $h\colon X\to W$ and $k\colon W\to Z$ such that the square below commutes and it is a pushout.
\[\xymatrix{X \ar[r]^{f} \ar[d]_{h}& Y \ar[d]^{g} \\ W \ar[r]_{k}& Z}\]
\end{definition}

\begin{example}
	In a generic category $\X$, pushout complements may not exist: in $\Set$ the arrows $?_{2}\colon \emptyset \to 2$ and $!_2\colon 2\to 1$ do not have a pushout complement.
	
	Moreover, composable arrows $f\colon X\to Y$ and $g\colon Y\to Z$ may have  pushout complements which are non-isomorphic: for instance, in $\Set$ the two squares below are both pushouts.
	
	\[\xymatrix{2 \ar[r]^{!_2} \ar[d]_{\id{2}}& 1 \ar[d]^{\id{1}} & 2 \ar[r]^{!_2} \ar[d]_{!_2}& 1 \ar[d]^{\id{1}}\\ 2 \ar[r]_{!_2}& 1 & 1 \ar[r]_{\id{1}}& 1}\]
\end{example}


In an $\mathcal{M}$-adhesive category we can approach pushout complements in a more abstract way, using the pullback functor of \Cref{rem:pb}.

\begin{definition}
Let $F\colon \X \to \Y$ be a functor and $Y$ an object of $\Y$. We say that $F$ has a \emph{partial right adjoint at $Y$} if there exists an object $G(Y)$ in $\X$ and an arrow $\epsilon_Y\colon F(G(Y))\to Y$ such that, for every  $f\colon F(X)\to Y$ there exists a unique $g\colon X\to G(Y)$ making the diagram below commutative.
\[\xymatrix@C=15pt{F(X) \ar[rr]^{F(g)} \ar[dr]_{f}&&F(G(Y)) \ar[dl]^{\epsilon_Y}\\ & Y}\]
\end{definition} 

\begin{remark}
	It is immediate to see that $F$ is a left adjoint if and only if it has a partial right adjoint at every $Y\in \Y$.
\end{remark}

We can prove two  useful properties of partial right adjoints.

\begin{proposition}\label{prop:uni}
	Let $F\colon \X\to \Y$ be a functor and suppose that it has a partial right adjoint $(G(Y), \epsilon_Y)$ at $Y$. Then the following hold true:
	\begin{enumerate}
\item there is a natural isomorphism $\psi\colon \X(-, G(Y))\to \Y(F(-), Y)$;
\item if $(G'(Y), \epsilon'_Y)$ is another partial right adjoint to $F$ at $Y$, then there exists a unique isomorphism $\phi_Y\colon G(Y)\to G'(Y)$ such that the diagram below commutes.
\[\xymatrix@C=15pt{F(G(Y)) \ar[rr]^{F(\phi_Y)} \ar[dr]_{\epsilon_Y} & &F(G'(Y)) \ar[dl]^{\epsilon'_Y}\\ & Y}\]
	\end{enumerate}
\end{proposition}
\begin{proof}
	\begin{enumerate}
		\item Given an object $X$ of $\X$ we can define $\phi_X$ putting
		\begin{align*}
			\phi_X \colon \X(X, G(Y))&\to \Y(F(X), Y)\\
			g&\mapsto \epsilon_Y\circ F(g)
		\end{align*}
		By definition of partial right adjoint $\phi_X$ is a bijection. Consider now an arrow $f\colon X_1\to X_2$, then, for every $g\colon X_2\to G(Y)$ we have:
		\begin{align*}
			\phi_X(g)\circ F(f)&=\epsilon_Y\circ F(g)\circ F(f)\\&= \epsilon_Y\circ F(g\circ f)\\&=\phi_X(g\circ f)
		\end{align*}
		Naturality now follows at once.
		
		\item Let $\phi_Y\colon G(Y)\to G'(Y)$ and $\phi'_Y\colon G(Y)\to G'(Y)$ be the unique arrows making the diagram below commutative.
\[\xymatrix@C=15pt{F(G(Y)) \ar[rr]^{F(\phi_Y)} \ar[dr]_{\epsilon_Y} & &F(G'(Y)) \ar[dl]^{\epsilon'_Y} && F(G'(Y)) \ar[rr]^{F(\phi_Y)} \ar[dr]_{\epsilon'_Y} & &F(G(Y)) \ar[dl]^{\epsilon_Y}\\ & Y &&&&Y}\]
		Now, we can notice that the diagram belows commute.
	
	\[\xymatrix{F(G(Y)) \ar[r]^{F(\phi_Y)} \ar[dr]_{\epsilon_Y} &F(G'(Y)) \ar[r]^{F(\phi_Y)} \ar[d]_{\epsilon'_Y} & F(G(Y)) \ar[dl]^{\epsilon_Y} \\ & Y \\ F(G'(Y)) \ar[r]_{F(\phi'_Y)} \ar[ur]^{\epsilon'_Y} &F(G(Y)) \ar[r]_{F(\phi'_Y)} \ar[u]_{\epsilon_Y} & F(G'(Y)) \ar[ul]_{\epsilon'_Y}}\]	
This implies at once that $\phi_Y$ and $\phi'_Y$ are mutally inverse. \qedhere 
	\end{enumerate}
	
\end{proof}

\begin{lemma}\label{lem:part}
	Let $\X$ be an  an $\mathcal{M}$-adhesive category  and cosinder a pair $(m,f)$ with $m\colon M\mto X $ in $\mathcal{M}$ and $f:X\to Y$. If $(h, k)$ is a pushout complement for the pair $(m,f)$, then there exists an isomorphism $\epsilon_k:f^*(k)\to m$ in $\mathcal{M}/X$ such that $(k, \epsilon_k)$ is  a partial right adjoint at $m$ to the pullback functor $f^*\colon \mathcal{M}/Y\to \mathcal{M}/X$.
\end{lemma}
\begin{proof}
	By \Cref{prop:pbpoad}, in the diagram below the inner square is both  a pushout and a pullback, yielding the dotted isomorphism $\epsilon_k\colon f^*(Q)\to M$.
\[\xymatrix{f^*(Q) \ar@{.>}[dr]^{\epsilon_k} \ar@{>->}@/^.3cm/[drr]^{f^*(k)} \ar@/_.3cm/[ddr]_{p}\\& M \ar@{>->}[r]^{m}\ar[d]_{h}& X \ar[d]^f \\&Q\ar@{>->}[r]_{k}&Y}\]

To see that in this way we get a partial right adjoint, let $g\colon f^*(n)\to m $ be a morphism in $\mathcal{M}/X$, then we have the solid diagram below
\[\xymatrix{f^*(N) \ar[dr]^{g} \ar@{>->}@/^.3cm/[drr]^{f^*(n)} \ar[ddd]_{q}\\& M \ar@{>->}[r]^{m}\ar[d]_{h}& X \ar[d]^f \\&Q\ar@{>->}[r]_{k}&Y\\N \ar@{>->}@/_.3cm/[urr]_{n} \ar@{.>}[ur]^{t}}\]
Now, since $n$ and $f^*(n)$ are in $\mathcal{M}$ we can form two other pullback squares:
\[\xymatrix{A \ar@{>->}[d]_{a}\ar@{>->}[r]^-{r}& f^*(N) \ar@{>->}[d]^{f^*(n)} & B \ar@{>->}[r]^{s} \ar@{>->}[d]_{b}& N \ar@{>->}[d]^{n}\\M \ar@{>->}[r]_{m}&X &Q \ar@{>->}[r]_k&Y\\}\]
Notice that 
\begin{align*}
	k\circ h\circ a&=f\circ m\circ a\\&=f\circ f^*(n)\circ r\\&=n\circ q \circ r
\end{align*}
Thus there exists a unique $c\colon A\to B$ as in the diagram
\[\xymatrix{A \ar@{>->}[d]_{a} \ar@{.>}[dr]^{c}\ar@{>->}[r]^-{r} &f^*(N) \ar@/^.2cm/[dr]^{q} \\ M \ar@/_.2cm/[dr]_{h}&B\ar@{>->}[r]^{s}  \ar@{>->}[d]_{b}& N \ar@{>->}[d]^{n}\\ &Q \ar@{>->}[r]_{k} & Y}\]
We can then consider the two diagrams below, noticing that the left one is $\mathcal{M}$-Van Kampen and in the cube on the right the front, right and left faces are pullbacks.
		\[\xymatrix@C=10pt@R=10pt{&&&&&A\ar@{>->}[dd]|\hole_(.65){a}\ar[rr]^{c} \ar@{>->}[dl]_{r} && B \ar@{>->}[dd]^{b} \ar@{>->}[dl]_{s} \\ M \ar@{>->}[dd]_{m}\ar[rr]^{h}&&Q\ar@{>->}[dd]^{k}&&f^*(N)  \ar@{>->}[dd]_{f^*(n)}\ar[rr]^(.7){q} & & N \ar@{>->}[dd]_(.3){n}\\&&&&&M\ar[rr]|\hole^(.65){h} \ar@{>->}[dl]^{m} && Q \ar@{>->}[dl]^{k} \\X\ar[rr]_{f} &&Y&&X \ar[rr]_{f} & & Y}\]
Now, we can also notice that the back square is a pullback too. To see this it is enough to apply \Cref{lem:pb1} to the diagram
\[\xymatrix{A  \ar@/^.4cm/[rr]^{a\circ r}\ar@{>->}[d]_{a} \ar[r]_{c}& \ar@{>->}[r]_{s}B \ar@{>->}[d]^{b}& N \ar[d]^{n}\\ M \ar@/_.4cm/[rr]_{f\circ m}\ar[r]^{h}& Q \ar@{>->}[r]^{k}& Y}\]

By hypothesis we have $g\colon f^*(n)\to m$ in $\mathcal{M}/X$, thus \Cref{lem:varie} yields a unique morphism $\bar{g}\colon n\to k$ in $\mathcal{M}/Y$. fitting in the diagram below.
 \[\xymatrix{f^*(N) \ar@/_.3cm/[ddr]_{q} \ar[r]^{f^*(\bar{g})}& f^*(Q) \ar[dr]^{\epsilon_k} \ar@{>->}@/^.3cm/[drr]^{f^*(k)} \ar@/_.3cm/[ddr]_{p}\\&& M \ar@{>->}[r]^{m}\ar[d]_{h}& X \ar[d]^f \\&N \ar@{>->}[r]^{\bar{g}} \ar@{>->}@/_.4cm/[rr]_{n} &Q\ar@{>->}[r]^{k}&Y}\]
In particular we have
\begin{align*}k\circ h\circ \epsilon_k\circ f^*(\bar{g})&=k\circ p\circ f^*(\bar{g})\\&=k\circ \bar{g}\circ q\\&=n\circ q\\&=k\circ h\circ g
\end{align*}
Therefore $ h\circ \epsilon_k\circ f^*(\bar{g})=h\circ g$. On the other hand we already know that $m\circ g=m\circ \epsilon_k\circ f^*(\bar{g})$, so that we can deduce $\epsilon_k \circ f^*(\bar{g})=g$ as wanted.
\end{proof}

We can deduce some properties of pushout complements from \Cref{lem:part}.

\begin{corollary}\label{lem:radj} Let $\X$ be an $\mathcal{M}$-adhesive category and consider two composable arrows $m\colon M \mto X$  and $f\colon X\to Y$, with $m\in \mathcal{M}$. Suppose moreover that the following square is a pushout, so that $(h,k)$ is a pushout complement of $(m,f)$:
	\[\xymatrix{M \ar@{>->}[r]^{m}\ar[d]_{h}& X \ar[d]^{f} \\Q\ar@{>->}[r]_{k}&Y}\]
	
Then the following properties hold true.
\begin{enumerate}
	\item  Given another arrow $n\colon N\to Y$ in $\mathcal{M}$ and the pullback square below,
	\[\xymatrix{P \ar[r]^{p_1} \ar@{>->}[d]_{p_2}& N \ar@{>->}[d]^{n}\\X \ar[r]_f&Y}\]
	$\mathcal{M}/Y(n, k)$ is non-empty  if and only $\mathcal{M}/X(p_2, m)$ is so.
	\item (Uniqueness of pushout complements) If $h'\colon M\to Q' $ and $k'\colon Q'\mto Y$ is a pushout complement for $(m,f)$, then there exists a unique isomorphism $\phi\colon Q\to Q'$ making the following diagram commutative.
	\[\xymatrix{&M \ar@{>->}[r]^{m} \ar[d]_{h} \ar@/_.3cm/[ddl]_{h'}& X \ar[d]^{f} \\ &Q \ar@{.>}[dl]^{\phi} \ar@{>->}[r]_{k} & Y \\ Q' \ar@{>->}@/_.3cm/[urr]_{k'}}\]
	\item The pair $(h,k)$ is a \emph{final $\mathcal{M}$-pullback complement} of $(m,f)$, i.e. the pushout square of $m$ along $h$ is also a pullback square for $f$ along $k$ and, moreover, given the solid part of the diagram
	\[\xymatrix{M' \ar[r]^{g}\ar[d]_{h'} & M \ar@{>->}[r]^{m}\ar[d]_{h}& X \ar[d]^{f}  \\  Q' \ar@{.>}[r]^{t} \ar@/_.4cm/@{>->}[rr]_{k'}&Q\ar@{>->}[r]^{k}&Y}\]
	in which the external rectangle is a pullback, then there exists a unique $t\colon Q'\to Q$ making the whole diagrams commutative.
\end{enumerate}
\end{corollary}
\begin{proof}
	\begin{enumerate}
		\item  This follows from point $1$ of \Cref{prop:uni} and \Cref{lem:part}.
		\item  By the second point  of \Cref{prop:uni} and \Cref{lem:part} we get an isomorphism $\phi:k\to k'$ in $\mathcal{M}/Y$. On the other hand we have
		\begin{align*}
			k'\circ \phi \circ  h_1&=k\circ h\\&=f\circ m\\&= k'\circ h'
		\end{align*}
		Since $k'$ is a monomorphism we conclude that $h'=\phi\circ h$.
		\item  This follows from the first point since $g$ belongs to $\mathcal{M}/X(m\circ g, m)$. \qedhere 
	\end{enumerate}
\end{proof}

\subsection{$\mathcal{M}$-adhesivity is not enough}\label{app:fill}
In \Cref{prop:equi} we proved that, in the linear case, the existence of an indpendence pair between two derivation is equivalent to that of a filler between them. This result can be further refined: in a\cite{baldan2011adhesivity} a class  $\mathbb{B}$ of (quasi)adhesive category is defined for which the local Church-Rosser Theorem holds even for left-linear DPO-rewriting system. In our language, and given \Cref{prop:fil} and \Cref{rem:locCR}, this amount to prove that, for elements of $\mathbb{B}$, every independence pair induces a filler. 

\begin{definition}Let $\X$ be a category, we say that $\X$ satisfies
	\begin{itemize}
		\item the \emph{mixed decomposition} property if for every diagram
		\[\xymatrix{X \ar[d]_{a} \ar[r]^{f}& \ar[r]^{g} Y \ar[d]^{b}& Z \ar[d]^{c}\\ A \ar[r]_{h}& B \ar[r]_{k}& C}\]
		whose outer boundary is a pushout and in which $k$ is a monomorphisms, 
		\item the \emph{pushout decomposition} property
	\end{itemize}
\end{definition}

\begin{lemma}
	contenuto...
\end{lemma}
\begin{proof}
	contenuto...\qedhere 
\end{proof}

\begin{corollary}
	contenuto...
\end{corollary}

The following result shows that the mixed and pushout decomposition properties guarantee that every independence pair gives rise to a filler.

\begin{theorem}
	\todo{filler e classe B+}
\end{theorem}
\begin{proof}\qedhere 
\end{proof}

Our next step is to identify sufficient conditions for a category $\X$ to satisfy the mixed and pushout decomposition properties.

\begin{definition}
	\todo{classe B e class B+}
\end{definition}

\begin{example}
	\todo{esempi}
\end{example}

\begin{example}
	\todo{esempi}
\end{example}

\begin{proposition}
	\todo{da B a B+}
\end{proposition}
\begin{proof}
	contenuto... \qedhere 
\end{proof}
\begin{lemma}\todo{due proprietà classe B}
\end{lemma}
\begin{proof}
	\qedhere 
\end{proof}

\begin{corollary}
	\todo{due proprietà classe B+}
\end{corollary}

\begin{corollary}
	\todo{filler e classe B+}
\end{corollary}

\section{Strict $\mathcal{M}$-adhesivity and $\mathcal{M}$-unions} 
 
 In this section we will prove that in a strict $\mathcal{M}$-adhesive category $\X$, the poset of $\mathcal{M}$-subobjects (see  \Cref{app:sub}) is a distributive lattice.
 
 We start by  proving the following a rather technical lemma \cite{garner2012axioms}, which will be needed afterward.
 
\begin{lemma}\label{lem:pb2}
	Let $\X$ be a category with pullbacks and $\mathcal{A}$  a class of arrows in it  stable under pullbacks. Suppose that  the following diagrams are given, with $r\colon W\to R$ and $q\colon Q'\to Q$ in $\mathcal{A}$.
	\[
	\xymatrix{Y\ar[r]^{f_2} \ar[d]_{f_1} & X_2 \ar[d]^{r_2} & Z_1 \ar[d]_{x_1}\ar[r]^{z_1} & W \ar[r]^{w} \ar[d]_{r} & Q'\ar[d]^{q} & Z_2 \ar[d]_{x_2} \ar[r]^{z_2}  & W  \ar[r]^{w} \ar[d]_{r}  & Q' \ar[d]^{q}\\ X_1 \ar[r]_{r_1} &R  & X_1 \ar[r]_{r_1} & R \ar[r]_{s}  & Q& X_2 \ar[r]_{r_2} & R \ar[r]_{s} & Q}\]
	
	If the first square is an $\mathcal{A}$-stable pushout and the whole rectangles and their left halves are pullbacks, then their common right half is a pullback too.
\end{lemma}
\begin{proof}Pulling back  $q$ along $s$ we get a square 
	\[\xymatrix{U \ar[r]^{u} \ar[d]_{h}& Q' \ar[d]^{q}\\ R \ar[r]_s & S}\]
	Notice that
	\[
	q\circ w\circ z_1=s\circ r_1\circ x_1 \qquad 
	q\circ w\circ z_2=s\circ r_2\circ x_2 \]
	Thus we get the dotted $u_1\colon Z_1\to U$ and $u_2\colon Z_2\to U$ fitting in the diagram
	\[\xymatrix{  Z_1  \ar@/^.4cm/[rr]^{w\circ z_1}\ar@{.>}[r]_{u_1}\ar[d]_{x_1}&  U \ar[d]_h \ar[r]_{u} & Q'  \ar[d]^{q}&  Z_2 \ar@/^.4cm/[rr]^{w\circ z_2} \ar@{.>}[r]_{u_2}\ar[d]_{x_2} &  U \ar[d]_h \ar[r]_{u}& Q' \ar[d]^{q} \\  X_1 \ar[r]_{r_1} & R \ar[r]_s& Q & X_2 \ar[r]_{r_2}& R \ar[r]_s & Q}\]
	which, by hypothesis and  \Cref{lem:pb1} have left halves which are pullbacks. Now,
	\[s\circ r_1\circ f_1 =s\circ r_2\circ f_2\]
	Pulling back $q$ along this arrow we get another square
	\[\xymatrix@C=40pt{Z'_0 \ar[r]^{t} \ar[d]_{y}& Q' \ar[d]^{q}\\ R \ar[r]_{s\circ r_1\circ f_1} & S}\]
	In particular, we obtain the dotted $b_1\colon Z'_0\to Z_1$ and $b_2\colon Z'_0\to Z_2$ in
	\[\xymatrix@C=30pt{ Z'_0 \ar@/^.4cm/[rrr]^{t}\ar[d]_{y} \ar@{.>}[r]_{b_1} & Z_1  \ar[r]_{u_1}\ar[d]_{x_1}&  U \ar[d]_h \ar[r]_{u} & Q'  \ar[d]^{q} &Z'_0 \ar@/^.4cm/[rrr]^{t} \ar[d]_{y}\ar@{.>}[r]_{b_2}  & Z_2  \ar[r]_{u_2}\ar[d]_{x_2} &  U \ar[d]_h \ar[r]_{u}& Q' \ar[d]^{q} \\ Y \ar[r]_{f_1}& X_1 \ar[r]_{r_1} & R \ar[r]_s& Q & Y \ar[r]_{f_2} & X_2 \ar[r]_{r_2}& R \ar[r]_s & Q}\]
	Notice that, again  by\Cref{lem:pb1}, all of the squares on the bottom rows of the two diagrams are pullbacks. 
	
	We are going to construct another row above the previous ones. By hypothesis we have that
	\[q\circ w = s\circ r\]
	Thus there exists a unique $g\colon W\to U$ such that
	\[r=h\circ g \qquad w=u\circ g\]
	Moreover, we also have that
	\[\begin{split}h\circ g \circ z_1&=r \circ z_1\\&= r_1\circ x_1 \\&=h\circ u_1
	\end{split}\qquad\begin{split}h\circ g \circ z_2&=r \circ z_2\\&= r_2\circ x_2 \\&=h\circ u_2
	\end{split}\]
	and 
	\[\begin{split}u\circ g \circ z_1&=w \circ z_1\\&= u\circ u_1 
	\end{split} \qquad \begin{split}u\circ g \circ z_1&=w \circ z_2\\&= u\circ u_2
	\end{split}\]
	which together show that
	\[g\circ z_1=u_1 \qquad g\circ z_2=u_2\]
	
	Summing up, we can depict all the arrows we have constucted so far in the following diagrams
	\[\xymatrix{Z'_0 \ar[r]^{b_1} \ar[d]_{\id{Z'}}& Z_1 \ar[r]^{z_1} \ar[d]_{\id{Z_1}} &  W \ar[d]_g \ar[r]^{w} & Q' \ar[d]^{\id{Q'}}&& Z'_0 \ar[r]^{b_2}  \ar[d]_{\id{Z'}}& Z_2 \ar[d]_{\id{Z_2}} \ar[r]^{z_2}&  W \ar[d]_g\ar[r]^{w}& Q' \ar[d]^{\id{Q'}}\\ Z'_0 \ar[d]_{y} \ar[r]^{b_1} & Z_1  \ar[r]^{u_1}\ar[d]_{x_1}&  U \ar[d]_h \ar[r]^{u} & Q'  \ar[d]^{q}& &Z'_0  \ar[d]_{y}\ar[r]^{b_2}  & Z_2  \ar[r]^{u_2}\ar[d]_{x_2} &  U \ar[d]_h \ar[r]^{u}& Q' \ar[d]^{q} \\ Y \ar[r]_{f_1}& X_1 \ar[r]_{r_1} & R \ar[r]_s& Q & &Y \ar[r]_{f_2} & X_2 \ar[r]_{r_2}& R \ar[r]_s & Q}\]
	If we show that $g$ is an isomorphism we are done. Consider the cubes
	\[\xymatrix@C=13pt@R=13pt{&Z_0'\ar[dd]|\hole_(.7){y}\ar[rr]^{b_2} \ar[dl]_{b_1} && Z_2 \ar[dd]^{x_2} \ar[dl]_{z_2}  &&Z'_0\ar[dd]|\hole_(.7){y}\ar[rr]^{b_2} \ar[dl]_{b_1} && Z_2 \ar[dd]^{x_2} \ar[dl]_{u_2}\\Z_1  \ar[dd]_{x_1}\ar[rr]^(.65){z_1} & & W \ar[dd]_(.3){r}&& Z_1  \ar[dd]_{x_1}\ar[rr]^(.65){u_1} & &U \ar[dd]_(.3){h}\\&Y\ar[rr]|\hole^(.65){f_2} \ar[dl]_{f_1} && X_2 \ar[dl]^{r_2} && Y\ar[rr]|\hole^(.65){f_2} \ar[dl]_{f_1} && X_2 \ar[dl]^{r_2}\\X_1 \ar[rr]_{r_1} & & R && X_1 \ar[rr]_{r_1} & & R}\]
	in which the vertical faces are pullbacks. The bottom face is an $\mathcal{A}$-stable pushout and all the vertical arrows are in $\mathcal{A}$, therefore the top faces are pushouts too. Now,  the arrow $g$ fits in the following  diagram
	\[\xymatrix{Z'_0\ar[r]^{b_2} \ar[d] _{b_1}& Z_2 \ar@/^.3cm/[ddr]^{u_2}\ar[d]^{z_2} \\ Z_1 \ar[r]_{z_1}  \ar@/_.3cm/[drr]_{u_1}& W \ar[dr]^{g} \\ &&U}\]
	and thus it is an isomorphism.
\end{proof} 


Given a strict $\mathcal{M}$-adhesive category with pullbacks, the previous result allows us to compute binary suprema of elements of the lattice of subobjects $\msub{X}{X}$ if, at least, one of the subobjects involved is in $\sub{M}{X}{X}$ .

\begin{proposition}\label{prop:uni2} Let $\X$ be a strict $\mathcal{M}$-adhesive category with pullbacks. Given  $m\colon M\mto X$ in $\mathcal{M}$  and another mono $n\colon N\to X$, consider the diagram 
	\[\xymatrix{P \ar[r]^{p_1} \ar@{>->}[d]_{p_2} & M\ar@{>->}[d]^{u_2} \ar@{>->}@/^.3cm/[ddr]^{m} \\ N \ar@/_.3cm/[drr]_{n}\ar[r]_{u_1} & U \ar@{.>}[dr]^{u} \\ && X}\]
	in which the outer boundary form a pullback and the inner square a pushout. Then the dotted arrow $u\colon U\to X$ is a monomorphism and, in $(\msub{X}{X}, \leq)$, $[u]$ is the supremum of $[m]$ and $[n]$.  
\end{proposition}
\begin{remark}
	Notice that the  $p_2$ and $p_1$ are both monos, moreover, $p_2$, as the pullback of $m$, is in $\mathcal{M}$. This, moreover, guarantees the existence of the inner pushout and thus of the dotted arrow $u$.  Finally, notice that also $u_2$, being the pushout of $p_2$ is in $\mathcal{M}$.
\end{remark}
\begin{proof}
	Consider the following two pullback squares
	\[\xymatrix{Q \ar[r]^{q_1} \ar[d]_{q_2} & N\ar[d]^{n} & W \ar[r]^{w_1} \ar@{>->}[d]_{w_2}& M\ar@{>->}[d]^{m}\\
		U \ar[r]_{u}& X & U \ar[r]_{u}& X}\]
	By construction we have the following equalities 
	\[\begin{split}
		n&=u\circ u_1  \qquad \\
		u\circ u_2 \circ p_1&=m\circ p_1\\&=n\circ p_2
	\end{split}\qquad \begin{split}
		m&=u\circ u_2\\ u\circ u_1\circ p_1&=n\circ p_2\\&=m\circ p_1
	\end{split}\]
	which give us the arrows $f_1\colon N\to Q$, $f_2\colon P\to Q$, $g_1\colon M\to W$, $g_2\colon P\to W$ making the following diagrams commute
	\[\xymatrix@C=20pt{N\ar@/^.4cm/[rr]^{\id{N}}\ar@{.>}[r]_{f_1}  \ar[d]_{\id{N}} &Q\ar[r]_{q_1} \ar[d]_{q_2} & N \ar[d]^{n}  &P\ar@{>->}@/^.4cm/[rr]^{p_2}\ar@{.>}[r]_{f_2}  \ar[d]_{p_1} &Q\ar[r]_{q_1} \ar[d]_{q_2} & N \ar[d]^{n}
		\\ N \ar@/_.4cm/[rr]_{n}\ar[r]^{u_1} & U \ar[r]^{u}  & X  & M\ar@{>->}@/_.4cm/[rr]_{m}\ar@{>->}[r]^{u_2}   &U\ar[r]^{u}  & X }\]
	\[\xymatrix@C=20pt{M\ar@/^.4cm/[rr]^{\id{M}}\ar@{.>}[r]_{g_1}  \ar[d]_{\id{M}} &W\ar[r]_{w_1} \ar@{>->}[d]_{w_2} & M \ar@{>->}[d]^{m} & P\ar@/^.4cm/[rr]^{p_1}\ar@{.>}[r]_{g_2}  \ar@{>->}[d]_{p_2} &W\ar[r]_{w_1} \ar@{>->}[d]_{w_2} & M \ar@{>->}[d]^{m}
		\\ M \ar@{>->}@/_.4cm/[rr]_{m}\ar@{>->}[r]^{u_2} & U \ar[r]^{u}  & X & N\ar@/_.4cm/[rr]_{n}\ar[r]^{u_1}  &U\ar[r]^{u} & X }\]	
		
	Their outer edges are pullbacks, thus in the following cubes, the vertical faces are pullbacks
	\[\xymatrix@C=13pt@R=13pt{&P\ar[dd]|\hole_(.65){\id{P}}\ar[rr]^{\id{P}} \ar@{>->}[dl]_{p_2} && P \ar[dd]^{p_1} \ar[dl]_{f_2} & & P\ar[dd]|\hole_(.65){\id{P}}\ar[rr]^{p_1} \ar[dl]_{\id{P}} && M \ar[dd]^{\id{M}} \ar[dl]_{g_1}\\ N  \ar[dd]_{\id{N}}\ar[rr]^(.65){f_1} & & Q \ar[dd]_(.3){q_2}&& P  \ar@{>->}[dd]_{p_2}\ar[rr]^(.65){g_2} & & W \ar@{>->}[dd]_(.3){w_2}\\&P\ar[rr]|\hole^(.65){p_1} \ar@{>->}[dl]^{p_2} && M \ar@{>->}[dl]^{u_2} && P\ar[rr]|\hole^(.65){p_1} \ar@{>->}[dl]^{p_2} && M \ar@{>->}[dl]^{u_2}\\N \ar[rr]_{u_1} & & U && N \ar[rr]_{u_1} & & U}\]

	We know that $p_2$ is an element of $\mathcal{M}$, thus  the top faces are pushouts and therefore $f_1$ and $g_1$ are isomorphisms with inverses given by $q_1$ and $w_1$.  But then, since
	\[u_1=q_2\circ f_1 \qquad u_2=w_2\circ g_1\]
	we can further deduce that the squares below  are both pullbacks.
	\[\xymatrix{N \ar[r]^{\id{N}} \ar[d]_{u_1} & N\ar[d]^{n} & M \ar[r]^{\id{M}} \ar@{>->}[d]_{u_2}& M\ar@{>->}[d]^{m}\\
		U \ar[r]_{u}& X & U \ar[r]_{u}& X}\]
	
	We have three diagrams 
	\[\xymatrix{P\ar@{>->}[d]_{p_2} \ar[r]^{p_1} & M \ar@{>->}[d]^{m}  & N \ar[d]_{\id{N}} \ar[r]^{u_1}& U\ar[d]_{\id{U}} \ar[r]^{\id{U}}  & U \ar[d]^{u} &  M \ar[d]_{\id{M}} \ar[r]^{u_2}& U \ar[r]^{\id{U}} \ar[d]_{\id{U}}  & U \ar[d]^{u}\\N \ar[r]_{n} & X & N \ar@/_.3cm/[rr]_{n}\ar[r]^{u_1} & U  \ar[r]^u  &X & M\ar@{>->}@/_.3cm/[rr]_{m}\ar@{>->}[r]^{u_2} & U  \ar[r]^u  &X }\]
	and we have just proved that the rectangles are pullbacks. Thus we can apply \Cref{lem:pb2} to deduce that 
	\[\xymatrix{ U \ar[r]^{\id{U}} \ar[d]_{\id{U}}  & U \ar[d]^{u}\\ U  \ar[r]_u  &X }\]
	is a pullback, but this means exactly that $u$ is a mono.
	
	For the second half: suppose that $k\colon K\to X$ is an upper bound for $m$ and $n$, thus there exists $k_1\colon M\to K$ and $k_2\colon N\to K$ such that
	\[m=k\circ k_1 \qquad n=k\circ k_2\] 
	But then
	\begin{align*}
		k\circ k_1 \circ p_1 &= m\circ p_1\\&= n\circ p_2\\&= k\circ k_2\circ p_2
	\end{align*}
	Since $k$ is mono, this implies that there exists a unique $h\colon U\to K$ such that 
	\[k_2=h\circ u_1 \qquad k_1=h\circ u_1\]
	and we have
	\[\begin{split}
		k\circ h \circ u_1&=k\circ k_2\\&=n \\&=u \circ u_1 
	\end{split}\qquad 
	\begin{split}
		k\circ h \circ u_2&=k\circ k_1\\&=m \\&=u \circ u_2 
	\end{split} \]
	showing that $u=k\circ h$, i.e.~that $[u]\leq [k]$.
\end{proof}
 
Our next step is to prove that, if $m\colon M\to X$ and $n\colon N\to X$ are in $\mathcal{M}$, then their supremum $u$ in $\msub{X}{X}$ belongs to $\mathcal{M}$ too. To do this we need another technical tool.
 
\begin{definition}\index{codiagonal} \index{codiagonal!$\mathcal{N}$-}
	Let $f\colon X\to Y$ be an arrow in a category $\X$ such that the pushout square below  exists.
	\[\xymatrix{ X \ar[r]^f \ar[d]_f& Y \ar[d]^{y_1}\\ Y \ar[r]_{y_2} & Q_f}\]
	The \emph{codiagonal} $\upsilon_f\colon Q_f\to Y$ is the unique arrow  fitting in the following diagram.
	\[\xymatrix{ X \ar[r]^f \ar[d]_f& Y \ar@/^.3cm/[ddr]^{\id{Y}} \ar[d]^{y_1}\\ Y \ar@/_.3cm/[drr]_{\id{Y}}\ar[r]_{y_2} & Q_f \ar@{.>}[dr]^{\upsilon_f}\\ && Y}\]
\end{definition}

\begin{remark}\label{rem:cod}
	Notice that, in a  $\mathcal{M}$-adhesive category,  for every $m\in \mathcal{M}$ there exists a pushout square of $m$ along itself and so also its codiagonal $\upsilon_m$.
\end{remark}

Let us list some useful properties of codiagonals.

\begin{lemma}\label{rem:coeq}
	Let $f\colon X\to Y$ be a morphism in a category $\X$ and suppose that $f$ admits a codiagonal $\upsilon_f\colon Q_f\to Y$, then the following hold true:
	\begin{enumerate}
		\item $\upsilon_f$ is the coequalizer of the pair of coprojections $y_1, y_2\colon Y\rightrightarrows Q_f$;
		\item suppose that the square below is a pullback $y_1$ along $y_2$
		\[\xymatrix{P \ar[r]^{p_1} \ar[d]_{p_2} & Y \ar[d]^{y_1}\\  Y\ar[r]_{y_2} &  Q_f}\]
		Then $p_1=p_2$ and the pair $y_1, y_2\colon Y\rightrightarrows Q_f$ has an equalizer $p_1\colon P\to Y$.
	\end{enumerate}
\end{lemma}
\begin{proof}\begin{enumerate}
		\item  Let $z\colon Q_f\to Z$ be such that
		\[z\circ y_1=z\circ y_2\]
		Then we also have the identities
		\[\begin{split}
			z\circ y_1 \circ \upsilon_f  \circ y_1 &=z\circ y_1 \circ \id{Y}\\&= z\circ y_1\\&
		\end{split}\qquad \begin{split}z\circ y_1\circ \upsilon \circ  y_2&= z\circ y_1 \circ \id{Y}\\&=z\circ y_1\\&=z\circ y_2
		\end{split}\]
		Therefore, \[z\circ y_1 \circ \nu_f=z\]	
		
		Uniqueness follows from the fact that $\upsilon_f$ is a split epi.
		\item  First of all we can notice that in every square, not necessarily a pullback one, as the one in the diagram below, the existence of the codiagonal implies $q_1=q_2$.
		\[\xymatrix{Q \ar[r]^{q_1} \ar[d]_{q_2} & Y \ar[d]^{y_1} \ar@/^.3cm/[ddr]^{\id{Y}}\\  Y\ar[r]_{y_2}  \ar@/_.3cm/[drr]_{\id{Y}}&  Q_f\ar[dr]^{\upsilon_f}\\ && Y}\]
		
		To see that $p_1\colon P \to Y$ is an equalizer for $y_1$ and $y_2$, consider an arrow $z\colon Z\to X$ such that
		\[y_1\circ z=y_2\circ z\]
		
		Then the universal property of the pullback of $y_1$ along $y_2$ yields a unique $g\colon Z\to P$ such that $z=p_1\circ g$.
		\qedhere
	\end{enumerate}
\end{proof}


\begin{lemma}\label{lem:fact2}
	Let $\X$ be a strict $\mathcal{M}$-adhesive category with pullbacks. Let $m\colon M\mto X$ and $n\colon N\mto X$ be two arrows in $\mathcal{M}$ and suppose that $u\colon U\to X$ is a representative for $[m]\vee[n]$ in $\msub{X}{X}$. Then:
	\begin{enumerate}
		\item $u$ admits pushouts  along itself (i.e.~it has a \emph{cokernel pair}); 
		\item if $\mathcal{M}$ the class of split monos, then there exists an epi $e_u\colon M\to E_u$ and an element $m_u\colon  E_u\to X$ of $\mathcal{M}$ such that   $u=m_u\circ e_u$.
	\end{enumerate}
\end{lemma}

\begin{proof}
	\begin{enumerate}
		\item 
		By \Cref{prop:uni2}, we can consider the following diagram	in which the outer edges form a pullback and the inner square is a pushout.
		\[\xymatrix{P \ar@{>->}[r]^{p_1} \ar@{>->}[d]_{p_2} & M\ar@{>->}[d]^{u_2} \ar@{>->}@/^.3cm/[ddr]^{m} \\ N \ar@{>->}@/_.3cm/[drr]_{n}\ar@{>->}[r]_{u_1} & U \ar[dr]^{u} \\ && X}\]
		
		By \Cref{rem:cod} we know that the codiagonal $\nu_n$ exists. We can then pull back $m$ along $\upsilon_n$, we obtaining the square below.
		\[\xymatrix{T\ar[r]^{t_1}  \ar@{>->}[d]_{t_2}& M\ar@{>->}[d]^{m}\\ Q_{n} \ar[r]_{\upsilon_n} & X}\]
	
		Now, we have identities
		\[\begin{split}
			m \circ \id{M}&=m \\&=\id{X} \circ  m \\&=\upsilon_n\circ n_1 \circ m
		\end{split} \qquad \begin{split}
			m \circ \id{M}&=m \\&=\id{X} \circ  m \\&=\upsilon_n\circ n_2\circ m
		\end{split}\]
		Thus, denoting by $n_1$ and $n_2$ the coprojections $X\rightrightarrows Q_n$, which are both in $\mathcal{M}$, there exist $l_1, l_2\colon M\rightrightarrows T$ as in the following diagram
		
			\[\xymatrix{ M\ar@{.>}[r]_{l_1} \ar@/^.4cm/[rr]^{\id{M}} \ar@{>->}[d]_{m}&T\ar[r]_{t_1}  \ar@{>->}[d]_{t_2}& M\ar@{>->}[d]^{m} & M\ar@{.>}[r]_{l_2} \ar@/^.4cm/[rr]^{\id{M}} \ar@{>->}[d]_{m}&T\ar[r]_{t_1}  \ar@{>->}[d]_{t_2}& M\ar@{>->}[d]^{m}\\ X  \ar@/_.4cm/[rr]_{\id{X}} \ar@{>->}[r]^{n_1}&Q_{n} \ar[r]^{\upsilon_n} & X & X \ar@/_.4cm/[rr]_{\id{X}} \ar@{>->}[r]^{n_2} &Q_{n} \ar[r]^{\upsilon_n} & X}\]
		
		In particular, by \Cref{lem:pb1}, the left halves of the above rectangles are pullbacks and both $l_1$ and $l_2$ belongs to $\mathcal{M}$. Now, notice that, by construction $t_1\circ l_1=t_1\circ l_2$, so that
		\[	t_1\circ l_1\circ p_1=	t_1\circ l_2\circ p_1\]
		
		Moreover, we can also compute to get
		\begin{align*}
		t_2\circ l_1\circ p_1&=n_1\circ m \circ p_1\\&=n_1\circ n\circ p_2\\&=n_2\circ n\circ p_2\\&=n_2\circ m\circ p_1\\&= t_2\circ l_2\circ p_1
		\end{align*} 
		
		Summing up, the cube below commutes and,  by the Van Kampen property, its top face is a pushout, proving that $t_1$ is a codiagonal of $p_1$.
		\[\xymatrix@C=13pt@R=13pt{&P\ar@{>->}[dd]|\hole_(.65){p_2}\ar@{>->}[rr]^{p_1} \ar@{>->}[dl]_{p_1} && M \ar@{>->}[dd]^{m} \ar@{>->}[dl]_{l_1}\\ M  \ar@{>->}[dd]_{m}\ar@{>->}[rr]^(.65){l_2} & & T\ar@{>->}[dd]_(.3){t_2}\\&N\ar@{>->}[rr]|\hole^(.65){n} \ar@{>->}[dl]_{n} && X \ar@{>->}[dl]^{n_1} \\X \ar@{>->}[rr]_{n_2} & & Q_n}\]
		
		As a next step, we can take the pushout of $t_2\colon T\mto Q_n$ alon $t_1$, obtaining the square below.
		\[\xymatrix{T \ar[r]^{t_1} \ar@{>->}[d]_{t_2}& M \ar@{>->}[d]^{q_1}\\ Q_n \ar[r]_{q_2} & Q }\]
		
		Now, if we compute on the one hand we have:
		\begin{align*}
			q_2\circ n_1\circ u \circ \circ u_1&=q_2\circ n_1\circ n\\&=q_2\circ n_2\circ n\\&=q_2\circ n_2 \circ u\circ u_1
		\end{align*}
		On the other hand a further computation yields
		\begin{align*}
		q_2\circ n_1\circ u \circ \circ u_2&=q_2\circ n_1\circ m\\&=q_2\circ t_2\circ l_1\\&=q_1\circ t_1 \circ l_1\\&=q_1\circ t_1\circ l_2\\&=q_2\circ t_2\circ l_2\\&=q_2\circ n_2\circ m\\&=q_2\circ n_2\circ u \circ \circ u_2
		\end{align*}
		
		Thus $q_2\circ n_1\circ u = q_2\circ n_2\circ u$ and the square below commutes. To conclude the proof, we have to show that it is a pushout.
		\[\xymatrix{U \ar[r]^u \ar[d]_u & X \ar[d]^{q_2\circ n_1}\\  X\ar[r]_{q_2\circ n_2} &  Q }\]
		
		Let now $z_1, z_2\colon X\rightrightarrows Z$ be two arrows such that 
		\[z_1\circ u=z_2\circ u\]
		
		If we precompose both sides of the previous equation  with $u_1$ and $u_2$ we get the following identities.
		\[\begin{split}
			z_1\circ m &= z_1\circ u\circ u_2\\&=z_2\circ u \circ u_2\\&=z_2\circ m
		\end{split} \qquad \begin{split}
			z_1\circ n &= z_1\circ u\circ u_1\\&=z_2\circ u \circ u_1\\&=z_2\circ n
		\end{split}\]
			
		The second chain of the equalities above allows us to deduce the existence of the dotted $w\colon Q_n\to Z$.
		\[\xymatrix{N \ar@{>->}[r]^{n} \ar@{>->}[d]_{n} & X\ar@{>->}[d]^{n_1} \ar@/^.3cm/[ddr]^{z_1} \\ X \ar@/_.3cm/[drr]_{z_2}\ar@{>->}[r]_{n_2} & Q_n \ar@{.>}[dr]^{w} \\ && Z}\]
		
		We compute to obtain
		\begin{align*}w\circ t_2\circ l_2&=w\circ n_2\circ m\\&=z_2\circ m\\&=z_1\circ m\\&=w\circ n_1\circ m\\&=w\circ t_2\circ l_1
		\end{align*}
		
	Now, we have already noted that $t_1$ is a codiagonal for $p_1$. Thus the first point of  \Cref{rem:coeq} entails the existence of a unique $k\colon M\to Z$ such that 
	\[k\circ t_1=w\circ t_2\]
	
	Thus the solid part in the diagram below commutes, yielding the existence of the dotted $z\colon Q\to Z$.
		\[\xymatrix{T \ar[r]^{t_1} \ar@{>->}[d]_{t_2}& M \ar@/^.3cm/[ddr]^{k}\ar@{>->}[d]^{q_1}\\ Q_n \ar@/_.3cm/[drr]_{w}\ar[r]_{q_2} & Q \ar@{.>}[dr]^{z}\\ && Z }\]

Computing further we have
		\[\begin{split}
		z\circ q_2\circ n_1&=z_1\\&=w\circ n_1
		\end{split}\qquad 
		\begin{split}
			z\circ q_2\circ n_2&=z_2\\&=w\circ n_2
		\end{split}\]
		
		Moreover, if $z'\colon Q\to Z$ is such that
		\[z_1=z'\circ q_2\circ n_1 \qquad z_2=z'\circ q_2\circ n_2\]
		then we also have 
		\[\begin{split}
			z'\circ q_2\circ n_1&=z_1\\
			&=w\circ n_1 
		\end{split}\qquad \begin{split}
			z'\circ q_2\circ n_2&=z_2\\
			&=w\circ n_2
		\end{split}\]
		which shows that $w=z'\circ q_2$. On the other hand
		\[\begin{split}
			z'\circ q_1\circ t_1 & = z'\circ q_2\circ t_2\\&=w\circ t_2
		\end{split}\]
		and so we also have that $z'\circ q_1=k$, allowing us to conclude that $z=z'$. 
		
		\item By the previous point, there exists a pushout square		
		\[\xymatrix{U \ar[r]^{u} \ar[d]_{u}& X \ar[d]^{a_1} \\ X \ar[r]_{a_2}  & A}\]
		
		Thus $u$ also has a codiagonal $\upsilon_{u}\colon A\to U$. In particular, the two coprojections $a_1$ and $a_2$  are split monos and thus are in $\mathcal{M}$. By the second point of \Cref{rem:coeq}, they have an equalizer $m_u\colon E_u\to X$ which, since $\mathcal{M}$ are stable under pullback, is also an element of $\mathcal{M}$. Since $a_1\circ u= a_2\circ u$, there exists also an arrow $e_u\colon U\to E_u$ such that $u=m_u\circ e_u$. 
		
		To show that this arrow is epi, we can start with the equalities
		\[\begin{split}
			m&=u\circ u_2\\&=m_u\circ e_u\circ u_2
		\end{split} \qquad \begin{split}
			n&=u\circ u_2\\&=m_u\circ e_u\circ u_1
		\end{split}  \]
		By \Cref{cor:deco}  $e_u\circ u_1$ and $e_u\circ u_2$ belong to $\mathcal{M}$. In particular they are monomorphisms and
		\[[e_u\circ u_1] \leq [e_u] \qquad [e_u\circ u_2] \leq [e_u]\]
		
		Now, we can also prove that,, in $(\msub{X}{E_u}, \leq)$ 
		\[[e_u]=[e_u\circ u_1]\vee [e_u\circ u_2]\]
		
		To see this, let  $b\colon B\to E_u$ be another mono such that 
		\[b\circ b_1 =e_u\circ u_1 \qquad b\circ b_2 =e_u\circ u_2\] 
		for some $b_1\colon N\to B$ and $b_2\colon M\to B$. Then 
		\begin{align*}
			b \circ b_1 \circ p_2 & = e_u \circ u_1\circ p_2\\&= e_u \circ  u_2\circ p_1\\&=b\circ b_2\circ p_1
		\end{align*}
		which, since $b$ is a mono, entails
		\[b_1\circ  p_2 =  b_2\circ p_1\]
		Thus there exists $\hat{b}\colon U\to B$ such that
		\[b_1 =\hat{b} \circ u_1 \qquad b_2=\hat{b}\circ u_2\]
		By computing further we get
		\[\begin{split}
			b\circ \hat{b} \circ u_1 &=b\circ b_1 \\&=e_u\circ u_1 
		\end{split}\qquad \begin{split}
			b\circ \hat{b} \circ u_2 &=b\circ b_2 \\&=e_u\circ u_2 
		\end{split}\]
	so that $e_u=b\circ \hat{b}$ and $[e_u] \leq [b]$. 
	
	By the previous point and point $2$ of \Cref{rem:coeq}, there exist a diagram in which the outer edges form a pushout, the inner square is a pullback  and $c$ is the equalizer of $c_1$ and $c_2$.
		\[\xymatrix{U  \ar@/^.5cm/[rr]^{e_u} \ar@/_.4cm/[dr]_{e_u}  \ar@{.>}[r]_{e}& C \ar[d]_{c} \ar[r]_{c} & E_u \ar[d]^{c_2} \\  & E_u \ar[r]_{c_1} & \hat{Q}}\]
		The existence of $e\colon U\to C$ can then be inferred from the universal property of pullbacks.
		
		Now, if we show that $c$ is invertible, then we are done: in such a case $c_1$ and $c_2$ must be equals , proving that $e_u$ is an epimorphism.  
		
		Being the coprojections of the pushout of $c$ along itself,  $c_1$ and $c_2$ are split monos and thus are in $\mathcal{M}$. Thus $c\in \mathcal{M}$ too. 
		
		Take two arrows $z_1, z_2\colon X\rightrightarrows Z $ such that
		\[z_1\circ m_u\circ c=z_2\circ m_u \circ c\]
			Then we have
		\begin{align*}
			z_1\circ u &= z_1\circ m_u\circ e_u \\&= z_1\circ m_u\circ c \circ e \\&= z_2\circ m_u\circ c \circ e \\&= z_2\circ m_u \circ e_u\\&= z_2\circ u 
		\end{align*}
		and thus there exists $z\colon A\to Z$ such that 
		\[z_1= z\circ a_1\qquad z_2=z\circ a_2\]
		
		Uniqueness of such a $z$ follows at once since $a_1$ and $a_2$ are the coprojections of the pushout of $u$ along itself.  We can then conclude that the square below is a pushout.
		\[\xymatrix@C=40pt{U \ar[r]^{m_u\circ c} \ar[d]_{m_u\circ c} & X \ar[d]^{a_1}\\  X\ar[r]_{a_2} &  Q }\]
		Now, $\mathcal{M}$ and $\mathcal{N}$ are closed under composition. Thus $m_u\circ c$ is in $\mathcal{M}$ and so, by \Cref{cor:rego} that $m_u\circ c$ is a regular mono. But regular monos are always equalizers of the coprojections into their cokernel pair. Since we already know that $m_u$ is the equalizer of $a_1, a_2\colon X \rightrightarrows A$, we conclude that $c$ is an isomorphism. 	\qedhere 
	\end{enumerate}
\end{proof}
 
With these tools available, we are ready to provide a way to construct binary suprema in $\sub{M}{\X}{X}$.
 
 \begin{theorem}[\cite{garner2012axioms,castelnovo2023thesis,johnstone2007quasitoposes}]\label{thm:un}Let $\X$ be a strict $\mathcal{M}$-adhesive category with pullbacks and suppose that $\mathcal{M}$ contains all split monomorphisms. Then for every object $X$, $(\sub{M}{X}{X}, \leq)$ has binary suprema, which are preserved by the inclusion $(\sub{M}{X}{X}, \leq)\to (\msub{X}{X}, \leq)$.
 \end{theorem}
\begin{proof}	Let $m\colon M\to X$  and $n\colon N\to X$ be elements of in $\mathcal{M}$. By \Cref{prop:uni2} we know that $[m]$ and $[n]$  have a supremum in $(\msub{X}{X}, \leq)$ given by $[u]$, where $u\colon U\to X$ fits in the  diagram
	\[\xymatrix{P \ar@{>->}[r]^{p_1} \ar@{>->}[d]_{p_2} & M\ar@{>->}[d]^{u_2} \ar@{>->}@/^.3cm/[ddr]^{m} \\ N \ar@{>->}@/_.3cm/[drr]_{n}\ar@{>->}[r]_{u_1} & U \ar[dr]^{u} \\ && X}\]
	in which the outer edges form a pullback and the inner square is a pushout.  
	
	 By the second point of \Cref{lem:fact2}, we also know that $u=m_u\circ e_u$ for some epi $e_u\colon Y\to E_u$ and $m_u\colon E_u\to X$ in $\mathcal{M}$. Our strategy to prove the theorem consists in showing that $e_u$ is an isomorphism.
	 

	First of all notice that $e_u$ is a mono because $u=m_u\circ e_u$ and \Cref{prop:uni2}. Thus in the following diagram every square is a pullback and, applying \Cref{lem:pb1}, we can deduce that the composite square is a pullback too.
	\[\xymatrix{P \ar@{>->}[r]^{p_1} \ar@{>->}[d]_{p_2}& M \ar[r]^{\id{M}} \ar@{>->}[d]_{u_2} & M \ar[d]^{u_2} \\ N \ar@{>->}[r]_{u_1}  \ar[d]_{\id{U}}& U \ar[r]_{\id{U}} \ar[d]_{\id{U}} & U \ar[d]^{e_u} \\ N  \ar[r]_{u_1}& U \ar[r]_{e_u} & E_u}\].
	
	 As we have noticed before, \Cref{cor:deco} entails that both $e_u\circ u_1$ and $e_u\circ u_2$ belongs to $\mathcal{M}$. We can then build the following two pushout squares, which, by \Cref{prop:pbpoad}, are also pullbacks.
	\[\xymatrix@C=40pt{N \ar@{>->}[r]^-{e_u\circ u_1} \ar@{>->}[d]_{e_u\circ u_1} & E_u \ar@{>->}[d]^{e_1} & N \ar@{>->}[r]^-{e_u\circ u_1} \ar@{>->}[d]_{ u_1} & E_u \ar@{>->}[d]^{a_1} \\ E_u   \ar@{>->}[r]_-{e_2} & Q_{e_u\circ u_1} & U \ar@{>->}[r]_{a_2}& A}\]
	
	By construction the upper rectangle in the following diagram is commutative. Thus the dotted arrow $a$ exists and, by \Cref{lem:po1}, the bottom rectangle is a pushout.
	\[\xymatrix{N \ar@{>->}[r]^{u_1} \ar@{>->}[d]_{u_1} & U \ar[r]^{e_u} & E_u \ar@{>->}[d]_{a_1} \ar@{>->}@/^.5cm/[dd]^{e_1} \\
		U \ar[d]_{e_u} \ar@{>->}[rr]^{a_2}&& A \ar@{.>}[d]_{a} \\ E_u \ar@{>->}[rr]_{e_2}&&Q_{e_u\circ u_1}}\]

	
	The arrow $e_2$ is the pushout of $e_u\circ u_1$ along itself, thus it is in $\mathcal{M}$, and so it is a mono. Together with \Cref{lem:pb1}, this entails that the following rectangle is a pullback.
	\[\xymatrix{U \ar[r]^{e_u}  \ar[d]_{\id{U}}& E_u \ar[r]^-{\id{E_u}} \ar[d]_{\id{E_u}} & E_u \ar@{>->}[d]^{e_2}\\ U \ar[r]_{e_u} & E_u \ar@{>->}[r]_-{e_2} & Q_{e_u\circ u_1}}\]
	
	Similarly, the arrow $a_2$ is in $\mathcal{M}$ as it is the pushout of $e_u\circ u_1$ along $u_1$. Thus we can apply \Cref{lem:pb2} to the diagrams
	\[\xymatrix{ N \ar@{>->}[rr]^-{e_u\circ u_1} \ar@{>->}[d]_{ u_1} && E_u \ar@{>->}[d]^{a_1} & N \ar@{>->}[d]_{e_u\circ u_1} \ar@{>->}[r]^{u_1} & U \ar@{>->}[d]_{a_2}\ar[r]^-{e_u} & E_u  \ar@{>->}[d]^{e_2}\\ U \ar@{>->}[rr]_{a_2}&& A & E_u \ar@{>->}@/_.4cm/[rr]_{e_1} \ar@{>->}[r]^{a_1} &  A \ar[r]^-{a} & Q_{e_u\circ u_1} }\]
	\[\xymatrix{U  \ar[r]^{\id{U}} \ar[d]_{\id{U}}& U \ar@{>->}[d]_{a_2} \ar[r]^-{e_u} & E_u \ar@{>->}[d]^{e_2} \\U \ar@{>->}[r]^{a_2} \ar@{>->}@/_.4cm/[rr]_{e_2\circ e_u} & A \ar[r]^-{a} & Q_{e_u\circ u_1}}\]
	to get that also the following square is a pullback.
	\[\xymatrix{
		U \ar[d]_{e_u} \ar@{>->}[r]^-{a_2}& A \ar[d]^{a} \\ E_u \ar@{>->}[r]_-{e_2}&Q_{e_u\circ u_1}}\]
	
	On the other hand,  $p_1$ is in $\mathcal{M}$, thus we can push it out along itself.
	\[\xymatrix{P \ar@{>->}[r]^{p_1} \ar@{>->}[d]_{p_1}& M \ar@{>->}[d]^{m_1} \\ M \ar@{>->}[r]_{m_2} & Q_{p_1}}\]
	We can then construct the solid part of the rightmost rectangle in the diagram below, inducing the dotted $b\colon Q_{p_1}\to A$. Notice that the first rectangle is a pushout by \Cref{lem:po1} so that  the right half of the second diagram also is a pushout, again because of \Cref{lem:po1}, proving that $b$ belongs to $\mathcal{M}$ too.
	\[\xymatrix{P \ar@{>->}[r]^{p_2} \ar@{>->}[d]_{p_1}& N \ar@{>->}[d]_{u_1} \ar@{>->}[r]^{u_1}& U \ar[r]^{e_u} & E_u \ar@{>->}[d]^{a_1} & P \ar@{>->}@/^.4cm/[rrr]^{e_u\circ u_1\circ p_2}\ar@{>->}[r]_{p_1} \ar@{>->}[d]_{p_1}& M \ar@{>->}[d]_{m_1} \ar@{>->}[r]_{u_2}& U \ar[r]_{e_u}& E_u \ar@{>->}[d]^{a_1}\\ M \ar@{>->}[r]_{u_2}& U \ar@{>->}[rr]_{a_2}&& A& M \ar@{>->}@/_.4cm/[rrr]_{a_2\circ u_2}\ar@{>->}[r]^{m_2} & Q_{p_1} \ar@{>.>}[rr]^{b} && A}\]
	
	We can compose with the codiagonal $\upsilon_{p_1}\colon Q_{p_1}\to M$ to obtain the solid part of the diagram below.
	\[\xymatrix{M \ar@/_.4cm/[dd]_{\id{M}}\ar@{>->}[r]^{u_2} \ar@{>->}[d]^{m_1} & U \ar[r]^-{e_u} & E_u \ar@{>->}[d]_{a_1} \ar@/^.4cm/[dd]^{\id{E_u}} \\
		Q_{p_1} \ar[d]^{\upsilon_{p_1}} \ar@{>->}[rr]^{b}&& A \ar@{.>}[d]_{r} \\ M\ar@{>->}[r]_{u_2} & U\ar[r]_-{e_u}&E_u}\]	
	Since the upper half of the square above is a pushout, the dotted $r\colon A\to Q_{e_u\circ u_1}$ exists. Moreover, since the outer boundary is a pushout square, the lower half is a pushout too, by \Cref{lem:po1}. By \Cref{prop:pbpoad} the bottom rectangle of the previous diagram is also a pullback.
	
	As a next step, since $e_1\circ u_1$ is in $\mathcal{M}$, then we can form the following pushout. 
	\[\xymatrix@C=40pt{N \ar@{>->}[r]^{e_u\circ u_1} \ar@{>->}[d]_{e_u\circ u_1}& E_u\ar@{>->}[d]^{e_1}\\
		E_u \ar@{>->}[r]_{e_2}& Q_{e_u\circ u_1} }\]
		
		Moreover, we can notice that
		\[\begin{split}
		a\circ b \circ m_1&=a\circ a_1\circ e_u\circ u_2\\&=e_1\circ e_u\circ u_2
		\end{split}\qquad \begin{split}
		a\circ b \circ m_2&=a\circ a_2\circ u_2\\&=e_2\circ e_u\circ u_2
		\end{split}\]
	
	Thus the following cube commutes, has an $\mathcal{M}$-pushout as a bottom and top face and the back and left faces are pullbacks. Hence the front and right faces are pullbacks too. 
	
	\[\xymatrix@C=15pt@R=15pt{&P\ar@{>->}[dd]|\hole_(.65){p_2}\ar@{>->}[rr]^{p_1} \ar@{>->}[dl]_{p_1} && M \ar@{>->}[dd]^{e_u\circ u_2} \ar@{>->}[dl]_{m_2}\\ M  \ar@{>->}[dd]_{e_u\circ u_2}\ar@{>->}[rr]^(.65){m_1} & & Q_{p_1}\ar[dd]_(.3){a\circ b}\\&N\ar@{>->}[rr]|\hole^(.7){e_u\circ u_1} \ar@{>->}[dl]^{e_u\circ u_1} && X \ar@{>->}[dl]^{e_2} \\X \ar@{>->}[rr]_{e_1} & & Q_{e_u\circ u_1}}\]
	
	Now, we can notice that for every $z_1\colon  Z\to M$ and $z_2\colon Z\to E_u$ such that 
	\[m\circ z_1=m_u\circ z_2\]
	we have the following chain of equalities
	\begin{align*}
		m_u\circ e_u\circ u_2\circ z_1&=u\circ u_2\circ z_1\\&=m\circ z_1\\&=m_u\circ z_2 
	\end{align*}
	which, since $m_u$ is mono, entails
	\[z_2=e_u\circ u_2\circ z_1\]
	But this means that the square below is a pullback. 
	\[\xymatrix{M \ar[r]^{\id{M}} \ar@{>->}[d]_{e_u\circ u_2}& M \ar@{>->}[d]^{m}\\ E_u \ar@{>->}[r]_{m_u} & X}\] 
	
	We can therefore apply \Cref{lem:pb2} to the pushout of $e_u\circ u_1$ along itself and to the following two diagrams, where $\upsilon_{e_u\circ u_1}$ is the codiagonal of $e_u\circ u_1$.
	\[
	\xymatrix@C=30pt@R=30pt{ M \ar@/^.4cm/[rrr]^{\id{M}} \ar@{>->}[d]_{e_u\circ u_2}\ar@{>->}[r]_{m_2} & Q_{p_1} \ar[rr]_{\upsilon_{p_1}} \ar[d]_{a\circ b} && M\ar@{>->}[d]^{m} \\ E_u \ar@{>->}@/_.4cm/[rrr]_{m_u}\ar@{>->}[r]^-{e_2} & Q_{e_u\circ u_1} \ar[r]^-{\upsilon_{e_u\circ u_1}} &E_u \ar@{>->}[r]^{m_u} &X\\ 
	M\ar@/^.4cm/[rrr]^{\id{M}} \ar@{>->}[d]_{e_u\circ u_2} \ar@{>->}[r]_{m_1}  & Q_{p_1} \ar[rr]_{\upsilon_{p_1}} \ar[d]_{a\circ b}  && M \ar@{>->}[d]^{m}\\ E_u \ar@{>->}@/_.4cm/[rrr]_{m_u}\ar@{>->}[r]^-{e_1}& Q_{e_u\circ u_1} \ar[r]^-{\upsilon_{e_u\circ u_1}} & E_u \ar@{>->}[r]^{m_u} &X }\]
	
	Hence the outer rectangle in the diagram below is a pullback, so that, in particular, $a\circ b\in \mathcal{M}$. We can also apply \Cref{lem:pb1} to deduce that the left half of the rectangle  is a pullback, too. 
	\[\xymatrix@C=40pt{Q_{p_1} \ar[r]^{\upsilon_{p_1}} \ar@{>->}[d]_{a\circ b} &M \ar[r]^{\id{M}} \ar@{>->}[d]_{e_u\circ u_2}& M \ar@{>->}[d]^{m}\\ Q_{e_u\circ u_1} \ar[r]_-{\upsilon_{e_u\circ u_1}} & E_u \ar@{>->}[r]_{m_u}& X}\]
	
	We can compute to obtain two other identities:
	\[\begin{split}
		\upsilon_{e_u\circ u_1}\circ a\circ b & = e_u\circ u_2\circ \upsilon_{p_1}\\&= r\circ b\\&
	\end{split}\qquad \begin{split}
		\upsilon_{e_u\circ u_1}\circ a\circ a_1 &= \upsilon_{e_u\circ u_1}\circ e_1\\&= \id{E_u}\\&=r\circ a_1
	\end{split}\]
	
	Therefore $r= \upsilon_{e_u\circ u_1}\circ a$. Applying again \Cref{lem:pb1} to the following rectangle, we get that its left half is a pullback.
	\[ \xymatrix@C=30pt{Q_{p_1} \ar@{>->}[d]_b \ar[r]^{\id{Q_{p_1}}} & Q_{p_1}  \ar@{>->}[d]_{a\circ b}\ar[r]^{\upsilon_{p_1}}& M \ar@{>->}[d]^{e_u\circ u_2}\\ A \ar@/_.4cm/[rr]_{r}\ar[r]^-a&Q_{e_u\circ u_1} \ar[r]^-{\upsilon_{e_u\circ u_1}}& E_u }\]
	
	We want now to show that the inner rectangle of the diagram below is a pushout.  Let thus $z_1\colon Q_{e_u\circ u_1}\to Z$ and $z_2\colon M\ to Z$ be arrows such that $z_1\circ a\circ b=z_2\circ \upsilon_{p_1}$. Since $\upsilon_{e_u\circ u_1}$ is an epimorphism it is enougH to construct the dotted $z\colon E_u \to Z$. 
	\[\xymatrix{Q_{p_1} \ar@{>->}[r]^{b} \ar[d]_{\upsilon_{p_1}}& A \ar[r]^-{a}& Q_{e_u\circ u_1} \ar[d]_{\upsilon_{e_u\circ u_1}} \ar@/^.3cm/[ddr]^{z_1} \\M \ar@/_.3cm/[drrr]_{z_2}\ar@{>->}[r]^-{u_2}&U \ar[r]_{e_u}& E_u \ar@{.>}[dr]^{z}\\ & & & Z}\]
	
	First of all we can notice that
	\[z_1\circ e_1\circ e_u\circ u_1 = z_1\circ e_2\circ e_u\circ u_1\]
	while we also have
	\begin{align*}
		z_1\circ e_1\circ e_u\circ u_2 &= z_1\circ a\circ a_1\circ e_u\circ u_2\\&=z_1\circ a \circ b\circ m_1\\&=z_2\circ \upsilon_{p_1} \circ m_1
		\\&=z_2\circ \upsilon_{p_1} \circ m_2
		\\&=z_1\circ a \circ b\circ m_2 
		\\&=z_1\circ a\circ a_2\circ u_2
		\\&=z_1\circ e_2\circ e_u \circ u_2
	\end{align*}
	which implies that
	\[z_1\circ e_1\circ e_u  = z_1\circ e_2\circ e_u\]
	which, since $e_u$ is an epimorphism, allows us to conclude that $z_1\circ e_1$ and $z_1\circ e_2$ are eual. So equipped, we can now compute:
	\[\begin{split}
		z_1\circ e_1\circ \upsilon_{e_u\circ u_1}\circ e_1&=z_1\circ e_1\circ \id{E_u}\\&=z_1\circ e_1\\&
	\end{split} \qquad \begin{split}
		z_1\circ e_1\circ \upsilon_{e_u\circ u_1}\circ e_2&=z_1\circ e_1\circ \id{E_u}\\&=z_1\circ e_1\\&=z_1\circ e_2
	\end{split}\]
	showing 
	\[z_1=z_1\circ e_1 \circ  \upsilon_{e_u\circ u_1}\]
	
	Moreover, computing again we obtain
	\begin{align*}
		z_2\circ \upsilon_{p_1}&=z_1\circ a \circ b \\&=z_1\circ \id{E_u} \circ a \circ b\\&=z_1\circ e_1 \circ \upsilon_{e_u\circ u_1} \circ a\circ b\\&=z_1\circ e_1\circ e_u\circ u_2 \circ \upsilon_{p_1} 
	\end{align*}
	and $\upsilon_{p_1}$ is an epimorphism, thus
	\[z_2=z_1\circ e_1\circ e_u \circ u_2\]
	Summing up, $z_1\circ e_1$ fills our original diagram, thus its inner rectangle is indeed a pushout.
	
	We are now ready to collect all our arrows in the following cube
	\[\xymatrix@C=20pt@R=10pt{ & &Q_{p_1}\ar@{>->}[ddll]_{b} \ar[ddd]^(.4){\id{Q_{p_1}}}|(.666666)\hole\ar[rrr]^{\upsilon_{p_1}} &&&M \ar[ddd]^{\id{M}} \ar@{>->}[dl]^{u_2}\\&  &&&U \ar[ddd]^{\id{U}} \ar[dl]^{e_u} \\A \ar[ddd]_{a} \ar[rrr]^{r}&&& E_u\ar[ddd]^{\id{E_u}}\\&&Q_{p_1} \ar[rrr]|(.34)\hole^{\upsilon_{p_1}}|(.67)\hole \ar@{>->}[dl]_{b}&&& M \ar@{>->}[dl]^{u_2}\\ & A  \ar[dl]_(.4){a}&&& U \ar[dl]^{e_u} \\ Q_{e_u\circ u_1}\ar[rrr]_{\upsilon_{e_u\circ u_1}} &&& E_u}\]
	
	The top and bottom faces of this cube are $\mathcal{M}$-pushouts and all faces beside the frontal one are pullbacks, hence, by strict $\mathcal{M}$-adhesivity it follows that also this last face is a pullback. By \Cref{lem:pb1} the rectangle
	\[\xymatrix@C=40pt{U\ar[r]^{a_2} \ar[d]_{e_u} & A \ar[r]^{r} \ar[d]_{a}& E_u \ar[d]^{\id{E_u}}\\
		E_u \ar[r]_{e_2}& Q_{e_u\circ u_1} \ar[r]_{\upsilon_{e_u\circ u_1}} & E_u}\]
	is a pullback. Thus $e_u$ is an isomorphism as it is the pullback of $\id{E_u}$.
\end{proof}

\begin{example}\label{ex:sem}In \cite{johnstone2007quasitoposes} it is shown that the category $\cat{SGraph}$ of simple graphs is not quasiadhesive. Too see this, first of all recall that a morphism in $\cat{SGraph}$ is a regular monos if and only if it is injective on vertex and reflects edges. Now, take 
	\[\mathcal{G}:=(1, \{0,1\}, \delta_0, \delta_1) \qquad \mathcal{H}:=( \emptyset, 1, ?_1, ?_1) \quad \]
	Then we have two regular monomorphisms $ (?_1, \delta_0), (?_1, \delta_1)\colon \mathcal{H}\rightrightarrows \mathcal{G} $. Their supremum in $\msub{SGraph}{\mathcal{G}}$ is represented by inclusion of $(\emptyset, \{0,1\}, ?_0, ?_1)$ into $\mathcal{G}$ which is not a regular monomorphism, thus, by \Cref{thm:un}, $\cat{SGraph}$ is not quasiadhesive. 
\end{example}

 
 \begin{proposition}\label{prop:distlat}Let $\X$ be a strict $\mathcal{M}$-adhesive with pullbacks. If $\mathcal{M}$ contains all split monos then for every object $X$, $(\sub{M}{X}{X})$ is a distributive lattice. 
 \end{proposition}
\begin{proof}
	Since $\X$ has pullbacks then $(\sub{M}{X}{X})$  has binary infima. The existence of binary suprema is guaranteed by \Cref{thm:un}. We only have to show that, given $m\colon M\mto X$, $n\colon N\mto X$ and $v\colon V\mto X$ we have
	\[[n]\wedge ([m]\vee[v])=([n]\wedge [m]) \vee ([n]\wedge [v])\]
	
	Let $u\colon \mto X$ be a representative of $[m]\vee [v]$, thus by \Cref{prop:uni2} and \Cref{thm:un} we know that it fits in the diagram below, whose outer boundary is a pullback and in which the inner square is a pushout.
		\[\xymatrix{P \ar@{>->}[r]^{p_1} \ar@{>->}[d]_{p_2} & M\ar@{>->}[d]^{u_2} \ar@{>->}@/^.3cm/[ddr]^{m} \\ V \ar@{>->}@/_.3cm/[drr]_{v}\ar@{>->}[r]_{u_1} & U \ar@{>->}[dr]^{u} \\ && X}\]
		
	Let also $r\colon R\mto X$,  $s\colon S\mto X$, $t\colon T\mto X$ and $q\colon Q\mto X$ be  representative for, respectively $[n]\wedge [u]$, $[n]  \wedge [m]$, $[n]\wedge [v]$ and $[s]\wedge [t]$,  then we have the four pullback squares below.
	\[\xymatrix{R \ar@{>->}[r]^{r_1} \ar@{>->}[d]_{r_2}\ar@{>->}[dr]^{r}& N \ar@{>->}[d]^{n} & S \ar@{>->}[r]^{s_1} \ar@{>->}[d]_{s_2}\ar@{>->}[dr]^{s}& N \ar@{>->}[d]^{n}&T \ar@{>->}[r]^{t_1} \ar@{>->}[d]_{t_2}\ar@{>->}[dr]^{t}& N \ar@{>->}[d]^{n}& Q \ar@{>->}[r]^{q_1} \ar@{>->}[d]_{q_2}\ar@{>->}[dr]^{q}& T \ar@{>->}[d]^{t}\\U \ar@{>->}[r]_{u} & X&M \ar@{>->}[r]_{m} & X&V \ar@{>->}[r]_{v} & X&S \ar@{>->}[r]_{s} & X}\]
	
Using the universal property of the first square, we can construct the two dotted arrows $m_1\colon T\to R$ and $m_2\colon S\to R$ fitting in the following diagrams. Notice that the left halves of these rectangles are pullbacks by \Cref{lem:pb1}.

\[\xymatrix{T \ar[d]_{t_2} \ar@/^.4cm/[rr]^{t_1} \ar@{.>}[r]_{m_1}& R \ar@{>->}[r]_{r_1} \ar@{>->}[d]_{r_2}& N \ar@{>->}[d]^{n} & S \ar[d]_{s_2} \ar@/^.4cm/[rr]^{s_1} \ar@{.>}[r]_{m_2}& R \ar@{>->}[r]_{r_1} \ar@{>->}[d]_{r_2}& N \ar@{>->}[d]^{n}\\ V \ar@/_.4cm/[rr]_{v} \ar@{>->}[r]^{u_1}&U \ar@{>->}[r]^{u} & X &  M \ar@/_.4cm/[rr]_{m} \ar@{>->}[r]^{u_2}&U \ar@{>->}[r]^{u} & X}\] 

On the other hand, we can also build the dotted $j\colon Q\to P$ as follows.

		\[\xymatrix{Q \ar[r]^{q_2} \ar[d]_{q_1} \ar@{.>}[dr]^{j}& S \ar@{>->}@/^.2cm/[dr]^{s_2}\\ T \ar@{>->}@/_.2cm/[dr]_{t_2}&P \ar@{>->}[r]^{p_1} \ar@{>->}[d]_{p_2} & M\ar@{>->}[d]^{u_2}  \\ &V \ar@{>->}[r]_{u_1} & U }\]
	
	Thus we have the commutative cube below, whose front and right faces are pullbacks and whose bottom face is an $\mathcal{M}$-pushout.
		\[\xymatrix@C=15pt@R=15pt{&Q\ar[dd]|\hole_(.65){j}\ar@{>->}[rr]^{q_2} \ar@{>->}[dl]_{q_1} && S \ar@{>->}[dd]^{s_2} \ar@{>->}[dl]_{m_2}\\ T \ar@{>->}[dd]_{t_2}\ar@{>->}[rr]^(.65){m_1} & & R \ar@{>->}[dd]_(.3){r_2}\\&P\ar@{>->}[rr]|\hole^(.7){p_1} \ar@{>->}[dl]_{p_2} && M \ar@{>->}[dl]^{u_2} \\V \ar@{>->}[rr]_{u_1} & & U}\]
		
		To show that the top face is a pushout it is enough to show that the left and back faces are pullbacks. Suppose thus that the solid part of the following two diagrams are given.
		\[\xymatrix{Z \ar@{.>}[dr]^{z} \ar@/^.3cm/[drr]^{z_1} \ar@/_.3cm/[ddr]_{z_2}&& &Y \ar@{.>}[dr]^{y} \ar@/^.3cm/[drr]^{y_1} \ar@/_.3cm/[ddr]_{y_2}\\&Q \ar@{>->}[r]^{q_2} \ar[d]_j & S \ar@{>->}[d]^{s_2}& &Q \ar@{>->}[r]^{q_1} \ar[d]_{j} & T \ar@{>->}[d]^{t_2}\\& P \ar@{>->}[r]_{p_1} & M && P \ar@{>->}[r]_{p_2}& V}\]
		
		By the universal property of pullbacks, we get the dotted arrows $z'\colon Z\to T$ and $y'\colon Y\to S$ fitting in the diagrams below.
		\[\xymatrix{Z \ar@{.>}[dr]^{z'} \ar[r]^{z_1} \ar[d]_{z_2}& S \ar@{>->}@/^.2cm/[dr]^{s_1}& &Y \ar@{.>}[dr]^{y'} \ar[r]^{y_1} \ar[d]_{y_2} & T  \ar@{>->}@/^.2cm/[dr]^{t_1}\\P \ar@{>->}@/_.2cm/[dr]_{p_2}&T \ar@{>->}[r]^{t_1} \ar@{>->}[d]_{t_2} & N \ar@{>->}[d]^{s_2}& P\ar@{>->}@/_.2cm/[dr]_{p_1}&S \ar@{>->}[r]^{s_1} \ar@{>->}[d]_{s_2} & N \ar@{>->}[d]^{n}\\& V \ar@{>->}[r]_{v} & X && M \ar@{>->}[r]_{m}& X}\]
		
		In turn, these two arrows yields $\hat{z}\colon Z \to Q$ and $\hat{y}\colon Y \to Q$ as below.
				\[\xymatrix{Z \ar@{.>}[dr]^{\hat{z}} \ar@/^.3cm/[drr]^{z_1} \ar@/_.3cm/[ddr]_{z'}&& &Y \ar@{.>}[dr]^{y} \ar@/^.3cm/[drr]^{y_1} \ar@/_.3cm/[ddr]_{y'}\\&Q \ar@{>->}[r]^{q_2} \ar@{>->}[d]_{q_1} & S \ar@{>->}[d]^{s}& &Q \ar@{>->}[r]^{q_1} \ar@{>->}[d]_{q_2} & T \ar@{>->}[d]^{t}\\& T \ar@{>->}[r]_{t} & X && S \ar@{>->}[r]_{s}& X}\]
	
	If we compute we get:
	\[\begin{split}
		p_2\circ j\circ \hat{z}&= t_2\circ q_1\circ \hat{z}\\&=t_2\circ z'\\&= p_2\circ z_2
	\end{split}\qquad \begin{split}
	p_1\circ j \circ \hat{y}&=s_2\circ q_2\circ \hat{y}\\&=s_2\circ y'\\&=p_1\circ y_2
	\end{split}\]
	
	Since $p_1$ and $p_2$ are mono it follows that 
	\[j\circ \hat{z}=z_2 \qquad j\circ \hat{y}=y_2\]
		
	Moreover, $q_2$ and $q_1$  are monos too, thus $\hat{z}$ and $\hat{y}$ are the unique arrows fitting in the two diagram with which we have begun, proving that the back and right faces of our original cubes are pullbacks.
	
	To end the proof it is now enough to see that 
	\[\begin{split}
	r\circ m_1&=u\circ r_2\circ m_1\\&=u\circ u_1\circ t_2\\&=v\circ t_2\\&=t
	\end{split}\qquad \begin{split}
	r\circ m_2&=u\circ r_2\circ m_2\\&=u\circ u_2\circ s_2\\&=m\circ s_2\\&=s
	\end{split}\]

Thus by \Cref{prop:uni2,thm:un} $[r]$ is the supremum of $[s]$ and $[t]$, which is exactly the thesis.	
\end{proof}





