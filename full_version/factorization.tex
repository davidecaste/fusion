
\chapter{Preliminaries on subobjects, images, pullbacks and factorization systems}\label{app:fact}
This first section is devoted to recall some classic notions related to subobjects, pullbacks and ways to factorize an arrow. The main source for the presentation is \cite[Ch.~1]{dikranjan2013categorical}.

\section{The pullback functor and right $\mathcal{M}$-images}
Let us start defining the \emph{pullback functor} and show how the existence of a left adjoint to it yields a notion of image of an arrow.

\begin{definition}
	Let $\mathcal{M}$ be a class of arrows in a category $\X$ with $\mathcal{M}$-pullbacks. Suppose that $\mathcal{M}$ is stable under pullbacks, then given an arrow $f\colon X\to Y$, we define the \emph{pullback functor} $f^*\colon \mathcal{M}/Y\to \mathcal{M}/X$ sending an arrow $m:M\to Y$ in $\mathcal{M}$ to a chosen pullback of it along $f$. Given a morphism $g:m\to n$ in $\mathcal{M}/Y$, the arrow $f^*(g)$ is the unique one fitting in the diagram below.
	\[\xymatrix{f^*(M)  \ar[dd]_{f^*(m)} \ar@{.>}[dr]_{f^*(g)}\ar[rr]^{p_m}&& M \ar[dr]^{g}\ar[dd]^(.3){m} \\&f^*(N)  \ar[rr]^(.35){p_n}|(.55)\hole \ar[dl]_{f^*(n)}&& N \ar[dl]^{n}\\ X \ar[rr]_{f}&& Y }\]
\end{definition}

\begin{notation}
	If $m\colon M\to X$ is an arrow in $\mathcal{M}$ and $f\colon X\to Y$ is another morphism, we will denote the chosen domain of $f^*(m)$ by $f^*(M)$.
\end{notation}

\begin{proposition} Let $f\colon X\to Y$ be an arrow in a category $\X$ and $\mathcal{M}$ a class of arrows stable under pullbacks and such that $f$ has a pullback along any arrow in $\mathcal{M}$. Let also $F\colon\D\to \mathcal{M}/Y$ be a diagram with a limiting cone $(l, \{l_D\}_{D\in \D})$ and suppose that  $L$ is the domain of $l$. Denoting $F(D)$ by  $m_D\colon M_D\to Y$, if  $(L, \{l_D\}_{D\in \D}+\{l\})$ is limiting for the diagram in $\X$ determined by the family of arrows $\{m_D\}_{D\in \D}$, then for every $f\colon X\to Y$, $f^*$ preserves the limit of $F$.
\end{proposition}
\begin{proof}
	Let $(a, \{a_D\}_{D\in \D})$ be a cocone in $\{\mathcal{M}/X\}$ over $f^*\circ F$. Then we have the solid part of the diagram below.
	\[\xymatrix{A \ar@/_.8cm/[dd]_{a}\ar[d]^{a_D}  \ar@{.>}[r]_-{b} \ar@/^.5cm/@{.>}[rrr]^{c}&f^*(L)  \ar@/^.7cm/[ddl]|(.37)\hole^(.6){f^*(l)} \ar[rr]_{q} \ar[dl]^{f^*(l_D)}&&L \ar@/^.3cm/[ddl]^{l} \ar[dl]_{l_D}\\f^*(M_D)  \ar[d]^{f^*(m_D)} \ar[rr]^(.7){p_{F(D)}}&& M_D \ar[d]_{m_D} \\ X \ar[rr]_{f}&& Y }\]
	
	Now, notice that, for every $D\in \D$ we have
	\begin{align*}
		m_D\circ p_{F(D)} \circ a_D&=f\circ f^*(m_D)\circ a_D \\&=f\circ a
	\end{align*}
	Moreover, given an arrow $g\colon D\to D'$ in $\D$ then we have
	\begin{align*}
		F(g)\circ p_{F(D)}\circ a_D&=p_{F(D')}\circ f^*(F(g))\circ a_D\\&=p_{F(D')}\circ a_{D'}
	\end{align*}
	
	Thus $(A, \{p_{F(D)}\circ a_D\}_{D\in \D}+ \{f\circ a\})$ is a cone over the diagram in $\X$ determined by $\{m_D\}_{D\in \D}$, so that we get the dotted arrow $b\colon A\to L$ in the diagram above. This arrow,  in turn, yields also $a\colon A\to f^*(L)$. Moreover, we also have that
	\[
	\begin{split}
		p_{F(D)}\circ f^*(l_D)\circ b&= l_D\circ q \circ b\\&=l_D\circ c\\&=p_{F(D)}\circ a_D
	\end{split}\qquad 
	\begin{split}
		f^*(m_D) \circ f^*(l_D)\circ b&= f^*(l)\circ b\\&=a\\&=f^*(m_D)\circ a_D
	\end{split}
	\]
	So that $b$ defines an arrow $a\to f^*(l)$ in $\mathcal{M}/X$ auch that $f^*(l_D)\circ b =a_D$
	
	To see that $b$ is unique, let $b'\colon a\to f^*(l)$ be another arrow such that $f^*(l_D)\circ b' =a_D$, then
	\[\begin{split}
		l\circ q \circ b'&=f\circ f^*(l)\circ b'\\&=f\circ a\\&=l\circ c
	\end{split} \qquad \begin{split}
		l_D\circ q \circ b'&=p_{F(D)}\circ f^*(l_D)\circ b'\\&=p_{F(D)}\circ a_D\\&=l_D\circ c
		3\end{split}\]
	Thus $q\circ b'=c$, implying that $b'=b$.
\end{proof}

The previous theorem gives us at once the following result.

\begin{corollary} \label{cor:finlim}
	Let $f\colon X\to Y$ be an arrow in a category $\X$ and $\mathcal{M}$ a class of arrows stable under pullbacks and such that $f$ has a pullback along any arrow in $\mathcal{M}$. Then $f^*$ preserves binary products. If, moreover, $\id{Y}$ belongs to $\mathcal{M}$ then $f^*$ preserves finite products.
\end{corollary}

Having examined the continuity properties of the pullback functor we can now turn to the notion of \emph{right $\mathcal{M}$-images}.

\begin{definition}
	Let $\mathcal{M}$ be a class of arrows in a category $\X$. A \emph{right $\mathcal{M}$-factorization
	} of a morphism $f\colon X\to Y$ is a pair of arrows $m\colon M\to Y$ and $e\colon X\to M$ such that:
	\begin{enumerate}
		\item $m$ belongs to $\mathcal{M}$ and $f=m\circ e$;
		\item given an arrow $n\colon N\to Z$ in $\mathcal{M}$ and two morphisms $g\colon Y\to Z$, $h\colon X\to N$ makint the solid part of the diagram below commutative, then there exists a unique dotted $v\colon M\to N$ fitting in it.
		\[\xymatrix{X \ar[r]_-{e} \ar@/^.4cm/[rr]^{f} \ar[d]_{h}& M \ar[r]_-{m} \ar@{.>}[dl]^{v} & Y \ar[d]^{g}\\N \ar[rr]_{n} && Z}\] 
	\end{enumerate}  
	
	The arrow $m$ will be called the \emph{right $\mathcal{M}$-image} of $f$.
\end{definition}

\begin{remark}\label{rem:uniq}
	Right $\mathcal{M}$-factorization are essentially unique. To see this, let $e\colon X\to M$ and $m\colon M\to Y$ be a right $\mathcal{M}$-factorization of an arrow $f\colon X\to Y$ and suppose that  $e'\colon X\to M'$ and $m'\colon M'\to Y$ is another, then we  get the existence of the dotted arrows in the following two diagrams
	\[\xymatrix{X \ar[r]^-{e}  \ar[d]_-{e'}& M \ar[r]^-{m} \ar@{.>}[dl]^{v} & Y \ar[d]^{\id{Y}} & X \ar[r]^-{e'}  \ar[d]_{e}& M' \ar[r]_-{m'} \ar@{.>}[dl]^{w} & Y \ar[d]^{\id{Y}}\\M' \ar[rr]_{m'} && Y & M \ar[rr]_{m} && Y}\] 
	On the other hand, from the commutativity of the diagram below we get at once that $v\circ w=\id{M'}$ and $w\circ v=\id{M}$.
	\[\xymatrix{X \ar[rr]^-{e}  \ar[dr]_-{e'} \ar[dd]_{e}&& M \ar[r]^-{m} \ar[dl]^{v} & Y \ar[dd]^{\id{Y}} & X \ar[dr]^{e} \ar[rr]^-{e'}  \ar[dd]_{e'}&& M' \ar[r]_-{m'} \ar[dl]^{w} &Y \ar[dd]^{\id{Y}}\\ & M' \ar[dl]^{w}  \ar[drr]^{m'} &&&&M\ar[dl]^{v} \ar[drr]^{m}\\M \ar[rrr]_{m} &&& Y & M' \ar[rrr]_{m'} &&& Y}\] 
\end{remark}


\begin{proposition}\label{prop:im}
	Let $\mathcal{M}$ be a class of arrows in a category $\X$ and suppose that every morphism in $\X$ has a right $\mathcal{M}$-factorization. Then for every arrow $f\colon X\to Y$, there is a functor $f_*\colon \mathcal{M}/X\to \mathcal{M}/Y$ sending $m\colon M\to X$ to a chosen right $\mathcal{M}$-image of $f\circ m$.
\end{proposition}
\begin{notation}
	As in the case of the pullback functor, if $m$ is an arrow $M\to X$, then we will denote the chosen domain of $f_*(m)$ by $f_*(M)$.
\end{notation}
\begin{proof}
	We only have to define $f_*$ on arrows. Let thus $g$ be a morphism in $\mathcal{M}/X$ between $m\colon M\to X$ and $n\colon N\to X$. Now, since \[f\circ m = f\circ n\circ g\] the solid part of the diagram below commutes, allowing us to define $f_*(g)$ as the dotted arrow.
	\[\xymatrix@C=30pt{M \ar[dr]_(.35){m}|(.55)\hole\ar[r]^-{e_m} \ar[d]_{g}& f_*(M)  \ar@/_.3cm/@{.>}[ddl]^(.7){f_*(g)}\ar[r]^-{f_*(m)}& Y \ar[dd]^{\id{Y}}\\N \ar[d]_{e_n} \ar[r]|(.32)\hole^(.6){n} & X  \ar[dr]^{f}\\f_*(N) \ar[rr]_{f_*(n)}&& Y}\]
	
	It is clear that, with the above definition, $f_*(\id{m})=\id{f_*(m)}$. Let $h\colon n\to o$ be another morphism in $\mathcal{M}/X$, then we can build the following diagram.
	\[\xymatrix@C=30pt{M \ar[r]^-{e_m} \ar[d]_{g}& f_*(M)  \ar[d]_{f_*(g)}\ar[r]^-{f_*(m)}& Y \ar[dd]^{\id{Y}}\\N \ar[r]^-{e_n} \ar[d]_{h} & f_*(N) \ar[d]_{f_*(h)} \ar[dr]^{f_*(n)}\\O\ar[r]_-{e_o}& f_*(O) \ar[r]_{f_*(o)}& Y}\]
	From this it follows at once that $f_*(h)\circ f_*(g)=f_*(g\circ h)$.
\end{proof}
\begin{remark} If $\id{X}$ belongs to $\mathcal{M}$, then $f_*(\id{X})$ is a right $\mathcal{M}$-image of $f$.
\end{remark}

The relationship of right $\mathcal{M}$-images to the pullback functors is explained by the following two results.

\begin{proposition}\label{prop:part}
	Let $\mathcal{M}$ be a class of arrows  in a category $\X$, containing  all identities and stable under pullbacks. Suppose, moreover, that every morphism has a right $\mathcal{M}$-factorization. Then $f_*\dashv f^*$ for every arrow $f\colon X\to Y$.  
\end{proposition}
\begin{proof}
	First of all we can notice that for every $m\colon M\to X$ in $\mathcal{M}/X$ the commutativity of the solid part of the following diagram yields the dotted morphism $\eta_n\colon m\to f^*(f_*(m))$.
	\[\xymatrix@C=30pt{M \ar@{.>}[dr]^{\eta_m} \ar@/^.4cm/[drr]^{e_m} \ar@/_.4cm/[ddr]_{m}\\ & f^*(f_*(M)) \ar[d]^{f^*(f_*(m))}\ar[r]^-{p_{f_*(m)}}& f_*(M) \ar[d]^{f_*(m)}\\ & X\ar[r]_f & Y}\]
	
	To see that $\eta_n$ is the component of a unit $\eta\colon \id{\mathcal{M}/X}\to f^*\circ f_*$, let $g$ be a morphism $m\to f^*(n)$ for some $n\colon N\to Y$ in $\mathcal{M}/Y$. If $p_n:f^*(N)\to N$ is the canonical projection, then 
	\begin{align*}
		n\circ p_n\circ g&=f\circ f^*(n)\circ g\\&=f\circ m
	\end{align*}
	Therefore we can consider the diagram below to get $h\colon f_*(m)\to n$ in $\mathcal{M}/Y$. 
	\[\xymatrix@C=30pt{M \ar[r]^-{e_m} \ar[d]_{g}& f_*(M) \ar[r]^-{f_*(m)}\ar@{.>}@/^.3cm/[ddl]^{h} & Y \ar[dd]^{\id{Y}}\\f^*(N) \ar[d]_{p_n}\\N \ar[rr]_{n}&& Y}\]
	
	Finally, the equations
	\[\begin{split}
		f^*(n)\circ f^*(h)\circ \eta_m&=f^*(f_*(m))\circ \eta_m\\&=m\\&=f^*(n)\circ g
	\end{split} \qquad \begin{split}
		p_n\circ f^*(h)\circ \eta_m&=h\circ p_m\circ \eta_m\\&=h\circ e_m\\&=p_n\circ g
	\end{split}\]
	entail that $f^*(h)\circ \eta_m=g$ as wanted.
	
	Let now $h'\colon f_*(m)\to n$ be an arrow such that $f*(h')\circ \eta_m=g$, then
	\begin{align*}
		h'\circ e_m&=h'\circ p_{f_*(m)}\circ \eta_m\\&=p_n\circ f^*(h')\circ \eta_m\\&=p_n\circ g
	\end{align*}
	Thus $h'$ fits in the diagram
	\[\xymatrix@C=30pt{M \ar[r]^-{e_m} \ar[d]_{g}& f_*(M) \ar[r]^-{f_*(m)}\ar@/^.3cm/[ddl]^{h'} & Y \ar[dd]^{\id{Y}}\\f^*(N) \ar[d]_{p_n}\\N \ar[rr]_{n}&& Y}\]
	showing that $h=h'$.	
\end{proof}

\begin{lemma}\label{lem:rfact}
	Let $\mathcal{M}$ be a class of arrows  in a category $\X$, containing all identities then the following are equivalent:
	\begin{enumerate}
		\item $\mathcal{M}$ is stable under pullbacks, $\X$ has all $\mathcal{M}$-pullbacks and every morphism has a right $\mathcal{M}$-factorization;
		\item $\mathcal{M}$ is stable under pullbacks, $\X$ has all $\mathcal{M}$-pullbacks and for every morphism $f\colon X\to Y$, $f^*$ has a left adjoint;
		\item every morphism has a right $\mathcal{M}$-factorization and for every arrow $f\colon X\to Y$, $f_*$ has a right adjoint.
	\end{enumerate}
\end{lemma}
\begin{proof}
	$(1\Rightarrow 2)$ This follows at once from \Cref{prop:part}.
	
	\smallskip \noindent $(2\Rightarrow 1)$ Let $L_f\colon \mathcal{M}/X\to \mathcal{M}/Y$ be the left adjoint to $f^*$. Since $\id{X}$ belongs to $\mathcal{M}$ we can define $m\colon M\to Y$ in $\mathcal{M}$ as  $L_f(\id{X})$. Moreover, let $\eta_{\id{X}}$ be the component at $\id{X}$ of the unit of the adjunction $L_f \dashv f^*$, if $p_m$ is the projection $f^*(M)\to M$, we can also define $e$ as $p_m\circ \eta_{\id{X}}$. Notice that, by construction, $m\circ e=f$.
	\[\xymatrix@C=30pt{X \ar@/^.4cm/[rr]^e \ar@/_.4cm/[dr]_{\id{X}} \ar[r]_-{\eta_{\id{X}}} & f^*(M) \ar[r]_-{p_m} \ar[d]^{f^*(m)}& M \ar[d]^{m}\\ & X \ar[r]_f& Y}\]
	
	We have to show that $e$ and $m$ so defined really give us a right $\mathcal{M}$-factorization of $f$. We thus start with the solid part of the diagram below, with $n\in \mathcal{M}$.
	\[\xymatrix@C=35pt{f^*(M)\ar[dr]^-{p_m}\\X \ar[r]^-{e} \ar[u]^{\eta_{\id{X}}} \ar|(.36)\hole@/_.5cm/[rr]_{f} \ar[d]_{h}&   M \ar[r]^-{m} \ar@{.>}[dl]^(.6){v} & Y \ar[d]^{g}\\N \ar[rr]_{n} && Z}\] 
	
	We can apply the pullback property of the inner rectangle below to get the dotted arrow $w\colon M\to f^*(g^*(N))$, which, by construction, defines an arrow $m\to f^*(g^*(n))$.
	\[\xymatrix@C=30pt{X \ar@/^.5cm/[drrr]^{h} \ar@/_1cm/[ddr]_{\id{X}} \ar@{.>}[dr]^{w}\\ &f^*(g^*(N)) \ar[d]_{f^*(g^*(n))}\ar[r]^-{p_{g^*(n)}}& g^*(N) \ar[d]_{g^*(n)} \ar[r]^{p_n} & N \ar[d]^{n} \\ &X \ar[r]_{f}& Y \ar[r]_g & Z}\]
	
	By construction $w$ defines a morphism $\id{X}\to f^*(g^*(n))$, thus by adjointness, this implies the existence of a unique $\hat{w}\colon m\to g^*(n)$ such that $f^*(\hat{w})\circ \eta_{\id{X}} =w$. Let $v$ be $p_n\circ \hat{w}$, then
	\[\begin{split}
		v\circ e &= p_n\circ \hat{w} \circ p_m\circ \eta_{\id{X}} \\&=p_n\circ p_{g^*(n)}\circ f^*(\hat{w})\circ \eta_{\id{X}}\\&=p_n\circ p_{g^*(n)}\circ w \\&=h
	\end{split}\qquad  \begin{split}
		n\circ v &=n\circ p_n\circ \hat{w}\\&=g\circ g^*(n)\circ \hat{w}\\&=g\circ m\\&
	\end{split}\] 
	
	To see that such $v$ is unique, let $v'$ be another arrow $M\to N$ such that
	\[v'\circ e=h \qquad n\circ v'=g\circ m\]
	The second of these equations entail the existence of the dotted $q\colon M\to g^*(N)$.
	\[\xymatrix{M \ar@{.>}[dr]^{q}\ar@/^.3cm/[drr]^{v'} \ar@/_.3cm/[ddr]_{m}\\ &g^*(N) \ar[d]^{g^*(n)}\ar[r]^{p_n}& N\ar[d]^{n}\\ &Y\ar[r]_{g} & Z}\]
	Notice that, in particular, $q$ is a morphism $m\to g^*(n)$ in $\mathcal{M}/Y$. Now, we can also consider the following diagram, which entails that $f^*(q)\circ \eta_{\id{X}}=w$.
	\[\xymatrix@C=30pt{f^*(M) \ar@/_1.2cm/[ddd]_{f^*(m)}\ar[dd]_{f^*(q)}\ar[rr]^{p_m}&&M \ar[dd]_{q}\ar@/^.9cm/[ddd]_(.4){m}|(.67)\hole \ar@/^.6cm/[ddr]^{v'}\\&X\ar@/^.8cm/[ddl]|(.33)\hole_(.2){\id{X}} \ar[drr]_(.3){h}|\hole|(.74)\hole\ar[ul]^{\eta_{\id{X}}} \ar[ur]_{e}\\ f^*(g^*(N)) \ar[d]^{f^*(g^*(n))}\ar[rr]_(.67){p_{g^*(n)}}&& g^*(N) \ar[d]_{g^*(n)} \ar[r]_(.65){p_n} & N \ar[d]^{n} \\ X \ar[rr]_{f}&& Y \ar[r]_g & Z}\]
	Thus $q$ must coincide with $\hat{w}$ and therefore $v'=v$.
	
	\smallskip \noindent $(1\Rightarrow 3)$ This also follows from \Cref{prop:part}.
	
	\smallskip \noindent $(3\Rightarrow 1)$ Let $R_f\colon \mathcal{M}/Y\to \mathcal{M}/Y$ be the right adjoint to $f_*$. Given $n\colon N\to Y$ in $\mathcal{M}$, we can build the rectangle below, where $f(R_f(n))$ and $e$ is a (chosen) right $\mathcal{M}$-factorization of $f\circ R_f(n)$ and $\epsilon\colon f_*\circ R_f\to \id{\mathcal{M}/Y}$ is the counit of the adjunction $f_* \dashv f^*$. 
	\[\xymatrix{R_f(N) \ar[d]_{R_f(n)} \ar[r]^-{e}& f_*(R_f(N))  \ar[r]^-{\epsilon_n} \ar[dr]_{f_*(R_f(n)) \hspace{3pt}}& N \ar[d]^{n}\\
		X \ar[rr]_{f}&& Y}\]
	
	We have to show that such a rectangle is a pullback. To do so, let $h\colon Z\to N$ and $g\colon Z\to X$ be arrows such that $n\circ h=f\circ g$ and consider the right $\mathcal{M}$-factorization of $g$ given by arrows $i\colon Z\to K$ and $k\colon K\to X$, with $k\in \mathcal{M}$. Thus there exists the dotted $w\colon K\to N$ in the diagram below.
	\[\xymatrix{Z \ar[r]_{i}\ar@/^.4cm/[rr]^{g}\ar[d]_{h}& K \ar@{.>}[dl]^{w} \ar[r]_k & X \ar[d]^{f}\\
		N \ar[rr]_{n}&& Y}\]
	
	In turn, we can factor $f\circ k$ to get the following rectangle and, henceforth, the existence of the dotted $u\colon f_*(K)\to N$.
	\[\xymatrix{K \ar[r]_-{j}\ar@/^.4cm/[rr]^-{f\circ k}\ar[d]_{w}& f_*(K) \ar@{.>}[dl]^{u} \ar[r]_{f_*(k)} & Y \ar[d]^{\id{Y}}\\
		N \ar[rr]_{n}&& Y}\]
	
	By construction $u$ is a morphism $f_*(k)\to n$, we can thus consider the unique morphism $\hat{u}\colon k\to R_f(n)$ such that
	\[\epsilon_n\circ f_*(\hat{u})=u\]
	
	Summing up we have constructed the diagram that follows.
	\[\xymatrix{Z \ar@/^.4cm/[drrrr]^{h}\ar[dr]^{i} \ar@/_1cm/[ddrr]_{g}\\& K \ar[r]^-{\hat{u}} \ar@/_.2cm/[dr]_{k}& R_f(N) \ar[d]_{R_f(n)} \ar[r]^-{e}& f_*(R_f(N))  \ar[r]_-{\epsilon_n} \ar[dr]_{f_*(R_f(n)) \hspace{3pt}}& N \ar[d]^{n}\\ && 
		X \ar[rr]_{f}&& Y}\]
	
	To conclude it is now enough to show that $\hat{u}\circ i$ is the unique morphism $Z\to R_f(N)$ such that 
	\[\epsilon_n\circ e \circ \hat{u}\circ i=h\qquad R_f(n)\circ \hat{u}\circ i=g\]
	
	Let, therefore, $t$ be an arrow $Z\to R_f(N)$ such that
	\[\epsilon_n\circ e \circ t=h\qquad R_f(n)\circ t=g\]
	The second equation provides us with the commutative rectangle below, guaranteeing the existence of the dotted $s\colon k\to R_f(n)$.
	\[\xymatrix{Z \ar[r]_-{i}\ar@/^.4cm/[rr]^-{g}\ar[d]_{t}& K \ar@{.>}[dl]^{s} \ar[r]_{k} & X \ar[d]^{\id{X}}\\
		R_f(N) \ar[rr]_{R_f(n)}&& X}\]
	
	If we compute we get:
	\[\begin{split}
		n\circ \epsilon_n\circ e\circ s&=f_*(R_f(n))\circ e\circ s\\&=f\circ R_f(n)\circ s\\&=f\circ k
	\end{split} \qquad \begin{split}
		\epsilon_n\circ e\circ s\circ i&=\epsilon_n\circ e\circ t\\&=h\\&
	\end{split} \]
	
	On the other hand, the definition of $f_*$ gives us the inner rectangle of the diagram below.
	\[\xymatrix@C=30pt{K \ar@/_1.2cm/[ddd]_{w}\ar[dr]_(.35){k}|(.55)\hole\ar[r]^-{j} \ar[d]_{s}& f_*(K)  \ar@/_.3cm/[ddl]^(.75){f_*(s)}\ar[r]^-{f_*(k)}& Y \ar[dd]^{\id{Y}}\\R_f(N) \ar[d]_{e} \ar[r]|(.34)\hole_(.6){R_f(n)} & X  \ar[dr]^{f}\\f_*(R_f(N)) \ar[d]_{\epsilon_n} \ar[rr]_{f_*(R_f(n))}&& Y\\N\ar@/_.8cm/[urr]_{n}}\]
	
	From the commutativity of the whole diagram it follows that $\epsilon_n\circ f_*(s)=u$ and thus $s=\hat{u}$, entailing the thesis.
\end{proof}

\section{Factorization systems}

We move now to the notion of \emph{factorization system} \cite{adamek2009abstract,bousfield1977constructions,kelly1991note,rosicky2007factorization,tholen1983factorizations}. Let us start by recalling its definition.

\begin{definition}\label{def:fs}
	Let $\X$ be a category and $\mathcal{E}$, $\mathcal{M}$ two classes of arrows, we will say that $(\mathcal{E},\mathcal{M})$ is a \emph{factorization system} if:
	\begin{enumerate}
		\item $\mathcal{E}$ and $\mathcal{M}$ are closed under composition with isomorphisms: if $f\colon X\to Y$ belongs to $\mathcal{E}$ (to $\mathcal{M}$) and $h\colon Y\to Z$ is an isomorphism then $h\circ f$ belongs to $\mathcal{E}$ (to $\mathcal{M}$);
		\item  every arrow $f\colon X\to Y$ of $\X$ admits a \emph{$(\mathcal{E}, \mathcal{M})$-factorization}, i.e.~there are arrows $e\in \mathcal{E}$ and $m\in \mathcal{M}$ with the property that $f=m\circ e$;
		\item every $e\in \mathcal{E}$ has the \emph{left lifting property} with respect to every $m\in \mathcal{M}$: for every commutative square as the one below, with $e\in \mathcal{E}$ and $m\in \mathcal{M}$ there exists a unique dotted $k\colon Y\rightarrow Z$ which fits in it.
		\[\xymatrix{X \ar[r]^g \ar[d]_{e}& Z \ar[d]^{m}\\ Y \ar[r]_{f} \ar@{.>}[ur]^{k}& V}\] 
	\end{enumerate}
	
	A factorization system is \emph{proper} if every $e\in \mathcal{E}$ is epi and every $m\in \mathcal{M}$ is mono; it is \emph{stable} if $\mathcal{E}$ is stable under pullbacks.
\end{definition}

\begin{remark}\label{rem:dual}It is immediate to notice that, given a factorization system $(\mathcal{E}, \mathcal{M})$ on $\X$, then $(\mathcal{M}, \mathcal{E})$ is a factorization system on $\X^{op}$. Such factorization system on $\X^{op}$ is proper if and only if $(\mathcal{E}, \mathcal{M})$ is so, it is stable if and only if $\mathcal{M}$ is stable under pushouts in $\X$.  
\end{remark}

We can now begin to prove some properties of factorization systems.

\begin{proposition}\label{prop:el}
	Let $(\mathcal{E}, \mathcal{M})$ be a factorization system on a category $\X$, then the following hold true:
	\begin{enumerate}
		\item $\mathcal{E}$ is the class of all arrows with the left lifting properties with respect to every $m\in \mathcal{M}$;
		\item every isomorphism belongs to $\mathcal{E}$;
		\item $\mathcal{E}$ is closed under composition.
	\end{enumerate}
\end{proposition}
\begin{proof}
	\begin{enumerate}
		\item By definition, every arrow in $\mathcal{E}$ has the left lifting property with respect to every arrow in $\mathcal{M}$. To prove the converse, let $f\colon X\to Y$ be an arrow with such a property with respect to every element of $\mathcal{M}$.  We can factorize it as $m\circ e$ with $m\colon M\to Y$ in $\mathcal{M}$ and $e\colon X\to M$ in $\mathcal{E}$. Thus, in the diagram below, the outer square commutes and so we get the dotted $k\colon Y\to M$.
		\[\xymatrix{X\ar[d]_f\ar[r]^e & M\ar[d]^{m}\\Y\ar[r]_{\id{Y}} \ar@{.>}[ur]^{k} & Y}\]
		
		On the other hand, $k\circ m$ fits in the diagram below, showing that $k$ and $m$ are mutually inverses. Thus $f=m\circ e$ belongs to $\mathcal{E}$.
		\[\xymatrix{X\ar[rr]^{e} \ar[dd]_{e} \ar[dr]^{f}&& M\ar[dd]^{m} \ar@{<-}[dl]^{m}\\& Y\ar@{<-}[dl]^{k}\\M \ar[rr]_{m}&&Y}\]
		\item By the previous point, it is enough to show that every isomorphism has the left lifting property. But this follows at once: indeed given a square
		\[\xymatrix{X\ar[d]_e\ar[r]^f & M\ar[d]^{m}\\E\ar[r]_{g} \ar@{.>}[ur]^{k} & Y}\]
		the unique $k\colon E\to M$ filling the diagonal in the diagram above is $f\circ e^{-1}$.
		
		\item Let $e\colon X\to Y$ and $e'\colon Y\to Z$ be two arrows in $\mathcal{E}$. Suppose that the outer rectangle below is given, with $m\in \mathcal{M}$.
		\[\xymatrix{X  \ar[r]^f \ar[d]_{e}& M \ar[dd]^{m}\\Y \ar@{.>}[ur]^{k} \ar[d]_{e'}\\Z \ar@{.>}[uur]_{h}\ar[r]_{g}& W}\]
		
		Then the first dotted arrow $k\colon Y\to M$ exists since $e$ has the left lifting property with respecto to $m$. Its existence, in turn yields the existence of the other dotted arrow $h\colon Z\to M$. 
		
		Let $h'\colon Z\to M$ be another arrow such that 
		\[m\circ h'=g \qquad h'\circ e'\circ e=f\]
		Then $h'\circ e'$ must coincide with $k$ by the uniqueness clause of the left lifting property and thus, by the same clause, $h'$ must be equal to $h$. By the first point $e'\circ e$ belongs to $\mathcal{E}$. \qedhere 
	\end{enumerate}
\end{proof}

Taking into account \Cref{rem:dual}, the previous proposition gives us the following results.

\begin{corollary}\label{cor:m}Let $(\mathcal{E}, \mathcal{M})$ be a factorization system on a category $\X$, then the following hold true:
	\begin{enumerate}
		\item every isomorphism belongs to $\mathcal{M}$;
		\item $\mathcal{M}$ is closed under composition.
	\end{enumerate}
\end{corollary}

In the presence of a factorization system, we also have right $\mathcal{M}$-factorizations.

\begin{proposition}\label{prop:rfact}
	Let $(\mathcal{E}, \mathcal{M})$ be a factorization system on a category $\X$. Let $f\colon X\to Y$ be a morphism with $e\colon X\to M$ and $m\colon M\to Y$ as an $(\mathcal{E}, \mathcal{M})$-factorization of $f$, then the pair $(e, m)$ is also a right $\mathcal{M}$-factorization of $f$.
\end{proposition}
\begin{proof}
	Suppose that the solid part of the rectangle below is given, with $n\colon N\to Z$ in $\mathcal{M}$. 
	\[\xymatrix{X \ar[d]_{h} \ar@/^.4cm/[rr]^{f} \ar[r]_{e}& M \ar@{.>}[dl]^{k} \ar[r]_{m} & Y \ar[d]^{g}\\N \ar[rr]_{n}&& Z}\]
	Bu the existence of the dotted $k\colon M\to N$ is guaranteed by the left lifting property of $e$ with respect to $n$.
\end{proof}

The previous proposition, together with \Cref{rem:uniq} gives us the following.

\begin{corollary}\label{prop:iso}
	Let $(\mathcal{E}, \mathcal{M})$ be a factorization system on a category $\X$. If $e\colon Y\to E$, $e'\colon Y\to E'$ and $m\colon Y\to X$, $m'\colon Y'\to X $ are arrows, respectively, in $\mathcal{E}$ and $\mathcal{M}$ such that
	$e'\circ m'=e\circ m$,  then  there exist a unique isomorphism $\phi\colon E \rightarrow E'$ such that the diagram below commutes.
	\[\xymatrix@C=40pt{Y \ar[r]^{e'} \ar[d]_{e}& E' \ar[d]^{m'}\\ E \ar[r]_{m} \ar@{.>}[ur]^{\phi}& X}\]
\end{corollary}

The previous corollary, in turn, allows us to deduce other useful properties of factorization systems.

\begin{proposition}\label{cor:iso} Given a factorization system $(\mathcal{E}, \mathcal{M})$ on a category $\X$, the following hold:
	\begin{enumerate}
		\item an arrow $f\colon X\to Y$ with $(\mathcal{E}, \mathcal{M})$-decomposition given by $e$ in $\mathcal{E}$ and $m\in \mathcal{M}$ is in $\mathcal{M}$ if and only if $e$ is an isomorphism;
		\item $f\in \mathcal{E}$ and $f\in \mathcal{M}$ if and only if $f$ is an isomorphism;
		\item $\mathcal{M}$ is stable under pullback.
		\item  if $(\mathcal{E}, \mathcal{M})$ is proper, then $g\circ f$ is in $\mathcal{M}$ implies $f\in \mathcal{M}$.
	\end{enumerate}
\end{proposition}
\begin{proof}
	\begin{enumerate}
		\item  ($\Rightarrow$) By hypothesis $f=f\circ \id{X}$ is a factorization with $\id{X}\in \mathcal{E}$ and $f\in \mathcal{M}$, thus the thesis follows from \Cref{prop:iso}.
		
		\smallskip \noindent
		($\Leftarrow$) Let $f=m\circ e$ be an $(\mathcal{E}, \mathcal{M})$-factorization. The thesis  follows from \Cref{cor:m} since, if $e$ is an isomorphism, then $f$ is the composition of two arrows in $\mathcal{M}$.
		
		\item This follows from the previous point, \Cref{cor:m} and \Cref{prop:el}.
		
		\item 	Suppose that the square below is a pullback, with $m\in \mathcal{M}$
		\[\xymatrix{X\ar[r]^{g} \ar[d]_{n} & M \ar[d]^{m} \\ Y \ar[r]_{f}  & Z}\] 
		We can factor  $n$ as $t\circ e$ for $e\colon X\to M$ in  $\mathcal{E}$ and $t\colon T\to Y$ in $ \mathcal{M}$.  By the left lifting property we get the dotted $l\colon T\to M$ in the diagram below.
		\[\xymatrix{X \ar[d]_{e}\ar[rr]^{g} & &V\ar[d]^{m}\\  T  \ar@{.>}[urr]^{l} \ar[r]_{t}&Y\ar[r]_{f} & Z}\]
		Therefore we get another diagram
		\[\xymatrix{ T \ar@/^.5cm/[drr]^{l}\ar@{.>}[dr]^{k} \ar@/_.5cm/[ddr]_{t}\\ & X \ar[d]_{n}\ar[r]^{g}& M  \ar[d]^{m} \\ &Y \ar[r]_{f} & Z }\]
		and thus we can deduce the existence of the dotted $k\colon E\to V$. On the one hand, computing we get
		\[\begin{split}
			n\circ 	k\circ e&=t  \circ  e\\ &=n
		\end{split}\qquad \begin{split}
			g\circ k\circ e &=l\circ e\\ &=g \end{split}\]
		and so $k\circ e= \id{X}$.  On the other hand 
		\[\begin{split}
			t\circ k \circ t &= \id{Y} \circ t \\&= t
		\end{split} \qquad 
		\begin{split}
			k \circ t \circ e &= k\circ n \\&= e
		\end{split}
		\]	
		so the following diagram commutes.
		\[
		\xymatrix@C=50pt{X \ar[d]_{e}\ar[r]^{e}& T \ar[d]^{t} \\
			T\ar[ur]^{e\circ k}\ar[r]_{t} & Y}\]
		Hence, by the uniqueness clause of the left lifting property, $e\circ k=\id{E}$. Therefore $e$ is an isomorphism and the thesis follows from  point $1$. 
		
		\item 
		Factor $f$ and $g$ as $m_f\circ e_f$ and $m_g\circ e_g$, respectively. Let also $h$ be $e_g\circ m_f$ and factor it as $m_h\circ e_h$ so that we get 
		\[\xymatrix@C=45pt{& C \ar[dr]^{m_h}&\\
			A\ar[ur]^{e_h} \ar[rr]^{h}  \ar[dr]^{m_f}& &B \ar[d]^{m_g}\\
			X \ar[u]^{e_f} \ar[r]_f& Y \ar[r]_g \ar[ur]^{e_g} & Z}\]
		B \Cref{prop:el,cor:m}, $\mathcal{E}$ and $\mathcal{M}$ are closed under composition, therefore $e_h\circ e_f\in \mathcal{E}$ and $m_g \circ m_h\in \mathcal{M}$. Thus these two arrows gives a $(\mathcal{E}, \mathcal{M})$-factorization of $g\circ f$. On the other hand $g\circ f\in \mathcal{M}$, thus point $1$ above  implies that  $e_h\circ e_f$ is an isomorphism, so $e_f$ is a epic split mono which is also an epimorphism. \qedhere
	\end{enumerate}
\end{proof}

With the help of \Cref{rem:dual} we can easy dualize the previous proposition.
\begin{corollary}\label{cor:iso2}
	Given a factorization system $(\mathcal{E}, \mathcal{M})$ on a category $\X$, the following hold:
	\begin{enumerate}
		\item an arrow $f\colon X\to Y$ is in $\mathcal{E}$ if and only if $m_f$ is an isomorphism;
		\item $\mathcal{E}$ is stable under pushouts;
		\item  if $(\mathcal{E}, \mathcal{M})$ is proper, then $g\circ f$ is in $\mathcal{E}$ implies $g\in \mathcal{E}$.
	\end{enumerate}
\end{corollary}

We conclude this section with a result linking closely factorization systems with the existence of right $\mathcal{M}$-factorizations.

\begin{lemma}\label{lem:fact}Let $\mathcal{M}$ be a class of arrows in a category $\X$ closed under composition with isomorphisms, then the following are equivalent:
	\begin{enumerate}
		\item every arrow $f\colon X\to Y$ has a right $\mathcal{M}$-factorization and $\mathcal{M}$ is closed under composition;
		\item there exists a class $\mathcal{E}$ such that $(\mathcal{E}, \mathcal{M})$ is a factorization system on $\X$.
	\end{enumerate}
\end{lemma}
\begin{proof} $(1\Rightarrow 2)$ Let $\mathcal{E}$ be the class of arrows which have the left lifting property with respect to every element of $\mathcal{M}$. Let us show that the three properties of \Cref{def:fs} hold.
	
	\begin{enumerate}
		\item $\mathcal{M}$ is closed under composition with isomorphisms by hypothesis. On the other hand, if $e\colon X\to Y$ is in $\mathcal{E}$ and $h\colon Y \to Z$ is an isomorphism, then $h\circ e$ still belongs to $\mathcal{E}$. To see this, consider the square below, with $m\in \mathcal{M}$.
		\[\xymatrix{X\ar[r]^{f} \ar[d]_{e}& M\ar[dd]^{m}  \\Y \ar@{.>}[ur]_{k}\ar[d]_h \\Z \ar[r]_{g}& V}\]
		
		The dotted $k\colon Y\to M$ exists, and it is unique, because $e\in \mathcal{E}$. The thesis now follows taking $k\circ h^{-1}$ as the lifting of $g$. 
		
		\item Given $f\colon X\to Y$, let $e\colon X\to M$ and $m\colon M\to Y$ be a right $\mathcal{M}$-factorization of it. We have to show that $e$ belongs to $\mathcal{E}$. We can further factor $e$ as $n\circ d$, with $n\in \mathcal{M}$. Then we can build the rectangle below.
		\[\xymatrix{X  \ar[dr]_(.3){e}\ar[d]_{d} \ar@/^.4cm/[rr]^{f}\ar[r]_{e} & M \ar@{.>}[dl]^(.3){k} \ar[r]_{m} & Y \ar[d]^{\id{Y}}\\
			D \ar[r]_{n} & M \ar[r]_{m}& Y}\]
		
		By hypothesis $m\circ n$ is in $\mathcal{M}$, so we can deduce the existence of the dotted $k\colon M\to D$.   Now, from the diagram below, we can deduce that $n\circ k=\id{M}$.
		\[\xymatrix{X \ar[rr]^{e} \ar[dr]^{d}\ar[dd]_{e}& &M \ar[r]^{m}\ar[dl]^{k}& Y \ar[dd]^{\id{Y}}\\ & D\ar[dl]^{n}\\M\ar[rrr]_{m}&&&Y}\]
		
		On the other hand, the previous result yields also the commutativity of the following diagram, entailing $k\circ n=\id{D}$.
		\[\xymatrix{X \ar[rr]^{d} \ar[dr]^{e}\ar[dd]_{d}& &D \ar[r]^{n}\ar[dl]^{n}& M \ar[dd]^{\id{M}}\\ & M\ar[dl]^{k} \ar[drr]^{\id{M}}\\D\ar[rrr]_{n}&&&M}\]
		
		Now, suppose that the outer square in the diagram below is given, with $v\in \mathcal{M}$.
		\[\xymatrix{X \ar[dd]_{e} \ar[dr]_{d}\ar[rr]^{t}&& V \ar[dd]^{v}\\ & D \ar[dl]^{n} \ar@{.>}[ur]_{h}\\M \ar[rr]_{s}&& Z}\]
		
		Then the dotted $h\colon N\to V$ exists and it is unique since $n$ is a right $\mathcal{M}$-image of  $e$. Thus $h\circ k$ is the unique lifting for $s$ along $v$, proving that $e\in \mathcal{E}$.
		\item This holds by construction.
	\end{enumerate}
	
	\smallskip \noindent $(2\Rightarrow 1)$  This follows from the second point of \Cref{cor:m} and from \Cref{prop:rfact}. \qedhere  
\end{proof}

Putting together \Cref{lem:fact,lem:rfact} we get the following result.

\begin{corollary}\label{cor:fact}
	Let $\mathcal{M}$ be a class of arrows in a category $\X$ closed under composition stable under pullbacks and containing the identities. If $\X$ has $\mathcal{M}$-pullbacks and for every arrow $f\colon X\to Y$ the pullback functor $f^*$ has a left adjoint, then there exists a, unique, class $\mathcal{E}$ such that $(\mathcal{E}, \mathcal{M})$ is a factorization system on $\X$.
\end{corollary}

\subsection{$\mathcal{M}$-subobjects and factorization systems}\label{app:sub}

In this section we focus on studying the category $\mathcal{M}/X$ when $\mathcal{M}$ is a class of monomorphism.

\begin{remark}
	If  $\mathcal{M}$ is a class of monos in a category $\X$, then $\mathcal{M}/X(f,g)$ has cardinality at most one for every two arrows $f$ and $g$ with codomain $X$. Thus it is a preordered class, in which $f\leq g$ if and only if  $\mathcal{M}/X(f,g)$ is non-empty. Now, it is immediate to notice that $f\leq g$ and $g\leq f$, then there exists a (unique) isomorphism fitting in the triangle below.
	\[\xymatrix{A\ar@{.>}[rr]^{\phi} \ar[dr]_{f} && B \ar[dl]^{g}\\ & X}\]
	In such a case we will write $f\equiv g$. It is immediate to see that in this way we get an equivalence relation on the class of objects of $\mathcal{M}/X$.
\end{remark}

Given the previous remark, we are led to the following definition.

\begin{definition} Let $\mathcal{M}$ be a class of monos in a category $\X$. For every object $X$ a \emph{$\mathcal{M}$-subobject} is an equivalence class $[f]$ of an object $f$ in $\mathcal{M}/X$ with respect to $\equiv$. Given two \emph{$\mathcal{M}$-subobjects} $[f]$ and $[g]$, represented by $f$ and $g$, respectively, we will define $[f]\leq [g]$ if and only if $f\leq g$. It is immediate to see that this defines a partially ordered class $(\sub{M}{X}{X}, \leq)$. 
	
	When $\mathcal{M}$ coincides with the class of all monos we will suppress the prefix, so that a \emph{subobject} is just an equivalence class of a mono and $(\msub{X}{X}, \leq)$ will denote the ordered class of such subobjects of $X$.
\end{definition}

\begin{remark}\label{rem:pbo} Let $f\colon X\to Y$ be an arrows admitting a pullback along arrows in $\mathcal{M}/Y$ and suppose that $\mathcal{M}$ is stable under pullbacks. The categories associated to $(\sub{M}{X}{X}, \leq)$ and $(\sub{M}{Y}{Y}, \leq)$ are equivalent to $\mathcal{M}/X$ and $\mathcal{M}/Y$, respectively, thus we get a monotone function $f^{-1}\colon \sub{M}{X}{Y}\to \sub{M}{X}{X}$ sending $[m]$ to $[f^*(m)]$.
	
	Similarly, if $f\circ m$ has a right $\mathcal{M}$-factorization for every $m\in \mathcal{M}$, then the functor $f_*$ induces a monotone function $\exists_f\colon \sub{M}{X}{X}\to \sub{M}{X}{Y}$.
	
	Notice that, whenever $\mathcal{M}$ is stable unr pullbacks, $\X$ has $\mathcal{M}$-pullbacks and  right $\mathcal{M}$-factorizations, then $\exists_f \dashv f^{-1}$ by \Cref{lem:rfact}.
\end{remark}

\Cref{cor:finlim} gives us immediately the following.

\begin{proposition}\label{prop:finlim}
	Let $f\colon X\to Y$ be an arrow in a category $\X$ and $\mathcal{M}$ a class of arrows stable under pullbacks and such that $f$ has a pullback along any arrow in $\mathcal{M}$. Then $f^{-1}$ preserves binary infima. If, moreover, $\id{Y}$ belongs to $\mathcal{M}$ then $f^{-1}$ preserves finite infima.
\end{proposition}

Now, let us recall the following lemma, which is the application of the Adjoint Functor Theorem \cite{freyd1964abelian,freyd1990categories} to the partial ordered case.

\begin{lemma}\label{lem:adj}
	Let $f\colon (P,\leq)\to (Q, \leq)$ be a monotone function. If $f$ preserves all infima then it has a left adjoint.
\end{lemma}
\begin{proof}
	For every $q\in Q$ define
	\[P_q:=\{p\in P \mid q \leq f(p)\}\]
	Clearly if $q'\leq q$ then $P_{q}\subseteq P_{q'}$. Let us define $g(q)$ as the infimum of $P_q$, the previous observation entails that in this way we get a monotone function $g\colon (Q, \leq)\to (P, \leq)$.
	We can also notice that, since $f$ preserves infima:
	\begin{align*}
		f(g(q))&=f(\inf(P_q))\\&=\inf(f(P_q))
	\end{align*}
	Now, by definition, $q$ is less or equal than every element in $f(P_q)$, thus $q\leq f(g(q))$, so that $g(q)$ is actually the infimum of $P_q$. 
	
	So equpped, we can now prove that $g$ is the right adjoint to $f$.
	\begin{itemize}
		\item Suppose that $g(q)\leq p$.  Then $f(g(q))\leq p$, but we already know that $q\leq f(g(q))$, therefore we can conclude that $q\leq f(p)$.
		\item  Suppose that $q\leq f(p)$. Thus $p$ belongs to $P_q$ and so $g(q)\leq p$. \qedhere 
	\end{itemize}
\end{proof}

The previous Lemma, together with \Cref{cor:fact} and \Cref{prop:finlim}, allows us to deduce the following result.

\begin{corollary}\label{cor:fin}
	Let $\mathcal{M}$ be a class of monomorphism in a category $\X$ stable under pullbacks, containing all identities and closed under composition. If $\sub{M}{X}{X}$ is finite for every object $X$ of $\X$, then there exists a factorization system $(\mathcal{E},\mathcal{M})$ on $\X$
\end{corollary}


