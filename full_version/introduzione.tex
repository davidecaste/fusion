\chapter{Introduction}

\section*{Summary (with dependencies)}

\subsection*{REWRITING}
\begin{itemize}
	\item DPO Rewriting in $\mathcal{M}$-adhesive categories, with left-$\mathcal{M}$-linear rules (from now on shortened in ``linear'')\\
	Observe that rewriting is deterministic (by uniqueness of
	PO-complement Corollary~\ref{lem:radj})
	
	\item Category of abstract derivation for DPO-rewriting system $(\X, \R)$
	\begin{itemize}
		\item \emph{DPO-derivation category} $\dpo$: objects are graphs, arrows between possibly empty, derivations.
		\item \emph{DPO-abstract decorated derivation category} $\dpi$
	\end{itemize}
	
	
	
	\item \emph{Consistent permutations} (Definition~\ref{def:permcon}):
	permutation on the steps of two abstract decorated derivations and isomorphism between the colimits such
	that arrows into the colimits ``commutes''.
	\begin{enumerate}
		\item It is unique for consuming rewriting systems
		(Corollary~\ref{cor:unique})
		\item They can be composed (Lemma~\ref{lem:sum})
	\end{enumerate}
	\item 
	
	
	\item Notions of independence\\
	We have several possibile notions of independence for two direct
	derivations $\dder{D} ; \dder{D'}$
	\begin{enumerate}
		
		\item \emph{Existence of a filler}
		There is a filler for $\dder{D} ; \dder{D'}$
		(Definition~\ref{def:filler})
		
		\item \emph{Switchable}
		There is $\dder{D}_1' ; \dder{D}_1$ which is a \emph{switch}
		of $\dder{D} ; \dder{D'}$ (Definition~\ref{de:switch}, being a
		switch is defined axiomatically) [actually, Definition~\ref{de:good-switchable} provides a definition of switchable which looks different to me, I would avoid this, to be checked] 
		
		\item \emph{(Weakly) Sequential independent} There is an
		\emph{independence pair}
		(Definition~\ref{de:sequential-independence}) They are called
		\emph{strongly sequential independent} if the independence pair is
		unique.
	\end{enumerate}
	
	We have that $(1) \Rightarrow (2)$ (Theorem~\ref{prop:fil}) and $(2) \Rightarrow (3)$ while
	we expect that in general $(3) \not\Rightarrow (1)$ (but no
	counterexample). This is instead known to hold for linear rules.
	
	
	Some terminology (Definition~\ref{de:good-switchable})
	\begin{itemize}
		
		
		\item \emph{Tame rewriting systems}\\
		An independence pairs induced by a filler is called \emph{good}
		A rewriting system where all indepedence pairs are good is called
		\emph{tame}
		
		NOTE: the categories in which condition $B$
		from~\cite{baldan2011adhesivity} identifies a large class of
		(quasi)adhesive categories where left-linear systems are all
		tame. Adapted here to M-adhesivity.
		
		\item \emph{Very Tame rewriting systems}\\
		Tame and existing indepence pairs are unique (hence switching is unique)
		(see later for examples)
	\end{itemize}
	
	
	\item \emph{Switch equivalent derivations} (concrete)\\
	Definition~\ref{de:switchable}
	
	THEOREM: Two switch equivalent derivations admit a consistent
	permutation (Lemma~\ref{lem:consperm} ??).
	
	A consequence of this used for proving that the slice of traces is a pre-order.
	
	The converse holds only when rules are linear [proof clear, but not written]
	
	
	
\end{itemize}

\subsection*{CONCURRENCY}

This part is dealt with for very tame rewriting systems and depends on this assumption. Note that
\begin{itemize}
	\item Linear rewriting systems on adhesive categories are very tame
	\item Left-linear rewritint systems on graph-like categories with generalisation of (i) no isolated nodes in $L$ and (ii) right leg injective on nodes is very tame
	\item Graphs with equivalences (Egg) with left-leg regular mono, right leg mono is very tame
\end{itemize}

We show
\begin{itemize}
	
	
	\item Three-steps lemma (Lemma~\ref{lem:iig1}), which is a weakened version of ``indepedence is global''.
	\begin{itemize}
		
		\item If in $abc$ I have that $ab$ are independent and $c$ is anticipated, they remain independent.
		\item  If in $abc$ I have that $bc$ are independent and $a$ is moved to the end, they remain independent only if there is an additional condition (essentially requiring that $a$ was not a cause of $c$)
	\end{itemize}
	
	
	
	\item No need of useless switchs (if two derivations are switch equivalent I can obtain one from another applying only inversions)
	
	\item weak prime domain e connected es semantics
\end{itemize}


\todo{A VERY NICE INTRODUCTION}

SOME IDEAS (BADLY WRITTEN, JUST TO START TO PUT IDEAS IN ORDER)

Graph-like structures are used in uncountably many setting for representing the dynamics of software systems: the graph represents the entities in the states and their relations, while rewriting steps, modifying the graph, model computational steps which produce state changes.

Rewriting over graph-like structures is elegantly captured by
DPO-rewriting over adhesive categories and their
variants~\cite{lack2005adhesive,ehrig2006weak}.  This has been largely
studied for so-called linear rewriting rules. Roughly speaking, with
linear rules a rewriting step can express the fact that some entities
in the current state are removed, some other are needed for the
rewriting step to occur but preserved and some new entities are
generated.

The computation described by a rewriting system can be naturally
interpreted as a distributed computation and a notion of independence
between rewriting steps naturally emerges. When different rules are
applied at ``distinct'' parts of the state, possibly overlapping on
preserved items, the rewriting steps can be naturally deemed as
independent and their application can occur in any order and possibly
concurrently. This phenomenon is formally captured by the so-called
Church-Rosser theorem of DPO rewriting and shift equivalence over derivations~\cite{CMREHL:AAGT}. This represent the basis for
viewing sequences of rewriting in the DPO approach as concurrent
computations and derive corresponding concurrent semantics. Derivation
traces can be quotiented by an equivalence relation which identifies
sequences of rewritings which differs only for the order of
independent steps, leading to so-called derivation traces. In turn,
the dependencies this setting can be captured by a classical structure
in concurrency theory, the so-called prime event structures~\cite{NPW:PNES} where
dependencies are reduced to causality and conflict~\cite{Handbook,Bal:PhD,Sch:RRSG}.


In some situations linear rules are not sufficiently expressive for capturing the dynamic of a system. This is the case when  a computational step is required not only to consume, preserve and generated, but also ``merge'' parts of the state. This happens, e.g.,
in nominal calculi where, as a result of name passing,
the received name is identified with a local one at the
receiver~\cite{CVY:ESSPE,Gad07} or in the modelling of bonding in biological/chemical processes~\cite{PUY:MBPE}.
add EGGS, something about string diagrams?

Technically, describing fusions, in the DPO jargon, requires to move from linear to left-linear rewriting rules. While left-linear rules have been considered in various instances (mainly for graph rewriting( and some results concerning the corresponding rewriting theory have been put forward, to the best of our knowledge the phenomena arising when rewriting with left-linear rules in the general setting of adhesive categories (which capture most ``applications'') have not been systematically explored.

Here we perform this exploration by observing that a number of basic
properties of rewriting which play a role in the corresponding
rewriting and concurrency theory need a deep rethinking for
left-linear rules and we propose solutions for the problems which
arise.

Firstly, Church-Rosser theorem does not work out of the box: the
independence condition for adhesive categories does not extend
smoothly to left-linear rules, i.e., existence of an independence pair
does not ensure the possibility of switching two derivation and a
refined notion, involving the concept of a so-called filler has to be
introduced. The problem was identified already
in~\cite{baldan2011adhesivity} where a suitable subclass of adhesive
categories is identified where independence as induced by a filler
coincide with independence as induced by an independence pair.

Having a notion of independence between computation steps sits at the core of the possibility of viewing an execution trace as a concurrent computation. This is obtained by considering traces up to an equivalence that identifies traces differing only for the order of independent steps (e.g., interchange law~\cite{Mes92} in term rewriting, shift
equivalence~\cite{CMREHL:AAGT} in graph rewriting or permutation
equivalence~\cite{JJL80} in $\lambda$-calculus). While for linear rewriting systems this works smoothly, for left-linear ones one has to face the problem that if two rewriting steps $r_a r_b$ are independent, if we switch $r_a$ after $r_a$ and then switch it back we can obtain a computation which is not the original one, making the theory of derivation traces inconsistent. In order to solve this problem, we work in a subclass of adhesive categories where derivation steps can be switched in an ``essentially unique way''. We also identify a large class of adhesive categories, arising as presheaves, where the condition is automatically satisfied. [this solves also the previous problem, right?]


Additionally, a central property for linear rewriting systems is the fact that independence of two rewriting steps is a global notion, independently of the context in which the rewritings take place. This is at the basis of the possibility of capturing the dependencies between rewriting steps in terms of two basic relations, causality and conflict, investigated in~~\cite{Handbook,Bal:PhD,Sch:RRSG}.

This fails for left-linear rules where, as observed
in~\cite{baldan2017domains}, for graph rewriting, the possibility of
merging parts of the state leads to a sort of disjunctive causality,
i.e., a rewriting step, say $r_c$ can be enabled by two different
rewriting steps $r_a$ and $r_b$, in a way that in a derivation
$r_a r_b r_c$ steps $r_b$ and $r_c$ are independent since the presence
of $r_a$ in the past is sufficient for enabling $r_a$, while in
$r_b r_c r_a$ , the steps $r_b$ and $r_c$ are not independent. The
paper shows that for left-linear rewriting systems a special form of
globality of independence can be recovered.


The concurrent semantics of linear graph rewriting systems over
adhesive categories can be captured by prime event structures and
prime algebraic domains. The idea consists in considering a partially
ordered domain, where elements are concurrent computations
(derivations up to shift equivalence), and the order relates two
computations when the latter is an extension of the former.
%
Events are then identified with computations consisting of a rewriting
step with all its causes. This is generalised to adhesive categories
in~\cite{heindel2009category}. [IS THIS IN TOBIAS?]

In~\cite{baldan2017domains} it is observed for graph rewriting systems that the presence of fusions, the same event can be enabled by different minimal sets of events, i.e., a central property of the domain of configuration, namely stability, fails. This prevents the identification of a proper notion of causality and makes prime event structures and prime domains inadequate to represent computations with fusions. It is argued that the right class of structures capturing the concurrent behaviour in the presence of fusion are connected event structures and weak prime algebraic domains.
%
An event is still a computation that cannot be decomposed
as the join of other computations, hence, in order theoretical terms,
it is an irreducible.
%
However, due to unstability, irreducibles are not necessarily primes:
two different irreducibles can represent the same computation step
with different minimal enablings.  In~\cite{baldan2017domains} this
is worked out for DPO rewriting systems over graphs. However the proof
relies on a wrong characterisation of independence which fails for non
linear rules. Here we rework in depth the result and extend it to DPO
rewriting over a large class of adhesive categories.